%%%% multifact-cfracs.tex : 
%%%% Article prepared for review at the 
%%%% Journal of Integer Sequences (JIS) in August of 2015l 
%%%% Author: Maxie D. Schmidt (maxieds@gmail.com) 

\documentclass[12pt,reqno]{article} 

\usepackage{amsmath,amsthm} 
\usepackage{amsfonts,amssymb,amscd} 
\usepackage{relsize,mathtools,cancel} 
\usepackage{url} 
\usepackage{fancyhdr} 
\usepackage{graphicx}

%\usepackage[amsfonts,amssymb]{concmath} 
%\usepackage[T1]{fontenc} 
%\usepackage{pbsi} 

%%%% Setup the labeling of subequations, i.e., 
%%%% (1.2.a) instead of (1.2a): 
\renewenvironment{subequations}{%
  \refstepcounter{equation}%
  \edef\theparentequation{\theequation}%
  \setcounter{parentequation}{\value{equation}}%
  \setcounter{equation}{0}%
  \def\theequation{\theparentequation.\alph{equation}}%
  %\def\theequation{\theparentequation\roman{equation}}%
  %\def\theequation{\theparentequation.\arabic{equation}}%
  \ignorespaces
}{%
  \setcounter{equation}{\value{parentequation}}%
  \ignorespacesafterend
}

%%%% Footnotes: 
\newcounter{pagefootnote}%[footnote] 
\setcounter{pagefootnote}{0} 
\newcounter{sfootnote}%[footnote] 
\setcounter{sfootnote}{0} 
\newcommand{\setpagefootnote}[0]{%
     \addtocounter{pagefootnote}{1}
     \ifnum\arabic{pagefootnote}=8\setcounter{pagefootnote}{1}\fi
}
\numberwithin{sfootnote}{section} 
\renewcommand{\thefootnote}{\fnsymbol{pagefootnote}\underline{\bf{F.\arabic{footnote}}}} 
\newcommand{\footnotelabeltext}[1]{%
     \ifthenelse{\equal{#1}{NoNoteTitle}}{}{%
     \addtocounter{sfootnote}{1}%
     {\ \em\bf Note \S\thesection.\arabic{sfootnote}}:\ \emph{#1}.\ \\ \noindent%
     }
} 

\usepackage{xstring}
\newcommand{\footnotemod}[2][NoNoteTitle]{
     \setpagefootnote\footnotemark%
     \footnotetext{\footnotelabeltext{#1}#2}
} 

%%%% Tables: 
\usepackage[font=small,labelfont=bf,textfont=rm,format=hang]{caption,subcaption} 
\captionsetup[subtable]{labelformat=simple} 
\captionsetup[subtable]{labelsep=period,width=0.95\textwidth,format=hang} 
\captionsetup[table]{font=small,labelfont=bf,textfont=rm,width=0.95\textwidth,format=hang} 
\renewcommand{\arraystretch}{1} 
\renewcommand{\thetable}{\thesection.\arabic{table}} 
\renewcommand{\thesubtable}{Table \thetable.\arabic{subtable}} 

\usepackage{rotating} 
\newcommand{\subtablewidth}{\textwidth} 
\newcommand{\subtableskip}{\bigskip} 
\newcommand{\tabletopstrut}[0]{\rule{0pt}{3ex}} 
\newcommand{\tablebottomstrut}[0]{\rule{0pt}{3ex}} 

\usepackage{diagbox} 
\newcommand{\trianglenk}[2]{$_{#1}\diagdown^{k#2}$} 
\renewcommand{\trianglenk}[2]{\smaller{\diagbox{#1}{#2}}} 

\newcommand{\tableref}[1]{Table \ref{#1} (page \pageref{#1})} 
\newcommand{\tablerefII}[1]{Table \ref{#1} starting on page \pageref{#1}} 
\newcommand{\tablerefIII}[2]{Table \ref{#1} starting on page \pageref{#2}} 
\newcommand{\tablerefIV}[1]{Table \ref{#1} on page \pageref{#1}} 
\newcommand{\ftref}[1]{\textsuperscript{\ref{#1}}} 

%%%% Hyperlinks: 
\usepackage[usenames,dvipsnames]{color} 
\usepackage[colorlinks=true,
            linkcolor=oeisblue,
            filecolor=webbrown,
            citecolor=oeis,
            urlcolor=oeis]{hyperref} 
\definecolor{oeis}{rgb}{0.852,0.0469,0.457} 
\definecolor{oeisblue}{rgb}{0.133,0.422,0.586} 
\definecolor{purple}{rgb}{0.7,0.2,0.7} 
\definecolor{penguinbook}{rgb}{0.0196,0.925,0.0196} 

%%%% : Customize the Bibliography and Citations for BibTeX: 
\usepackage[square,semicolon,numbers,sort&compress]{natbib} 
\bibpunct{[}{]}{;}{}{,}{,~} 

%%%% : set paragraph layout / spacing for the document: 
\usepackage[compact]{titlesec} 
\setlength{\parindent}{0.15in} 
\setlength{\parsep}{0in} 
\setlength{\headsep}{0.15in} 
\setlength{\topskip}{0in} 
\setlength{\topmargin}{0in} 
\setlength{\topsep}{0in} 
\setlength{\partopsep}{0in} 
\setlength{\pdfpagewidth}{8.5in} 
\setlength{\pdfpageheight}{11in} 
\setlength{\textwidth}{7in}
\setlength{\oddsidemargin}{-0.25in}
\setlength{\evensidemargin}{-0.25in}
\setlength{\topmargin}{-.65in}
\setlength{\textheight}{9.5in}

%%%% Change Various Labelings and Counter Defines: 
\usepackage{enumitem} 
\numberwithin{equation}{section} 
\newcommand{\tagonce}[0]{
     \addtocounter{equation}{1}
     \tag{\theequation}
} 
\newcommand{\tagtext}[1]{\tag*{\underline{\textit{#1}}}} 
%\renewcommand{\tagtext}[1]{\tag{\textit{#1}}} 

\renewcommand{\labelenumi}{\textbf{\arabic{enumi}.}} 
\newcommand{\itemlabel}[1]{\textbf{#1}: \\ } 
\renewcommand{\labelenumi}{$\mathsmaller{\blacktriangleright}$ } 
\newcommand{\sublabel}[1]{\noindent{\smaller{\underline{\textbf{\textit{#1}}}.\ }} \\ } 
\newcommand{\sublabelII}[1]{\noindent{\smaller{\underline{\textbf{\textit{#1}}}.\ }}\smallskip} 

%%%% AMS Theorem-Related Environments: 
\newtheoremstyle{DefaultTheoremStyle}{\parskip}{\parskip}{}{0in}{}{}{\newline}
                {\textbf{\thmname{#1} \thmnumber{#2}} (\textit{\thmnote{#3}}).} 
\theoremstyle{DefaultTheoremStyle} 
\newtheorem{theorem}{Theorem}[section]
\newtheorem{prop}[theorem]{Proposition}
\newtheorem{cor}[theorem]{Corollary}
\newtheorem{lemma}[theorem]{Lemma}

\theoremstyle{definition}
\newtheorem{definition}[theorem]{Definition}
\newtheorem{remark}[theorem]{Remark}
\newtheorem{example}[theorem]{Example} 
\newtheorem{claim}[theorem]{Claim}

%%%% : Custom Theorem Environment End Markers: 
\usepackage{mboxfill} 
\newcommand{\eolqedsymbol}[1]{{\hrulefill\ensuremath{\ #1}}}
\renewcommand{\eolqedsymbol}[1]{{\mboxfill{ }\ensuremath{\ #1}}}
\renewcommand{\qedsymbol}{$\boxdot$} 

%% NOTE: Modifying and/or removing these custom end marker macros: 
\usepackage{qsymbols}
\newcommand{\DefinitionQEDSymbol}{`(\text{\rm{\bf D}})} 
\newcommand{\RemarkQEDSymbol}{`(\text{\rm{\bf R}})} 
\newcommand{\ExampleQEDSymbol}{`(\text{\rm{\bf E}})} 
\newcommand{\DefinitionQED}{\eolqedsymbol{\DefinitionQEDSymbol}} 
\newcommand{\RemarkQED}{\eolqedsymbol{\RemarkQEDSymbol}} 
\newcommand{\ExampleQED}{\eolqedsymbol{\ExampleQEDSymbol}} 

%%%% : JIS LaTeX Macros: 
\newcommand{\quotetext}[1]{``#1''} 
\newcommand{\keywordemph}[1]{\emph{#1}} 
\newcommand{\cf}[0]{cf.\ } 
\newcommand{\ie}[0]{i.e.,\ } 
\newcommand{\ul}{\underline} 

\newcommand{\seqnum}[1]{\href{http://oeis.org/#1}{\texttt{\underline{#1}}}} 
\newcommand{\OEIS}[1]{{\color{oeis}\texttt{#1}}} 
\newcommand{\OEISII}[1]{{\color{oeisblue}\texttt{#1}}} 
\newcommand{\seqmapsto}[2][]{%
     \xrightarrow[\text{ \OEISII{#1} }]{\text{ \OEISII{#2} }}\quad%
}  

%\usepackage{nowtoaux} 
%\newcommand{\WriteLogFile}[1]{\writeaux{#1}} 

\usepackage{pgffor}
\def\citeOEISGetList#1{%
     \gdef\seqargctr{1}%
     \foreach \seq in {#1}{%
          \ifnum\seqargctr=1[\fi%
          \ifnum\seqargctr=-1; \fi\seqnum{\seq}%
          \gdef\seqargctr{-1}%
     }]%
}
\newcommand{\citeOEIS}[1]{\citeOEISGetList{#1}} 
\newcommand{\OEISRef}[1]{\href{http://oeis.org}{\texttt{\underline{#1}}}} 

\newcommand{\defequals}{\ensuremath{\vcentcolon=}} 
\newcommand{\defmapsto}{\ensuremath{\vcentcolon\mapsto}} 
\newcommand{\undersetbrace}[2]{\ensuremath{\underset{\mathlarger{#1}}{\mathsmaller{\underbrace{#2}}}}} 
\newcommand{\undersetbraceII}[2]{\ensuremath{\underset{#1}{\mathsmaller{\underbrace{#2}}}}} 
\newcommand{\undersetline}[2]{\ensuremath{\underset{\mathlarger{#1}}{\mathsmaller{\underline{#2}}}}} 

\newcommand{\TODO}{\textbf{TODO}} 
\newcommand{\TheSummaryNBFile}[0]{\texttt{multifact-cfracs-summary.nb}}
\newcommand{\TheSummaryNBFileGoogleDriveLink}[0]{https://drive.google.com/file/d/0B6na6iIT7ICZRFltbTVVcmVpVk0/view?usp=drivesdk}  
\renewcommand{\TheSummaryNBFile}{ 
     \href{\TheSummaryNBFileGoogleDriveLink}{\texttt{multifact-cfracs-summary.nb}}%
}
\newcommand{\Mm}[0]{\emph{Mathematica}} 
\newcommand{\SigmaPkg}[0]{%
     \href{http://www.risc.jku.at/research/combinat/software/Sigma/index.php}{%
     \texttt{Sigma}}
}

\newcommand{\tagTODO}[1]{{\bf \color{red}{#1}}} 
\newcommand{\StartGroupingSubEquations}{\begin{subequations}} 
\newcommand{\EndGroupingSubEquations}{\end{subequations}} 

%%%% : Concrete Math book special triangles: 
\newcommand{\gkpSI}[2]{\ensuremath{\genfrac{\lbrack}{\rbrack}{0pt}{}{#1}{#2}}} 
\newcommand{\gkpSII}[2]{\ensuremath{\genfrac{\{}{\}}{0pt}{}{#1}{#2}}} 
\newcommand{\gkpEI}[2]{\ensuremath{\genfrac{\langle}{\rangle}{0pt}{}{#1}{#2}}} 
\newcommand{\gkpEII}[2]{\ensuremath{\left\langle\genfrac{\langle}{\rangle}{
            0pt}{}{#1}{#2}\right\rangle}} 

%%%% : Other custom macros (functions with correct input variable spacings): 
\newcommand{\Fcf}[3]{\ensuremath{F_{#1}\left(#2,\ #3\right)}} 
\newcommand{\FcfII}[3]{\ensuremath{\gkpSI{#2}{#3}_{#1}}} 
\newcommand{\FFact}[3]{\ensuremath{(#1|#2)^{\underline{#3}}}} 
\newcommand{\FFactII}[2]{\ensuremath{#1^{\underline{#2}}}} 
\newcommand{\RFact}[3]{\ensuremath{(#1|#2)^{\overline{#3}}}} 
\newcommand{\RFactII}[2]{\ensuremath{#1^{\overline{#2}}}} 
\newcommand{\Pochhammer}[2]{\ensuremath{\left(#1\right)_{#2}}} 
\newcommand{\Iverson}[1]{\ensuremath{\left[#1\right]_{\delta}}} 
\newcommand{\KDelta}[2]{\ensuremath{\delta_{#1, #2}}} 
\newcommand{\MultiFactorial}[2]{\ensuremath{#1!_{\left(#2\right)}}} 
\newcommand{\AlphaFactorial}[2]{\ensuremath{\left(#1\right)!_{\left(#2\right)}}} 
\newcommand{\pn}[3]{\ensuremath{p_{#1}\left(#2, #3\right)}} 
\newcommand{\HypU}[3]{\ensuremath{U\left(#1, #2, #3\right)}} 
\newcommand{\HypM}[3]{\ensuremath{M\left(#1, #2, #3\right)}} 
\newcommand{\ConvGF}[4]{\ensuremath{\Conv_{#1}\left(#2, #3; #4\right)}} 
\newcommand{\ConvFP}[4]{\ensuremath{\FP_{#1}\left(#2, #3; #4\right)}} 
\newcommand{\ConvFQ}[4]{\ensuremath{\FQ_{#1}\left(#2, #3; #4\right)}} 

\newcommand{\FiGF}[1]{\ensuremath{F_{#1}}} 
\newcommand{\HNumGFFactoredDenomFn}[3]{
     \Denom_{#1, \bmod{#2}}\left\llbracket #3 \right\rrbracket 
} 
\def\?{\hbox{!`}} % subfactorial: $n\textsuperscript{\rotatebox{180}{!}}$, 

%%%% : Standard math function names: 
\DeclareMathOperator{\T}{T} 
\DeclareMathOperator{\U}{U} 
\DeclareMathOperator{\I}{I} 
\DeclareMathOperator{\F}{F} 
\DeclareMathOperator{\Log}{Log} 
\DeclareMathOperator{\FP}{FP} 
\DeclareMathOperator{\FQ}{FQ} 
\DeclareMathOperator{\Conv}{Conv} 
\DeclareMathOperator{\as}{as} 
\DeclareMathOperator{\bs}{bs} 
\DeclareMathOperator{\cs}{cs} 
\DeclareMathOperator{\af}{af} 
\DeclareMathOperator{\bfcf}{bf} 
\DeclareMathOperator{\cfcf}{cf} 
\DeclareMathOperator{\Sf}{sf} 
\DeclareMathOperator{\remainderrem}{rem} 
\DeclareMathOperator{\E}{E} 
\DeclareMathOperator{\Li}{Li} 
\DeclareMathOperator{\Denom}{Denom} 

\DeclareMathOperator{\WT}{WT} 
\DeclareMathOperator{\CT}{CT} 
\DeclareMathOperator{\SPT}{SPT} 
\newcommand{\Wilson}{\mathsmaller{\WT}} 
\newcommand{\Clement}{\mathsmaller{\CT}} 
\newcommand{\SPTriple}{\mathsmaller{\SPT}} 

\DeclareMathOperator{\XT}{T} 
\DeclareMathOperator{\XK}{K} 
\DeclareMathOperator{\XP}{P} 
\newcommand{\xt}{x_{\mathsmaller{\XT}}} 
\newcommand{\xk}{x_{\mathsmaller{\XK}}} 
\newcommand{\xp}{x_{\mathsmaller{\XP}}} 

\DeclareMathOperator{\quot}{quot} 
\newcommand{\WilsonQuotient}[1]{W_{\mathsmaller{\quot}}\left(#1\right)} 

\DeclareMathOperator{\WilsonPrime}{Wilson} 
\DeclareMathOperator{\WolstPrime}{Wolstenholme} 
\DeclareMathOperator{\WieferichPrime}{Wieferich} 
\newcommand{\WilsonPrimeSet}{\mathbb{P}_{\mathsmaller{\WilsonPrime}}} 
\newcommand{\WolstPrimeSet}{\mathbb{P}_{\mathsmaller{\WolstPrime}}} 
\newcommand{\WieferichPrimeSet}{\mathbb{P}_{\mathsmaller{\WieferichPrime}}} 

%%%% Page Header Setup for Drafts: 


\usepackage{lastpage} 
\pagestyle{fancy} 
\fancyhead[LE,LO]{\small{\textbf{Maxie D. Schmidt}\ \hspace{0.05in} 
                  \texttt{maxieds@gmail.com}}} 
\fancyhead[CE,CO]{\hspace{2in} 
                  \small{\textit{Draft Revised: \today}}} 
\fancyhead[RE,RO]{\small{Page \thepage\ of \pageref{LastPage}}}

\usepackage[yyyymmdd]{datetime} 
\renewcommand{\dateseparator}{.} 
\fancyfoot[LE,LO]{\small{\textrm{\today\ @\ \currenttime}}} 


\def\lastworkingpage{1} 
\newcommand{\CheckMarkBox}{\ensuremath{`[\text{\rm{\bf \checkmark}}]}} 
\newcommand{\EmptyBox}{\ensuremath{`[\text{\rm{\bf \ }}]}} 

\fancyfoot[RE,RO]{
   \Large{%
     \ifnum\lastworkingpage>\thepage{\EmptyBox\ \CheckMarkBox}\fi
     \ifnum\lastworkingpage=\thepage{\EmptyBox\ \CheckMarkBox}\fi
     \ifnum\lastworkingpage<\thepage{\smaller\EmptyBox}\fi
   }%
} 

\title{Jacobi Type Continued Fractions for the 
       Ordinary Generating Functions of 
       Generalized Factorial Functions} 
\author{Maxie D. Schmidt \\ 
        \href{mailto:maxieds@gmail.com}{\nolinkurl{maxieds@gmail.com}}} 
\date{} 

\allowdisplaybreaks

\begin{document}

%\tableofcontents 

\maketitle

\begin{abstract} 
The article studies a class of generalized factorial functions 
and symbolic product sequences 
through Jacobi type continued fractions (J-fractions) that 
formally enumerate the divergent ordinary generating functions of 
these sequences. 
The more general definitions of 
these J-fractions extend the known expansions of the 
continued fractions originally proved by Flajolet 
that generate the rising factorial function, or 
Pochhammer symbol, $(x)_n$, at any fixed non--zero 
indeterminate $x \in \mathbb{C}$. 
The rational convergents of these generalized J-fractions 
provide formal power series approximations to the 
ordinary generating functions that enumerate many specific classes of 
factorial--related integer product sequences. 

The article also provides applications to a number of specific identities, 
new integer congruence relations satisfied by 
generalized factorial-related product sequences and the 
$r$-order harmonic numbers, 
restatements of classical congruence properties concerning the 
primality of integer subsequences, among several other notable 
motivating examples as immediate applications of the new results. 
In this sense, the article serves as a 
semi--comprehensive, detailed survey reference that 
introduces applications to many established and 
otherwise well--known combinatorial identities, 
new cases of generating functions for factorial--function--related 
product sequences, and other examples of the generalized 
integer--valued multifactorial, or $\alpha$--factorial, function sequences. 
%% 
The convergent--based generating function techniques illustrated by the 
particular examples cited within the article 
are easily extended to enumerate the 
factorial-like product sequences arising in the context of 
many other specific applications. 

The article is prepared with a more extensive set of 
computational data and software routines to be tentatively 
released as open source software to accompany the examples and 
numerous other applications suggested as topics for future 
research and investigation within the article. 
It is highly encouraged, and expected, that the 
interested reader obtain a copy of the summary notebook reference and 
computational documentation prepared in this format to 
assist with computations in a multitude of special case examples cited as 
particular applications of the new results. 
\end{abstract} 

\section{Introduction} 
\label{Section_project_overview} 
\label{Section_Intro} 

The primary focus of the new results established by this article is to 
enumerate new properties of the 
generalized symbolic product sequences, $\pn{n}{\alpha}{R}$ defined by 
\eqref{eqn_GenFact_product_form}, 
which are generated by the convergents to 
\emph{Jacobi--type continued fractions} (\emph{J--fractions}) 
that represent formal power series expansions of the 
otherwise divergent ordinary generating functions (OGFs) 
for these sequences\footnotemod[Iverson's Convention]{ 
     The use of \emph{Iverson's convention}, 
     $\Iverson{\mathtt{n = k}} \defequals \delta_{n,k}$, 
     is consistent with its usage in the references \citep{GKP}. 
     See \S \ref{Section_Notation_and_Convs} 
     of the article for explanations of other notation and conventions 
     within the article. 
}. 
\begin{align} 
\label{eqn_GenFact_product_form} 
\pn{n}{\alpha}{R} & \defequals 
     \prod\limits_{0 \leq j < n} (R+\alpha j) + \Iverson{n = 0} \\ 
\notag 
     & \phantom{:} = 
     R (R + \alpha) (R + 2\alpha) \times \cdots \times (R + (n-1)\alpha) + 
     \Iverson{n = 0}. 
\end{align} 
%% 
The related integer--valued cases of the multiple factorial sequences, 
or \emph{$\alpha$--factorial} functions, 
$\MultiFactorial{n}{\alpha} \defequals \pn{n}{-\alpha}{n}$, 
of interest in the applications of 
this article are defined recursively for any fixed 
$\alpha \in \mathbb{Z}^{+}$ and $n \in \mathbb{N}$ by the 
following equation \citep[\S 2]{MULTIFACTJIS}: 
\begin{equation} 
\label{eqn_nAlpha_Multifact_variant_rdef} 
\MultiFactorial{n}{\alpha} = 
     \begin{cases} 
     n \cdot (n-\alpha)!_{(\alpha)}, & \text{ if $n > 0$;} \\ 
     1, & \text{ if $-\alpha < n \leq 0$; } \\ 
     0, & \text{ otherwise. } 
     \end{cases} 
\end{equation} 
%% 
The particular new results studied within the article generalize the 
known series proved in the references \citep{FLAJOLET80B,FLAJOLET82}, 
including expansions of the series for generating functions enumerating the 
\emph{rising} and \emph{falling} \emph{factorial} functions, 
$\RFactII{x}{n} = (-1)^{n} \FFactII{(-x)}{n}$ and 
$\FFactII{x}{n} = x! / (x-n)! = \pn{n}{-1}{x}$, and the 
\emph{Pochhammer symbol}, $\Pochhammer{x}{n} = \pn{n}{1}{x}$, 
expanded by the 
\emph{Stirling numbers of the first kind}, $\gkpSI{n}{k}$, as 
\citeOEIS{A130534} 
\footnotemod[Notation for the Rising and Falling Factorial Functions]{ 
     The generalized rising and falling factorial functions 
     denote the products, 
     $\RFact{x}{\alpha}{n} = \Pochhammer{x}{n,\alpha}$ and 
     $\FFact{x}{\alpha}{n} = \Pochhammer{x}{n,-\alpha}$, 
     defined in the reference, where the products, 
     $\Pochhammer{x}{n} = \RFact{x}{1}{n}$ and 
     $\FFactII{x}{n} = \FFact{x}{1}{n}$, 
     correspond to the particular special cases of these functions cited below 
     \citep[\S 2]{MULTIFACTJIS} 
     (see footnote \ftref{footnote_PHkSymb_CvlPolyRef} on page 
     \pageref{footnote_PHkSymb_CvlPolyRef}). 
     %% 
     Note that 
     Roman's \emph{Umbral Calculus} book employs the 
     alternate, and less standard, notation of 
     $\left(\frac{x}{a}\right)_n \defequals 
      \frac{x}{a} \left(\frac{x}{a}-1\right) \times \cdots \times 
      \left(\frac{x}{a}-n+1\right)$ to denote the 
     sequences of \emph{lower factorial polynomials}, and 
     $x^{(n)}$ in place of the Pochhammer symbol 
     to denote the \emph{rising factorial polynomials} 
     \citep[\S 1.2]{UC}. 
} 
\begin{align*} 
\Pochhammer{x}{n} 
     & = 
     x(x+1)(x+2) \cdots (x+n-1) \Iverson{n \geq 1} + \Iverson{n = 0} \\ 
     & = 
     \left(\sum_{k=1}^{n} \gkpSI{n}{k} x^{k}\right) \times \Iverson{n \geq 1} 
     + \Iverson{n = 0}. 
\end{align*} 
%% 
The generalized product sequences in \eqref{eqn_GenFact_product_form} 
also correspond to the definition of the \emph{Pochhammer $\alpha$--symbol}, 
$\Pochhammer{x}{n,\alpha} = \pn{n}{\alpha}{x}$, 
defined as in the references 
\citep{ON-HYPGEOMFNS-PHKSYMBOL} \citep[Examples]{CVLPOLYS} 
\citep[\cf \S 5.4]{GKP} 
for any fixed $\alpha \neq 0$ and non--zero indeterminate, 
$x \in \mathbb{C}$, by the following analogs expansions involving the 
generalized $\alpha$--factorial coefficient triangles defined in 
\eqref{eqn_Fa_rdef} of the next section in this article 
\citep[\S 3]{MULTIFACTJIS}: 
\begin{align*} 
\Pochhammer{x}{n,\alpha} 
     & = 
     x(x+\alpha)(x+2\alpha) \cdots (x+ (n-1) \alpha) \Iverson{n \geq 1} + 
     \Iverson{n = 0} \\ 
     & = 
     \left(\sum_{k=1}^{n} \gkpSI{n}{k} \alpha^{n-k} x^{k}\right) \times 
     \Iverson{n \geq 1} + \Iverson{n = 0} \\ 
     & = 
     \left(\sum_{k=0}^{n} \FcfII{\alpha}{n+1}{k+1} (x-1)^{k}\right) \times 
     \Iverson{n \geq 1} + \Iverson{n = 0}. %\\ 
\end{align*} 
We are especially interested in using the new results established in this 
article to formally enumerate the 
factorial--function--like product sequences, 
$\pn{n}{\alpha}{R}$ and $\pn{n}{\alpha}{\beta n + \gamma}$, for some fixed 
parameters $\alpha, \beta, \gamma \in \mathbb{Q}$, when the 
initially fixed symbolic indeterminate, $R$, depends linearly on $n$. 
The particular forms of the 
generalized product sequences of interest in the 
applications of this article are related to the \emph{Gould polynomials}, 
$G_n(x; a, b) = \frac{x}{x-an} \cdot \FFactII{\left(\frac{x-an}{b}\right)}{n}$, 
in the form of the following equation 
\citep[\S 3.4.2]{MULTIFACTJIS} \citep[\S 4.1.4]{UC}: 
\begin{align} 
\label{eqn_pnAlphaBetanpGamma_GouldPolyExp_Ident-stmt_v1} 
\pn{n}{\alpha}{\beta n + \gamma} & = 
     \frac{(-\alpha)^{n+1}}{\gamma-\alpha-\beta} \times 
     G_{n+1}\left(\gamma-\alpha-\beta; -\beta, -\alpha\right). 
\end{align} 
%% 
Whereas the first results proved in the first articles 
\citep{FLAJOLET80B,FLAJOLET82} 
are focused on establishing properties of divergent forms of the 
ordinary generating functions for a number of special sequence cases 
through more combinatorial interpretations of these 
continued fraction series, the emphasis in this article 
is more enumerative in flavor. 
%% 
The new identities involving the integer--valued cases of the 
multiple, $\alpha$--factorial functions, $\MultiFactorial{n}{\alpha}$, 
defined in \eqref{eqn_nAlpha_Multifact_variant_rdef} 
obtained by this article 
extend the study of these sequences motivated by the 
distinct symbolic polynomial expansions of these functions 
originally considered in the reference \citep{MULTIFACTJIS}. 
This article extends a number of the examples considered as 
applications of the results from the 2010 article %\citep{MULTIFACTJIS} 
briefly summarized in the next section. 

\subsection{Polynomial expansions of generalized $\alpha$--factorial functions} 
\label{subSection_GenAlphaFactorialTriangle_exps} 

For any fixed integer $\alpha \geq 1$ and $n, k \in \mathbb{N}$, the 
coefficients defined by the triangular recurrence relation in 
\eqref{eqn_Fa_rdef} provides one approach to enumerating the 
symbolic polynomial expansions of the 
generalized factorial function product sequences 
defined as special cases of \eqref{eqn_GenFact_product_form} 
defined above 
(see \tablerefIV{table_FcfAlphankCoeffs}). 
\begin{equation} 
\label{eqn_Fa_rdef} 
\FcfII{\alpha}{n}{k} = (\alpha n+1-2\alpha)\FcfII{\alpha}{n-1}{k} + 
     \FcfII{\alpha}{n-1}{k-1} + \Iverson{n = k = 0} 
\end{equation} 
The combinatorial interpretations of these coefficients 
motivated in the reference \citep{MULTIFACTJIS} leads to polynomial 
expansions in $n$ of the multiple factorial function sequence variants in 
\eqref{eqn_nAlpha_Multifact_variant_rdef} that generalize the 
known formulas for the 
single and double factorial functions, $n!$ and $n!!$, 
involving the Stirling numbers of the first kind, 
$\gkpSI{n}{k} = \FcfII{1}{n}{k} = (-1)^{n-k} s(n, k)$, 
expanded in the forms of the following equations 
\citep[\S 6]{GKP} \citep[\S 26.8]{NISTHB} 
\citeOEIS{A000142,A006882}
\footnotemod[Other Finite Sums Involving the Stirling and Eulerian Number Triangles]{ 
     The Stirling numbers of the first kind similarly provide 
     non--polynomial exact 
     finite sum formulas for the single and double factorial 
     functions in the following forms for $n \geq 1$ 
     where $(2n)!! = 2^{n} \times n!$ 
     \citep[\S 6.1]{GKP} \citep[\S 5.3]{DBLFACTFN-COMBIDENTS-SURVEY}: 
     \begin{align*} 
     n! & = 
     \mathsmaller{ 
          \sum\limits_{k=0}^{n} \gkpSI{n}{k} 
     }  
     \quad \text{ and } \quad 
     (2n-1)!! = 
     \mathsmaller{ 
          \sum\limits_{k=0}^{n} \gkpSI{n}{k} 2^{n-k} 
     }. 
     \end{align*} 
     Related finite sums generating the 
     single factorial, double factorial, and 
     $\alpha$--factorial functions are expanded respectively through the 
     \emph{first--order Euler numbers}, $\gkpEI{n}{m}$, as 
     $n! = \sum_{k=0}^{n-1} \gkpEI{n}{k}$ \citep[\S 26.14(iii)]{NISTHB}, 
     through the \emph{second--order Euler numbers}, $\gkpEII{n}{m}$, as 
     $(2n-1)!! = \sum_{k=0}^{n-1} \gkpEII{n}{k}$ \citep[\S 6.2]{GKP}, and 
     by the generalized cases of these triangles defined in the 
     reference \citep[\S 6.2.4]{MULTIFACTJIS}. 
}: 
\StartGroupingSubEquations
\label{eqn_AlphaNm1pd_AlphaFactFn_PolyCoeffSum_Exp_formula-v1} 
\begin{align*} 
n! & = \sum_{m=0}^{n} \gkpSI{n}{m} (-1)^{n-m} n^m,\ \forall n \geq 1 \\ 
n!! & = \sum_{m=0}^{n} \gkpSI{\lfloor \frac{n+1}{2} \rfloor}{m} 
     (-2)^{\lfloor \frac{n+1}{2} \rfloor - m} n^{m},\ \forall n \geq 1 \\ 
\tagonce\label{eqn_AlphaNm1pd_AlphaFactFn_PolyCoeffSum_Exp_formula-eqns_v3} 
%\MultiFactorial{n}{\alpha} & = 
%     \sum_{m=0}^{n} 
%     \gkpSI{\lfloor \frac{n-1+\alpha}{\alpha} \rfloor}{m} 
%     (-\alpha)^{\lfloor \frac{n-1+\alpha}{\alpha} \rfloor - m} n^{m} \\ 
\MultiFactorial{n}{\alpha} & = 
     \sum_{m=0}^{n} 
     \gkpSI{\lceil n / \alpha \rceil}{m} 
     (-\alpha)^{\lceil \frac{n}{\alpha} \rceil - m} n^{m},\ 
     \forall n \geq 1, \alpha \in \mathbb{Z}^{+}. 
\end{align*} 
The polynomial expansions of the first two classical sequences 
in the previous equations are then generalized to the 
more general $\alpha$--factorial function cases 
through the triangles defined as in \eqref{eqn_Fa_rdef} 
from the reference \citep{MULTIFACTJIS} through the 
next explicit finite sum formulas when $n \geq 1$. 
\begin{align*} 
\tagonce\label{eqn_AlphaNm1pd_AlphaFactFn_PolyCoeffSum_Exp_formula-eqns_v1} 
\MultiFactorial{n}{\alpha} & = 
     \sum_{m=0}^{n} 
     \FcfII{\alpha}{\lfloor \frac{n-1+\alpha}{\alpha} \rfloor + 1}{m+1} 
     (-1)^{\lfloor \frac{n-1+\alpha}{\alpha} \rfloor - m} 
     (n+1)^{m},\ 
     \forall n \geq 1, \alpha \in \mathbb{Z}^{+} 
\end{align*} 
The polynomial expansions in $n$ of the generalized 
$\alpha$--factorial functions, $\AlphaFactorial{\alpha n-d}{\alpha}$, 
for fixed $\alpha \in \mathbb{Z}^{+}$ and $0 \leq d < \alpha$, 
are obtained similarly from 
\eqref{eqn_AlphaNm1pd_AlphaFactFn_PolyCoeffSum_Exp_formula-eqns_v1} 
through the generalized coefficients in 
\eqref{eqn_Fa_rdef} as follows \citep[\cf \S 2]{MULTIFACTJIS}: 
\begin{align*} 
\tagonce\label{eqn_AlphaNm1pd_AlphaFactFn_PolyCoeffSum_Exp_formula-eqns_v2} 
(\alpha n-d)!_{(\alpha)} 
     & = 
     (\alpha - d) \times 
     \sum_{m=1}^{n} \FcfII{\alpha}{n}{m} (-1)^{n-m} 
     (\alpha n + 1 - d)^{m-1} \\ 
%% 
%(\alpha n-d)!_{(\alpha)} 
     & = 
     \phantom{(\alpha - d) \times} 
     \sum_{m=0}^{n} \FcfII{\alpha}{n+1}{m+1} (-1)^{n-m} 
     (\alpha n + 1 - d)^{m},\ 
     \forall n \geq 1, \alpha \in \mathbb{Z}^{+}, 0 \leq d < \alpha. 
\end{align*} 
\EndGroupingSubEquations
A binomial--coefficient--themed phrasing of the products underlying the 
expansions of the more general factorial function sequences of this type 
(each formed by dividing through by a normalizing factor of $n!$) 
is suggested by the next expansions of these coefficients by 
the Pochhammer symbol \citep[\S 5]{GKP}: 
\begin{align} 
\label{eqn_BinomCoeff_AlphaNm1pd_AlphaFactFn_Exp_formula-v2} 
\binom{\frac{s-1}{\alpha}}{n} & = 
     \frac{(-1)^{n}}{n!} \cdot \Pochhammer{\frac{s-1}{\alpha}}{n} = 
     \frac{1}{\alpha^{n} \cdot n!} \prod_{j=0}^{n-1} \left( 
     s - 1 -\alpha j 
     \right). 
\end{align} 
When the initially fixed indeterminate $s \defequals s_n$ 
is considered modulo $\alpha$ in the form of 
$s_n \defequals \alpha n + d$ for some fixed least integer residue, 
$0 \leq d < \alpha$, the 
prescribed setting of this offset $d$ completely determines the 
numerical $\alpha$--factorial function sequences of the forms in 
\eqref{eqn_AlphaNm1pd_AlphaFactFn_PolyCoeffSum_Exp_formula-eqns_v2} 
generated by these products 
(see the examples cited below in 
Section \ref{subsubSection_Intro_Examples_Fact-RelatedSeqs_GenByTheConvFns} 
and the tables in the reference 
\citep[\cf \S 6.1.2, Table 6.1]{MULTIFACTJIS}). 

For any lower index $n \geq 1$, the 
binomial coefficient formulation of the multiple factorial function 
products in 
\eqref{eqn_BinomCoeff_AlphaNm1pd_AlphaFactFn_Exp_formula-v2} 
provides the next several expansions by the 
exponential generating functions %(EGF) 
for the generalized coefficient triangles in \eqref{eqn_Fa_rdef}, and 
their corresponding generalized 
Stirling polynomial analogues, $\sigma_k^{(\alpha)}(x)$, 
defined in the references 
\citep[\S 5]{MULTIFACTJIS} \citep[\cf \S 6, \S 7.4]{GKP}
\footnotemod[Generalized Stirling and $\alpha$--Factorial Polynomial Sequences]{ 
     The generalized forms of the 
     \emph{Stirling convolution polynomials}, $\sigma_n(x)$, and the 
     \emph{$\alpha$--factorial polynomials}, $\sigma_n^{(\alpha)}(x)$, 
     studied in the reference are defined for each $n \geq 0$ by the 
     triangle in \eqref{eqn_Fa_rdef} as follows 
     \citep[\S 5.2]{MULTIFACTJIS}: 
     \begin{align*} 
     \tagtext{Generalized Stirling Polynomials} 
     x \cdot \sigma_n^{(\alpha)}(x) & \defequals 
          \FcfII{\alpha}{x}{x-n} \frac{(x-n-1)!}{(x-1)!} = 
          [z^n] \left( 
          e^{(1-\alpha) z} \left(\frac{\alpha z e^{\alpha z}}{e^{\alpha z}-1} 
          \right)^{x} \right). 
     \end{align*} 
     \tableref{table_GenStirlingAlphaCvlPolys} 
     provides listings of the first several examples of 
     these polynomials and the corresponding special case 
     corresponding to the Stirling polynomials, $\sigma_n(x)$, 
     when $\alpha \defequals 1$. 
}: 
\StartGroupingSubEquations 
\begin{align} 
\label{eqn_binom_FaSPoly_exp_ident} 
\binom{\frac{s-1}{\alpha}}{n} & = 
     \sum_{m=0}^{n} 
     \FcfII{\alpha}{n+1}{n+1-m} \frac{(-1)^{m} s^{n-m}}{\alpha^{n} n!} \\ 
\notag 
     & = 
     \sum_{m=0}^{n} 
     \frac{(-1)^{m} \cdot (n+1) \sigma_m^{(\alpha)}(n+1)}{\alpha^{m}} \times 
     \frac{\left(s / \alpha\right)^{n-m}}{(n-m)!} \\ 
\label{eqn_fx_EGF_result} 
\binom{\frac{s-1}{\alpha}}{n} & = 
     %& = 
     [z^n] \left( 
     e^{(s-1+\alpha) z / \alpha} \left( 
     \frac{-z e^{-z}}{e^{-z} - 1} 
     \right)^{n+1} 
     \right) \\ 
\notag 
     & = 
     [z^n w^n] \left( 
     - \frac{z \cdot e^{(s-1+\alpha) z / \alpha}}{1+wz-e^{z}} 
     \right). 
\end{align} 
\EndGroupingSubEquations 
A more extensive treatment of the properties and generating function 
relations satisfied by the triangular coefficients defined by 
\eqref{eqn_Fa_rdef}, including their similarities to the 
Stirling number triangles, Stirling polynomial sequences, and the 
generalized Bernoulli polynomials, among relations to several other notable 
special sequences, is provided in the references. 
%\citep{MULTIFACTJIS,GKP,CVLPOLYS,UC}. 
%% 
The results in 
Section \ref{subsubSection_MoreGeneralExps_congruences_multiple_factfns} and 
Section \ref{subsubSection_S1TripleSums_GenSPolyExps} 
of the article suggest further related 
applications of the generalized forms of 
these triangles enumerated in the references. 

\subsection{Divergent ordinary generating functions 
            approximated by the convergents to 
            infinite Jacobi--type and Stieltjes--type 
            continued fraction expansions} 
\label{subSection_Intro_Examples_DivergentCFracOGFs} 

\subsubsection{Infinite J--fraction expansions generating the 
               rising factorial function} 

Another approach to enumerating the symbolic expansions of the 
generalized $\alpha$--factorial function 
sequences outlined above is constructed 
as a new generalization of the continued fraction series 
representations of the ordinary generating function for the 
rising factorial function, or Pochhammer symbol, 
$\Pochhammer{x}{n} = \Gamma(x+n) / \Gamma(x)$, 
proved by Flajolet in the references \citep{FLAJOLET80B,FLAJOLET82}. 
For any fixed non--zero indeterminate, $x \in \mathbb{C}$, the 
ordinary power series enumerating the rising factorial 
sequence is defined through the 
next infinite Jacobi--type J--fraction 
expansion \citep[\S 2, p.\ 148]{FLAJOLET80B}: 
\begin{equation} 
\label{eqn_PochhammerSymbol_InfCFrac_series_Rep_example-v1} 
R_0(x, z) : = 
     %\sum_{n \geq 0} (x)_n z^n = 
     %\cfrac{1}{1-xz-\cfrac{1 \cdot x z^2}{1-(x+2)z-
     %\cfrac{2(x+1)z^2}{\cdots}.}} 
     \sum_{n \geq 0} (x)_n z^n = 
     \cfrac{1}{1-xz-\cfrac{1 \cdot x z^2}{1-(x+2)z-
     \cfrac{2(x+1)z^2}{1-(x+4) z - \cfrac{3(x+2) z^2}{\cdots}.}}} 
\end{equation} 
Since we know symbolic polynomial expansions of the functions, $(x)_n$, 
through the Stirling numbers of the first kind, 
we notice that the terms in a convergent 
power series defined by 
\eqref{eqn_PochhammerSymbol_InfCFrac_series_Rep_example-v1} 
correspond to the coefficients of the following well--known two--variable 
\quotetext{\emph{double}}, or \quotetext{\emph{super}}, 
exponential generating functions (EGFs) 
for the Stirling number triangle 
when $x$ is taken to be a fixed, formal parameter 
with respect to these series 
\citep[\S 7.4]{GKP} \citep[\S 26.8(ii)]{NISTHB} 
\citep[\cf Prop.\ 9]{FLAJOLET80B}
\footnotemod[EGFs for the Generalized $\alpha$--Factorial Function Coefficient Triangles]{ 
     For natural numbers $m \geq 1$ and fixed $\alpha \in \mathbb{Z}^{+}$, the 
     coefficients defined by the generalized triangles in 
     \eqref{eqn_Fa_rdef} are enumerated similarly by the 
     generating functions \citep[\cf \S 3.3]{MULTIFACTJIS} 
     (see also \eqref{eqn_FcfIIAlphanp1mp1_two-variable_EGFwz} in 
     Section \ref{subsubSection_MoreGeneralExps_congruences_multiple_factfns}) 
     \begin{align*} 
     \tagonce\label{eqn_FcfIIAlphanp1mp1_two-variable_EGFwz-footnote_v1} 
     \sum_{m,n \geq 0} \FcfII{\alpha}{n+1}{m+1} \frac{w^m z^n}{n!} & = 
          (1- \alpha z)^{-(w+1) / \alpha}. %\\ 
     \end{align*} 
}: 
\begin{align*} 
\sum_{n \geq 0} \Pochhammer{x}{n} \frac{z^n}{n!} & = \frac{1}{(1-z)^{x}} 
     \qquad \text{ and } \qquad 
\sum_{n \geq 0} \FFactII{x}{n} \cdot \frac{z^{n}}{n!} = (1+z)^{x}. 
\end{align*} 
When $x$ depends linearly on $n$, the ordinary generating 
functions for the numerical factorial functions formed by 
$(x)_n$ do not converge for $z \neq 0$. 
However, the convergents of the continued fraction representations of 
these series still lead to partial, truncated series approximations 
enumerating these generalized product sequences, 
which in turn immediately satisfy a number of 
combinatorial properties, recurrence relations, and other 
established integer congruence properties implied by the 
rational convergents to the first continued fraction expansion given in 
\eqref{eqn_PochhammerSymbol_InfCFrac_series_Rep_example-v1}. 

\subsubsection{Examples} 

Two particular divergent ordinary generating functions 
for the single factorial function sequences, 
$f_1(n) \defequals n!$ and $f_2(n) \defequals (n+1)!$, 
are cited in the references as examples of the 
Jacobi--type J--fraction results proved in 
Flajolet's articles 
\citep{FLAJOLET80B,FLAJOLET82} \citep[\cf \S 5.5]{GFLECT}. 
The next pair of series expansions serve to illustrate the utility to 
enumerating each sequence formally with respect to $z$ 
required by the results in this article 
\citep[Thm. 3A; Thm. 3B]{FLAJOLET80B}. 
\begin{align*} 
\tagtext{Single Factorial J-Fractions} 
F_{1,\infty}(z) & \defequals 
     \sum_{n \geq 0} n! \cdot z^n && = 
     \cfrac{1}{1-z-\cfrac{1^2 \cdot z^2}{ 
     1-3z-\cfrac{2^2 z^2}{\cdots}}} \\ 
F_{2,\infty}(z) & \defequals 
     \sum_{n \geq 0} (n+1)! \cdot z^n && = 
     \cfrac{1}{1-2z-\cfrac{1 \cdot 2 z^2}{ 
     1-4z-\cfrac{2 \cdot 3 z^2}{\cdots}}} 
\end{align*} 
In each of these respective formal power series expansions, 
we immediately see that for each finite $h \geq 1$, the 
$h^{th}$ convergent functions, denoted $F_{i,h}(z)$ for $i = 1,2$, 
satisfy $f_i(n) = [z^n] F_{i,h}(z)$ whenever $1 \leq n \leq 2h$. 
We also have that 
$[z^n] F_{i,h}(z) \equiv f_i(n) \pmod{p}$ 
for any $n \geq 0$ whenever $p$ is a divisor of $h$ 
\citep{FLAJOLET82} \citep[\cf \S 5]{GFLECT}. 

Similar expansions of other factorial--related 
continued fraction series are given in the references 
\citep{FLAJOLET80B} \citep[\cf \S 5.9]{GFLECT}. 
For example, the next known 
\emph{Stieltjes--type} continued fractions (\emph{S--fractions}), 
formally generating the double factorial function, 
$(2n-1)!!$, and the \emph{Catalan numbers}, 
$C_n$, respectively, 
are expanded through the convergents of the following 
infinite continued fractions 
(see Section \ref{subsubSection_footnote_CatalanNumber_S-Fraction_Apps}) 
\citep[Prop.\ 5; Thm.\ 2]{FLAJOLET80B} \citep[\S 5.5]{GFLECT} 
\citeOEIS{A001147,A000108}: 
\begin{align*} 
\tagtext{Double Factorial S-Fractions} 
F_{3,\infty}(z) & \defequals 
     \sum_{n \geq 0} \undersetbrace{(2n-1)!!}{1 \cdot 3 \cdots (2n-1)} 
     \times z^{2n} && = 
     \cfrac{1}{
     1-\cfrac{1 \cdot z^2}{ 
     1-\cfrac{2 \cdot z^2}{
     1-\cfrac{3 \cdot z^2}{\cdots}}}} \\ 
F_{4,\infty}(z) & \defequals 
     \sum_{n \geq 0} \undersetbrace{C_n = \binom{2n}{n} \frac{1}{(n+1)}}{
     \frac{2^{n} (2n-1)!!}{(n+1)!}} 
     \times z^{2n} && = 
     \cfrac{1}{
     1-\cfrac{z^2}{ 
     1-\cfrac{z^2}{
     1-\cfrac{z^2}{\cdots}.}}} 
\end{align*} 
For comparison, some related forms of 
regularized ordinary power series in $z$ generating the 
single and double factorial function sequences from the 
previous examples are stated in terms of the 
\emph{incomplete gamma function}, 
$\Gamma(a, z) = \int_z^{\infty} t^{a-1} e^{-t} dt$, 
as follows \citep[\S 8; \cf \S 18.5--18.6]{NISTHB}
\footnotemod[Remarks on Laplace--Borel Transformations and 
             Regularized Sums for $\alpha$--Factorial Function OGFs]{ 
     Since $\pn{n}{\alpha}{R} = \alpha^{n} \Pochhammer{R / \alpha}{n}$, the 
     exponential generating function for the 
     generalized product sequences corresponds to the series 
     \citep[\cf \S 7.4, eq.\ (7.55)]{GKP} \citep{CVLPOLYS} 
     \begin{align*} 
     \tagtext{Generalized Product Sequence EGFs} 
     \widehat{P}(\alpha, R; z) & \defequals 
          \sum_{n=0}^{\infty} \pn{n}{\alpha}{R} \frac{z^n}{n!} = 
          (1-\alpha z)^{-R / \alpha}, 
     \end{align*} 
     where for each fixed $\alpha \in \mathbb{Z}^{+}$ and $0 \leq r < \alpha$, 
     we have the identities, 
     $\AlphaFactorial{\alpha n-r}{\alpha} = \pn{n}{\alpha}{\alpha - r} = 
      \alpha^{n} \Pochhammer{1 - \frac{r}{\alpha}}{n}$. 
     The form of this exponential generating function 
     leads to the next forms of the regularized sums 
     (see Remark \ref{remark_Formal_Laplace-Borel_Transforms}) 
     \citep[\cf \S 8.6(i)]{NISTHB}. 
     \begin{align*} 
     \widetilde{B}_{\alpha,-r}(z) & \defequals 
     \sum_{n \geq 0} \AlphaFactorial{\alpha n - r}{\alpha} z^n = 
     %\tagtext{Borel Regularized Sums} 
          %& \phantom{:} = 
     \int_0^{\infty} \frac{e^{-t}}{(1-\alpha tz)^{1 - r / \alpha}} dt = 
     \frac{e^{-\frac{1}{\alpha z}}}{(-\alpha z)^{1 - r / \alpha}} \times 
          \Gamma\left(\frac{r}{\alpha}, -\frac{1}{\alpha z}\right) 
     \end{align*} 
}: 
\begin{align*} 
\sum_{n \geq 0} n! \cdot z^{n} & = 
     -\frac{e^{-1/z}}{z} \times 
     \Gamma\left(0, -\frac{1}{z}\right) \\ 
\sum_{n \geq 0} (n+1)! \cdot z^{n} & = 
     -\frac{e^{-1/z}}{z^2} \times 
     \Gamma\left(-1, -\frac{1}{z}\right) \\ 
\tagonce\label{eqn_RelatedFormsOf_SgAndDblFact_OGFs-intro_examples-stmts_v1} 
\sum_{n \geq 1} (2n-1)!! \cdot z^{n} & = 
     -\frac{e^{-1 / 2 z}}{(-2z)^{3/2}} \times 
     \Gamma\left(-\frac{1}{2}, -\frac{1}{2 z}\right). 
\end{align*} 
The remarks given in 
Section \ref{subSection_AltExps_of_the_GenConvFns} 
suggest similar approximations to the 
$\alpha$--factorial functions generated by the 
generalized convergent functions defined in the next section, and their 
relations to the confluent hypergeometric functions and the 
associated Laguerre polynomial sequences 
\citep[\cf \S 18.5(ii)]{NISTHB} \citep[\S 4.3.1]{UC}. 

\subsection{Generalized convergent functions enumerating 
            factorial--related integer product sequences} 
\label{subSection_Intro_GenConvFn_Defs_and_Properties} 

\subsubsection{Definitions of the generalized J--fraction expansions and 
               the generalized convergent function series} 

We state the next definition as a 
generalization of the result for the 
rising factorial function due to Flajolet cited in 
\eqref{eqn_PochhammerSymbol_InfCFrac_series_Rep_example-v1} 
to form the analogs series enumerating the 
multiple, $\alpha$--factorial product 
sequence cases defined by \eqref{eqn_GenFact_product_form} and 
\eqref{eqn_nAlpha_Multifact_variant_rdef}. 

\begin{definition}[Generalized J-Fraction Convergent Functions] 
\label{def_GenConvFns_PFact_Phz_eqn_QFact_Qhz-defs_intro_v1} 
%% 
Suppose that the parameters 
$h \in \mathbb{N}$, $\alpha \in \mathbb{Z}^{+}$ and $R \defequals R(n)$ 
are defined in the notation of the product--wise sequences from 
\eqref{eqn_GenFact_product_form}. 
For $h \geq 0$ and $z \in \mathbb{C}$, let the component 
numerator and denominator convergent functions, denoted  
$\FP_h(\alpha, R; z)$ and $\FQ_h(\alpha, R; z)$, respectively, 
be defined by the next equations.
\StartGroupingSubEquations 
\begin{align} 
\label{eqn_PFact_Phz} 
\FP_h(\alpha, R; z) & \defequals 
     \begin{cases} 
     \mathsmaller{(1-(R+2\alpha(h-1))z)\FP_{h-1}(\alpha, R; z)- 
     \alpha(R+\alpha(h-2))(h-1) z^2 \FP_{h-2}(\alpha, R; z)}, & 
     \text{if $h \geq 2$;} \\ 
     1, & \text{if $h = 1$;} \\ 
     0, & \text{otherwise.} 
     \end{cases} \\ 
\label{eqn_QFact_Qhz} 
\FQ_h(\alpha, R; z) & \defequals 
     \begin{cases} 
     \mathsmaller{(1-(R+2\alpha(h-1))z)\FQ_{h-1}(\alpha, R; z)- 
     \alpha(R+\alpha(h-2))(h-1) z^2 \FQ_{h-2}(\alpha, R; z)}, & 
     \text{if $h \geq 2$;} \\ 
     1-Rz, & \text{if $h = 1$;} \\ 
     1, & \text{if $h = 0$;} \\ 
     0, & \text{otherwise.} 
     \end{cases} 
\end{align} 
\EndGroupingSubEquations 
The corresponding convergent functions, 
$\ConvGF{h}{\alpha}{R}{z}$, defined in the 
next equation provide the 
rational, formal power series approximations in $z$ to the 
divergent ordinary generating functions of many factorial--related sequences 
formed as special cases of the symbolic products in 
\eqref{eqn_GenFact_product_form}. 
\begin{align} 
\notag 
\ConvGF{h}{\alpha}{R}{z} & \phantom{:} = 
     \cfrac{1}{1 - R \cdot z - 
     \cfrac{\alpha R \cdot z^2}{ 
            1 - (R+2\alpha) \cdot z -
     \cfrac{2\alpha (R + \alpha) \cdot z^2}{ 
            1 - (R + 4\alpha) \cdot z - 
     \cfrac{3\alpha (R + 2\alpha) \cdot z^2}{ 
     \cfrac{\cdots}{1 - (R + 2 (h-1) \alpha) \cdot z}}}}} \\ 
\label{eqn_ConvGF_notation_def} 
\ConvGF{h}{\alpha}{R}{z} & \defequals 
     \frac{\FP_h(\alpha, R; z)}{\FQ_h(\alpha, R; z)} = 
     \sum_{n=0}^{2h-1} p_n(\alpha, R) z^n + 
     \sum_{n=2h}^{\infty} \widetilde{e}_{h,n}(\alpha, R) z^n 
\end{align} 
The first series coefficients on the right--hand--side of 
\eqref{eqn_ConvGF_notation_def} enumerate the products, 
$p_n(\alpha, R)$, from \eqref{eqn_GenFact_product_form}, where the 
remaining forms of the power series coefficients, 
$\widetilde{e}_{h,n}(\alpha, R)$, 
correspond to \quotetext{error terms} in the 
truncated formal series approximations to the 
exact sequence generating functions 
obtained from these convergent functions, and where 
$\pn{n}{\alpha}{R} \equiv \widetilde{e}_{h,n}(\alpha, R) \pmod{h}$ 
for all $h \geq 2$ and $n \geq 2h$. 
%% 
\DefinitionQED 
\end{definition} 

\subsubsection{Properties of the generalized J--fraction convergent functions} 

%% 
A number of the immediate, noteworthy properties satisfied by these 
generalized convergent functions 
are apparent from inspection of the first few special cases provided in 
\tableref{table_SpCase_Listings_Of_PhzQhz_ConvFn} and in 
\tableref{table_RelfectedConvNumPolySeqs_sp_cases}. 
The most important of these properties relevant 
to the new interpretations of the 
$\alpha$--factorial function sequences proved in the next sections of the 
article are briefly summarized in the points stated below. 

\begin{enumerate} 

\item \itemlabel{Rationality of the convergent functions in $\alpha$, $R$, and $z$} 
For any fixed $h \geq 1$, it is easy to show that the 
component convergent functions, $\FP_h(z)$ and $\FQ_h(z)$, 
defined by \eqref{eqn_PFact_Phz} and \eqref{eqn_QFact_Qhz}, respectively, 
are polynomials of finite degree in each of $z$, $R$, and $\alpha$ 
satisfying 
\begin{equation} 
\notag 
\deg_{z,R,\alpha}\bigl\{ \FP_h(\alpha, R; z) \bigr\} = h-1 
     \quad \text{ and } \quad 
\deg_{z,R,\alpha}\bigl\{ \FQ_h(\alpha, R; z) \bigr\} = h. 
\end{equation} 
For any $h, n \in \mathbb{Z}^{+}$, if $R \defequals R(n)$ 
denotes some linear function of $n$, the product sequences, 
$p_n(\alpha, R)$, 
generated by the generalized convergent functions 
always correspond to polynomials in $n$ (in $R$) 
of predictably finite degree with integer coefficients determined by the 
choice of $n \geq 1$ expanded by the results in 
Section \ref{subSection_FiniteDiffEqns_for_the_GenFactFns}. 

\item \itemlabel{Expansions of the convergent functions by 
                 special functions} 
%% 
For all $h \geq 0$, and fixed non--zero parameters $\alpha$ and $R$, the 
power series in $z$ generated by the generalized $h^{th}$ 
convergents, $\ConvGF{h}{\alpha}{R}{z}$, are characterized by the 
representations of the convergent denominator functions, 
$\FQ_h(\alpha, R; z)$, through the 
confluent hypergeometric functions, 
$\HypU{a}{b}{w}$ and $\HypM{a}{b}{w}$, and the 
associated Laguerre polynomial sequences, $L_n^{(\beta)}(x)$, 
as follows \citep[\S 13; \S 18]{NISTHB} \citep[\S 4.3.1]{UC}: 
\begin{align*} 
\tagonce\label{eqn_PFact_Qhz_Exp_idents-stmts_v1} 
\undersetbrace{\widetilde{\FQ}_h(\alpha, R; z)}{ 
     z^{h} \cdot \FQ_h\left(\alpha, R; z^{-1}\right) 
} & = 
     \alpha^{h} \times \HypU{-h}{\frac{R}{\alpha}}{\frac{z}{\alpha}} \\ 
     & = 
     (-\alpha)^{h} \Pochhammer{R / \alpha}{h} \times 
     \HypM{-h}{\frac{R}{\alpha}}{\frac{z}{\alpha}} \\ 
     & = 
     (-\alpha)^{h} \cdot h! \times 
     L_h^{(R / \alpha - 1)}\left(\frac{z}{\alpha}\right). 
\end{align*} 
The special function expansions of the reflected convergent denominator 
function sequences above lead to the statements of 
addition theorems, multiplication theorems, and 
several additional auxiliary recurrence relations for these functions 
proved in Section \ref{subsubSection_Properties_Of_ConvFn_Qhz}. 

\item \itemlabel{Corollaries: 
       New exact formulas and congruence properties for the 
       $\alpha$--factorial functions and the generalized product sequences} 
If some ordering of the $h$ zeros of 
\eqref{eqn_PFact_Qhz_Exp_idents-stmts_v1} is fixed at each $h \geq 1$, 
we can define the next sequences 
forming special cases the zeros studied in the references 
\citep{LGWORKS-ASYMP-SPFNZEROS2008,PROPS-ZEROS-CHYPFNS80}. 
In particular, each of the following special zero sequence definitions 
provide factorizations over $z$ of the denominator sequences, 
$\FQ_h(\alpha, R; z)$, parameterized by $\alpha$ and $R$: 
\begin{align} 
\tagtext{Special Function Zeros} 
\left( \ell_{h,j}(\alpha, R) \right)_{j=1}^{h} & \defequals 
     \left\{ z_j : 
     \alpha^{h} \times \HypU{-h}{R / \alpha}{\frac{z}{\alpha}} = 0,\ 
     1 \leq j \leq h 
     \right\} \\ 
\notag 
     & \phantom{:} = 
     \left\{ z_j : 
     \alpha^{h} \times L_h^{(R / \alpha - 1)}\left(\frac{z}{\alpha}\right) = 0,\ 
     1 \leq j \leq h 
     \right\}. 
\end{align} 
Let the sequences, $c_{h,j}(\alpha, R)$, denote a shorthand 
for the coefficients corresponding to an 
expansion of the generalized convergent functions, 
$\ConvGF{h}{\alpha}{R}{z}$, by 
partial fractions in $z$, \ie the coefficients 
defined so that \citep[\S 1.2(iii)]{NISTHB} 
\begin{align*} 
\ConvGF{h}{\alpha}{R}{z} & \defequals 
     \sum_{j=1}^{h} \frac{c_{h,j}(\alpha, R)}{ 
     (1-\ell_{h,j}(\alpha, R) \cdot z)}. 
\end{align*} 
%% 
\StartGroupingSubEquations 
For $n \geq 1$ and any fixed integer $\alpha \neq 0$, these 
rational convergent functions 
provide the following formulas exactly generating the 
respective sequence cases in \eqref{eqn_GenFact_product_form} and 
\eqref{eqn_nAlpha_Multifact_variant_rdef}: 
\begin{align} 
\label{eqn_AlphaFactFn_Exact_PartialFracsRep_v1} 
p_n(\alpha, R) & = 
     \sum_{j=1}^{n} c_{n,j}(\alpha, R) \times 
     \ell_{n,j}(\alpha, R)^{n} \\ 
\notag 
n!_{(\alpha)} & = 
     \sum_{j=1}^{n} c_{n,j}(-\alpha, n) \times 
     \ell_{n,j}(-\alpha, n)^{\lfloor \frac{n-1}{\alpha} \rfloor}. 
\end{align} 
The corresponding congruences satisfied by each of these 
generalized sequence cases obtained from the $h^{th}$ 
convergent function expansions in $z$ are stated similarly 
modulo any prescribed integers $h \geq 2$ and fixed $\alpha \geq 1$ 
in the next forms. 
\begin{align} 
\label{eqn_AlphaFactFn_Exact_PartialFracsRep_v2} 
p_n(\alpha, R) & \equiv 
     \sum_{j=1}^{h} c_{h,j}(\alpha, R) \times 
     \ell_{h,j}(\alpha, R)^{n} 
     && \pmod{h} \\ 
\notag 
n!_{(\alpha)} & \equiv 
     \sum_{j=1}^{h} c_{h,j}(-\alpha, n) \times 
     \ell_{h,j}(-\alpha, n)^{\lfloor \frac{n-1}{\alpha} \rfloor} 
     && \pmod{h, h\alpha, \cdots, h\alpha^{h}} 
\end{align} 
\EndGroupingSubEquations 
The notation $g_1(n) \equiv g_2(n) \pmod{n_1, n_2}$ in the 
previous equations denotes that the stated 
congruences hold modulo either base, $n_1$ or $n_2$. 
Section \ref{subsubSection_Examples_NewCongruences} and 
Section \ref{subSection_NewCongruence_Relations_Modulo_Integer_Bases} 
provide several particular special case examples of the new 
congruence properties expanded by 
\eqref{eqn_AlphaFactFn_Exact_PartialFracsRep_v2}. 

\end{enumerate} 

\section{Notation and other conventions in the article} 
\label{Section_Notation_and_Convs} 

\subsection{Notation and special sequences} 

Most of the conventions in the article are consistent with the 
notation employed within the \emph{Concrete Mathematics} reference, and 
the conventions defined in the introduction to the first article 
\citep{MULTIFACTJIS}. 
These conventions 
include the following particular notational variants: 
\begin{enumerate} 
     \renewcommand{\labelenumi}{$\mathsmaller{\blacktriangleright}$ } 
     \setlength{\itemsep}{-1mm} 
     \newcommand{\localitemlabel}[1]{\textbf{#1}.\ } 

\item \localitemlabel{Extraction of formal power series coefficients} 
The special notation for formal 
power series coefficient extraction, 
$[z^n] \left( \sum_{k} f_k z^k \right) \defmapsto f_n$; 

\item \localitemlabel{Iverson's convention} 
The more compact usage of Iverson's convention, 
$\Iverson{i = j} \equiv \delta_{i,j}$, where 
$\Iverson{n = k = 0} \equiv \delta_{n,0} \delta_{k,0}$; 

\item \localitemlabel{Bracket notation for the Stirling and 
                      Eulerian number triangles} 
The alternate bracket notation for the Stirling number triangles, 
$\gkpSI{n}{k} = (-1)^{n-k} s(n, k)$ and 
$\gkpSII{n}{k} = S(n, k)$, as well as 
$\gkpEI{n}{m}$ to denote the first--order Eulerian number triangle, and 
$\gkpEII{n}{m}$ to denote the second--order Eulerian numbers; 

\item \localitemlabel{Harmonic number sequences} 
Use of the notation for the first--order harmonic numbers, $H_n$ or 
$H_n^{(1)}$, defines the sequence 
\[
H_n \defequals 1+\frac{1}{2}+\frac{1}{3}+\cdots+\frac{1}{n}, 
\] 
and the notation for the partial sums defining the more general cases of the 
$r$--order harmonic numbers defined as 
\[ 
\mathsmaller{H_n^{(r)}} \defequals 1 + 2^{-r} + 3^{-r} + \cdots + n^{-r}, 
\]
when $r, n \geq 1$ are integer--valued; 

\item \localitemlabel{Rising and falling factorial functions} 
The convention of denoting the 
falling factorial function by $\FFactII{x}{n} = x! / (x-n)!$, the 
rising factorial function as $\RFactII{x}{n} = \Gamma(x+n) / \Gamma(x)$, 
and the Pochhammer symbol, $\Pochhammer{x}{n}$; 

\item \localitemlabel{Shorthand notation in integer congruences and modular arithmetic} 
Within the article the notation 
$g_1(n) \equiv g_2(n) \pmod{N_1, N_2, \ldots, N_k}$ is understood to 
mean that the congruence, $g_1(n) \equiv g_2(n) \pmod{N_j}$, holds 
modulo any of the bases, $N_j$, for $1 \leq j \leq k$. 

\end{enumerate} 
%% 
Within the article, the standard set notation for 
$\mathbb{Z}$, $\mathbb{Q}$, and $\mathbb{R}$ 
denote the sets of integers, rational numbers, and real numbers, respectively, 
where the set of natural numbers, $\mathbb{N}$, is defined by 
$\mathbb{N} \defequals \{0, 1, 2, \ldots \} = \mathbb{Z}^{+} \bigcup \{0\}$. 
Other more standard notation for the special functions 
cited within the article is consistent with the definitions 
employed within the \emph{NIST Handbook} reference. 

\subsection{Conventions in preparing and formatting the article} 

The following listings explain other conventions used in the 
typesetting and formatting of the article: 
\begin{enumerate} 
     \renewcommand{\labelenumi}{$\mathsmaller{\blacktriangleright}$ } 
     \setlength{\itemsep}{-1mm} 
     \newcommand{\localitemlabel}[1]{\textbf{#1}.\ } 

\item \localitemlabel{End marker symbols for theorem--like environments} 
The respective symbols 
$\DefinitionQEDSymbol$, $\ExampleQEDSymbol$, and $\RemarkQEDSymbol$ 
are taken to denote the 
end markers following the conclusions for the 
definition, example, and remark environments expanded 
in manuscript text below. 
It is easy to redefine, or simply choose to not display, 
these end marker symbols in the latex source code for the article. 

\item \localitemlabel{Inline footnote references within the article} 
The labeled footnotes cited in the next sections of the article 
are numbered by section to include statements of noteworthy identities 
from the references, and to provide short explanations of other 
formulas cited within the examples discussed by the applications below. 

\end{enumerate} 

\section{Examples of the new results in the article} 
\label{subSection_Intro_Examples} 

\subsection{Factorial--related finite product sequences 
            enumerated by the 
            generalized convergent functions} 
\label{subsubSection_Intro_Examples_Fact-RelatedSeqs_GenByTheConvFns} 
\label{prop_Conv-Based_Defs_for_FactFn_Variants} 
\label{cor_NumericalMultiFactSeqsEnum_Alpha1234} 

\subsubsection{Generating functions for arithmetic progressions of the 
               $\alpha$--factorial functions} 
%% 
There are a couple of noteworthy subtleties that arise in defining the specific 
numerical forms of the $\alpha$--factorial function sequences 
defined by 
\eqref{eqn_nAlpha_Multifact_variant_rdef} and 
\eqref{eqn_AlphaNm1pd_AlphaFactFn_PolyCoeffSum_Exp_formula-eqns_v1}. 
%% 
First, since the generalized convergent functions generate the distinct 
symbolic products that characterize the forms of these expansions, 
we see that the following convergent--based enumerations of the 
multiple factorial sequence variants hold at each 
$\alpha, n \in \mathbb{Z}^{+}$, and 
some fixed choice of the prescribed offset, $0 \leq d < \alpha$: 
\StartGroupingSubEquations 
\begin{align*} 
\tagonce\label{eqn_MultFactFn_ConvSeq_def_v1} 
\left(\alpha n-d\right)!_{(\alpha)} & = 
     \undersetbrace{\pn{n}{-\alpha}{\alpha n-d}}{ 
     \left(-\alpha\right)^{n} \cdot \Pochhammer{\frac{d}{\alpha} - n}{n} 
     } = 
     [z^n] \ConvGF{n}{-\alpha}{\alpha n-d}{z} \\ 
     & = 
     \undersetbrace{\pn{n}{\alpha}{\alpha-d}}{ 
     \alpha^{n} \cdot \Pochhammer{1 - \frac{d}{\alpha}}{n} 
     } 
     \phantom{- n} = 
     [z^n] \ConvGF{n}{\alpha}{\alpha - d}{z}. 
\end{align*} 
For example, 
some variants of the arithmetic progression sequences formed by the 
single factorial and double factorial functions, $n!$ and $n!!$, in 
Section \ref{subsubSection_Apps_ArithmeticProgs_of_the_SgFactFns} 
are generated by the particular shifted inputs to these functions 
highlighted by the special cases in the next equations 
\citeOEIS{A000142,A000165,A001147}: 
\begin{align*} 
\left( n! \right)_{n=1}^{\infty} & = 
     %\left( p_n(-1, n) \right)_{n=1}^{\infty} && = 
     %\left( p_n(1, 1) \right)_{n=1}^{\infty} 
     \left( \Pochhammer{1}{n} \right)_{n=1}^{\infty} 
     && \seqmapsto{A000142} 
     \left(1, 2, 6, 24, 120, 720, 5040, \ldots \right) \\ 
\left( (2n)!! \right)_{n=1}^{\infty} & = 
     %\left( p_n(-2, 2n) \right)_{n=1}^{\infty} && = 
     %\left( p_n(2, 2) \right)_{n=1}^{\infty} 
     \left( 2^{n} \cdot \Pochhammer{1}{n} \right)_{n=1}^{\infty} 
     && \seqmapsto{A001147}  
     \left(2, 8, 48, 384, 3840, 46080, \ldots \right) \\ 
\left( (2n-1)!! \right)_{n=1}^{\infty} & = 
     %\left( p_n(-2, 2n-1) \right)_{n=1}^{\infty} && = 
     %\left( p_n(2, 1) \right)_{n=1}^{\infty} 
     \left( 2^{n} \cdot \Pochhammer{1/2}{n} \right)_{n=1}^{\infty} 
     && \seqmapsto{A000165} 
     \left(1, 3, 15, 105, 945, 10395, \ldots \right). 
\end{align*} 
The next few special case variants of the $\alpha$--factorial function 
sequences corresponding to $\alpha \defequals 3, 4$, also expanded in 
Section \ref{subsubSection_Apps_ArithmeticProgs_of_the_SgFactFns}, 
are given in the following sequence forms 
\citeOEIS{A032031,A008544,A007559,A047053,A007696}: 
\begin{align*} 
\left( (3n)!!! \right)_{n=1}^{\infty} & = 
     \left( 3^{n} \cdot \Pochhammer{1}{n} \right)_{n=1}^{\infty} 
     && \seqmapsto{A032031} 
     \left(3, 18, 162, 1944, 29160, \ldots \right) \\ 
\left( (3n-1)!!! \right)_{n=1}^{\infty} & = 
     \left( 3^{n} \cdot \Pochhammer{2/3}{n} \right)_{n=1}^{\infty} 
     && \seqmapsto{A008544} 
     \left(2, 10, 80, 880, 12320, 209440, \ldots \right) \\ 
\left( (3n-2)!!! \right)_{n=1}^{\infty} & = 
     \left( 3^{n} \cdot \Pochhammer{1/3}{n} \right)_{n=1}^{\infty} 
     && \seqmapsto{A007559} 
     \left(1, 4, 28, 280, 3640, 58240, \ldots \right) \\ 
%%%% 
\left( \AlphaFactorial{4n}{4} \right)_{n=0}^{\infty} & = 
     \left( 4^{n} \cdot \Pochhammer{1}{n} \right)_{n=0}^{\infty} 
     && \seqmapsto{A047053} 
     \left(1, 4, 32, 384, 6144, 122880, \ldots \right) \\ 
\left( \AlphaFactorial{4n+1}{4} \right)_{n=0}^{\infty} & = 
     \left( 4^{n} \cdot \Pochhammer{5/4}{n} \right)_{n=0}^{\infty} 
     && \seqmapsto{A007696} 
     \left(1, 5, 45, 585, 9945, 208845, \ldots \right). 
\end{align*} 
For each $n \in \mathbb{N}$ and prescribed constants 
$r, c \in \mathbb{Z}$ defined such that $c \mid n+r$, we also obtain 
rational convergent--based generating functions enumerating the modified 
multiple factorial function sequences given by 
\begin{align} 
\label{eqn_PPlusROverAlpha_FactFn_Conv_Ident-stmt_v2} 
\mathlarger{\mathsmaller{\left(\frac{n+r}{c}\right)\mathlarger{!}}} & = 
     [z^{n}] 
     \ConvGF{h}{-c}{n+r}{\frac{z}{c}} + 
     \Iverson{\frac{r}{c} = 0} \Iverson{n = 0}, 
     \text{ $\forall$ $h \geq \lfloor (n+r) / c \rfloor$}. 
\end{align} 
The identity corresponding to the special case of 
\eqref{eqn_PPlusROverAlpha_FactFn_Conv_Ident-stmt_v2} 
when $c \defequals 2$ leads to the convergent--based generating function 
expansions of the congruence for the 
primes of the form $p = 4k+1$ cited by \eqref{eqn_WThm_p4akp1_gen} 
in the examples given below in 
Section \ref{subSection_Wthm_CThm_SpCase_Apps}. 

\subsubsection{Generating functions for 
               multi--valued integer product sequences} 
%% 
Likewise, 
given any $n \geq 1$ and fixed $\alpha \in \mathbb{Z}^{+}$, 
we can enumerate the somewhat less obvious full forms of the 
generalized $\alpha$--factorial function sequences defined piecewise for the 
distinct residues, $n \in \{0,1,\ldots, \alpha -1\}$, modulo $\alpha$ by 
\eqref{eqn_nAlpha_Multifact_variant_rdef} and in 
\eqref{eqn_MultFactFn_ConvSeq_def_v1}. 
The multi--valued products defined by 
\eqref{eqn_GenFact_product_form} for these functions are 
generated as follows: 
\begin{align*} 
\tagonce\label{eqn_MultFactFn_ConvSeq_def_v2} 
n!_{(\alpha)} & = 
     \left[z^{\lfloor (n+\alpha-1) / \alpha \rfloor}\right] 
     \ConvGF{n}{-\alpha}{n}{z} \\ 
     & = 
     [z^{n}] \left( 
     \sum_{0 \leq d < \alpha} z^{-d} \cdot 
     \ConvGF{n}{\alpha}{\alpha - d}{z^{\alpha}} 
     \right) \\ 
\tagonce\label{eqn_MultFactFn_ConvSeq_def_v3} 
    & = 
     [z^{n+\alpha-1}] \left( 
     \frac{1-z^{\alpha}}{1-z} \times 
     \ConvGF{n}{-\alpha}{n}{z^{\alpha}} 
     \right). 
\end{align*} 
\EndGroupingSubEquations 
%% 
The complete sequences over the multi--valued symbolic products 
formed by the special cases of the 
double factorial function, the \emph{triple factorial} function, $n!!!$, the 
\emph{quadruple factorial} function, 
$n!!!! = \MultiFactorial{n}{4}$, the 
\emph{quintuple factorial} (\emph{$5$--factorial}) function, 
$\MultiFactorial{n}{5}$, and the 
\emph{$6$--factorial} function, $\MultiFactorial{n}{6}$, 
respectively, are generated by the convergent generating function 
approximations expanded in the next equations 
\citeOEIS{A006882,A007661,A007662,A085157,A085158}. 
\begin{align*} 
\left( n!! \right)_{n=1}^{\infty} & = \left( 
     \left[z^{\lfloor (n+1) / 2 \rfloor}\right] 
     \ConvGF{n}{-2}{n}{z} 
     \right)_{n=1}^{\infty} && %\longmapsto 
     \seqmapsto{A006882} 
     \left(1, 2, 3, 8, 15, 48, 105, 384, 945, 3840, \ldots\right) \\ 
\left( n!!! \right)_{n=1}^{\infty} & = \left( 
     \left[z^{\lfloor (n+2) / 3 \rfloor}\right] 
     \ConvGF{n}{-3}{n}{z} 
     \right)_{n=1}^{\infty} && %\longmapsto 
     \seqmapsto{A007661} 
     \left(1, 2, 3, 4, 10, 18, 28, 80, 162, 280, \ldots\right) \\ 
\left( \MultiFactorial{n}{4} \right)_{n=1}^{\infty} & = 
     \left( 
     \left[z^{\lfloor (n+3) / 4 \rfloor}\right] 
     \ConvGF{n}{-4}{n}{z} 
     \right)_{n=1}^{\infty} && %\longmapsto 
     \seqmapsto{A007662} 
     \left(1, 2, 3, 4, 5, 12, 21, 32, 45, 120, 231, \ldots\right) \\ 
\left( \MultiFactorial{n}{5} \right)_{n=1}^{\infty} & = 
     \left( 
     \left[z^{\lfloor (n+4) / 5 \rfloor}\right] 
     \ConvGF{n}{-5}{n}{z} 
     \right)_{n=1}^{\infty} && %\longmapsto 
     \seqmapsto{A085157} 
     \left(1, 2, 3, 4, 5, 6, 14, 24, 36, 50, 66, 168, \ldots\right) \\ 
\left( \MultiFactorial{n}{6} \right)_{n=1}^{\infty} & = 
     \left( 
     \left[z^{\lfloor (n+5) / 6 \rfloor}\right] 
     \ConvGF{n}{-6}{n}{z} 
     \right)_{n=1}^{\infty} && %\longmapsto 
     \seqmapsto{A085158} 
     \left(1, 2, 3, 4, 5, 6, 7, 16, 27, 40, 55, 72, 91, \ldots\right) 
\end{align*} 

\subsubsection{Examples of new convergent--based generating function identities 
               for the binomial coefficients} 
%% 
The rationality in $z$ of the generalized convergent functions, 
$\ConvGF{h}{\alpha}{R}{z}$, for all $h \geq 1$ also provides 
several of the new forms of generating function identities for many 
factorial--related product sequences and 
related expansions of the binomial coefficients that are easily 
established from the 
diagonal--coefficient, or Hadamard product, generating function 
results established in 
Section \ref{subSection_DiagonalGFSequences_Apps}. 
%% 
For example, for natural numbers $n \geq 1$, the next variants of the 
binomial--coefficient--related product sequences are enumerated by the 
following coefficient identities 
\citeOEIS{A009120,A001448}: 
\begin{align*} 
\tagonce\label{eqn_HybridDiagCoeffHPGFs_BinomCoeff_Examples-exps_v1} 
\frac{(4n)!}{(2n)!} & = 
     \frac{4^{4n} 
           \bcancel{\Pochhammer{1}{n}} \bcancel{\Pochhammer{\frac{2}{4}}{n}} 
           \Pochhammer{\frac{1}{4}}{n} \Pochhammer{\frac{3}{4}}{n}}{ 
           2^{2n} \Pochhammer{1}{n} \Pochhammer{\frac{1}{2}}{n}} \\ 
     & = 
     4^{n} \times 4^{n} \Pochhammer{1/4}{n} \times 4^{n} \Pochhammer{3/4}{n} \\ 
     & = 
     [x_1^0 z^n]\left( 
     \ConvGF{n}{4}{3}{\frac{4z}{x_1}} \ConvGF{n}{4}{1}{x_1} 
     \right) \\ 
     & = 
     4^{n} \times \AlphaFactorial{4n-3}{4} \AlphaFactorial{4n-1}{4} \\ 
     & = 
     [x_1^0 z^n]\left( 
     \ConvGF{n}{-4}{4n-3}{\frac{4z}{x_1}} \ConvGF{n}{-4}{4n-1}{x_1} 
     \right) \\ 
\binom{4n}{2n} & = 
     [x_1^0 x_2^0 z^n]\Biggl( 
     \ConvGF{n}{4}{3}{\frac{4z}{x_2}} \ConvGF{n}{4}{1}{\frac{x_2}{x_1}} \times 
     \undersetbrace{\widehat{E}_2(x_1) = E_{2,1}(x_1)}{ 
     \cosh\left(\sqrt{x_1}\right) 
     } 
     \quad\Biggr) \\ 
     & = 
     [x_1^0 x_2^0 z^n]\Biggl( 
     \ConvGF{n}{-4}{4n-3}{\frac{4z}{x_2}} 
     \ConvGF{n}{-4}{4n-1}{\frac{x_2}{x_1}} \times 
     \undersetbrace{\widehat{E}_2(x_1) = E_{2,1}(x_1)}{ 
     \cosh\left(\sqrt{x_1}\right) 
     } 
     \quad\Biggr). 
\end{align*} 
The examples given in 
Section \ref{subsubSection_remark_HybridDiagonalHPGFs} 
provide examples of related constructions 
of the hybrid rational convergent--based generating function products that 
generate the central binomial coefficients and 
several other notable cases of related sequence expansions. 

\subsection{New congruences for the 
            $\alpha$--factorial functions, the 
            Stirling numbers of the first kind, and the 
            $r$--order harmonic number sequences} 
\label{subsubSection_Examples_NewCongruences} 

\subsubsection{Congruences for the $\alpha$--factorial functions modulo $2$} 
%% 
For any fixed $\alpha \in \mathbb{Z}^{+}$ and natural numbers $n \geq 1$, the 
generalized multiple, $\alpha$--factorial functions, $n!_{(\alpha)}$, 
defined by \eqref{eqn_nAlpha_Multifact_variant_rdef} 
satisfy the following congruences 
modulo $2$ (and $2\alpha$): 
\begin{align} 
\label{eqn_cor_Congruences_for_AlphaFactFns_modulo2} 
n!_{(\alpha)} & \equiv 
     \frac{n}{2} \left(\left(n-\alpha + \sqrt{\alpha 
     (\alpha -n)}\right)^{\left\lfloor \frac{n-1}{\alpha }\right\rfloor } + 
     \left(n-\alpha - \sqrt{\alpha (\alpha -n)}\right)^{\left\lfloor 
     \frac{n-1}{\alpha }\right\rfloor }\right) \pmod{2, 2\alpha}. 
\end{align} 
Given that the definition of the single factorial function implies that 
$n! \equiv 0 \pmod{2}$ whenever $n \geq 2$, the statement of 
\eqref{eqn_cor_Congruences_for_AlphaFactFns_modulo2} 
provides somewhat less obvious results for the 
generalized $\alpha$--factorial function sequence cases when $\alpha \geq 2$. 
\tableref{table_AlphaFactFns_Modulo245_spcase_examples} 
provides specific listings of the result in 
\eqref{eqn_cor_Congruences_for_AlphaFactFns_modulo2} satisfied by the 
$\alpha$--factorial functions, $\MultiFactorial{n}{\alpha}$, 
for $\alpha \defequals 1, 2, 3, 4$. 
The corresponding, closely--related new forms of 
congruence properties satisfied by 
these functions expanded through analogs exact algebraic formulas 
modulo $3$ ($3\alpha$) and modulo $4$ ($4\alpha$) are also cited 
as special cases in the next examples. 

\subsubsection{New forms of congruences for the $\alpha$--factorial functions 
               modulo $3$, modulo $4$, and modulo $5$} 
%% 
To simplify notation, we first define the next shorthand for the respective 
(distinct) roots, $r_{p,i}^{(\alpha)}(n)$ for $1 \leq i \leq p$, 
corresponding to the special cases of the convergent denominator 
functions, $\ConvFQ{p}{\alpha}{R}{z}$, 
factorized over $z$ for any fixed integers $n, \alpha \geq 1$ 
when $p \defequals 3, 4, 5$ 
\citep[\S 1.11(iii); \cf \S 4.43]{NISTHB}: 
\begin{align} 
\notag 
\left( r_{3,i}^{(\alpha)}(n) \right)_{i=1}^{3} & \defequals 
     \left\{ z_i : z_i^3 -3 z_i^2 (2 \alpha +n)+3 z_i (\alpha +n) 
     (2 \alpha +n) - n (\alpha +n) (2 \alpha +n) = 0,\ 
     1 \leq i \leq 3 \right\} \\ 
\notag 
\left( r_{4,j}^{(\alpha)}(n) \right)_{j=1}^{4} & \defequals 
     \bigl\{ z_j : 
     z_j^4 - 4 z_j^3 (3 \alpha +n) + 
     6 z_j^2 (2 \alpha +n) (3 \alpha +n) - 
     4 z_j (\alpha +n) (2 \alpha +n) (3 \alpha +n) \\ 
\notag 
     & \phantom{\defequals \bigl( z_j : z_j^4\ } + 
     n (\alpha +n) (2 \alpha +n) (3 \alpha +n) = 0,\ 
     1 \leq j \leq 4 \bigr\} \\ 
\notag 
\left( r_{5,k}^{(\alpha)}(n) \right)_{k=1}^{5} & \defequals 
     \bigl\{ z_k : 
     z_k^5 - 5 (4 \alpha + n) z_k^4 + 
     10 (3 \alpha + n) (4 \alpha + n) z_k^3 - 
     10 (2 \alpha + n) (3 \alpha + n) (4 \alpha + n) z_k^2 \\ 
\notag 
     & \phantom{\defequals \bigl( z_k : z_k^5\ } + 
     5 (\alpha + n) (2 \alpha + n) (3 \alpha + n) (4 \alpha + n) z_k \\ 
\label{eqn_cor_AlphaFactMod3_orig_roots_eqn} 
     & \phantom{\defequals \bigl( z_k : z_k^5\ } - 
     n (\alpha + n) (2 \alpha + n) (3 \alpha + n) (4 \alpha + n) = 0,\ 
     1 \leq k \leq 5 \bigr\}. 
\end{align} 
Similarly, we define the following functions for any fixed 
$\alpha \in \mathbb{Z}^{+}$ and $n \geq 1$ to 
simplify the notation in stating next the congruences in 
\eqref{eqn_AlphaFactFnModulo3_congruence_stmts} below: 
%%%% 
\newcommand{\rootri}[2]{\ensuremath{r_{#1,#2}^{(-\alpha)}(n)}} 
%%%% 
\begin{align*} 
\widetilde{R}_3^{(\alpha)}(n) & \defequals 
     \frac{\left(6 \alpha ^2+\alpha 
     \left(6 \rootri{3}{1}-4 n\right)+\left(n-\rootri{3}{1}
     \right){}^2\right) 
     \rootri{3}{1}{}^{\left\lfloor \frac{n-1}{\alpha }\right\rfloor+1}}{ 
     \left(\rootri{3}{1}-\rootri{3}{2}\right) 
     \left(\rootri{3}{1}-\rootri{3}{3}\right)} \\ 
   & \phantom{\equiv\ } \quad + 
     \frac{\left(6 \alpha ^2+\alpha  \left(6 \rootri{3}{3}-4 n\right)+ 
     \left(n-\rootri{3}{3}\right){}^2\right)
      \rootri{3}{3}{}^{\left\lfloor \frac{n-1}{\alpha }\right\rfloor +1}}{ 
      \left(\rootri{3}{3}-\rootri{3}{1}\right) 
      \left(\rootri{3}{3}-\rootri{3}{2}\right)} \\ 
   & \phantom{\equiv\ } \quad + 
     \frac{\left(6 \alpha ^2+\alpha \left(6 \rootri{3}{2}-4 n\right) + 
     \left(n-\rootri{3}{2}\right){}^2\right) \rootri{3}{2}{}^{\left\lfloor 
     \frac{n-1}{\alpha }\right\rfloor +1}}{ 
     \left(\rootri{3}{2}-\rootri{3}{1}\right)
     \left(\rootri{3}{2}-\rootri{3}{3}\right)} \\ 
%%%% 
%%%% 
C_{4,i}^{(\alpha)}(n) & \defequals 
     24 \alpha^3-18 \alpha ^2 \left(n-2 \cdot \rootri{4}{i} 
     \right)+\alpha  \left(7 n-12 \cdot \rootri{4}{i}\right) 
     \left(n-\rootri{4}{i}\right) \\ 
   & \phantom{\defequals 24 \alpha^3\ } - 
     \left(n-\rootri{4}{i}\right)^3,\ 
     \text{ for } 
     1 \leq i \leq 4 \\ 
C_{5,k}^{(\alpha)}(n) & \defequals 
     120 \alpha^4 + 2 \alpha^2 \left(23 n^2-79 n \cdot \rootri{5}{k} + 
     60 \cdot \rootri{5}{k}^2\right) + 
     48 \alpha ^3 (2n - 5 \cdot \rootri{5}{k}) \\ 
     & \phantom{\defequals 120 \alpha^4\ } + 
     \alpha  (11 n-20 \rootri{5}{k}) (n-\rootri{5}{k})^2 + 
     (n-\rootri{5}{k})^4,\ 
     \text{ for } 
     1 \leq k \leq 5. 
\end{align*} 
%% 
For fixed $\alpha \in \mathbb{Z}^{+}$ and $n \geq 0$, 
we obtain the following analogues to the first congruence result modulo $2$ 
expanded by \eqref{eqn_cor_Congruences_for_AlphaFactFns_modulo2} 
for the $\alpha$--factorial functions, $n_{(\alpha)}$, 
when $n \geq 1$: 
\begin{align} 
\label{eqn_AlphaFactFnModulo3_congruence_stmts} 
n!_{(\alpha)} & \equiv \widetilde{R}_3^{(\alpha)}(n) && \pmod{3, 3\alpha} \\ 
\notag 
n!_{(\alpha)} & \equiv 
     \undersetbrace{ \defequals R_4^{(\alpha)}(n)}{
     \sum\limits_{1 \leq i \leq 4} 
     \frac{C_{4,i}^{(\alpha)}(n)}{\prod\limits_{j \neq i} 
     \left(\rootri{4}{i} - \rootri{4}{j}\right)} 
     \rootri{4}{i}^{\left\lfloor \frac{n+\alpha-1}{\alpha} \right\rfloor} 
     } 
     && \pmod{4, 4\alpha} \\ 
\notag 
n!_{(\alpha)} & \equiv 
     \undersetbrace{ \defequals R_5^{(\alpha)}(n)}{ 
     \sum\limits_{1 \leq k \leq 5} 
     \frac{C_{5,k}^{(\alpha)}(n)}{\prod\limits_{j \neq k} 
     \left(\rootri{5}{k} - \rootri{5}{j}\right)} 
     \rootri{5}{k}^{\left\lfloor \frac{n+\alpha-1}{\alpha} \right\rfloor} 
     } 
     && \pmod{5, 5\alpha}. 
\end{align} 
Several particular concrete examples illustrating the 
results cited in 
\eqref{eqn_cor_Congruences_for_AlphaFactFns_modulo2} 
modulo $2$ (and $2\alpha$), and in 
\eqref{eqn_AlphaFactFnModulo3_congruence_stmts} 
modulo $p$ (and $p\alpha$) for $p \defequals 3, 4, 5$, 
corresponding to the first few cases of 
$\alpha \geq 1$ and $n \geq 1$ appear in 
\tableref{table_AlphaFactFns_Modulo245_spcase_examples}. 

%% 
Further computations of the congruences given in 
\eqref{eqn_AlphaFactFnModulo3_congruence_stmts} 
modulo $p\alpha^{i}$ (for some $0 \leq i \leq p$) are contained in the 
\Mm{} summary notebook 
included as a supplementary file with the 
submission of this article 
(see Section \ref{subSection_MmSummaryNBInfo} and the 
reference document \citep{SUMMARYNBREF-STUB}). 
%% 
The results in 
Section \ref{subSection_NewCongruence_Relations_Modulo_Integer_Bases} 
provide statements of these new integer congruences 
for fixed $\alpha \neq 0$ modulo any integers $p \geq 2$. 
The analogs formulations of 
the new relations for the factorial--related product sequences 
modulo any $p$ and $p\alpha$ 
are easily established for the subsequent cases of integers $p \geq 6$ 
from the properties of the 
convergent functions, $\ConvGF{h}{\alpha}{R}{z}$, 
cited in the particular listings in 
\tableref{table_SpCase_Listings_Of_PhzQhz_ConvFn} and in 
\tableref{table_RelfectedConvNumPolySeqs_sp_cases} 
through the generalized convergent function properties proved in 
Section \ref{Section_Props_Of_CFracExps_OfThe_GenFactFnSeries}. 

\subsubsection{New congruence properties for the 
               Stirling numbers of the first kind} 
%% 
The results given in 
Section \ref{subSection_NewCongruence_Relations_Modulo_Integer_Bases} also 
provide new congruences for the generalized Stirling number triangles 
in \eqref{eqn_Fa_rdef}, as well as several new forms of rational generating 
functions that enumerate the scaled factorial--power variants of the 
\emph{$r$--order harmonic numbers}, 
$(n!)^{r} \times H_n^{(r)}$, 
modulo integers $p \geq 2$ \citep[\S 6.3]{GKP}. 
For example, the known congruences for the 
Stirling numbers of the first kind proved by the 
generating function techniques enumerated in the reference 
\citep[\S 4.6]{GFOLOGY} imply the next new 
congruence results satisfied by the binomial coefficients modulo $2$ 
(see Section \ref{subsubSection_remark_New_Congruences_for_GenS1Triangles_and_HNumSeqs}) 
\citeOEIS{A087755}. 
\begin{align*} 
\binom{\lfloor \frac{n}{2} \rfloor}{m - \lceil \frac{n}{2} \rceil} & 
     \Iverson{1 \leq m \leq 6} & \\ 
     & \equiv 
   \begin{rcases*} 
     \begin{cases} 
     \frac{2^{n}}{4} & 
     \text{ if $m = 1$; } \\ 
     \frac{3 \cdot 2^{n}}{16} (n-1) & 
     \text{ if $m = 2$; } \\ 
     \frac{2^{n}}{128} (9n-20) (n-1) & 
     \text{ if $m = 3$; } \\ 
     \frac{2^{n}}{512} (3n-10) (3n-7) (n-1) & 
     \text{ if $m = 4$; } \\ 
     \frac{2^{n}}{8192} (27n^3-279n^2+934n-1008) (n-1) & 
     \text{ if $m = 5$; } \\ 
     \frac{2^{n}}{163840} (9n^2-71n+120) (3n-14) (3n-11) (n-1) & 
     \text{ if $m = 6$; } \\ 
     0 & \text{ otherwise. } 
     \end{cases} 
   \end{rcases*} \times & \Iverson{n > m} \\ 
     & \phantom{\qquad 1 } + 
     \Iverson{1 \leq m \leq 6} \Iverson{n = m} 
     & \pmod{2}
\end{align*} 

\subsubsection{New congruences and rational generating functions for the 
               $r$--order harmonic numbers} 
The next results state several additional new congruence properties 
satisfied by the 
first--order, second--order, and third--order harmonic number sequences, 
each expanded by the rational generating functions enumerating these 
sequences modulo the first few small cases of integer--valued $p$ 
constructed from the generalized convergent functions in 
Section \ref{subsubSection_remark_New_Congruences_for_GenS1Triangles_and_HNumSeqs} 
of the article 
\citeOEIS{A001008,A002805,A007406,A007407,A007408,A007409}. 
\begin{align*} 
(n!)^{3} \times H_n^{(3)} 
     & \equiv 
     [z^{n}] \left( 
     \mathsmaller{ 
     \frac{z \left(1-7 z+49 z^2-144 z^3+192 z^4\right)}{(1-8z)^2} 
     } 
     \right) 
     && \pmod{2} \\ 
(n!)^{2} \times H_n^{(2)} 
     & \equiv 
     [z^{n}] \left( 
     \mathsmaller{ 
     \frac{z \left(1-61 z+1339 z^2-13106 z^3+62284 z^4-144264 z^5+ 
     151776 z^6-124416 z^7+41472 z^8\right)}{(1-6 z)^3 
     \left(1-24 z+36 z^2\right)^2}
     } 
     \right) 
     && \pmod{3} \\ 
(n!) \times H_n^{(1)} 
     & \equiv 
     [z^{n+1}] \left( 
     \mathsmaller{ 
     \frac{36 z^2 - 48z + 325}{576} + 
     \frac{17040 z^2+1782 z+6467}{576 \left(24 z^3-36 z^2+12 z-1\right)}+\frac{78828 z^2-33987 z+3071}{288 \left(24
        z^3-36 z^2+12 z-1\right)^2} 
     } 
     \right) && \pmod{4} \\ 
     & \equiv 
     [z^{n\phantom{-1}}]\left( 
     \mathsmaller{ 
     \frac{3z-4}{48} + 
     \frac{1300 z^2+890 z+947}{96 \left(24 z^3-36 z^2+12 z-1\right)}+\frac{24568 z^2-10576 z+955}{96 \left(24 z^3-36 z^2+12 z-1\right)^2}
     } 
     \right) && \pmod{4} \\ 
     & \equiv 
     [z^{n-1}] \left( 
     \mathsmaller{ 
     \frac{1}{16} + 
     \frac{-96 z^2+794 z+397}{48 \left(24 z^3-36 z^2+12 z-1\right)}+\frac{5730 z^2-2453 z+221}{24 \left(24 z^3-36 z^2+12 z-1\right)^2} 
     } 
     \right) && \pmod{4}. 
\end{align*} 

\subsection{Restatements of classical congruences and other 
            necessary conditions concerning primality} 
\label{subSection_Wthm_CThm_SpCase_Apps} 

\subsubsection{Applications in Wilson's theorem and 
               Clement's theorem concerning the twin primes} 
%% 
The first examples given in this section provide restatements of the 
necessary and sufficient integer--congruence--based conditions 
imposed in both statements of Wilson's theorem and Clement's theorem 
through the exact expansions of the factorial functions defined above. 
%% 
For odd integers $p \geq 3$, the congruences 
implicit to each of \emph{Wilson's theorem} and \emph{Clement's theorem} 
are enumerated as follows 
\citep[\S 4.3]{PRIMEREC} \citep[\S 6.6]{HARDYWRIGHTNUMT} \citep{CLEMENTPRIMES}: 
\begin{align} 
\tag{Wilson} 
\text{ $p$ prime } & \iff (p-1)! + 1 && \equiv 0 \pmod{p} \\ 
\notag 
   & \iff 
     [z^{p-1}] \ConvGF{p}{-1}{p-1}{z} + 1 && \equiv 0 \pmod{p} \\ 
\notag 
   & \iff 
     [z^{p-1}] \ConvGF{p}{1}{1}{z} + 1 && \equiv 0 \pmod{p} \\ 
\tag{Clement} 
\text{ $p, p+2$ prime } & \iff 
     4\left((p-1)! + 1\right) + p && \equiv 0 \pmod{p(p+2)} \\ 
\notag 
   & \iff 
     4 [z^{p-1}] \ConvGF{p(p+2)}{-1}{p-1}{z} + p + 4 && \equiv 0 \pmod{p(p+2)} \\ 
\notag 
   & \iff 
     4 [z^{p-1}] \ConvGF{p(p+2)}{1}{1}{z} + p + 4 && \equiv 0 \pmod{p(p+2)}. 
\end{align} 
These particular congruences involving the expansions of the 
single factorial function are considered in the reference 
\citep[\S 6.1.6]{MULTIFACTJIS} as an example of the 
first product--based symbolic factorial function expansions implicit to both 
Wilson's theorem and Clement's theorem. 
Section \ref{subSection_FiniteDiffEqns_for_the_GenFactFns} 
of this article considers the particular cases of 
these two classically--phrased congruence statements as 
applications of the new polynomial expansions for the 
generalized product sequences, $\pn{n}{\alpha}{\beta n+\gamma}$, 
derived from the expansions of the 
convergent function sequences by finite difference equations. 
Related formulations of conditions concerning the 
primality of prime pairs, $(p, p+d)$, and then of 
other prime $k$--tuples, 
are similarly straightforward to obtain by elementary methods 
starting from the statement of Wilson's theorem 
\citep{ONWTHM-AND-POLIGNAC-CONJ}. 

For example, the new results proved in 
Section \ref{subSection_FiniteDiffEqns_for_the_GenFactFns} 
are combined with the known congruences established in the reference 
\citep[\S 3, \S 5]{ONWTHM-AND-POLIGNAC-CONJ} to obtain the 
cases of the next particular forms of 
alternate necessary and sufficient conditions for the 
twin primality of the 
odd positive integers $p_1 \defequals 2n+1$ and $p_2 \defequals 2n+3$ 
when $n \geq 1$ \citeOEIS{A001359,A001097}: 
\begin{align*} 
\tagonce\label{eqn_TwinPrime_NewExpsOfKnownCongruenceResults-stmts_v1} 
 & 2n+1, 2n+3 \text{ odd primes } & \\ 
 & \quad \iff 
   \mathsmaller{ 
     2\left(\sum\limits_{i=0}^{n} 
     \binom{(2n+1)(2n+3)}{i}^2 (-1)^i i! (n-i)!\right)^2 + 
     (-1)^{n} (10n+7) 
   } && \equiv 0 \pmod{(2n+1)(2n+3)} \\ 
 & \quad \iff
   \mathsmaller{
     4 \left(\sum\limits_{i=0}^{2n} 
     \binom{(2n+1)(2n+3)}{i}^2 (-1)^i i! (2n-i)!\right) + 
     2n+5 
   } && \equiv 0 \pmod{(2n+1)(2n+3)}. 
\end{align*} 

\subsubsection{Congruences for the Wilson primes and the 
               single factorial function, $(n-1)!$, modulo $n^2$} 
%% 
The sequence of \emph{Wilson primes}, %$\mathbb{P}_{\Wilson}$, 
or the subsequence of odd integers $p \geq 5$ satisfying 
$n^2 \mid (n-1)! + 1$, is characterized through each of the 
following additional divisibility 
requirements placed on the expansions of the 
single factorial function implicit to Wilson's theorem 
cited by the applications of the new results given below in 
Section \ref{subsubSection_Examples-remarks_RelatedCongruences} 
of the article \citeOEIS{A007540}: 
\begin{align*} 
\undersetbrace{\equiv\ (n-1)! \pmod{n^2}}{
     \sum_{i=0}^{n-1} \binom{n^2}{i}^{2} (-1)^{i} i! (n-1-i)! 
     } 
     \phantom{\qquad\qquad\qquad\qquad} 
     & \equiv -1 && \pmod{n^2} \\ 
\undersetbrace{\equiv (n-1)! \pmod{n^2}}{
     \sum_{i=0}^{n-1} \binom{n^2}{i} 
     \FFactII{(n^2-n)}{i} \times (-1)^{n-1-i} \FFactII{(n-1)}{n-1-i} 
     } 
     \phantom{\qquad} 
     & \equiv -1 && \pmod{n^2} \\ 
\mathsmaller{ 
     \sum\limits_{s=0}^{n-1} \sum\limits_{i=0}^{s} \sum\limits_{v=0}^{i} 
     \left( 
     \sum\limits_{k=0}^{n-1} \sum\limits_{m=0}^{k} 
     \binom{n^2}{k} \binom{m}{s} \binom{i}{v} \binom{n^2+v}{v} 
     \gkpSI{k}{m} \gkpSII{s}{i} (-1)^{i-v} 
     \FFactII{(n-1)}{n-1-k} (-n)^{m-s} i! 
     \right) 
     } 
     & \equiv -1. && \pmod{n^2} 
\end{align*} 
The results providing the new congruence properties for the 
$\alpha$--factorial functions modulo the integers 
$p$, and $p \alpha^{i}$ for some $0 \leq i \leq p$, expanded in 
Section \ref{subSection_FiniteDiffEqns_for_the_GenFactFns} 
also lead to alternate phrasings of the 
necessary and sufficient conditions on the 
primality of several notable subsequences of the 
odd positive integers $n \geq 3$. 

\subsubsection{Generating terms in 
               congruences involving squares of factorial functions} 
%% 
The rationality in $z$ of the convergent functions, 
$\ConvGF{h}{\alpha}{R}{z}$, 
at each $h$ leads to further alternate formulations of other well--known 
congruence statements concerning the divisibility of factorial functions. 
%% 
For example, we may characterize the primality of the 
odd integers, $p > 3$, of the form $p = 4k+1$ 
(\ie the so--termed subset of \quotetext{\emph{Pythagorean primes}}) 
according to the next condition 
\citep[\S 7]{HARDYWRIGHTNUMT} \citeOEIS{A002144}: 
\begin{align} 
\label{eqn_WThm_p4akp1_gen}  
     \mathsmaller{\left(\frac{p-1}{2}\right)\mathlarger{!}}^{2} 
     \equiv -1 \pmod{p} & 
     \iff %\quad\iff\quad 
\text{ $p$ is a prime of the form $4k+1$.} 
\end{align} 
For an odd integer $p > 3$ to be both prime and 
satisfy $p \equiv 1 \pmod{4}$, the 
congruence statement in \eqref{eqn_WThm_p4akp1_gen} 
requires that the diagonals of the following rational two--variable 
convergent generating functions satisfy the following 
equivalent conditions where $p_i$ is chosen so that 
$2^{p} p_i \mid p$ for each $i = 1, 2$: 
\begin{align} 
\notag 
\mathsmaller{\left(\frac{p-1}{2}\right)\mathlarger{!}}^{2} & = 
     [z^{(p-1)/2}][x^0] \left( 
     \ConvGF{p_1}{-1}{\frac{p-1}{2}}{x} 
     \ConvGF{p_2}{-1}{\frac{p-1}{2}}{\frac{z}{x}} 
     \right) && \equiv -1 \pmod{p} \\ 
\notag 
\mathsmaller{\left(\frac{p-1}{2}\right)\mathlarger{!}}^{2} & = 
     [z^{(p-1)/2}][x^0] \left( 
     \ConvGF{p_1}{-2}{p-1}{x} 
     \ConvGF{p_2}{-2}{p-1}{\frac{z}{4 x}} 
     \right) && \equiv -1 \pmod{p} \\ 
\notag 
\mathsmaller{\left(\frac{p-1}{2}\right)\mathlarger{!}}^{2} & = 
     [z^{(p-1)/2}][x^0] \left( 
     \ConvGF{p_1}{-1}{\frac{p-1}{2}}{x} 
     \ConvGF{p_2}{-2}{p-1}{\frac{z}{2 x}} 
     \right) && \equiv -1 \pmod{p}. 
\end{align} 
The reference provides remarks on the harmonic--number--related 
fractional power series expansions of the 
convergent--based generating functions, 
$\ConvGF{n}{1}{1}{z/x} \times \ConvGF{n}{1}{1}{x}$ and 
$\ConvGF{n}{2}{1}{z/x} \times \ConvGF{n}{1}{1}{x}$, 
related to the single factorial function squares enumerated by the 
identities in the previous equations \citep{SUMMARYNBREF-STUB}. 

\subsubsection{Expansions of a necessary condition implied by Wolstenholme's theorem} 
%% 
Another related example is given by the 
next necessary condition following from 
\emph{Wolstenholme's theorem}, which provides that 
$p^2 \mid (p-1)! \cdot H_{p-1}$ whenever $p > 3$ is prime, 
where $H_n = H_n^{(1)}$ is a \emph{first--order harmonic number} 
\citep[\S 7.8]{HARDYWRIGHTNUMT} 
\citeOEIS{A001008,A002805}. 
This requirement on the primality of the odd integers $p > 3$ 
is rewritten in this case as the 
following statements involving the generalized 
convergent functions and the corresponding 
exponential generating function for the 
Stirling numbers, $H_n^{(1)} = \gkpSI{n+1}{2} \frac{1}{n!}$ 
\citep[\cf \S 6, 7.4]{GKP}: 
\begin{align} 
\label{eqn_POddPrime_NecessaryCond_HNumGF_example-stmt_v1} 
p > 3 \text{ prime } & \quad\implies\quad 
     [z^{p-1}] [x^0] \left( 
     \ConvGF{p^2}{-1}{p-1}{\frac{z}{x}} \frac{\Log(1-x)}{(1-x)} 
     \right) && \equiv 0 \pmod{p^2} \\ 
\notag 
     & \quad\implies\quad 
     [x^0 z^{p-1}] \left( 
     \ConvGF{p^2}{1}{1}{\frac{z}{x}} \frac{\Log(1-x)}{(1-x)} 
     \right) && \equiv 0 \pmod{p^2}. 
\end{align} 
For odd primes $p > 3$ and non--negative integers, 
$2v < p-3$, a theorem stated in the reference similarly implies that 
$(p-1)!^{2v+1} \times H_{p-1}^{(2v+1)} \equiv 0 \pmod{p^2}$ 
(see Section \ref{subsubSection_remark_New_Congruences_for_GenS1Triangles_and_HNumSeqs}) 
\citep[\S 8.8, Thm.\ 131]{HARDYWRIGHTNUMT}. 

\subsection{Expansions of congruences for the 
            $\alpha$--factorial functions and related sequences} 

\subsubsection{Polynomial congruences for double factorial functions and the 
               central binomial coefficients} 
%% 
The integer congruences satisfied by the double factorial function, 
$(2n-1)!!$, and the Pochhammer symbol cases, 
$2^{n} \times \Pochhammer{\frac{1}{2}}{n}$, expanded in 
Section \ref{subsubSection-example_OtherRelatedCongruences_DblFactFns} 
provide the next variants of the polynomial congruences for the 
\emph{central binomial coefficients}, 
$\binom{2n}{n} = 2^{n} \times (2n-1)!! / n!$, 
reduced modulo the respective integer multiples of 
$2n+1$ and the polynomial powers, $n^{p}$, 
for fixed integers $p \geq 2$ in the following equations 
\citeOEIS{A000984} 
(see Section \ref{subsubSection_remark_HybridDiagonalHPGFs} and the 
computations in the reference \citep{SUMMARYNBREF-STUB}): 
\begin{align*} 
\binom{2n}{n} & \equiv 
     \left\lbrace 
     \sum_{i=0}^{n} 
     \undersetline{\mod{2x+1} \quad \looparrowright \quad x \defmapsto n}{
     \binom{2x+1}{i} 
     (-2)^{i} \FFactII{\left(1/2 + 2x\right)}{i} \Pochhammer{1/2}{n-i} 
     \times \frac{2^{2n}}{n!}
     } 
     \right\rbrace 
     && \pmod{2n+1} \\ 
\binom{2n}{n} & \equiv 
     \left\lbrace 
     \undersetline{\mod{x^p} \quad \looparrowright \quad x \defmapsto n}{
     \sum_{i=0}^{n} \binom{x^p}{i} \binom{2n-2i}{n-i} 
     \Pochhammer{1/2 - x^p}{i} \times 
     \frac{8^{i} \cdot (n-i)!}{n!} 
     } 
     \right\rbrace 
     && \pmod{n^p}. 
\end{align*} 

\subsubsection{Expansions of other congruences for the 
               double and triple factorial functions} 
%% 
A few representative examples of the 
new congruences for the double and triple factorial 
functions obtained from the statements of the propositions in 
Section \ref{subSection_FiniteDiffEqns_for_the_GenFactFns} 
also include the following 
particular expansions for integers $p_1,p_2\geq 2$, and where 
$0 \leq s \leq p_1$ and $0 \leq t \leq p_2$ 
assume some prescribed values over the non--negative integers 
(see the computations contained in the reference \citep{SUMMARYNBREF-STUB}): 
\begin{align*} 
(2n-1)!! 
     & \equiv 
     \sum_{i=0}^{n} \binom{p_1}{i} 
     2^{n} (-2)^{(s+1)i} \Pochhammer{1/2-p_1}{i} \Pochhammer{1/2}{n-i} 
     && \pmod{2^{s} p_1} \\ 
     & \equiv 
     \sum_{i=0}^{n} \binom{p_1}{i} 
     (-2)^{n} 2^{(s+1)i} \Pochhammer{1/2+n-p_1}{i} 
     \Pochhammer{1/2-n+i}{n-i} 
     && \pmod{2^{s} p_1} \\ 
(3n-1)!!! 
     & \equiv 
     \sum_{i=0}^{n} \binom{p_2}{i} 
     3^{n} (-3)^{(t+1)i} \Pochhammer{1/3-p_2}{i} \Pochhammer{2/3}{n-i} 
     && \pmod{3^{t} p_2} \\ 
     & \equiv 
     \sum_{i=0}^{n} \binom{p_2}{i} 
     (-3)^{n} 3^{(t+1)i} \Pochhammer{1/3+n-p_2}{i} 
     \Pochhammer{1/3-n+i}{n-i} 
     && \pmod{3^{t} p_2} \\ 
(3n-2)!!! 
     & \equiv 
     \sum_{i=0}^{n} \binom{p_2}{i} 
     3^{n} (-3)^{(t+1)i} \Pochhammer{2/3-p_2}{i} \Pochhammer{1/3}{n-i} 
     && \pmod{3^{t} p_2} \\ 
     & \equiv 
     \sum_{i=0}^{n} \binom{p_2}{i} 
     (-3)^{n} 3^{(t+1)i} \Pochhammer{2/3+n-p_2}{i} 
     \Pochhammer{2/3-n+i}{n-i} 
     && \pmod{3^{t} p_2}. 
\end{align*} 

\section{Organization of the article} 

\subsection{Mathematica summary notebook document and 
            computational reference information} 
\label{subSection_MmSummaryNBInfo} 

The prepared summary notebook file, \TheSummaryNBFile, 
attached to the submission of this manuscript 
contains working \Mm{} code to verify the formulas, 
propositions, and other identities cited within the article 
\citep{SUMMARYNBREF-STUB}. 
Given the length of the article, 
the \Mm{} summary notebook included with this submission 
is intended to help the reviewer to process the 
article more quickly, and to help the reader with verifying and 
modifying the examples presented as 
applications of the new results cited below. 
The summary notebook also contains numerical data 
corresponding to computations of multiple examples and 
congruences specifically referenced in several places by the 
applications given in the next subsections of the article 
(see also Section \ref{subsubSection_FutureResTopics_Rmks_in_SummaryNB}). 

\subsection{A quick note on the page length of the article} 

%% 
The presentation of the multiple applications to 
special integer sequences in this manuscript is intended to give a 
more or less cohesive overview of the 
new convergent--based generating function and formal power series 
techniques, as enumerated through specific applications that 
highlight the new results proved in 
Section \ref{subSection_EnumProps_of_JFractions} and 
Section \ref{Section_Props_Of_CFracExps_OfThe_GenFactFnSeries}. 
The multitude of differing special case identities and 
integer congruence properties given in the context of the sequences 
considered below illustrate by example the 
more significant applications of these new results that 
are not readily partitioned into multiple separate articles 
related to the subject matter contained in the next few subsections. 

\subsection{Organization of the article contents by section} 

The content in the next several sections of the article is 
roughly organized according to the 
next topics summarized in the listings below. 
Several appropriate inline references related to the motivating examples and 
applications cited in these next few subsections of the article 
also suggest further topics for investigation and future research 
on the new results and factorial--related expansions of the 
integer congruence properties proved in this article. 
The final section after the bibliography section concluding the article 
(starting on page \pageref{page_StartOfTableData}) 
is an appendix containing the listings of all of the 
tables referenced within the article. 

\newcommand{\sectionpageref}[1]{
     \smaller{\textbf{\textit{(Page \pageref{#1})}}}
} 
\begin{enumerate}[leftmargin=\parindent,itemsep=-1mm]  
     \renewcommand{\labelenumi}{$\mathsmaller\blacktriangleright$ } 
     \renewcommand{\labelenumii}{$\mathsmaller\vartriangleright$ } 

     \renewcommand{\itemlabel}[3][Section]{ 
          \item {\normalsize \textbf{\underline{#1 \ref{#2}}}.\ 
          \textrm{\textbf{#3}}.\ \sectionpageref{#2}} 
     } 

\itemlabel{Section_Proofs_of_the_GenCFracReps}{
           J--fraction expansions of the 
           generalized convergent functions} \\ 
Section \ref{subSection_EnumProps_of_JFractions} 
provides a brief overview of the enumerative 
J--fraction properties 
established in Flajolet's articles that summarize the properties 
needed to prove the new results within this article. 
A short, direct proof of the convergent function representations for the 
more general product sequence expansions defined in 
\eqref{eqn_ConvGF_notation_def} is given in 
Section \ref{subSection_GenCFrac_Reps_for_GenFactFns}. 
This proof follows as a straightforward adaptation of the known 
J-fraction expansions enumerating the Pochhammer symbol, 
$\Pochhammer{x}{n}$, and the two--variable generating function for the 
Stirling numbers of the first kind from the reference 
\citep{FLAJOLET80B}. \\ 

\itemlabel{Section_Props_Of_CFracExps_OfThe_GenFactFnSeries}{
           Properties of the generalized convergent functions} \\ 
Section \ref{subsubSection_Properties_Of_ConvFn_Qhz} and 
Section \ref{subsubSection_Properties_Of_ConvFn_Phz} 
prove several additional properties and exact formulas satisfied by the 
respective convergent numerator and denominator functions, 
$\ConvFQ{h}{\alpha}{R}{z}$ and $\ConvFP{h}{\alpha}{R}{z}$. 
The results for the convergent denominator functions, 
$\ConvFQ{h}{\alpha}{R}{z}$, 
stated by the propositions in 
Section \ref{subsubSection_Properties_Of_ConvFn_Qhz} 
provide characterizations of these sequences by well--known special 
functions and orthogonal polynomial sequences, 
additional auxiliary recurrence relations, and analogues to the known 
addition and multiplication theorems for the 
confluent hypergeometric functions, 
$\HypU{-h}{b}{z}$ and $\HypM{-h}{b}{z}$. 
%% 
Besides the short direct proof given in 
Section \ref{subSection_GenCFrac_Reps_for_GenFactFns}, 
this section provides careful proofs of the convergent function 
properties needed to rigorously justify the results for the examples 
cited as particular applications elsewhere within the article. 

\itemlabel{Section_Apps_and_Examples}{Applications and motivating examples} 
\begin{enumerate}[leftmargin=\parindent]  
     \renewcommand{\itemlabel}[1]{\item \textbf{\textrm{#1}} \\ } 

\itemlabel{Statements and proofs of several lemmas. 
           \sectionpageref{subSection_Apps_and_Examples_StmtsOfLemmas}} 
\itemlabel{New congruences for the 
           $\alpha$--factorial functions, 
           generalized cases of the symbolic product sequences, the 
           Stirling number triangles, and the 
           $r$--order harmonic number sequences:
           \sectionpageref{subSection_NewCongruence_Relations_Modulo_Integer_Bases}} 
Section \ref{subSection_NewCongruence_Relations_Modulo_Integer_Bases} 
treats the new forms of the integer congruence properties satisfied by 
special cases of the product sequences, $\pn{n}{\alpha}{R}$, 
and their expansions by the special function zeros from 
\eqref{eqn_PFact_Qhz_Exp_idents-stmts_v1} in 
Section \ref{subSection_Intro_GenConvFn_Defs_and_Properties} 
studied in the references 
\citep{LGWORKS-ASYMP-SPFNZEROS2008,PROPS-ZEROS-CHYPFNS80}. 
This subsection states several new congruence identities 
satisfied by the Stirling numbers of the first kind, $\gkpSI{n}{k}$, 
modulo $2$ and modulo $3$ 
for comparison with some of the other known congruences for the 
triangle expanded in the references 
\citep[\S 4.6]{GFOLOGY} \citep[\cf \S 5.8]{ADVCOMB}. 
These results also imply the 
rational generating function expansions enumerating the 
scaled $r$--order harmonic number sequences, $(n!)^{r} \times H_n^{(r)}$, 
modulo fixed integers $p$ for the first few integer cases of 
$p \geq 2$ and $r \geq 1$ expanded in 
Section \ref{subsubSection_remark_New_Congruences_for_GenS1Triangles_and_HNumSeqs}. \\ 
{ 
     \smaller
     \textbf{\underline{Sequence References}}: \\ 
     \seqnum{A001008}, \seqnum{A002805}, \seqnum{A007406}, \seqnum{A007407}, 
     \seqnum{A007408}, \seqnum{A007409}, \seqnum{A008275}, \seqnum{A087755}, 
     and \seqnum{A130534}. 
} 

\itemlabel{Convergent--based generating function transformations and 
           identities enumerating factorial--related sums and 
           product sequences:
           \sectionpageref{subSection_DiagonalGFSequences_Apps}} 
The examples cited in 
Section \ref{subSection_DiagonalGFSequences_Apps} through 
Section \ref{subsubSection_Apps_Example_SumsOfPowers_Seqs} 
are intended to provide a number of 
specific representative applications that arise in 
enumerating factorial--related sums, 
binomial power sequences, and other special products through the 
rational convergent function approximations, $\ConvGF{h}{\alpha}{R}{z}$, 
provided by \eqref{eqn_ConvGF_notation_def} through the proof in 
Section \ref{subSection_GenCFrac_Reps_for_GenFactFns}. 
These examples illustrate expansions 
corresponding to several well--known sequences and other related identities 
formally enumerated by special cases of the 
diagonal--coefficient, or Hadamard product, generating functions defined in 
\eqref{eqn_HProdGFs_kGen_def_v1} and 
\eqref{eqn_kGenHProdGFs_RationalDiagonalGF_Idents-stmts_v1} 
of the overview to 
Section \ref{subSection_DiagonalGFSequences_Apps}. 

A number of the examples given in this section provide 
specific applications of the formal Laplace--Borel--like 
generating function transformation technique outlined below in 
Remark \ref{remark_Formal_Laplace-Borel_Transforms}. 
These formal series transformations are 
performed by diagonal coefficient extractions 
involving the rational convergent--function--based 
generating functions that enumerate the single and double 
factorial function terms implicit to the expansions of many 
factorial--related sums, products, and related 
special sequence identities. 

{ 
     \smaller\noindent 
     \textbf{\underline{Sequence References}}: \\ 
     \seqnum{A000166}, \seqnum{A000178}, \seqnum{A000225}, \seqnum{A000918}, 
     \seqnum{A000984}, \seqnum{A001044}, \seqnum{A001142}, 
     \seqnum{A002109}, \seqnum{A003422}, 
     \seqnum{A005165}, \seqnum{A008277}, \seqnum{A008292}, \seqnum{A009445}, 
     \seqnum{A010050}, \seqnum{A024023}, \seqnum{A024036}, \seqnum{A024049}, 
     \seqnum{A027641}, \seqnum{A027642}, \seqnum{A033312}, \seqnum{A061062}, 
     \seqnum{A066802}, \seqnum{A100043}, \seqnum{A100089}, \seqnum{A100732}, 
     \seqnum{A104344}, \seqnum{A166351}, and \seqnum{A184877}. 

} 

\end{enumerate} 

\itemlabel{subSection_FiniteDiffEqns_for_the_GenFactFns}{ 
           Applications of new 
           expansions by finite difference equations} \\ 
The last topics considered in 
Section \ref{subSection_FiniteDiffEqns_for_the_GenFactFns} 
provide further new applications illustrated by example of the 
$h$--order finite difference equations suggested by the 
expansions of the rational $h^{th}$ convergents, 
$\ConvGF{h}{\alpha}{R}{z}$, when the parameter, $R$, 
denotes some initially fixed indeterminate parameter with respect to the 
generalized product sequences defined by \eqref{eqn_GenFact_product_form}. 

The corresponding finite--degree, rational polynomial expansions in $n$ of the 
sequences, $\pn{n}{\alpha}{R}$, when $R$ depends linearly on $n$ 
offer a dual interpretation to the algebraic structure of the 
previous formulas for these functions 
given in terms of the special function zeros 
defined above. 
The resulting multiple sum expansions suggest 
many new applications to classical congruences and otherwise 
noteworthy special case identities concerning primality of 
subsequences of the odd positive integers and prime pairs. \\ 
{ 
     \smaller
     \textbf{\underline{Sequence References}}: \\ 
     \seqnum{A000043}, \seqnum{A000108}, \seqnum{A000215}, \seqnum{A000668}, 
     \seqnum{A000978}, \seqnum{A000984}, \seqnum{A001008}, \seqnum{A001097}, 
     \seqnum{A001220}, \seqnum{A001348}, \seqnum{A001359}, \seqnum{A002234}, 
     \seqnum{A002496}, \seqnum{A002805}, \seqnum{A002981}, \seqnum{A002982}, 
     \seqnum{A005109}, \seqnum{A005384}, \seqnum{A007406}, \seqnum{A007407}, 
     \seqnum{A007408}, \seqnum{A007409}, \seqnum{A007540}, \seqnum{A007619}, 
     \seqnum{A019434}, \seqnum{A022004}, \seqnum{A022005}, \seqnum{A023200}, 
     \seqnum{A023201}, \seqnum{A023202}, \seqnum{A023203}, \seqnum{A046118}, 
     \seqnum{A046124}, \seqnum{A046133}, \seqnum{A080075}, \seqnum{A088164}, 
     and \seqnum{A123176}. 
} 

\itemlabel{Section_ConcludingRemarks}{Conclusions and 
           remarks on future research topics} 
\itemlabel{Section_Acks}{Acknowledgements} 

\itemlabel[Appendix]{Section_appendix_StartOfTableData}{ 
          Listings of tables referenced in the article}
\begin{enumerate}[leftmargin=\parindent,itemsep=-1mm] 
     \renewcommand{\itemlabel}[3][Section]{ 
          \item {\normalsize \textbf{\underline{#1 \ref{#2}}}.\ 
          \textrm{\textbf{#3}}.\ } 
     } 

\itemlabel[Table]{table_FcfAlphankCoeffs}{
     The generalized $\alpha$--factorial coefficient triangles} %\\ 

\itemlabel[Table]{table_GenStirlingAlphaCvlPolys}{ 
     The generalized Stirling and 
     $\alpha$--factorial polynomial sequences} %\\ 

\itemlabel[Table]{table_SpCase_Listings_Of_PhzQhz_ConvFn}{
     The generalized convergent function subsequences} 

\itemlabel[Table]{table_RelfectedConvNumPolySeqs_sp_cases}{
     The reflected convergent numerator function sequences} 

\itemlabel[Table]{table_AlphaFactFns_Modulo245_spcase_examples}{
     The $\alpha$--factorial functions modulo $h$ (and $h\alpha$)} 

\itemlabel[Table]{table_ConvGF_Examples_for_PthPowerSeqs}{
     The convergent generating functions for 
     $p^{th}$ power sequences} 

\itemlabel[Table]{table_ConvNumFnSeqs_Chn_AlphaR_SpCaseListings}{
     The auxiliary convergent numerator function sequences, 
     $C_{h,n}(\alpha, R)$} 

\itemlabel[Table]{table_ConvNumFnSeqs_Rhk_Alphaz_SpCaseListings}{
     The auxiliary convergent numerator function sequences, 
     $R_{h,k}(\alpha; z)$} 

\end{enumerate} 

\end{enumerate} 

\section{The Jacobi type J--fractions for 
         generalized factorial function sequences} 
\label{Section_Proofs_of_the_GenCFracReps} 

\subsection{Enumerative properties of 
            Jacobi type J--fractions}
\label{subSection_EnumProps_of_JFractions} 

To simplify the exposition in this article, we adopt the 
notation for the Jacobi type continued fractions, or J--fractions, 
in the references 
\citep{FLAJOLET80B,FLAJOLET82} 
\citep[\cf \S 3.10]{NISTHB} \citep[\cf \S 5.5]{GFLECT}. 
Given some application--specific choices of the prescribed 
sequences, $\{a_k, b_k, c_k\}$, we consider the 
formal power series whose coefficients are generated by the 
rational convergents, $J^{[h]}(z) \defequals J^{[h]}(\{a_k, b_k, c_k\}; z)$, 
of the infinite continued fractions, 
$J(z) \defequals J^{[\infty]}(\{a_k, b_k, c_k\}; z)$, defined as follows: 
\begin{align} 
\label{eqn_CF_exp_general_form} 
%J^{[\infty]}(\{a_k, b_k, c_k\}; z) \defequals 
J(z) & = 
     \cfrac{1}{1-c_0z-\cfrac{a_0b_1 z^2}{1-c_1z- 
     \cfrac{a_1b_2 z^2}{\cdots}.}} 
%\notag 
%     & = 
%     \frac{1}{1-c_0 z} \frac{a_0b_1 z^2}{1-c_1 z} 
%     \frac{a_1b_2 z^2}{1-c_2 z} \cdots . 
\end{align} 
We briefly summarize the other enumerative properties from the 
references that are 
relevant in constructing the new factorial--function--related results 
given in the next subsections of this article 
\citep{FLAJOLET82,FLAJOLET80B} \citep[\S 5.5]{GFLECT} \citep[\cf \S 6.7]{GKP}. 

\begin{enumerate} 
     \setlength{\itemsep}{-1mm} 

\item \itemlabel{Definitions of the $h$--order convergent series} 
When $h \geq 1$, the $h^{th}$ convergent functions, 
given by the equivalent notation of 
$J^{[h]}(z)$ and $J^{[h]}(\{a_k, b_k, c_k\}; z)$ within this section, 
of the infinite continued fraction expansions of 
\eqref{eqn_CF_exp_general_form} 
are defined as the ratios, $J^{[h]}(z) \defequals P_h(z) / Q_h(z)$. 

The component functions corresponding to the convergent 
numerator and denominator sequences, $P_h(z)$ and $Q_h(z)$, 
each satisfy second--order 
finite difference equations (in $h$) of the respective forms 
defined by the next two equations. 
\begin{align*} 
P_h(z) & = (1-c_{h-1} z) P_{h-1}(z)-a_{h-2}b_{h-1} z^2 P_{h-2}(z) + 
           \Iverson{h = 1} \\ 
Q_h(z) & = (1-c_{h-1} z) Q_{h-1}(z)-a_{h-2}b_{h-1} z^2 Q_{h-2}(z) + 
           (1-c_0 z) \Iverson{h = 1}+\Iverson{h=0} 
\end{align*} 

\item \itemlabel{Rationality of truncated convergent function approximations} 
Let $p_n = p_n(\{a_k, b_k, c_k\}) \defequals [z^n] J(z)$ 
denote the expected term corresponding to the coefficient of $z^n$ in the 
formal power series expansion defined by the infinite J--fraction from 
\eqref{eqn_CF_exp_general_form}. 
For all $n \geq 0$, we know that the $h^{th}$ convergent functions 
have truncated power series expansions that satisfy 
\begin{equation*} 
p_n(\{a_k, b_k, c_k\}) = [z^n] J^{[h]}(\{a_k, b_k, c_k\}; z),\ 
     \forall n \leq h. 
\end{equation*} 
In particular, the series coefficients of the $h^{th}$ convergents are 
always at least $h$--order accurate as formal power series 
expansions in $z$ that exactly enumerate the expected sequence terms, 
$\left(p_n\right)_{n \geq 0}$. 

The resulting \quotetext{\emph{eventually periodic}} nature 
suggested by the approximate sequences enumerated by 
the rational convergent functions in $z$ is formalized in the 
congruence properties given below in 
\eqref{eqn_EnumProps_Of_JTypeCFracs_congruence_rels-stmt_v1} 
\citep{FLAJOLET82} \citep[See \S 2, \S 5.7]{GFLECT}. 

\item \itemlabel{Congruence properties modulo integer bases} 
Let $\lambda_k \defequals a_{k-1} b_{k}$ and suppose that the 
corresponding bases, $M_h$, are formed by the products 
$M_h \defequals \lambda_1 \lambda_2 \cdots \lambda_h$ for $h \geq 1$. 
Whenever $N_h \mid M_h$, and for any $n \geq 0$, we have that 
\begin{equation} 
\label{eqn_EnumProps_Of_JTypeCFracs_congruence_rels-stmt_v1} 
p_n(\{a_k, b_k, c_k\}) \equiv [z^n] J^{[h]}(\{a_k, b_k, c_k\}; z) \pmod{N_h}, 
\end{equation} 
which is also true of all partial sequence terms enumerated by the 
$h^{th}$ convergent functions modulo any integer divisors of the $M_h$ 
\citep[\cf \S 5.7]{GFLECT}. 

\end{enumerate} 

\subsection{A short direct proof of the J--fraction representations for the 
            generalized product sequence generating functions} 
\label{subSection_GenCFrac_Reps_for_GenFactFns} 

We omit the details to a more combinatorially flavored proof that the 
J--fraction series defined by the convergent functions in 
\eqref{eqn_ConvGF_notation_def} do, in fact, correctly enumerate the expected 
symbolic product sequences in \eqref{eqn_GenFact_product_form}. 
Instead, a short direct proof following from the J--fraction results 
given in Flajolet's first article is sketched below. 
Even further combinatorial interpretations of the sequences generated by 
these continued fraction series, 
their relations to the Stirling number triangles, and other 
properties tied to the coefficient triangles studied in depth by the article 
\citep{MULTIFACTJIS} based on the properties of these new 
J--fractions is suggested as a topic for later investigation. 

\begin{definition} 
%% 
The prescribed sequences in the J--fraction expansions defined by 
\eqref{eqn_CF_exp_general_form} in the previous section, corresponding to 
\textbf{(\emph{i})} the 
Pochhammer symbol, $\Pochhammer{x}{n}$, and 
\textbf{(\emph{ii})} the 
convergent functions enumerating the generalized products, 
$\pn{n}{\alpha}{R}$, or equivalently, the 
Pochhammer $k$--symbols, $\Pochhammer{R}{n,\alpha}$, 
over any fixed $\alpha \in \mathbb{Z}^{+}$ and 
indeterminate, $R$, are defined as follows: 
\begin{align*} 
\tag{\em Pochhammer Symbol} 
\{\as_k(x), \bs_k(x), \cs_k(x)\} & 
     \quad\overset{\text{ (i)}}{\defequals}\quad 
     \left\{ x+k, k, x+2k \right\} \\ 
\tag{\em Generalized Products} 
\{\af_k(\alpha, R), \bfcf_k(\alpha, R), \cfcf_k(\alpha, R)\} & 
     \quad\overset{\text{ (ii)}}{\defequals}\quad 
     \left\{ 
     R+k\alpha, k, \alpha \cdot (R+2k \alpha) 
     \right\}.  
\end{align*} 
\textbf{\textit{Claim}:} 
We claim that the modified J--fractions, $R_0(R / \alpha, \alpha z)$, 
that generate the corresponding series expansions in 
\eqref{eqn_PochhammerSymbol_InfCFrac_series_Rep_example-v1} 
enumerate the analogs terms of the generalized 
symbolic product sequences, $\pn{n}{\alpha}{R}$, 
defined by \eqref{eqn_GenFact_product_form}. 
%% 
\end{definition} 
%%%% 
\begin{proof}[Proof of the Claim.] 
%% 
First, an appeal to the polynomial expansions of both the 
Pochhammer symbol, $\Pochhammer{x}{n}$, and 
then of the products, $\pn{n}{\alpha}{R}$, 
defined by \eqref{eqn_GenFact_product_form} by the 
Stirling numbers of the first kind, yields the following sums: 
\begin{align} 
\notag 
p_n(\alpha, R) \times z^n & = 
     \left( 
     \prod_{j=0}^{n-1} (R+j\alpha) + \Iverson{n = 0} 
     \right) \times z^n \\ 
\notag 
   & = 
     \left( 
     \sum_{k=0}^{n} \gkpSI{n}{k} \alpha^{n-k} R^{k} 
     \right) \times z^{n} \\ 
\label{eqn_gen_pnAlphaR_seq_PochhammerSymbol_exps} 
   & = 
     \undersetbrace{\Pochhammer{R}{n,\alpha} \times z^{n}}{ 
     \alpha^{n} \cdot \Pochhammer{\frac{R}{\alpha}}{n} 
     \times z^{n}
     }. 
\end{align} 
Finally, after a parameter substitution of 
$x \defmapsto R / \alpha$ together with the change of variable 
$z \defmapsto \alpha z$ in the first results from 
Flajolet's article \citep{FLAJOLET80B}, 
we obtain identical forms of the convergent--based definitions for the 
generalized J--fraction definitions given in 
Definition \ref{def_GenConvFns_PFact_Phz_eqn_QFact_Qhz-defs_intro_v1}. 
%% 
\end{proof} 

\subsection{Alternate exact expansions of the generalized convergent functions} 
\label{subSection_AltExps_of_the_GenConvFns} 

%% 
The notational inconvenience introduced in the inner sums of 
\eqref{eqn_AlphaFactFn_Exact_PartialFracsRep_v1} and 
\eqref{eqn_AlphaFactFn_Exact_PartialFracsRep_v2}, 
implicitly determined by the shorthand for the coefficients, 
$c_{h,j}(\alpha, R)$, 
each of which depend on the more difficult terms of the 
convergent numerator function sequences, $\FP_h(\alpha, R; z)$, 
is avoided in place of alternate recurrence relations for each 
finite $h^{th}$ convergent function, $\ConvGF{h}{\alpha}{R}{z}$, 
involving paired products of the 
denominator polynomials, $\FQ_h(\alpha, R; z)$. 
These denominator function sequences are related to generalized forms of the 
Laguerre polynomial sequences as follows 
(see Section \ref{subsubSection_Properties_Of_ConvFn_Qhz}): 
\begin{align} 
\label{eqn_PFact_Qhz_Uident_LPoly_exp-stmt_v0} 
\FQ_h(\alpha, R; z) & = 
     (-\alpha z)^{h} \cdot h! \cdot 
     L_h^{(R / \alpha - 1)}\left((\alpha z)^{-1}\right). 
\end{align} 
%% 
The expansions provided by the formula in 
\eqref{eqn_PFact_Qhz_Uident_LPoly_exp-stmt_v0} 
suggest useful alternate formulations of the 
congruence results given below in 
Section \ref{subSection_NewCongruence_Relations_Modulo_Integer_Bases} 
when the Laguerre polynomial, or corresponding 
confluent hypergeometric function, zeros 
are considered to be less complicated in form than the 
more involved sums expanded through the numerator functions, 
$\FP_h(\alpha, R; z)$. 

\begin{example}[Recurrence Relations for the Convergent Functions and Laguerre Polynomials] 
%% 
The well--known cases of the enumerative properties satisfied by the 
expansions of the convergent function sequences given in the references 
immediately yield the following relations 
\citep[\S 3]{FLAJOLET80B} \citep[\S 1.12(ii)]{NISTHB}: 
\begin{align} 
\notag 
\ConvGF{h}{\alpha}{R}{z} & = 
     \ConvGF{h-k}{\alpha}{R}{z} \\ 
\notag 
     & \phantom{= \quad } + 
     \sum_{i=0}^{k-1} 
     \frac{\alpha^{h-i-1} (h-i-1)! \cdot p_{h-i-1}(\alpha, R) \cdot 
     z^{2(h-i-1)}}{ 
     \FQ_{h-i}(\alpha, R; z) \FQ_{h-i-1}(\alpha, R; z)},\ 
     h > k \geq 1 \\ 
\label{eqn_AltExps_of_the_GenConvFns-rec_properties-stmts_v1} 
\ConvGF{h}{\alpha}{R}{z} & = 
     \sum_{i=0}^{h-1} 
     \frac{\alpha^{h-i-1} (h-i-1)! \cdot p_{h-i-1}(\alpha, R) \cdot 
     z^{2(h-i-1)}}{ 
     \FQ_{h-i}(\alpha, R; z) \FQ_{h-i-1}(\alpha, R; z)} \\ 
\notag 
     & = 
     \sum_{i=0}^{h-1} \binom{\frac{R}{\alpha}+i-1}{i} \times 
     \frac{(-\alpha z)^{-1}}{(i+1) \cdot 
     L_{i}^{(R / \alpha - 1)}\left((\alpha z)^{-1}\right) 
     L_{i+1}^{(R / \alpha - 1)}\left((\alpha z)^{-1}\right)},\ 
     h \geq 2. 
\end{align} 
%% 
The convergent function recurrences expanded by 
\eqref{eqn_AltExps_of_the_GenConvFns-rec_properties-stmts_v1} 
provide identities for particular cases of the generalized 
Laguerre polynomial sequences, $L_n^{(\beta)}(x)$, 
expressed as finite sums over paired reciprocals of the 
sequence at the same choices of the $x$ and $\beta$. 
%% 
\ExampleQED 
\end{example} 

Topics aimed at finding new results obtained from other known, strictly 
continued--fraction--related properties of the 
convergent function sequences beyond the proofs given in 
Section \ref{Section_Props_Of_CFracExps_OfThe_GenFactFnSeries} 
are suggested as a further avenue to approach the 
otherwise divergent ordinary generating functions for these 
more general cases of the integer--valued 
factorial--related product sequences 
(see Section \ref{subsubSection_footnote_CatalanNumber_S-Fraction_Apps}) 
\citep[\cf \S 10]{HARDYWRIGHTNUMT}. 


\begin{remark}[Related Convergent Function Expansions] 
%% 
For $h-i \geq 0$, 
we have a noteworthy Rodrigues--type formula 
satisfied by the Laguerre polynomial sequences in 
\eqref{eqn_PFact_Qhz_Uident_LPoly_exp-stmt_v0} 
stated by the reference as follows \citep[\S 18.5(ii)]{NISTHB}: 
\begin{align} 
\notag 
(-\alpha z)^{h-i} \cdot (h-i)! \times 
     L_{h-i}^{(\beta)}\left(\frac{1}{\alpha z}\right) & = 
     \alpha^{h-i} \cdot z^{2h-2i+\beta+1} e^{1 / \alpha z} \times 
     \left\{ 
     \frac{e^{-1 / \alpha z}}{z^{\beta+1}} 
     \right\}^{(h-i)}. 
\end{align} 
The multiple derivatives implicit to the statement in the previous 
equation then have the additional expansions through the 
product rule analogue provided by the formula of Halphen 
from the reference given in the form of the following equation 
for natural numbers $n \geq 0$, and where the 
notation for the functions, $F(z)$ and $G(z)$, employed in the 
previous equation corresponds to any prescribed choice of these functions 
that are each $n$ times continuously differentiable at the point $z \neq 0$ 
\citep[\S 3 Exercises, p.\ 161]{ADVCOMB}: 
\begin{align*} 
\tagtext{Halphen's Product Rule} 
\Biggl\{ F\left(\frac{1}{z}\right) G(z) \Biggr\}^{(n)} & = 
     \sum_{k=0}^{n} \binom{n}{k} \frac{(-1)^{k}}{z^{k}} \cdot 
     F^{(k)}\left(\frac{1}{z}\right) 
     \left\{ \frac{G(z)}{z^{k}} \right\}^{(n-k)}. 
\end{align*} 
The next particular restatements of 
\eqref{eqn_AltExps_of_the_GenConvFns-rec_properties-stmts_v1} 
then follow easily from the last two equations as 
\begin{align} 
\notag 
& \ConvGF{h}{\alpha}{R}{z} = 
     \ConvGF{h-k}{\alpha}{R}{z} \\ 
\notag 
     & \phantom{\Conv_h } + 
     \sum_{i=0}^{k-1} 
     \frac{(h-i-1)!}{(\alpha z) \cdot \Pochhammer{R / \alpha}{h-i}} \times 
     \frac{1}{_1F_1\left(-(h-i); \frac{R}{\alpha}; \frac{1}{\alpha z}\right) 
     {_1F_1}\left(-(h-i-1); \frac{R}{\alpha}; \frac{1}{\alpha z}\right)},\ 
     1 \leq k < h \\ 
\notag 
& \ConvGF{h}{\alpha}{R}{z} = 
     \sum_{i=0}^{h-1} 
     \frac{(h-i-1)!}{(\alpha z) \cdot \Pochhammer{R / \alpha}{h-i}} \times 
     \frac{1}{_1F_1\left(-(h-i); \frac{R}{\alpha}; \frac{1}{\alpha z}\right) 
     {_1F_1}\left(-(h-i-1); \frac{R}{\alpha}; \frac{1}{\alpha z}\right)} \\ 
\label{eqn_AltExps_of_the_GenConvFns-rec_properties-stmts_v2} 
& \phantom{\ConvGF{h}{\alpha}{R}{z}} = 
     \sum_{i=1}^{h} 
     \binom{\frac{R}{\alpha}+i-1}{i-1}^{-1} \times 
     \frac{(-R z)^{-1}}{
     {_1F_1}\left(1-i; \frac{R}{\alpha}; \frac{1}{\alpha z}\right) 
     {_1F_1}\left(-i; \frac{R}{\alpha}; \frac{1}{\alpha z}\right)}, 
\end{align} 
where the $h^{th}$ convergents, $\ConvGF{h}{\alpha}{R}{z}$, 
are rational in $z$ for all $h \geq 1$. 
The functions $_1F_1(a; b; z)$, or 
$\HypM{a}{b}{z} = \sum_{s \geq 0} \frac{(a)_s}{(b)_s s!} z^{s}$, 
in the previous equations denote 
\emph{Kummer's confluent hypergeometric function} 
\citep[\S 13.2]{NISTHB} \citep[\S 5.5]{GKP}. 
%% 
\RemarkQED 
\end{remark} 

\section{Properties of the generalized convergent functions} 
\label{Section_Props_Of_CFracExps_OfThe_GenFactFnSeries} 

We first focus on the comparatively simple factored expressions for the 
series coefficients of the denominator sequences in the 
results proved in 
Section \ref{subsubSection_Properties_Of_ConvFn_Qhz}. 
The identification of the convergent denominator functions as 
special cases of the confluent hypergeometric function 
yields additional identities providing 
analogs addition and multiplication theorems for these functions 
with respect to the parameter $z$, as well as a number of further, new 
recurrence relations derived from established relations 
stated in the references, 
such as those provided by Kummer's transformations. 
These properties form a superset of 
extended results beyond the immediate, more combinatorial, 
known relations for the J--fractions summarized in 
Section \ref{subSection_EnumProps_of_JFractions} and in 
Section \ref{subSection_AltExps_of_the_GenConvFns}. 
%% 
The numerator convergent sequences considered in 
Section \ref{subsubSection_Properties_Of_ConvFn_Phz} 
have less obvious 
expansions through special functions, or otherwise more well--known 
polynomial sequences\footnotemod{ 
     A point concerning the relative simplicity of the 
     expressions of the denominator convergent polynomials 
     compared to the numerator convergent sequences is 
     also mentioned 
     in {\S 3.1} of Flajolet's article \citep{FLAJOLET80B}. 
}. 

%% 
The last characterization of the generalized convergent functions by the 
Laguerre polynomials in 
Proposition \ref{prop_Closed-Form_Rep_for_ConvFn_Qhz} 
below provides the factorizations over the 
zeros of the classical orthogonal polynomial sequences studied in the 
references \citep{LGWORKS-ASYMP-SPFNZEROS2008,PROPS-ZEROS-CHYPFNS80}, 
required to state the results provided by 
\eqref{eqn_AlphaFactFn_Exact_PartialFracsRep_v1} and 
\eqref{eqn_AlphaFactFn_Exact_PartialFracsRep_v2} from 
Section \ref{subSection_Intro_GenConvFn_Defs_and_Properties}, and 
more generally by 
\eqref{eqn_AlphaFactFnModulop_congruence_stmts} in 
Section \ref{subSection_Congruences_for_Series_ModuloIntegers_p}. 
The expansions of the convergent numerator functions stated in 
Section \ref{subsubSection_Properties_Of_ConvFn_Phz}, and 
of the auxiliary subsequences defined by 
Section \ref{subsubSection_Properties_Of_ConvFn_Phz-AuxNumFn_Subsequences}, 
provide further identities for the particular congruences satisfied by 
many of the prime--related identities and notable prime number subsequences 
cited as the applications of the new results expanded in 
Section \ref{subSection_FiniteDiffEqns_for_the_GenFactFns}. 


\subsection{The convergent denominator function sequences} 
\label{subsubSection_Properties_Of_ConvFn_Qhz} 

In contrast to the convergent numerator functions, 
$\FP_h(\alpha, R; z)$, discussed next in 
Section \ref{subsubSection_Properties_Of_ConvFn_Phz}, the 
corresponding denominator functions, $\FQ_h(\alpha, R; z)$, 
are readily expressed through well--known special functions. 
The first several special cases given in 
\tableref{table_SpCase_Listings_Of_Qhz_ConvFn} 
suggest the next identity, which is proved following 
Proposition \ref{prop_Closed-Form_Rep_for_ConvFn_Qhz} below. 
\begin{align} 
\label{eqn_PFact_Qhz_product_ident} 
\FQ_h(\alpha, R; z) & = 
     \sum_{k=0}^{h} \binom{h}{k} (-1)^{k} \left(\prod_{j=0}^{k-1} 
     (R + (h-1-j)\alpha)\right) z^k 
\end{align} 
The convergent denominator functions are expanded by the 
\emph{confluent hypergeometric functions}, 
$\HypU{-h}{b}{w}$ and $\HypM{-h}{b}{w}$, 
and equivalently by the 
\emph{associated Laguerre polynomials}, $L_h^{(b-1)}(w)$, 
when $b \defmapsto R / \alpha$ and $w \defmapsto (\alpha z)^{-1}$ 
through the relations proved by the next proposition 
\citep[\cf \S 18.6(iv); \S 13.9(ii)]{NISTHB}. 

\begin{prop}[Exact Representations by Special Functions] 
\label{prop_Closed-Form_Rep_for_ConvFn_Qhz} 
%% 
The convergent denominator functions, $\FQ_h(\alpha, R; z)$, are 
expanded in terms of the confluent hypergeometric function and the 
associated Laguerre polynomials through the following results: 
\StartGroupingSubEquations 
\begin{align} 
\label{eqn_PFact_Qhz_Uident} 
\FQ_h(\alpha, R; z) 
     & = 
     \left(\alpha z\right)^{h} \times 
     \HypU{-h}{R / \alpha}{(\alpha z)^{-1}} \\ 
\label{eqn_PFact_Qhz_Mident} 
     & = 
     \left(-\alpha z\right)^{h} \Pochhammer{\frac{R}{\alpha}}{h} \times 
     \HypM{-h}{R / \alpha}{(\alpha z)^{-1}} \\ 
\label{eqn_PFact_Qhz_Uident_LPoly_exp-stmt_v1} 
   & = 
     (-\alpha z)^{h} \cdot h! \times 
     L_h^{(R / \alpha - 1)}\left((\alpha z)^{-1}\right). 
\end{align} 
\EndGroupingSubEquations 
%% 
\end{prop} 
%%%% 
\begin{proof} 
%% 
We proceed to prove the first identity in 
\eqref{eqn_PFact_Qhz_Uident} by induction. 
It easy to verify by computation 
(see Table \ref{table_SpCase_Listings_Of_Qhz_ConvFn}) 
that the left--hand--side and right--hand--sides of 
\eqref{eqn_PFact_Qhz_Uident} coincide when $h = 0$ and $h = 1$. 
For $h \geq 2$, we apply the recurrence relation 
from \eqref{eqn_QFact_Qhz} to write the 
right--hand--side of \eqref{eqn_PFact_Qhz_Uident} as 
\begin{align} 
\label{eqn_FQhz_HyperFactU_proof_intermed_eqn-v1} 
\FQ_h(\alpha, R; z) & = (1-(R+2\alpha (h-1)) z) 
     \HypU{-h+1}{R / \alpha}{(\alpha z)^{-1}} (\alpha z)^{h-1} \\ 
\notag 
   & \phantom{= (1\ \ } - 
     \alpha (R+\alpha (h-2)) (h-1) z^2 
     \HypU{-h+2}{R / \alpha}{(\alpha z)^{-1}} (\alpha z)^{h-2}. 
\end{align} 
The proof is completed using the 
known recurrence relation for the confluent hypergeometric function 
stated in reference as \citep[\S 13.3(i)]{NISTHB} 
\begin{equation} 
\label{eqn_HyperFactU_proof_recurrence_ident} 
\HypU{-h}{b}{u} = (u-b-2(h-1)) \HypU{-h+1}{b}{u} - 
     (h-1) (b+h-2) \HypU{-h+2}{b}{u}. 
\end{equation} 
In particular, we can rewrite 
\eqref{eqn_FQhz_HyperFactU_proof_intermed_eqn-v1} as 
\begin{align} 
\notag 
\FQ_h(\alpha, R; z) & = 
     (\alpha z)^{h} \Biggl[\left((\alpha z)^{-1}-\left( 
     \frac{R}{\alpha} + 2(h-1)\right)\right) 
     \HypU{-h+1}{R / \alpha}{(\alpha z)^{-1}} \\ 
   & \phantom{= (\alpha z)^{h} \Biggl[ \Bigl(} - 
     \left(\frac{R}{\alpha}+h-2\right) (h-1) 
     \HypU{-h+2}{R / \alpha}{(\alpha z)^{-1}}\Biggr], 
\end{align} 
which implies \eqref{eqn_PFact_Qhz_Uident} in the special case of 
\eqref{eqn_HyperFactU_proof_recurrence_ident} where 
$(b, u) \defequals (R / \alpha, (\alpha z)^{-1})$. 
The second characterizations of $\FQ_h(\alpha, R; z)$ by 
Kummer's confluent hypergeometric function, $M(a, b, z)$, 
in \eqref{eqn_PFact_Qhz_Mident}, and by the 
Laguerre polynomials stated in 
\eqref{eqn_PFact_Qhz_Uident_LPoly_exp-stmt_v1} follow from the 
first result whenever $h \geq 0$ \citep[\S 13.6(v), \S 18.11(i)]{NISTHB}. 
%% 
\end{proof} 

\begin{proof}[Proof of Equation \eqref{eqn_PFact_Qhz_product_ident}] 
The first identity for the denominator functions, $\FQ_h(\alpha, R; z)$, 
conjectured from the special case table listings by 
\eqref{eqn_PFact_Qhz_product_ident}, follows from the 
first statement of the previous proposition. 
We cite the particular expansions of $U(-n, b, z)$ when 
$n \geq 0$ is integer--valued involving the Pochhammer symbol, $(x)_n$, 
stated as follows \citep[\S 13.2(i)]{NISTHB}: 
\begin{align} 
\notag 
U(-n, b, z) & = \sum_{k=0}^{n} \binom{n}{k} (b+k)_{n-k} (-1)^n (-z)^k \\ 
\label{eqn_HyperFactU_proof_sum_ident-v2} 
   & = 
     \sum_{k=0}^{n} \binom{n}{k} (b+n-k)_{k} (-1)^{k} z^{n-k} 
\end{align} 
The second sum for the confluent hypergeometric function given in 
\eqref{eqn_HyperFactU_proof_sum_ident-v2} 
then implies that the right--hand--side of 
\eqref{eqn_PFact_Qhz_Uident} can be expanded as follows: 
\begin{align*} 
\FQ_h(\alpha, R; z) & = 
     \left(\alpha z\right)^{h} 
     \HypU{-h}{R / \alpha}{(\alpha z)^{-1}} \\ 
   & = 
     (\alpha z)^h \sum_{k=0}^{h} \binom{h}{k} (-1)^k 
     \left(\frac{R}{\alpha} + h-k\right)_{k} (\alpha z)^{k-h} \\ 
   & = \phantom{(\alpha z)^h} 
     \sum_{k=0}^{h} \binom{h}{k} 
     \left(\frac{R}{\alpha} + h-k\right)_{k} (-\alpha z)^{k} \\ 
   & = \phantom{(\alpha z)^h} 
     \sum_{k=0}^{h} \binom{h}{k} 
     \underset{
     \mathlarger{(\pm 1)^{k} \pn{k}{\mp \alpha}{\pm R \pm (h-1) \alpha}}
     }{ 
     \underbrace{ 
     \left( 
     (-1)^{k} \times \prod_{j=0}^{k-1} \left(R+(h-1-j) \alpha \right) 
     \right) 
     } 
     } 
     z^{k}. 
\end{align*} 
The last line of previous equations 
provides the required expansion to complete a proof of the 
first identity cited in \eqref{eqn_PFact_Qhz_product_ident}. 
%% 
\end{proof} 

\begin{proof}[Corollaries] 
%% 
The coefficients of $z$ from \eqref{eqn_PFact_Qhz_product_ident} 
also yield the next identities involving the product sequences from 
\eqref{eqn_GenFact_product_form} 
that are employed in formulating several of the new results given 
in Section \ref{subSection_FiniteDiffEqns_for_the_GenFactFns}. 
In particular, we obtain the alternate restatements of these 
coefficients provided by the following equations: 
\begin{align*} 
\tagonce\label{eqn_CoeffsOfzk_FQhaRz_restmts_by_GenProductSeqs-v1} 
[z^k] \FQ_p(\alpha, R; z) & = 
     \binom{p}{k} (-1)^{k} 
     p_k(-\alpha, R + (p-1) \alpha) 
     \cdot \Iverson{0 \leq k \leq p} \\ 
     & = 
     \binom{p}{k} \alpha^{k} \Pochhammer{1-p-R / \alpha}{k} 
     \cdot \Iverson{0 \leq k \leq p} \\ 
[z^k] \FQ_p(\alpha, R; z) & = 
     \binom{p}{k} p_k(\alpha, -R - (p-1) \alpha) 
     \cdot \Iverson{0 \leq k \leq p} \\ 
     & = 
     \binom{p}{k} (-\alpha)^{k} \FFactII{\left(R / \alpha + p-1\right)}{k} 
     \cdot \Iverson{0 \leq k \leq p}. 
\qedhere 
\end{align*} 
%% 
\end{proof} 

\begin{cor}[Recurrence Relations] 
\label{cor_HypU-Based_recurrences_for_FQhz} 
%% 
For $h \geq 0$ and any integers $s > -h$, the convergent denominator functions, 
$\FQ_h(\alpha, R; z)$, satisfy the reflection identity, or 
analogue to \emph{Kummer's transformation} for the 
confluent hypergeometric function, given by 
\StartGroupingSubEquations 
\begin{equation} 
\label{eqn_RecurrenceRelations_HypU_for_FQhz_stmt-v1} 
\FQ_h(\alpha, \alpha s; z) = \FQ_{h+s-1}(\alpha; \alpha (2-s); z). 
\end{equation} 
Additionally, for $h \geq 0$ these functions satisfy 
recurrence relations of the following forms: 
\begin{align} 
\label{eqn_RecurrenceRelations_HypU_for_FQhz_stmt-v2} 
(R+(h-1)\alpha) z \FP_h(\alpha, R-\alpha; z) + ((\alpha-R) z-1) 
     \FQ_h(\alpha, R; z) + \FQ_h(\alpha, R+\alpha; z) & = 0 \\ 
\notag 
\FQ_h(\alpha, R; z) + \alpha h z \FQ_{h-1}(\alpha, R; z) - 
     \FQ_h(\alpha, R-\alpha; z) & = 0 \\ 
\notag 
(R+\alpha h) z \FQ_h(\alpha, R; z) + \FQ_{h+1}(\alpha, R; z) - 
     \FQ_{h}(\alpha, R+\alpha; z) & = 0 \\ 
\notag 
(1-\alpha h z) \FQ_h(\alpha, R; z) - \FQ_h(\alpha, R+\alpha; z) - 
     \alpha h (R + (h - 1) \alpha) z^2 \FQ_{h-1}(\alpha, R; z) & = 0 \\ 
\notag 
(1-(h+1) \alpha z) \FQ_h(\alpha, R; z) - \FQ_{h+1}(\alpha, R; z) - 
     (R + (h-1) \alpha) z \FQ_h(\alpha, R-\alpha; z) & = 0 \\ 
\notag 
\alpha (h-1) z \FQ_{h-2}(\alpha, R+2\alpha; z) - (1-Rz) 
     \FQ_{h-1}(\alpha, R+\alpha; z) + \FQ_h(\alpha, R; z) & = \Iverson{h = 0} 
\end{align} 
\EndGroupingSubEquations 
%% 
\end{cor} 
%%%% 
\begin{proof} 
%% 
The first equation results from \textit{Kummer's transformation} 
for the confluent hypergeometric function, $\HypU{a}{b}{z}$, 
given by \citep[\S 13.2(vii)]{NISTHB} 
\begin{equation*} 
\HypU{a}{b}{z} = z^{1-b} \HypU{a-b+1}{2-b}{z}. 
\end{equation*} 
In particular, when $R \defequals \alpha s$ and $h+s-1 \geq 0$ 
Proposition \ref{prop_Closed-Form_Rep_for_ConvFn_Qhz} 
implies that 
\begin{align*} 
\FQ_h(\alpha, R; z) & = \left(\alpha z\right)^{h+R/\alpha - 1} 
     \HypU{-(h+\frac{R}{\alpha}-1)}{\frac{2\alpha-R}{\alpha}}{ 
     (\alpha z)^{-1}} \\ 
   & = 
     \FQ_{h+R/\alpha - 1}(\alpha, 2\alpha-R; z). 
\end{align*} 
The recurrence relations stated in 
\eqref{eqn_RecurrenceRelations_HypU_for_FQhz_stmt-v2} follow 
similarly consequences of the first proposition by 
applying the known results for the 
confluent hypergeometric functions cited in the reference 
\citep[\S 13.3(i)]{NISTHB}. 
%% 
\end{proof} 

\begin{prop}[Addition and Multiplication Theorems] 
\label{cor_HypU_fn_FiniteSums_Involving_FQhz} 
%% 
Let $z, w \in \mathbb{C}$ with $z \neq w$ and suppose that $z \neq 0$. 
For a fixed $\alpha \in \mathbb{Z}^{+}$ and $h \geq 0$, the 
following finite sums provide two 
addition theorem analogues satisfied by the 
sequences of convergent denominator functions: 
\begin{align*} 
\tagonce\label{eqn_FQhaRz_HypU_FiniteSum_Idents_stmt} 
\FQ_h(\alpha, R; z-w) & = 
     \sum_{n=0}^{h} \frac{(-h)_n (-w)^n (z-w)^{h-n}}{z^{h} \cdot n!} 
     \FQ_{h-n}(\alpha, R+\alpha n; z) \\ 
\FQ_h(\alpha, R; z-w) & = 
     \sum_{n=0}^{h} \frac{(-h)_n \left(1-h-\frac{R}{\alpha}\right)_n 
     (\alpha w)^{n}}{n!} 
     \FQ_{h-n}(\alpha, R; z) \\ 
     & = 
     \sum_{n=0}^{h} \binom{h}{n} \binom{h+\frac{R}{\alpha}-1}{n} 
     \times (\alpha w)^{n} n! \times 
     \FQ_{h-n}(\alpha, R; z). 
\end{align*} 
The corresponding multiplication theorems for the 
denominator functions are stated similarly for $h \geq 0$ 
in the form of the following equations: 
\begin{align*} 
\tagonce\label{eqn_FQhaRz_HypU_FiniteSum_Idents-v_MultThms_stmts} 
%\FQ_h(\alpha, R; zw) & = \sum_{n=0}^{h} \frac{(-h)_n (-1)^n}{n!} 
%     w^{h} \left(w^{-1}-1\right)^{n} \FQ_{h-n}(\alpha, R+\alpha n; z) \\ 
\FQ_h(\alpha, R; zw) & = 
     \sum_{n=0}^{h} \frac{(-h)_n (w-1)^n w^{h-n}}{n!} 
     \FQ_{h-n}(\alpha, R+\alpha n; z) \\ 
%\FQ_h(\alpha, R; zw) & = \sum_{n=0}^{h} \frac{(-h)_n 
%     \left(1-h-\frac{R}{\alpha}\right)_n}{n!} (\alpha z)^n 
%     w^n \left(w^{-1}-1\right)^{n} \FQ_{h-n}(\alpha, R; z). 
\FQ_h(\alpha, R; zw) & = 
     \sum_{n=0}^{h} \frac{(-h)_n \left(1-h-\frac{R}{\alpha}\right)_n 
     (1-w)^{n} (\alpha z)^{n}}{n!} 
     \FQ_{h-n}(\alpha, R; z) \\ 
     & = 
     \sum_{n=0}^{h} 
     \binom{h}{n} \binom{n-h-\frac{R}{\alpha}}{n} \times 
     \left(\alpha z (w-1)\right)^{n} n! \times 
     \FQ_{h-n}(\alpha, R; z) 
\end{align*} 
%% 
\end{prop} 
%%%% 
\begin{proof}[Proof of the Addition Theorems] 
%% 
The sums stated in \eqref{eqn_FQhaRz_HypU_FiniteSum_Idents_stmt} 
follow from special cases of established addition theorems for the 
confluent hypergeometric function, $\HypU{a}{b}{x+y}$, cited in 
\citep[\S 13.13(ii)]{NISTHB}. 
The particular addition theorems required in the proof are 
provided as follows: 
\begin{align} 
\label{eqn_AdditionMultThm_for_HypU_proof_stmts-v1} 
\HypU{a}{b}{x+y} & = \sum_{n=0}^{\infty} \frac{(a)_n (-y)^n}{n!} 
     \HypU{a+n}{b+n}{x},\ |y| < |x| \\ 
\notag 
\HypU{a}{b}{x+y} & = \left(\frac{x}{x+y}\right)^{a} \sum_{n=0}^{\infty} 
     \frac{(a)_n (1+a-b)_n y^n}{n! (x+y)^n} \HypU{a+n}{b}{x},\ 
     \Re[y / x] > -\frac{1}{2}. 
\end{align} 
First, 
observe that in the special case inputs to $\HypU{a}{b}{z}$ 
resulting from the application of 
Proposition \ref{prop_Closed-Form_Rep_for_ConvFn_Qhz} 
involving the functions 
\begin{align*} 
\FQ_h(\alpha, R; z) & = \HypU{-h}{R / \alpha}{(\alpha z)^{-1}}, 
\end{align*} 
in the infinite sums of 
\eqref{eqn_AdditionMultThm_for_HypU_proof_stmts-v1} 
lead to to finite sum identities corresponding to the inputs, $h$, to 
$\FQ_h(\alpha, R; z)$ where $h \geq 0$. 
More precisely, the definition of the convergent denominator sequences 
provided by \eqref{eqn_QFact_Qhz} requires that 
$\FP_h(\alpha, R; z) = 0$ whenever $h < 0$. 

To apply the cited results for $\HypU{a}{b}{x+y}$ in these cases, 
let $z \neq w$, assume that both $\alpha, z \neq 0$, and 
suppose the parameters corresponding to $x$ and $y$ 
in \eqref{eqn_FQhaRz_HypU_FiniteSum_Idents_stmt} are defined so that 
\begin{equation} 
\label{eqn_AdditionMultThm_analogues_proof_stmt-v1} 
x \defequals (\alpha z)^{-1},\ 
y \defequals \frac{1}{\alpha}\left((z-w)^{-1} - z^{-1}\right),\ 
x + y = \left(\alpha (z-w)\right)^{-1} 
\end{equation} 
Since each of the sums in \eqref{eqn_FQhaRz_HypU_FiniteSum_Idents_stmt} 
involve only finitely--many terms, 
we ignore treatment of the convergence conditions given on the 
right--hand--sides of the equations in 
\eqref{eqn_AdditionMultThm_for_HypU_proof_stmts-v1} 
to justify these two restatements of the addition theorem analogues 
provided above. 
%% 
\end{proof} 
%%%% 
\begin{proof}[Proof of the Multiplication Theorems] 
%% 
The second pair of identities stated in 
\eqref{eqn_FQhaRz_HypU_FiniteSum_Idents-v_MultThms_stmts} 
are formed by the 
multiplication theorems for $\HypU{a}{b}{z}$ noted as in 
\citep[\S 13.13(iii)]{NISTHB}. 
The proof is derived similarly from the first parameter 
definitions of $x$ and $y$ given in the addition theorem proof, 
with an additional adjustment employed in these cases 
corresponding to the change of variable 
$\widehat{y} \mapsto (y-1) x$, which is selected so that 
$x + \widehat{y} \defmapsto xy$ in the above proof. 
%% 
The analogue to 
\eqref{eqn_AdditionMultThm_analogues_proof_stmt-v1} 
that results in these two cases then 
yields the parameters, 
$x \defequals (\alpha z)^{-1}$ and $y \defequals (w^{-1}-1) \cdot (\alpha z)^{-1}$, 
in the first identities for the 
confluent hypergeometric function, $\HypU{a}{b}{x+y}$, 
given by \eqref{eqn_AdditionMultThm_for_HypU_proof_stmts-v1}. 
%% 
\end{proof} 

\begin{remark} 
%% 
The expansions of the addition and multiplication theorem analogues to the 
established relations for the confluent hypergeometric function, 
$\HypU{a}{b}{w}$, are also compared to the known expansions of the 
\emph{duplication formula} for the associated Laguerre polynomial 
sequence stated in the following form 
\citep[\S 5.1]{UC} \citep[\cf \S 18.18(iii)]{NISTHB}: 
\begin{align*} 
h! \times L_h^{(\beta)}(wx) & = 
     \sum_{k=0}^{h} \binom{h+\beta}{h-k} \left(\frac{h!}{k!}\right)^{2} 
     \times w^{k} (1-w)^{h-k} \times k! L_k^{(\beta)}(x). 
\end{align*} 
The second expansion of the convergent denominator functions, 
$\ConvFQ{h}{\alpha}{R}{z}$, by the 
confluent hypergeometric function, $\HypM{a}{b}{w}$, stated in 
\eqref{eqn_PFact_Qhz_Mident} of the first proposition in this section 
also suggests additional identities for these sequences generated by the 
multiplication formula analogues in 
\eqref{eqn_FQhaRz_HypU_FiniteSum_Idents-v_MultThms_stmts} 
when the parameter $z \defmapsto \pm 1 / \alpha$, for example, as in the 
simplified cases of the identities expanded in the references 
\citep[\S 5.5 -- \S 5.6; Ex.\ 5.29]{GKP} 
\citep[\cf \S 15]{NISTHB}. 
%% 
\RemarkQED 
\end{remark} 

\subsection{The convergent numerator function sequences} 
\label{subsubSection_Properties_Of_ConvFn_Phz} 

The most direct expansion of the convergent numerator functions, 
$\FP_h(\alpha, R; z)$, is obtained from the 
\emph{erasing operator}, defined as in Flajolet's first article, 
which performs the formal power series 
truncation operation defined by the next equation 
\citep[\S 3]{FLAJOLET80B}. 
\begin{align*} 
\tagtext{Erasing Operator} 
\E_m\left\llbracket\sum_{i} g_i z^i\right\rrbracket 
     & \defequals 
     \sum_{i \leq m} g_i z^{i} %\cdot \Iverson{i \leq m} 
\end{align*} 
The numerator polynomials are then given through this notation by the 
expansions in the following equations: 
\begin{align*} 
\FP_h(\alpha, R; z) & = 
     \E_{h-1} \Bigl\llbracket 
     \FQ_h(\alpha, R; z) \cdot \ConvGF{h}{\alpha}{R}{z} 
     \Bigr\rrbracket \\ 
     & = 
     \sum_{n=0}^{h-1} 
     \undersetbrace{ 
     C_{h,n}(\alpha, R) \defequals [z^{n}] \ConvFP{h}{\alpha}{R}{z}}{ 
     \left( 
     \sum_{i=0}^{n} 
     [z^{i}] \ConvFQ{h}{\alpha}{R}{z} \times 
     \pn{n-i}{\alpha}{R} 
     \right) 
     } \times z^{n}. 
\end{align*} 
The coefficients of $z^{n}$ 
expanded in the last equation are rewritten slightly in terms of 
\eqref{eqn_CoeffsOfzk_FQhaRz_restmts_by_GenProductSeqs-v1} and the 
Pochhammer symbol representations of the product sequences, 
$\pn{n}{\alpha}{R}$, to arrive at a pair of formulas 
expanded as follows: 
\begin{subequations} 
\label{eqn_Vandermonde-like_PHSymb_exps_of_PhzCfs} 
\begin{align} 
\label{eqn_Chn_formula_stmt_v1} 
C_{h,n}(\alpha, R) 
     & = 
     \sum_{i=0}^{n} \binom{h}{i} (-1)^{i} 
     \pn{i}{-\alpha}{R + (h-1) \alpha} 
     \pn{n-i}{\alpha}{R}, && 
     h > n \geq 0 \\ 
\label{eqn_Vandermonde-like_PHSymb_exps_of_PhzCfs-stmt_v1} 
     & = 
     \sum_{i=0}^{n} \binom{h}{i} 
     \Pochhammer{1-h-R / \alpha}{i} 
     \Pochhammer{R / \alpha}{n-i} \times \alpha^{n}, && 
     h > n \geq 0. 
\end{align} 
\end{subequations} 
These sums are remarkably similar in form to the next 
binomial--type convolution formula, or \emph{Vandermonde identity}, 
stated as follows \citep[\S 1.13(I)]{ADVCOMB} 
\citep{WOLFRAMFNSSITE-INTRO-FACTBINOMS} 
\citep[\S 1.2.6, Ex.\ 34]{TAOCPV1}: 
\begin{align*} 
\tagtext{Vandermonde Convolution} 
\Pochhammer{x+y}{n} & = 
     \sum_{i=0}^{n} \binom{n}{i} \Pochhammer{x}{i} \Pochhammer{y}{n-i} \\ 
     %& = 
     %\sum_{i=0}^{n} \binom{n}{i} \tagTODO{(-1)^{i}} 
     %\Pochhammer{x+i}{n-i} \Pochhammer{-y}{i} \\ 
     & = 
     \sum_{i=0}^{n} \binom{n}{i} x \cdot \Pochhammer{x-iz+1}{i-1} 
     \Pochhammer{y+iz}{n-i},\ x \neq 0. 
\end{align*} 
A separate treatment of other properties implicit to the 
more complicated expansions of these convergent function 
subsequences is briefly 
explored through the definitions of the three additional forms of 
auxiliary coefficient sequences, denoted in respective order by 
$C_{h,n}(\alpha, R)$, $R_{h,k}(\alpha; z)$, and $T_h^{(\alpha)}(n, k)$, 
considered in the subsection below. 

\subsubsection{Reflected convergent numerator function sequences} 

%% 
The special cases of the reflected numerator polynomials given in 
\tableref{table_RelfectedConvNumPolySeqs_sp_cases} also 
suggest a consideration of the numerator convergent functions 
factored with respect to powers of $\pm (z-R)$ 
by expanding these sequences with respect to 
another formal auxiliary variable, $w$, when $R \defmapsto z \mp w$. 
The tables contained in the attached summary notebook 
provide working \Mm{} 
code to expand and factor these modified forms of the 
reflected numerator polynomial sequences employed in stating the 
generalized congruence results for the $\alpha$--factorial functions, 
$\MultiFactorial{n}{\alpha}$, from the examples cited in 
Section \ref{subsubSection_Examples_NewCongruences}, and 
more generally by the results proved in 
Section \ref{subSection_NewCongruence_Relations_Modulo_Integer_Bases}, 
satisfied by the $\alpha$--factorial functions and 
generalized product sequence expansions modulo integers $p \geq 2$. 
%% 

\subsubsection{Alternate forms of the 
               convergent numerator function subsequences}
\label{subsubSection_Properties_Of_ConvFn_Phz-AuxNumFn_Subsequences} 

%% 
The next results summarize three semi--triangular 
recurrence relations satisfied by the 
particular variations of the numerator function subsequences considered, 
respectively, as polynomials with respect to $z$ and $R$. 
For $h \geq 2$, fixed $\alpha \in \mathbb{Z}^{+}$, and 
$n, k \geq 0$, we consider the following forms of these auxiliary 
numerator coefficient subsequences: 
\begin{align*} 
\tagonce\label{eqn_Chk_aux_numerator_subseqs-def_v1}
C_{h,n}(\alpha, R) & \defequals [z^n] \ConvFP{h}{\alpha}{R}{z},\ 
       \text{ for \ } 
       0 \leq n \leq h-1 \\ 
       & \phantom{:} = 
C_{h-1,n}(\alpha, R) - (R+2\alpha(h-1)) C_{h-1,n-1}(\alpha, R) \\ 
     & \phantom{= \quad} - 
     \alpha(R+\alpha(h-2))(h-1) C_{h-2,n-2}(\alpha, R) \\ 
R_{h,k}(\alpha; z) & \defequals [R^k] \ConvFP{h}{\alpha}{R}{z},\ 
       \text{ for \ } 
       0 \leq k \leq h-1 \\ 
       & \phantom{:} = 
     (1-2\alpha(h-1)z)R_{h-1,k}(\alpha; z) - 
     \alpha^2(h-1)(h-2)z^2 R_{h-2,k}(\alpha; z) - 
     zR_{h-1,k-1}(\alpha; z) \\ 
       & \phantom{\defequals\ }  
     - \alpha(h-1)z^2 R_{h-2,k-1}(\alpha; z) \\ 
T_h^{(\alpha)}(n, k) & \defequals [z^n R^k] \ConvFP{h}{\alpha}{R}{z},\ 
       \text{ for \ } 
       0 \leq n, k \leq h-1 \\ 
       & \phantom{:}= 
T_{h-1}^{(\alpha)}(n, k) -T_{h-1}^{(\alpha)}(n-1, k-1) - 
     2\alpha (h-1) T_{h-1}^{(\alpha)}(n-1 ,k) \\ 
   & \phantom{\defequals\ } - 
     \alpha (h-1) T_{h-2}^{(\alpha)}(n-2, k-1) - 
     \alpha^2 (h-1)(h-2) T_{h-2}^{(\alpha)}(n-2, k) \\ 
   & \phantom{\defequals\ } + 
     \left([z^n R^0] \FP_h(z)\right) 
     \Iverson{h \geq 1} \Iverson{n \geq 0} \Iverson{k = 0}. 
\end{align*} 
Each of the recurrence relations for the triangles 
cited in the previous equations 
are derived from \eqref{eqn_PFact_Phz} by a 
straightforward application of the coefficient extraction method 
first motivated in \citep{MULTIFACTJIS}. 
%% 
\tableref{table_ConvNumFnSeqs_Chn_AlphaR_SpCaseListings} and 
\tableref{table_ConvNumFnSeqs_Rhk_Alphaz_SpCaseListings} 
list the first few special cases of the first two 
auxiliary forms of these component polynomial subsequences. 

%%%% 
%% See : "FPhz-ChnRhk-idents-summary-2013.03.09-v1.*": 
%% See : "FPhz-ChnRhk-idents-summary-tables-2015.07.28-v2.*": 
We also state, without proof, 
a number of multiple, alternating sums involving the Stirling number 
triangles that generate these auxiliary subsequences 
for reference in the next several equations. 
In particular, for $h \geq 1$ and $0 \leq n < h$, the sequences, 
$C_{h,n}(\alpha, R)$, are expanded by the following sums: 
\begin{subequations} 
\label{eqn_Chn_formula_stmts} 
\begin{align} 
\label{eqn_Chn_formula_stmts-exp_v1}
C_{h,n}(\alpha, R) & = 
     \sum\limits_{\substack{0 \leq m \leq k \leq n \\ 
                            0 \leq s \leq n} 
                 } 
     \left( 
     \binom{h}{k} \binom{m}{s} \gkpSI{k}{m} (-1)^{m} \alpha^{n} 
     \Pochhammer{\frac{R}{\alpha}}{n-k} 
     \left(\frac{R}{\alpha} - 1\right)^{m-s} 
     \right) \times h^{s} \\ 
\label{eqn_Chn_formula_stmts-exp_v2}
  & = 
     \sum\limits_{\substack{0 \leq m \leq k \leq n \\ 
                            0 \leq t \leq s \leq n} 
                 } 
     \left( 
     \binom{h}{k} \binom{m}{t} \gkpSI{k}{m} \gkpSI{n-k}{s-t} 
     (-1)^{m} \alpha^{n-s} (h-1)^{m-t} 
     \right) \times R^{s} \\ 
\label{eqn_Chn_formula_stmts-exp_v3}
   & = 
     \sum\limits_{\substack{0 \leq m \leq k \leq n \\ 
                            0 \leq i \leq s \leq n} 
                 } 
     \binom{h}{k} \binom{h}{i} \binom{m}{s} \gkpSI{k}{m} \gkpSII{s}{i} 
     (-1)^{m} \alpha^{n} 
     \Pochhammer{\frac{R}{\alpha}}{n-k} 
     \left(\frac{R}{\alpha} - 1\right)^{m-s} \times i! \\ 
\label{eqn_Chn_formula_stmts-exp_v4}
   & = 
     \sum\limits_{\substack{0 \leq m \leq k \leq n \\ 
                            0 \leq v \leq i \leq s \leq n} 
                 } 
     \binom{h}{k} \binom{m}{s} \binom{i}{v} \binom{h+v}{v} 
     \gkpSI{k}{m} \gkpSII{s}{i} (-1)^{m+i-v} \alpha^{n} \times \\ 
\notag 
     & \phantom{= \sum \binom{h}{k} \quad } \times 
     \Pochhammer{\frac{R}{\alpha}}{n-k} 
     \left(\frac{R}{\alpha} - 1\right)^{m-s} \times i!. 
\end{align} 
Since the powers of $R$ in the second identity 
are expanded by the Stirling numbers of the second kind as 
\citep[\S 6.1]{GKP} 
\begin{align*} 
R^{p} & = \alpha^{p} \times 
     \sum_{i=0}^{p} \gkpSII{p}{i} (-1)^{p-i} \Pochhammer{\frac{R}{\alpha}}{i}, 
\end{align*} 
for all natural numbers $p \geq 0$, the multiple sum identity in 
\eqref{eqn_Chn_formula_stmts-exp_v2} also implies the next finite multiple sum 
expansion for these auxiliary coefficient subsequences. 
\begin{align} 
\label{eqn_Chn_formula_stmts-exp_v5}
C_{h,n}(\alpha, R) & = 
     \sum_{i=0}^{n} 
     \underset{\mathlarger{\text{polynomial function of $h$ only } \defequals 
               \frac{(-1)^{n} m_{n,h}}{n!} \times \binom{n}{i} p_{n,i}(h)}}{ 
               \underbrace{ 
     \left( 
     \sum\limits_{\substack{0 \leq m \leq k \leq n \\ 
                            0 \leq t \leq s \leq n} 
                 } 
     \binom{h}{k} \binom{m}{t} \gkpSI{k}{m} \gkpSI{n-k}{s-t} \gkpSII{s}{i} 
     (-1)^{m+s-i} (h-1)^{m-t} 
     \right) 
     } 
     } 
     \times \alpha^{n} \Pochhammer{\frac{R}{\alpha}}{i} 
\end{align} 
\end{subequations} 
%% 
Similarly, for all $h \geq 1$ and $0 \leq k < h$, the sequences, 
$R_{h,k}(\alpha, R)$, are expanded as follows: 
\begin{align} 
\label{eqn_Rhk_formula_stmts} 
R_{h,k}(\alpha; z) & = 
     \sum\limits_{\substack{0 \leq m \leq i \leq n < h \\ 
                            0 \leq t \leq k} 
                 } 
     \left( 
     \binom{h}{i} \binom{m}{t} \gkpSI{i}{m} \gkpSI{n-i}{k-t} 
     (-1)^{m} \alpha^{n-k} (h-1)^{m-t} 
     \right) \times z^{n} \\ 
\notag 
   & = 
     \sum\limits_{\substack{0 \leq m \leq i \leq n < h \\ 
                            0 \leq t \leq k \\ 0 \leq p \leq m-t} 
                 } 
     \left( 
     \binom{h}{i} \binom{m}{t} \binom{h-1}{p} 
     \gkpSI{i}{m} \gkpSI{n-i}{k-t} \gkpSII{m-t}{p} 
     (-1)^{m} \alpha^{n-k} \times p! 
     \right) \times z^{n}. 
\end{align} 
%% 
A more careful immediate 
treatment of the properties satisfied by these subsequences is omitted 
from this section for brevity. 
A number of the new congruence results cited in the next sections do, 
at any rate, 
have alternate expansions given by the more involved termwise structure 
implicit to these finite multiple sums modulo some 
application--specific prescribed functions of $h$. 

\section{Applications and motivating examples} 
\label{Section_Apps_and_Examples} 

\subsection{Lemmas} 
\label{subSection_Apps_and_Examples_StmtsOfLemmas} 

%% 
We require the next lemma 
to formally enumerate the generalized products and factorial function sequences 
already stated without proof in the examples from 
Section \ref{subSection_Intro_Examples}. 

\begin{lemma}[Sequences Generated by the Generalized Convergent Functions] 
\label{lemma_GenConvFn_EnumIdents_pnAlphaRSeq_idents_combined_v1} 
%% 
For fixed integers $\alpha \neq 0$, $0 \leq d < \alpha$, and each 
$n \geq 1$, the generalized $\alpha$--factorial sequences 
defined in \eqref{eqn_nAlpha_Multifact_variant_rdef} 
satisfy the following expansions by the generalized products in 
\eqref{eqn_GenFact_product_form}: 
\StartGroupingSubEquations 
\begin{align} 
(\alpha n - d)!_{(\alpha)} 
     & = p_n(-\alpha, \alpha n-d) \\ 
     & = p_n(\alpha, \alpha -d) \\ 
\label{eqn_MultFactFn_ConvSeq_def-stmts_v1.c} 
n!_{(\alpha)} 
     & = p_{\lfloor (n+\alpha-1) / \alpha \rfloor}(-\alpha, n). 
\end{align} 
\EndGroupingSubEquations 
%% 
\end{lemma} 
%%%% 
\begin{proof} 
%% 
The related cases of each of these identities cited in the 
equations above correspond proving to the 
equivalent expansions of the product--wise representations for the 
$\alpha$--factorial functions given in each of the next equations: 
\begin{align*} 
\label{eqn_MultFactFn_ConvSeq_def-proof_stmts_v1.i} 
\tag{a} 
\AlphaFactorial{\alpha n-d}{\alpha} & = 
     \prod_{j=0}^{n-1} \left(\alpha n - d - j\alpha\right) \\ 
\label{eqn_MultFactFn_ConvSeq_def-proof_stmts_v1.ii} 
\tag{b} 
   & = 
     \prod_{j=0}^{n-1} \left(\alpha - d + j \alpha\right) \\ 
\tag{c} 
\label{eqn_MultFactFn_ConvSeq_def-proof_stmts_v2.iii} 
\MultiFactorial{n}{\alpha} & = 
     \prod\limits_{j=0}^{\left\lfloor (n+\alpha-1) / \alpha 
     \right\rfloor - 1} (n-i\alpha). 
\end{align*} 
The first product in \eqref{eqn_MultFactFn_ConvSeq_def-proof_stmts_v1.i} 
is easily obtained from \eqref{eqn_nAlpha_Multifact_variant_rdef} by 
induction on $n$, which then implies the second result in 
\eqref{eqn_MultFactFn_ConvSeq_def-proof_stmts_v1.ii}. 
Similarly, an inductive argument applied to the definition provided by 
\eqref{eqn_nAlpha_Multifact_variant_rdef} 
proves the last product representation given in 
\eqref{eqn_MultFactFn_ConvSeq_def-proof_stmts_v2.iii}. 
%% 
\end{proof} 

%%%% 
\begin{proof}[Corollaries] 
%% 
The proof of the lemma provides immediate corollaries 
to the special cases of the $\alpha$--factorial functions, 
$(\alpha n-d)!_{(\alpha)}$, expanded by the results from 
\eqref{eqn_MultFactFn_ConvSeq_def_v1}. 
We explicitly state the 
following particular special cases of the lemma corresponding to 
$d \defequals 0$ in \eqref{eqn_AlphaFactFn_anm1_SpCase_SeqIdents-stmts_v0} 
below, and then to $d \defequals 1$ in 
\eqref{eqn_AlphaFactFn_anm1_SpCase_SeqIdents-stmts_v1} below, respectively, 
for later use in 
Section \ref{subsubSection_Apps_ArithmeticProgs_of_the_SgFactFns} and 
Section \ref{subSection_FiniteDiffEqns_for_the_GenFactFns} 
of the article: 
\StartGroupingSubEquations 
\begin{align} 
\label{eqn_AlphaFactFn_anm1_SpCase_SeqIdents-stmts_v0} 
(\alpha n)!_{(\alpha)} 
     & = \alpha^{n} \cdot \Pochhammer{1}{n} 
       = [z^n] \ConvGF{n+n_0}{-\alpha}{\alpha n}{z},\ 
       \forall n_0 \geq 0 \\ 
\notag 
    & = \alpha^{n} \cdot n! 
       \phantom{(_{n}} = 
       [z^n] \ConvGF{n+n_0}{-1}{n}{\alpha z},\ 
       \forall n_0 \geq 0 \\ 
%%%% 
\label{eqn_AlphaFactFn_anm1_SpCase_SeqIdents-stmts_v1} 
(\alpha n-1)!_{(\alpha)} & = 
     p_n(-\alpha, \alpha n-1) = 
     (-\alpha)^{n} \Pochhammer{\frac{1}{\alpha}-n}{n} \\ 
\notag 
     & = 
     p_n(\alpha, \alpha - 1) 
     \phantom{-n} = 
     \alpha^{n} \Pochhammer{1-\frac{1}{\alpha}}{n}. 
\end{align} 
\EndGroupingSubEquations 
The first two results are employed by the convergent--based formulations to the 
applications given in 
Section \ref{subsubSection_Apps_ArithmeticProgs_of_the_SgFactFns}. 
The last pair of results given in 
\eqref{eqn_AlphaFactFn_anm1_SpCase_SeqIdents-stmts_v1} 
are employed to phrase the generalized expansions for the identities stated in 
Example \ref{example_GenDblFactFnSumIdents_FiniteSumsInvolving_AlphaFactFns} 
of 
Section \ref{subSection_FiniteDiffEqns_for_the_GenFactFns-ExactFormulas_Stmts}. 
%% 
\end{proof} 

\begin{remark} 
%% 
Lemma \ref{lemma_GenConvFn_EnumIdents_pnAlphaRSeq_idents_combined_v1} 
provides proofs of the convergent--function--based 
generating function identities enumerating the $\alpha$--factorial 
sequences given in 
\eqref{eqn_MultFactFn_ConvSeq_def_v1} and 
\eqref{eqn_MultFactFn_ConvSeq_def_v2} of the introduction. 
The last convergent--based generating function identity that enumerates the 
$\alpha$--factorial functions, $\MultiFactorial{n}{\alpha}$, when 
$n > \alpha$ expanded in the form of 
\eqref{eqn_MultFactFn_ConvSeq_def_v3} from 
Section \ref{subSection_Intro_Examples} 
follows from the product function expansions provided in 
\eqref{eqn_MultFactFn_ConvSeq_def-stmts_v1.c} of the lemma by 
applying a result proved in the exercises section of the reference 
\citep[\S 7, Ex.\ 7.36; p.\ 569]{GKP}. 

In particular, for any fixed $m \geq 1$ and some sequence, 
$\left(a_n\right)_{n \geq 0}$, a generating function for the 
modified sequences, $\left(a_{\lfloor n/m \rfloor}\right)_{n \geq 0}$, is 
given by 
\begin{align*} 
\widehat{A}_m(z) & \defequals 
     \sum_{n \geq 0} a_{\lfloor \frac{n}{m} \rfloor} z^{n} = 
     \frac{1-z^{m}}{1-z} \times \widetilde{A}\left(z^{m}\right) = 
     \left(1 + z + \cdots + z^{m-2} + z^{m-1}\right) \times 
     \widetilde{A}\left(z^{m}\right), 
\end{align*} 
where $\widetilde{A}(z) \defequals \sum_{n} a_n z^{n}$ denotes the 
ordinary power series generating function formally enumerating the 
prescribed sequence over $n \geq 0$. 
%% 
\RemarkQED 
\end{remark} 

\begin{lemma}[Identities and Other Formulas Connecting Pochhammer Symbols] 
\label{lemma_footnote_PHSymbol_BinomIdents} 
%% 
The next equations provide statements of several known identities 
from the references involving the 
falling factorial function, the Pochhammer symbol, 
or rising factorial function in the next identities, and the 
binomial coefficients required by the applications given in the 
next sections of the article. 
\begin{enumerate} 
     \setlength{\itemsep}{-2mm} 

\item \itemlabel{Relations between rising and falling factorial functions} 
The following identities provide known relations between the 
rising and falling factorial functions for fixed $x \neq \pm 1$ and 
integers $m,n \geq 0$ 
\citep[\S 4.1.2, \S 5; \cf \S 4.3.1]{UC} 
\citep[\S 2, Ex.\ 2.17, 2.9, 2.16; \S 5.3; \S 6, Ex.\ 6.31, p.\ 552]{GKP}: 
\begin{align*} 
\tag{\em Connection Formulas} 
\FFactII{x}{n} 
     & = 
     \sum_{k=1}^{n} 
     \binom{n-1}{k-1} \frac{n!}{k!} \times \Pochhammer{x}{k} \\ 
     & = 
     (-1)^{n} \Pochhammer{-x}{n} = 
     \Pochhammer{x-n+1}{n} = 
     \frac{1}{\RFactII{(x+1)}{-n}} \\ 
\Pochhammer{x}{n} 
     & = 
     \sum_{k=0}^{n} 
     \binom{n}{k} \FFactII{(n-1)}{n-k} \times \FFactII{x}{k} \\ 
     & = 
     (-1)^{n} \FFactII{(-x)}{n} = 
     \FFactII{(x+n-1)}{n} = 
     \frac{1}{\FFactII{(x-1)}{-n}} \\ 
\tag{\em Generalized Exponent Laws} 
\FFactII{x}{m+n} 
     & = 
     \FFactII{x}{m} \FFactII{(x-m)}{n} \\ 
\RFactII{x}{m+n} 
     & = 
     \RFactII{x}{m} \RFactII{(x+m)}{n} \\  
\tag{\em Negative Rising and Falling Powers} 
\RFactII{x}{-n} 
     & = 
     \frac{1}{\Pochhammer{x-n}{n}} = 
     \frac{1}{\FFactII{(x-1)}{n}} \\ 
\FFactII{x}{-n} 
     & = 
     \frac{1}{\Pochhammer{x+1}{n}} = 
     \frac{1}{n! \cdot \binom{x+n}{n}} = 
     \frac{1}{(x+1)(x+2) \cdots (x+n)}. 
\end{align*} 

\item \itemlabel{Expansions of polynomial powers by the 
                 Stirling numbers of the second kind} 
For any fixed $x \neq 0$ and integers $n \geq 0$, the 
polynomial powers of $x^n$ are expanded as follows 
\citep[\S 6.1]{GKP}: 
\begin{align*} 
\tag{\em Expansions of Polynomial Powers} 
x^{n} 
     & = 
     \sum_{k=0}^{n} \gkpSII{n}{n-k} \FFactII{x}{n-k} = %\\ 
     %& = 
     \sum_{k=0}^{n} \gkpSII{n}{k} (-1)^{n-k} \RFactII{x}{n}. 
\end{align*} 

\item \itemlabel{Binomial coefficient identities expanded by the 
                 Pochhammer symbol} 
For fixed $x \neq 0$ and integers $n \geq 1$, the 
binomial coefficients are expanded by 
\citep{CVLPOLYS,WOLFRAMFNSSITE-INTRO-FACTBINOMS} 
\begin{align*} 
\tag{\em Binomial Coefficient Identities} 
\Pochhammer{x}{n} 
     & = 
     \binom{-x}{n} \times (-1)^{n} n! \\ 
     & = 
     \binom{x+n-1}{n} \times n!. 
\end{align*} 

\item \itemlabel{Connection formulas for products and 
                 ratios of Pochhammer symbols} 
The next identities connecting products and ratios of the 
Pochhammer symbols are stated 
for integers $m, n, i \geq 0$ and fixed $x \neq 0$
\citep[Ex.\ 1.23, p.\ 83]{ADVCOMB} 
\citep{WOLFRAMFNSSITE-INTRO-FACTBINOMS}. 
\begin{align*} 
\tag{\em Connection Formulas for Products} 
\Pochhammer{x}{n} \Pochhammer{x}{m} 
     & = 
     \sum_{k=0}^{\min(m, n)} \binom{m}{k} \binom{n}{k} k! \cdot 
     \Pochhammer{x}{m+n-k}, n \neq m \\ 
\tag{\em Fractions of Pochhammer Symbols} 
\frac{\Pochhammer{x}{n}}{\Pochhammer{x}{i}} 
     & = 
     \Pochhammer{x+i}{n-i},\ n \geq i 
\end{align*} 

\end{enumerate} 
%% 
\end{lemma} 
%%%% 
\begin{proof} 
%% 
The formulas given in the lemma provide inline citations to the 
references for proofs of these 
results, which are restated as lemmas within this article by 
the previous equations above. 
%% 
\end{proof} 

\subsection{New congruences for the 
            $\alpha$--factorial functions, the 
            generalized Stirling number triangles, and 
            Pochhammer $k$--symbols} 
\label{subSection_NewCongruence_Relations_Modulo_Integer_Bases} 
\label{subSection_Congruences_for_Series_ModuloIntegers_p} 

The results stated in this section follow immediately 
from the congruences properties modulo integer divisors of the $M_h$ 
summarized by Section \ref{subSection_EnumProps_of_JFractions} 
considered as in the references \citep{FLAJOLET80B,FLAJOLET82} 
\citep[\cf \S 5.7]{GFLECT}. 
The particular cases of the J--fraction representations enumerating the 
product sequences defined by \eqref{eqn_GenFact_product_form} 
always yield a factor of $h \defequals N_h \mid M_h$ in the statement of 
\eqref{eqn_EnumProps_Of_JTypeCFracs_congruence_rels-stmt_v1} 
(see Remark \ref{remark_Congruences_for_Rational-Valued_Params}). 
One consequence of this property implicit to each of the generalized 
factorial--like sequences observed so far, is that it is straightforward to 
formulate new congruence relations for these sequences modulo any 
fixed integers $p \geq 2$. 

\begin{remark}[Congruences for Rational--Valued Parameters] 
\label{remark_Congruences_for_Rational-Valued_Params} 
%% 
The J--fraction parameters, 
$\lambda_h = \lambda_h(\alpha, R)$ and $M_h = M_h(\alpha, R)$, 
defined as in the summary of the enumerative properties from 
Section \ref{subSection_EnumProps_of_JFractions}, 
corresponding to the 
expansions of the generalized convergents defined by the proof in 
Section \ref{subSection_GenCFrac_Reps_for_GenFactFns} 
satisfy 
\begin{align*} 
\lambda_k(\alpha, R) & \defequals 
     \as_{k-1}(\alpha, R) \cdot \bs_{k}(\alpha, R) \\ 
   & \phantom{:} = 
     \alpha (R + (k-1) \alpha) \cdot k \\ 
M_h(\alpha, R) & \defequals 
     \lambda_1(\alpha, R) \cdot \lambda_2(\alpha, R) 
     \times \cdots \times 
     \lambda_h(\alpha, R) \\ 
   & \phantom{:} = 
     \alpha^{h} \cdot h! \times p_h(\alpha, R) \\ 
   & \phantom{:} = 
   \alpha^{h} \cdot h! \times \Pochhammer{R}{h,\alpha}, 
\end{align*} 
so that for integer divisors, $N_h(\alpha, R) \mid M_h(\alpha, R)$, 
we have that 
\begin{align*} 
p_n(\alpha, R) & \equiv 
     [z^n] \ConvGF{h}{\alpha}{R}{z} \pmod{N_h(\alpha, R)}. 
\end{align*} 
So far we have restricted ourselves to examples of the 
particular product sequence cases, $p_n(\alpha, R)$, 
where $\alpha \neq 0$ is integer--valued, \ie 
so that $p, p\alpha^{i} \mid M_p(\alpha, R)$ for $1 \leq i \leq p$ 
whenever $p \geq 2$ is a fixed natural number. 
%% 
Identities arising in some 
related applications that involve finding results analogs to the 
explicit new congruence properties stated so far for other 
factorial--related sequence cases, specifically when the choice of 
$\alpha \neq 0$ is strictly rational--valued, 
are intentionally not treated in the examples cited as applications 
above and in this section. 
%% 
\RemarkQED 
\end{remark} 

\subsubsection{Congruences for the Stirling numbers of the first kind and the 
               $r$--order harmonic number sequences} 
\label{subsubSection_remark_New_Congruences_for_GenS1Triangles_and_HNumSeqs} 

\sublabel{Generating the Stirling numbers 
          as series coefficients of the 
          generalized convergent functions} 
For integers $h \geq 3$ and $n \geq m \geq 1$, the (unsigned) 
Stirling numbers of the first kind, $\gkpSI{n}{m}$, 
are generated by the polynomial expansions of the 
rising factorial function, or Pochhammer symbol, 
$\RFactII{x}{n} = \Pochhammer{x}{n}$, as follows 
\citep[\S 7.4; \S 6]{GKP} \citeOEIS{A130534,A008275}: 
\StartGroupingSubEquations 
\label{eqn_S1FcfAlpha_GenConvFn_Coeffs_and_CongruencesResultStmts} 
\begin{align*} 
\tagonce 
\gkpSI{n}{m} & = 
     [R^m] R (R+1) \cdots (R+n-1) \\ 
     & = 
     [z^n] [R^m] \ConvGF{h}{1}{R}{z},\ 
     \text{ for } 
     1 \leq m \leq n \leq 2h. 
\end{align*} 
analogs formulations of new congruence results for the 
$\alpha$--factorial triangles defined by \eqref{eqn_Fa_rdef}, and the 
corresponding forms of the generalized harmonic number sequences 
employed in stating the results in 
Section \ref{subsubSection_MoreGeneralExps_congruences_multiple_factfns} 
of the article below, are expanded by noting that 
for all $n,m \geq 1$, and integers $p \geq 2$, we have the 
following expansions 
(see Section \ref{subsubSection_FutureResTopics_GenSNumCongResults}) 
\citep{MULTIFACTJIS}: 
\begin{align*} 
\tagonce 
\FcfII{\alpha}{n}{m} 
     & = 
     [s^{m-1}] (s+1) (s+1+\alpha) \cdots (s+1 + (n-2) \alpha) \\ 
     & = 
     [s^{m-1}] \pn{n-1}{\alpha}{s+1} \\ 
\FcfII{\alpha}{n}{m} & \equiv 
     [z^{n-1}] [R^{m-1}] \ConvGF{p}{\alpha}{R+1}{z} \pmod{p}. 
\end{align*} 
\EndGroupingSubEquations 
The coefficients of $R^{m}$ in the 
series expansions of the convergent functions, 
$\ConvGF{h}{1}{R}{z}$, in the formal variable $R$ are 
rational functions of $z$ with denominators given by $m^{th}$ powers of the 
reflected polynomials defined in the next equation. 

\sublabel{Definitions of the quasi--polynomial expansions for the 
          Stirling numbers of the first kind} 
For a fixed $h \geq 3$, let the zeros, $\omega_{h,i}$, be 
defined as follows: 
\begin{align*} 
\tagonce\label{eqn_S1CoeffsModuloh_DenomReflectedRoots_defs} 
\left(\omega_{h,i}\right)_{i=1}^{h-1} & \defequals 
     \left\{ \omega_j : 
     \sum_{i=0}^{h-1} \binom{h-1}{i} \frac{h!}{(i+1)!} (-\omega_j)^{i} = 0,\ 
     1 \leq j < h \right\}. 
\end{align*} 
The forms of both exact formulas and congruences for the 
Stirling numbers of the first kind modulo any prescribed integers $h \geq 3$ 
are then expanded as 
\begin{align*} 
\gkpSI{n}{m} & = 
     \left(\sum_{i=0}^{h-1} p_{h,i}^{[m]}(n) \times \omega_{h,i}^{n} 
     \right) \Iverson{n > m} + \Iverson{n = m},\ 
     \phantom{\pmod{h}} 
     m \leq n \leq 2h-1 \\ 
\gkpSI{n}{m} & \equiv 
     \left(\sum_{i=0}^{h-1} p_{h,i}^{[m]}(n) \times \omega_{h,i}^{n} 
     \right) \Iverson{n > m} + \Iverson{n = m} \pmod{h},\ 
     \forall n \geq m, 
\end{align*} 
where the functions, $p_{h,i}^{[m]}(n)$, denote fixed 
polynomials of degree $m$ in $n$ for each $h$, $m$, and $i$ 
\citep[\S 7.2]{GKP}. 
%% 
For example, 
when $h \defequals 2, 3$, the respective reflected roots 
defined by the previous equations in 
\eqref{eqn_S1CoeffsModuloh_DenomReflectedRoots_defs} are given exactly by 
\begin{equation*} 
\left\{ \omega_{2,1} \right\} \defequals \{2\} 
     \qquad \text{ and } \qquad 
\left(\omega_{3,i}\right)_{i=1}^{2} \defequals 
     \left\{3-\sqrt{3}, 3+\sqrt{3}\right\}. 
\end{equation*} 

\sublabel{Comparisons to known congruences for the Stirling numbers} 
%% 
The special case of 
\eqref{eqn_cor_Congruences_for_AlphaFactFns_modulo2} from the 
examples given in the introduction 
when $\alpha \defequals 1$ corresponding to the single factorial function, 
$n!$, agrees with the known congruence for the 
Stirling numbers of the first kind derived in the reference 
\citep[\S 4.6]{GFOLOGY} \citep[\cf \S 5.8]{ADVCOMB}. 
In particular, for all $n \geq 1$ we can prove that 
\begin{align*} 
n! & \equiv 
     \sum_{m=1}^{n} 
     \binom{\lfloor n/2 \rfloor}{m - \lceil n/2 \rceil} 
     (-1)^{n-m} n^m + \Iverson{n = 0} 
     \pmod{2}. 
\end{align*} 
For comparison with the known result for the Stirling numbers of the 
first kind modulo $2$ expanded as in the result from the reference 
stated above, 
several particular cases of these congruences for the Stirling numbers, 
$\gkpSI{n}{m}$, modulo $2$ are given by 
\begin{align*} 
\gkpSI{n}{1} & \equiv 
     \frac{2^{n}}{4} \Iverson{n \geq 2} + \Iverson{n = 1} && \pmod{2} \\ 
\gkpSI{n}{2} & \equiv 
     \frac{3 \cdot 2^{n}}{16} (n-1) \Iverson{n \geq 3} + 
     \Iverson{n = 2} && \pmod{2} \\ 
\gkpSI{n}{3} & \equiv 
     2^{n-7} (9n-20) (n-1) \Iverson{n \geq 4} + 
     \Iverson{n = 3} && \pmod{2} \\ 
\gkpSI{n}{4} & \equiv 
     2^{n-9} (3n-10) (3n-7) (n-1) \Iverson{n \geq 5} + 
     \Iverson{n = 4} && \pmod{2} \\ 
\gkpSI{n}{5} & \equiv 
     2^{n-13} (27n^3-279n^2+934n-1008) (n-1) \Iverson{n \geq 6} + 
     \Iverson{n = 5} && \pmod{2} \\ 
\gkpSI{n}{6} & \equiv 
     \frac{2^{n-15}}{5} (9n^2-71n+120) (3n-14) (3n-11) (n-1) 
     \Iverson{n \geq 7} + \Iverson{n = 6} && \pmod{2}, 
\end{align*} 
where 
\begin{align*} 
\gkpSI{n}{m} & \equiv 
     \binom{\lfloor n/2 \rfloor}{m - \lceil n/2 \rceil} = 
     [x^m] \left( 
     x^{\lceil n/2 \rceil} (x+1)^{\lfloor n/2 \rfloor} 
     \right) && \pmod{2}, 
\end{align*} 
for all $n \geq m \geq 1$ \citep[\S 4.6]{GFOLOGY} 
\citeOEIS{A087755}. 
For comparison, 
the termwise expansions of the row generating functions, $\Pochhammer{x}{n}$, 
for the Stirling number triangle considered modulo $3$ 
with respect to the non--zero indeterminate $x$ similarly imply the next 
properties of these coefficients for any $n \geq m > 0$. 
\begin{align*} 
\gkpSI{n}{m} & \equiv 
     [x^m] \left( 
     x^{\lceil n/3 \rceil} (x+1)^{\lceil (n-1)/3 \rceil} 
     (x+2)^{\lfloor n/3 \rfloor} 
     \right) && \pmod{3} \\ 
     & \equiv 
     \sum_{k=0}^{m} \binom{\lceil (n-1)/3 \rceil}{k} 
     \binom{\lfloor n/3 \rfloor}{m-k - \lceil n/3 \rceil} \times 
     2^{\lceil n/3 \rceil + \lfloor n/3 \rfloor -(m-k)} && \pmod{3} 
\end{align*} 
The next few particular examples of the 
special case congruences 
satisfied by the Stirling numbers of the first kind modulo $3$ 
are obtained from the results in 
\eqref{eqn_S1CoeffsModuloh_DenomReflectedRoots_defs} 
above are expanded in the following forms: 
\begin{align*} 
\gkpSI{n}{1} & \equiv 
     \mathsmaller{ 
     \sum\limits_{j = \pm 1} 
     \frac{1}{36} \left(9-5 j\sqrt{3}\right) 
     \times \left(3+j\sqrt{3}\right)^{n} 
     } 
     \Iverson{n \geq 2} + \Iverson{n = 1} && \pmod{3} \\ 
\gkpSI{n}{2} & \equiv 
     \mathsmaller{ 
     \sum\limits_{j = \pm 1} 
     \frac{1}{216} \left((44n-41)-(25n-24) \cdot j\sqrt{3}\right) 
     \times \left(3+j\sqrt{3}\right)^{n} 
     } 
     \Iverson{n \geq 3} + \Iverson{n = 2} && \pmod{3} \\ 
\gkpSI{n}{3} & \equiv 
     \mathsmaller{ 
     \sum\limits_{j = \pm 1} 
     \frac{1}{15552} \left((1299n^2-3837n+2412)- 
     (745n^2-2217n+1418) \cdot j\sqrt{3}\right) 
     } \times && \\ 
     & \phantom{\equiv \sum\limits_{j = \pm 1} \frac{1}{15552}} \times 
     \mathsmaller{ 
     \left(3+j\sqrt{3}\right)^{n} 
     } 
     \Iverson{n \geq 4} + \Iverson{n = 3} && \pmod{3} \\ 
\gkpSI{n}{4} & \equiv 
     \mathsmaller{ 
     \sum\limits_{j = \pm 1} 
     \frac{1}{179936} \bigl((6409n^3-383778n^2+70901n-37092) 
     } && \\ 
     & \phantom{\equiv\mathsmaller{\sum\limits_{j=\pm 1} 
                \frac{1}{179936}}\bigl(} - 
     \mathsmaller{ 
     (3690n^3-22374n^2+41088n-21708) \cdot j\sqrt{3}\bigr) 
     } \times && \\ 
     & \phantom{\equiv \sum\limits_{j = \pm 1} \quad } \times 
     \mathsmaller{ 
     \left(3+j\sqrt{3}\right)^{n} 
     } 
     \Iverson{n \geq 5} + \Iverson{n = 4}. && \pmod{3} 
\end{align*} 
Additional congruences for the Stirling numbers of the first kind 
modulo $4$ and modulo $5$ are straightforward to expand by 
related formulas with exact 
algebraic expressions for the roots of the third--degree and 
fourth--degree equations defined as in 
\eqref{eqn_S1CoeffsModuloh_DenomReflectedRoots_defs}. 

\sublabel{Rational generating function expansions enumerating the 
          first--order harmonic numbers} 
The next several particular cases of the congruences 
satisfied by the first--order harmonic numbers, $H_n$, or $H_n^{(1)}$, 
are stated exactly in terms the rational generating functions in $z$ that 
lead to generalized forms of the congruences in the last equations 
modulo the integers $p \defequals 2, 3$ for these next cases of the 
integers $p \geq 4$ 
\citep[\S 6.3]{GKP} \citeOEIS{A001008,A002805}. 
\begin{align*} 
n! \times H_n^{(1)} 
     & = 
     \gkpSI{n+1}{2} && \\ 
n! \times H_n^{(1)}     
     & \equiv 
     [z^{n+1}] \left( 
     \mathsmaller{ 
     \frac{36 z^2 - 48z + 325}{576} + 
     \frac{17040 z^2+1782 z+6467}{576 \left(24 z^3-36 z^2+12 z-1\right)}+\frac{78828 z^2-33987 z+3071}{288 \left(24
        z^3-36 z^2+12 z-1\right)^2} 
     } 
     \right) && \pmod{4} \\ 
     & \equiv 
     [z^{n}]\left( 
     \mathsmaller{ 
     \frac{3z-4}{48} + 
     \frac{1300 z^2+890 z+947}{96 \left(24 z^3-36 z^2+12 z-1\right)}+\frac{24568 z^2-10576 z+955}{96 \left(24 z^3-36 z^2+12 z-1\right)^2}
     } 
     \right) && \pmod{4} \\ 
     & \equiv 
     [z^{n-1}] \left( 
     \mathsmaller{ 
     \frac{1}{16} + 
     \frac{-96 z^2+794 z+397}{48 \left(24 z^3-36 z^2+12 z-1\right)}+\frac{5730 z^2-2453 z+221}{24 \left(24 z^3-36 z^2+12 z-1\right)^2} 
     } 
     \right) && \pmod{4} \\ 
n! \times H_n^{(1)} 
     & \equiv 
     [z^{n}]\Bigl( 
     \mathsmaller{ 
     \frac{12z-29}{300} + 
     \frac{80130 z^3+54450 z^2+79113 z+108164}{ 
     900 \left(120 z^4-240 z^3+120 z^2-20 z+1\right)} 
     } && \\ 
     & \phantom{\equiv [z^{n}]\Bigl( \quad} + 
     \mathsmaller{ 
     \frac{17470170 z^3-11428050 z^2+2081551 z-108077}{900
     \left(120 z^4-240 z^3+120 z^2-20 z+1\right)^2} 
     } 
     \Bigr) && \pmod{5} \\ 
n! \times H_n^{(1)} 
     & \equiv 
     [z^{n}]\Bigl( 
     \mathsmaller{ 
     \frac{10z-37}{360} + 
     \frac{1419408 z^4+903312 z^3+1797924 z^2+2950002 z+4780681}{2160 
     \left(720 z^5-1800 z^4+1200 z^3-300 z^2+30 z-1\right)} 
     } && \\ 
     & \phantom{\equiv [z^{n}]\Bigl( \quad } + 
     \mathsmaller{ 
     \frac{5581246248 z^4-4906594848z^3+1347715644 z^2-140481648 z+ 
     4780903}{2160 \left(720 z^5-1800 z^4+1200 z^3-300 z^2+30 z-1\right)^2}
     } 
     \Bigr) && \pmod{6} 
\end{align*} 
The expansions of the integer--order harmonic number sequences 
cited in the reference \citep[\S 4.3.3]{MULTIFACTJIS} also 
yield additional related expansions of congruences for the 
terms, $(n!)^{r} \times H_n^{(r)}$, 
provided by the noted identities for these functions involving the 
Stirling numbers of the first kind modulo any fixed integers $p \geq 2$. 

\sublabel{Rational generating functions enumerating congruences for the 
          $r$--order harmonic numbers} 
%% 
The \emph{second--order} and \emph{third--order} \emph{harmonic numbers}, 
$H_n^{(2)}$ and $H_n^{(3)}$, respectively, 
are expanded exactly through the following formulas involving the 
Stirling numbers of the first kind modulo any fixed integers $p \geq 2$, and
where the Stirling number sequences, $\gkpSI{n}{m} \pmod{p}$ at each 
fixed $m \defequals 1,2,3,4$, are generated by the predictably 
rational functions of $z$ enumerated through the identities stated above 
\citep[\S 4.3.3]{MULTIFACTJIS} 
\citeOEIS{A007406,A007407,A007408,A007409}: 
\begin{align*} 
(n!)^{2} \times H_n^{(2)} 
     & = 
     (n!)^{2} \times \sum_{k=1}^{n} \frac{1}{k^2} \\ 
     & \equiv 
     \gkpSI{n+1}{2}^{2} - 2 \gkpSI{n+1}{1} \gkpSI{n+1}{3} && \pmod{p} \\ 
(n!)^{3} \times H_n^{(3)} 
     & = 
     (n!)^{3} \times \sum_{k=1}^{n} \frac{1}{k^3} \\ 
     & \equiv 
     \gkpSI{n+1}{2}^{3} - 3 \gkpSI{n+1}{1} \gkpSI{n+1}{2} \gkpSI{n+1}{3} + 
     3 \gkpSI{n+1}{1}^{2} \gkpSI{n+1}{4}. 
     && \pmod{p} 
\end{align*} 
The Hadamard product, or diagonal--coefficient, generating function 
constructions formulated in the examples introduced by 
Section \ref{subSection_DiagonalGFSequences_Apps} below 
give expansions of rational convergent--function--based 
generating functions in $z$ that 
generate these corresponding $r$--order sequence cases modulo any 
fixed integers $p \geq 2$. 

The proof of {Theorem 2.4} given in the reference \citep[\S 2]{GFLECT} 
suggests the direct method for obtaining the next 
rational generating functions for these sequences in the 
working source code documented in the reference \citep{SUMMARYNBREF-STUB}, 
each of which generate series coefficients for these particular 
harmonic number sequence variants (modulo $p$) that 
always satisfy some finite--degree 
linear difference equation with constant coefficients over $n$ 
when $\alpha$ and $R$ are constant parameters. 
\begin{align*} 
(n!)^{2} \times H_n^{(2)} 
     & \equiv 
     [z^{n}] \left( 
     \mathsmaller{ 
     \frac{z \left(1-3z+9z^2-8z^3\right)}{(1-4z)^2} 
     } 
     \right) 
     && \pmod{2} \\ 
     & = 
     [z^{n}] \left(
     \mathsmaller{ 
     \frac{5z}{16}-\frac{z^2}{2} + \frac{11}{64 (1-4z)^2} - 
     \frac{11}{64 (1-4z)} 
     } 
     \right) 
     && \\ 
     & = 
     [z^{n}] \left( 
     \mathsmaller{ 
     z + 5z^2+33z^3+176z^4+880z^5+4224z^6+19712z^7+ \cdots 
     } 
     \right) 
     && \\ 
(n!)^{2} \times H_n^{(2)}     
     & \equiv 
     [z^{n}] \left( 
     \mathsmaller{ 
     \frac{z \left(1-61 z+1339 z^2-13106 z^3+62284 z^4-144264 z^5+ 
     151776 z^6-124416 z^7+41472 z^8\right)}{(1-6 z)^3 
     \left(1-24 z+36 z^2\right)^2}
     } 
     \right) 
     && \pmod{3} \\ 
     & = 
     [z^{n}] \Bigl( 
     \mathsmaller{ 
     -\frac{13}{324}+\frac{14 z}{81}-\frac{4 z^2}{27} + 
     \frac{25}{1944 (-1+6 z)^3}+\frac{115}{1944 (-1+6 z)^2} + 
     \frac{5}{162 (-1+6 z)} 
     } && \\ 
     & \phantom{= [z^{n}] \Bigl( \quad} + 
     \mathsmaller{ 
     \frac{-787+17624 z}{216
     \left(1-24 z+36 z^2\right)^2}+\frac{2377+3754 z}{648 
     \left(1-24 z+36 z^2\right)} 
     } 
     \Bigr) 
     && \\ 
     & = 
     [z^{n}] \left( 
     \mathsmaller{ 
     z + 5z^2+49z^3+820z^4+18232z^5+437616z^6+10619568z^7 + \cdots 
     } 
     \right) 
     && \\ 
(n!)^{3} \times H_n^{(3)} 
     & \equiv 
     [z^{n}] \left( 
     \mathsmaller{ 
     \frac{z \left(1-7 z+49 z^2-144 z^3+192 z^4\right)}{(1-8z)^2} 
     } 
     \right) 
     && \pmod{2} \\ 
     & = 
     [z^{n}] \left( 
     \mathsmaller{ 
     \frac{11z}{32}-\frac{3z^2}{2}+3z^3 + 
     \frac{21}{256 (1-8z)^2} - \frac{21}{256 (1-8z)} 
     } 
     \right) 
     && \\ 
     & = 
     [z^{n}] \left( 
     \mathsmaller{ 
     z +9z^2+129z^3+1344z^4+13440z^5+129024z^6+1204224z^7 + \cdots 
     } 
     \right) 
     && \\ 
(n!)^{3} \times H_n^{(3)}     
     & \equiv 
     [z^{n}] \Bigl( 
     \mathsmaller{ 
     -\frac{143}{5832}+\frac{625 z}{2916}-\frac{4 z^2}{9}+ 
     \frac{4 z^3}{3}+\frac{115 (-6719+711956 z)}{93312 
     \left(1-108 z+216 z^2\right)^2} 
     } 
     && \pmod{3} \\ 
     & \phantom{\equiv [z^{n}] \Bigl( \quad} + 
     \mathsmaller{ 
     \frac{774079+1459082 z}{93312 \left(1-108 z+216 z^2\right)} - 
     \frac{125 (-11+312 z)}{11664 \left(1-36 z+216 z^2\right)^4} 
     } && \\ 
     & \phantom{\equiv [z^{n}] \Bigl(\quad} - 
     \mathsmaller{ 
     \frac{10 (1+306z)}{729 \left(1-36 z+216 z^2\right)^3} + 
     \frac{-20677+269268 z}{93312 \left(1-36 z+216 z^2\right)^2} + 
     \frac{11851+89478 z}{93312 \left(1-36 z+216z^2\right)} 
     } 
     \Bigr) 
     && \\ 
     & = 
     [z^{n}] \left( 
     \mathsmaller{ 
     z+9z^2+251z^3+16280z^4+1586800z^5+171547200z^6+\cdots 
     %18958757184z^7 + \cdots 
     } 
     \right) 
     && 
\end{align*} 
%% 
The next cases of the rational generating functions enumerating the 
terms of these two sequences modulo $4$ and $5$ 
lead to less compact formulas expanded in partial fractions over $z$, 
roughly approximated in form by the 
generating function expansions from the previous formulas. 
The factored denominators, 
denoted $\HNumGFFactoredDenomFn{r}{p}{z}$ immediately below, 
of the rational generating functions over the 
respective second--order and third--order cases of the $r$--order sequences 
modulo $p \defequals 4,5$ are provided in the following equations: 
\begin{align*} 
\HNumGFFactoredDenomFn{2}{4}{z} & = 
     \mathsmaller{ 
     \left(-1+72 z-720 z^2+576 z^3\right)^2 
     \left(-1+36 z-288 z^2+576 z^3\right)^3 
     } \\ 
\HNumGFFactoredDenomFn{2}{5}{z} & = 
     \mathsmaller{ 
     \left(1-160 z+5040 z^2-28800 z^3+14400 z^4\right)^2 \times 
     } \\ 
     & \phantom{= \quad } \times 
     \mathsmaller{ 
     \left(1-120 z+4680 z^2-76800 z^3+561600 z^4-1728000 z^5+ 
     1728000 z^6\right)^3 
     } \\ 
\HNumGFFactoredDenomFn{3}{4}{z} & = 
     \mathsmaller{ 
     (1-24z)^4 \left(-1+504 z-17280 z^2+13824 z^3\right)^2 \times 
     } \\ 
     & \phantom{= \quad} \times 
     \mathsmaller{ 
     \left(-1+144 z-5184 z^2+13824 z^3\right)^4 
     } 
     \left(-1+216 z-3456 z^2+13824 z^3\right)^4 \\ 
\HNumGFFactoredDenomFn{3}{5}{z} & = 
     \mathsmaller{ 
     \left(1-1520 z+273600 z^2-4320000 z^3+1728000 z^4\right)^2 \times 
     } \\ 
     & \phantom{= \quad } \times 
     \mathsmaller{ 
     \left(1-240 z+14400 z^2-288000 z^3+1728000 z^4\right)^4 \times 
     } \\ 
     & \phantom{= \quad } \times 
     \mathsmaller{ 
     \bigl(1-1680 z+1051200 z^2-319776000
        z^3+51914304000 z^4 
     } \\ 
     & \phantom{= \quad \times \bigl(1} 
     \mathsmaller{ - 
     4764026880000 z^5+251795865600000 z^6- 
        7537618944000000 z^7 
     } \\ 
     & \phantom{= \quad \times \bigl(1} 
     \mathsmaller{ + 
     121956544512000000 z^8-998751928320000000 z^9 
     } \\ 
     & \phantom{= \quad \times \bigl(1} 
     \mathsmaller{ + 
     4084826112000000000 z^{10} 
     -7739670528000000000 z^{11} 
     } \\ 
     & \phantom{= \quad \times \bigl(1} 
     \mathsmaller{ + 
     5159780352000000000 z^{12} 
     \bigr)^4 
     }. 
\end{align*} 
The summary notebook reference contains further complete expansions of the 
rational generating functions enumerating these $r$--order sequence 
cases for $r \defequals 1,2,3,4$ modulo the next few prescribed cases of the 
integers $p \geq 6$ 
\citep[\cf \S 4.3.3]{MULTIFACTJIS} \citep{SUMMARYNBREF-STUB}. 

\subsubsection{Generalized expansions of the new integer congruences for the 
               $\alpha$--factorial functions and the 
               symbolic product sequences} 

\sublabelII{Generalized forms of the special case results 
            expanded in the introduction} 
\begin{example}[The Special Cases Modulo $2$, $3$, and $4$] 
%% 
The first congruences for the $\alpha$--factorial functions, 
$n!_{(\alpha)}$, modulo the prescribed integer bases, $2$ and $2\alpha$, 
cited in \eqref{eqn_cor_Congruences_for_AlphaFactFns_modulo2} 
from the introduction result by applying 
Lemma \ref{lemma_GenConvFn_EnumIdents_pnAlphaRSeq_idents_combined_v1} 
to the series for the generalized convergent function, 
$\ConvGF{2}{\alpha}{R}{z}$, expanded by following equations: 
\begin{align*} 
p_n(\alpha, R) & \equiv 
     [z^n] \left( 
     \frac{1-z (2\alpha+R)}{R (\alpha+R) z^2 -2(\alpha+R) z + 1} 
     \right) 
     && \pmod{2, 2\alpha} \\ 
   & = 
     \sum_{b=\pm 1} 
     \frac{\left(\sqrt{\alpha  (\alpha +R)} -b \cdot \alpha\right) 
     \left(\alpha + b \cdot \sqrt{\alpha  (\alpha +R)}+R\right)^n}{ 
     2 \sqrt{\alpha (\alpha +R)}} 
     && \pmod{2, 2\alpha}. 
\end{align*} 
The next congruences for the $\alpha$--factorial function sequences 
modulo $3$ ($3\alpha$) and modulo $4$ ($4\alpha$) 
cited as particulars example in 
Section \ref{subSection_Intro_Examples} 
are established similarly by applying the previous lemma to the 
series coefficients of the next cases of the convergent functions, 
$\ConvGF{p}{\alpha}{R}{z}$, for $p \defequals 3,4$ and where 
$\alpha \defmapsto -\alpha$ and $R \defmapsto n$. 
\begin{align*} 
p_n(\alpha, R) & \equiv 
     [z^n] \left( 
     \frac{1 - 2(3\alpha+R) z + z^2 \left(R^2 + 4\alpha R + 6\alpha^2\right)}{ 
     1 - 3 (2\alpha+R) z + 3 (\alpha+R)(2\alpha+R) z^2 - 
     R (\alpha+R) (2\alpha+R) z^3} 
     \right) \phantom{\quad\ \ } \pmod{3, 3\alpha} \\ 
p_n(\alpha, R) & \equiv 
     [z^n] \left( 
     \mathsmaller{ 
     \frac{1-3(R+4 \alpha )z + z^2 \left(3 R^2+19 R \alpha +36 \alpha ^2\right)- (R+4 \alpha ) \left(R^2+3 R \alpha +6 \alpha ^2\right) z^3}{ 
     1-4(R+3 \alpha) z + 6 (R+2 \alpha ) (R+3 \alpha ) z^2 - 4(R+\alpha ) (R+2 \alpha ) (R+3 \alpha ) z^3 + R (R+\alpha ) (R+2 \alpha ) (R+3 \alpha ) z^4} 
     } 
     \right) \pmod{4, 4\alpha} 
\end{align*} 
The particular cases of the new congruence properties satisfied 
modulo $3$ ($3\alpha$) and $4$ ($4\alpha$) cited in 
\eqref{eqn_AlphaFactFnModulo3_congruence_stmts} from 
Section \ref{subSection_Intro_Examples} of the 
introduction also phrase results that are expanded 
through exact algebraic formulas involving the reciprocal zeros 
of the convergent denominator functions, 
$\ConvFQ{3}{-\alpha}{n}{z}$ and $\ConvFQ{4}{-\alpha}{n}{z}$, given in 
\tableref{table_SpCase_Listings_Of_Qhz_ConvFn}. 
\citep[\cf \S 1.11(iii), \S 4.43]{NISTHB}. 
The congruences cited in the example cases from the 
introduction then correspond to the respective special cases of the reflected 
numerator polynomial sequences provided in 
\tableref{table_RelfectedConvNumPolySeqs_sp_cases}. 
%% 
\ExampleQED 
\end{example} 

\sublabelII{Definitions related to the 
            reflected convergent numerator and denominator function 
            sequences} 
%% 
\begin{definition} 
%% 
For any $h \geq 1$, let the reflected convergent numerator and 
denominator function sequences be defined as follows: 
\begin{align*} 
\tagtext{Reflected Numerator Functions} 
\widetilde{\FP}_h(\alpha, R; z) & \defequals 
     z^{h-1} \times \ConvFP{h}{\alpha}{R}{z^{-1}} \\ 
\tagtext{Reflected Denominator Polynomials} 
\widetilde{\FQ}_h(\alpha, R; z) & \defequals 
     z^{h} \times \ConvFQ{h}{\alpha}{R}{z^{-1}} 
\end{align*} 
The listings given in 
\tableref{table_RelfectedConvNumPolySeqs_sp_cases} 
provide the first few simplified cases of the 
reflected numerator polynomial sequences 
which lead to the explicit formulations of the congruences 
modulo $p$ (and $p\alpha$) 
at each of the particular cases of $p \defequals 4, 5$ given in 
\eqref{eqn_AlphaFactFnModulo3_congruence_stmts}, and 
then for the next few small special cases for subsequent 
cases of the integers $p \geq 6$. 
%% 
\DefinitionQED 
\end{definition} 

\begin{definition} 
%% 
More generally, 
let the respective sequences of $p$--order roots, 
and then the corresponding sequences of 
reflected partial fraction coefficients, 
be defined as sequences over any $p \geq 2$ and each 
$1 \leq i \leq p$ through some fixed orderings of the 
special function zeros defined by the next equations. 
\begin{align*} 
\tagtext{Sequences of Reflected Roots} 
\left( \ell_{p,i}^{(\alpha)}(R) \right)_{i=1}^{p} & \defequals 
     \left\{ z_i : \widetilde{\FQ}_h(\alpha, R; z_i) = 0,\ 
     1 \leq i \leq p \right\} \\ 
\tagtext{Sequences of Reflected Coefficients} 
\left( C_{p,i}^{(\alpha)}(R)  \right)_{i=1}^{p} & \defequals 
     \left( 
     \widetilde{\FP}_h\left(\alpha, R; \ell_{p,i}^{(\alpha)}(R)\right) 
     \right)_{i=1}^{p}. 
\end{align*} 
The sequences of reflected roots 
defined by the previous equations above in terms of the 
reflected denominator polynomials 
correspond to special zeros of the 
confluent hypergeometric function, $\HypU{-h}{b}{w}$, and the 
associated Laguerre polynomials, $L_p^{(\beta)}(w)$, defined as in the 
special zero sets from 
Section \ref{subSection_Intro_GenConvFn_Defs_and_Properties} 
of the introduction 
\citep[\S 18.2(vi), \S 18.16]{NISTHB} 
\citep{LGWORKS-ASYMP-SPFNZEROS2008,PROPS-ZEROS-CHYPFNS80}. 
%% 
\DefinitionQED 
\end{definition} 
%% 

\sublabel{Generalized statements of the exact formulas and 
          new congruence expansions} 
%% 
The notation in the previous definition is then employed by the 
next restatements of the 
exact formula expansions and congruence properties cited in the 
initial forms of the expansions given in 
\eqref{eqn_AlphaFactFn_Exact_PartialFracsRep_v1} and 
\eqref{eqn_AlphaFactFn_Exact_PartialFracsRep_v2} of 
Section \ref{subSection_Intro_GenConvFn_Defs_and_Properties}. 
In particular, for any $p \geq 2$, $h \geq n \geq 1$, and 
where $\alpha, R \neq 0$ correspond to fixed 
integer--valued (or symbolic indeterminate) parameters, the 
expansions of the generalized product sequence cases defined by 
\eqref{eqn_GenFact_product_form} satisfy the following relations 
(see Remark \ref{remark_Congruences_for_Rational-Valued_Params} above): 
%%%% 
\renewcommand{\rootri}[4]{ 
     \ensuremath{\ell_{#1,#2}^{\left(#3\right)}\left(#4\right)}
} 
%%%% 
\StartGroupingSubEquations 
\label{eqn_AlphaFactFnModulop_congruence_stmts} 
\begin{align*} 
\tagonce\label{eqn_AlphaFactFnModulop_congruence_stmts-pn_stmts_v1} 
\pn{n}{\alpha}{R} & = 
     \sum\limits_{1 \leq i \leq h} 
     \frac{C_{h,i}^{(\alpha)}(R)}{\prod\limits_{j \neq i} 
     \left(\rootri{h}{i}{\alpha}{R} - \rootri{h}{j}{\alpha}{R}\right)} 
     \times \left( \rootri{h}{i}{\alpha}{R} \right)^{n+1},\ 
     && \quad \forall h \geq n \geq 1 \\ 
\pn{n}{\alpha}{R} & \equiv 
     \sum\limits_{1 \leq i \leq p} 
     \frac{C_{p,i}^{(\alpha)}(R)}{\prod\limits_{j \neq i} 
     \left(\rootri{p}{i}{\alpha}{R} - \rootri{p}{j}{\alpha}{R}\right)} 
     \times \left( \rootri{p}{i}{\alpha}{R} \right)^{n+1} 
     && \pmod{p, p\alpha, p\alpha^{2}, \ldots, p\alpha^{p}}. 
\end{align*} 
The expansions in the next equations similarly state the 
desired results stating the generalized forms of the 
congruence formulas for the $\alpha$--factorial functions, 
$\MultiFactorial{n}{\alpha}$, 
given by the particular special case expansions in 
\eqref{eqn_AlphaFactFnModulo3_congruence_stmts} from 
Section \ref{subsubSection_Examples_NewCongruences} in the 
introduction. 
\begin{align*} 
\tagonce 
n!_{(\alpha)} & = 
     \sum\limits_{1 \leq i \leq h} 
     \frac{C_{h,i}^{(-\alpha)}(n)}{\prod\limits_{j \neq i} 
     \left(\rootri{h}{i}{-\alpha}{n} - \rootri{h}{j}{-\alpha}{n}\right)} 
     \times \left( \rootri{h}{i}{-\alpha}{n} 
     \right)^{\left\lfloor \frac{n-1}{\alpha} \right\rfloor + 1},\ 
     && \quad \forall h \geq n \geq 1 && \\ 
n!_{(\alpha)} & \equiv 
     \undersetbrace{\defequals R_p^{(\alpha)}(n) 
     \text{ in Section \ref{subsubSection_Examples_NewCongruences} and 
            in \tableref{table_AlphaFactFns_Modulo245_spcase_examples}}}{ 
     \sum\limits_{1 \leq i \leq p} 
     \frac{C_{p,i}^{(-\alpha)}(n)}{\prod\limits_{j \neq i} 
     \left(\rootri{p}{i}{-\alpha}{n} - \rootri{p}{j}{-\alpha}{n}\right)} 
     \times \left( \rootri{p}{i}{-\alpha}{n} 
     \right)^{\left\lfloor \frac{n-1}{\alpha} \right\rfloor + 1} 
     } 
     && \pmod{p, p\alpha, p\alpha^{2}, \ldots, p\alpha^{p}}. 
\end{align*} 
\EndGroupingSubEquations 
The first pair of expansions given in 
\eqref{eqn_AlphaFactFnModulop_congruence_stmts-pn_stmts_v1} 
for the generalized product sequences, $\pn{n}{\alpha}{R}$, 
provide exact formulas and the corresponding 
new congruence properties for the Pochhammer symbol and 
Pochhammer $k$--symbol, 
in the respective special cases where 
$\Pochhammer{x}{n} \defmapsto \alpha^{-n} \pn{n}{\alpha}{\alpha x}$ and 
$\Pochhammer{x}{n,\alpha} \defmapsto \pn{n}{\alpha}{x}$ in the 
equations above. 

\subsection{Applications of 
            rational diagonal--coefficient generating functions and 
            Hadamard product sequences 
            involving the generalized convergent functions} 
\label{subSection_DiagonalGFSequences_Apps} 

\subsubsection{Generalized definitions and 
               coefficient extraction formulas for 
               sequences involving products of rational generating functions} 

We define the next extended notation for the 
\emph{Hadamard product} generating functions, 
$\left(\FiGF{1} \odot \FiGF{2}\right)(z)$ and 
$\left(\FiGF{1} \odot \cdots \odot \FiGF{k}\right)(z)$, 
at some fixed, formal $z \in \mathbb{C}$. 
Phrased in slightly different wording, 
we define \eqref{eqn_HProdGFs_kGen_def_v1} as an alternate notation for the 
\emph{diagonal} \emph{generating functions} that enumerate the 
corresponding product sequences 
generated by the diagonal coefficients of the multiple--variable 
product series in 
$k$ formal variables treated as in the reference \citep[\S 6.3]{ECV2}. 
\begin{align} 
\label{eqn_HProdGFs_kGen_def_v1} 
\FiGF{1} \odot \FiGF{2} \odot \cdots \odot \FiGF{k} & 
     \defequals 
     \sum_{n \geq 0} f_{1,n} f_{2,n} \cdots f_{k,n} \times z^{n} 
     \quad \text{ where } \quad 
     \FiGF{i}(z) \defequals \sum_{n \geq 0} f_{i,n} z^{n} 
     \text{ for } 
     1 \leq i \leq k 
\end{align} 
When $\FiGF{i}(z)$ is a rational function of $z$ for each $1 \leq i \leq k$, 
we have particularly nice expansions of the 
coefficient extraction formulas of the 
rational diagonal generating functions from the references 
\citep[\S 6.3]{ECV2} \citep[\S2.4]{GFLECT}. 
In particular, 
when $\FiGF{i}(z)$ is rational in $z$ at each respective $i$,
these rational generating functions are expanded through the 
next few useful formulas: 
\begin{align*} 
\tagtext{Diagonal Coefficient Extraction Formulas} 
\FiGF{1} \odot \FiGF{2} & = 
     [x_1^0] \Biggl( 
     \FiGF{2}\left(\frac{z}{x_1}\right) \cdot 
     \FiGF{1}(x_1) 
     \Biggr) \\ 
\tagonce\label{eqn_kGenHProdGFs_RationalDiagonalGF_Idents-stmts_v1} 
\FiGF{1} \odot \FiGF{2} \odot \FiGF{3} & = 
     [x_2^0 x_1^0] \Biggl( 
     \FiGF{3}\left(\frac{z}{x_2}\right) \cdot 
     \FiGF{2}\left(\frac{x_2}{x_1}\right) \cdot 
     \FiGF{1}(x_1) 
     \Biggr) \\ 
\FiGF{1} \odot \FiGF{2} \odot \cdots \odot \FiGF{k} & = 
     [x_{k-1}^0 \cdots x_2^0 x_1^0] \Biggl( 
     \FiGF{k}\left(\frac{z}{x_{k-1}}\right) \cdot 
     \FiGF{k-1}\left(\frac{x_{k-1}}{x_{k-2}}\right) 
     \times \cdots \times 
     \FiGF{2}\left(\frac{x_{2}}{x_{1}}\right) \cdot 
     \FiGF{1}(x_1) 
     \Biggr). 
\end{align*} 
Analytic formulas for the Hadamard products, 
$\FiGF{1} \odot \FiGF{2} = \FiGF{1}(z) \odot \FiGF{2}(z)$, 
when the component sequence generating functions are 
well enough behaved in some neighborhood of $z_0 = 0$ 
are given in the references 
\citep[\S 1.12(V); Ex.\ 1.30, p.\ 85]{ADVCOMB} \citep[\S 6.10]{ACOMB-BOOK} 
\footnotemod[Hadamard Product Integral Formulas]{ 
     Compare with the next known formula 
     when both sequence generating functions, 
     $\FiGF{1}(z)$ and $\FiGF{2}(z)$, are absolutely convergent for 
     some $|z| \leq r < 1$: 
     \begin{align*} 
     \left(\FiGF{1} \odot \FiGF{2}\right)(z^2) & = 
          \frac{1}{2\pi} \int_0^{2\pi} 
          \FiGF{1}\left(z e^{\imath t}\right) 
          \FiGF{2}\left(z e^{-\imath t}\right) dt. 
     \end{align*} 
}. 

%% 
We regard the rational convergents approximating the 
otherwise divergent 
ordinary generating functions for the generalized factorial function 
sequences strictly as formal power series in $z$ 
whenever possible in this article. 
The remaining examples in this section illustrate this more 
formal approach taken with the generating functions enumerating the 
factorial--related product sequences considered here. 
The next several subsections aim to provide several concrete applications and 
some notable special cases illustrating the utility of this 
approach to the more general formal sequence products enumerated through the 
rational convergent functions, 
especially when combined with other 
generating function techniques discussed elsewhere and in the references. 

\subsubsection{Examples: Constructing hybrid rational 
               generating function approximations from the 
               convergent functions enumerating the generalized 
               factorial product sequences} 
\label{subsubSection_remark_HybridDiagonalHPGFs} 

When one of the generating functions of an individual sequence from the 
Hadamard product representations in 
\eqref{eqn_kGenHProdGFs_RationalDiagonalGF_Idents-stmts_v1} 
is not rational in $z$, we still proceed, however slightly more carefully, to 
formally enumerate the terms of these sequences that arise in applications. 
For example, 
the \emph{central binomial coefficients} 
are enumerated by the next convergent--based generating functions 
whenever $n \geq 1$ 
\citep[\cf \S 5.3]{GKP} \citeOEIS{A000984}. 
\begin{align*} 
\tagtext{Central Binomial Coefficients} 
\binom{2n}{n} & = 
     \frac{2^{2n}}{n!} \times \Pochhammer{1 / 2}{n} \ \ \ = 
     [z^{n}] [x^{0}] \Biggl( 
     e^{2x} \ConvGF{n}{2}{1}{\frac{z}{x}} 
     \Biggr) \\ 
     & = 
     \frac{2^{n}}{n!} \times (2n-1)!! = 
     [z^{n}] [x^{1}] \Biggl( 
     e^{2x} \ConvGF{n}{-2}{2n-1}{\frac{z}{x}} 
     \Biggr). 
\end{align*} 

\sublabel{Other generating function identities involving the 
          central binomial coefficients} 
Since the reciprocal factorial squared terms, $(n!)^{-2}$, 
are generated by the power series for the 
\emph{modified Bessel function of the first kind}, 
$I_0(2\sqrt{z}) = \sum_{n \geq 0} z^{n} / (n!)^{2}$, these 
central binomial coefficients are also enumerated as the 
diagonal coefficients of the 
following convergent function products 
\citep[\S 10.25(ii)]{NISTHB} \citep[\S 5.5]{GKP} 
\citep[\cf \S 1.13(II)]{ADVCOMB}: 
\begin{align*} 
\binom{2n}{n} & = 
     \frac{2^{2n} \Pochhammer{1}{n} \Pochhammer{\frac{1}{2}}{n}}{(n!)^2} && = 
     [x_1^0 x_2^0 z^n] \left( 
     \ConvGF{n}{2}{2}{\frac{z}{x_2}} \ConvGF{n}{2}{1}{\frac{x_2}{x_1}} 
     I_0\left(2\sqrt{x_1}\right) 
     \right) \\ 
     & = 
     \frac{(2n)!! (2n-1)!!}{(n!)^2} && = 
     [x_1^0 x_2^0 z^n] \left( 
     \ConvGF{n}{-2}{2n}{\frac{z}{x_2}} \ConvGF{n}{-2}{2n-1}{\frac{x_2}{x_1}} 
     I_0\left(2\sqrt{x_1}\right) 
     \right). 
\end{align*} 
The next binomial coefficient product sequence is enumerated through a 
similar construction of the convergent--based generating function 
identities expanded in the previous equations: 
\begin{align*} 
\tagtext{Paired Binomial Coefficient Products} 
\binom{3n}{n} \binom{2n}{n} & = 
     \frac{3^{3n} \Pochhammer{\frac{1}{3}}{n} 
     \Pochhammer{\frac{2}{3}}{n}}{(n!)^2} \\ 
     & = 
     [x_1^0 x_2^0 z^n] \left( 
     \ConvGF{n}{3}{2}{\frac{3z}{x_2}} \ConvGF{n}{3}{1}{\frac{x_2}{x_1}} 
     I_0\left(2\sqrt{x_1}\right) 
     \right) \\ 
     & = 
     \frac{3^n}{(n!)^2} \times 
     \AlphaFactorial{3n-1}{3} \AlphaFactorial{3n-2}{3} \\ 
     & = 
     [x_1^0 x_2^0 z^n] \left( 
     \ConvGF{n}{-3}{3n-1}{\frac{3z}{x_2}} 
     \ConvGF{n}{-3}{3n-2}{\frac{x_2}{x_1}} 
     I_0\left(2\sqrt{x_1}\right) 
     \right). 
\end{align*} 

\sublabel{Generating ratios of factorial functions and 
          binomial coefficients} 
The next few identities for the convergent generating function products 
over the binomial coefficient variants cited in 
\eqref{eqn_HybridDiagCoeffHPGFs_BinomCoeff_Examples-exps_v1} from the 
introduction are generated as the diagonal coefficients 
of the corresponding products of the convergent functions convolved with 
arithmetic progressions extracted from the exponential series 
in the form of the following equation, 
where $\omega_a \defequals \exp\left(2\pi\imath / a\right)$ denotes the 
primitive $a^{th}$ root of unity for integers $a \geq 2$ 
\citep[\S 1.2.9]{TAOCPV1} \citep[Ex.\ 1.26, p.\ 84]{ADVCOMB}
\footnotemod[Modified Exponential Series Generating Functions]{ 
     The modified generating functions, $\widehat{E}_a(z) = E_{a,1}(z)$, 
     correspond to special cases of the 
     \emph{Mittag--Leffler function}, $E_{a,b}(z)$, defined as in the 
     reference by the following series \citep[\S 10.46]{NISTHB}: 
     \begin{align*} 
     E_{a,b}(z) & \defequals 
          \sum_{n \geq 0} \frac{z^n}{\Gamma(an+b)},\ a > 0. 
     \end{align*} 
     These modified exponential series generating functions then denote the 
     power series expansions of arithmetic progressions over the 
     coefficients of the ordinary generating function 
     for the exponential series sequences, $f_n \defequals 1 / n!$ and 
     $f_{an} = 1 / (an)!$. 
     For $a \defequals 2, 3, 4$, the 
     particular cases of these exponential series generating functions 
     are given by 
     \begin{equation*} 
     \widehat{E}_2(z) = \cosh\left(\sqrt{z}\right), 
     \widehat{E}_3(z) = 
          \frac{1}{3}\left( 
          e^{z^{1/3}} + 2 e^{-\frac{z^{1/3}}{2}} 
          \mathsmaller{\cos\left(\frac{\sqrt{3} \cdot z^{1/3}}{2}\right)} 
          \right), 
          \text{ and } 
     \widehat{E}_4(z) = 
          \frac{1}{2}\left( 
          \cos\left(z^{1/4}\right) + \cosh\left(z^{1/4}\right) 
          \right), 
     \end{equation*} 
     where the powers of the $a^{th}$ roots of unity in these 
     special cases satisfy 
     $\omega_2 = -1$, 
     $\omega_3 = \frac{\imath}{2}\left(\imath + \sqrt{3}\right)$, 
     $\omega_3^2 = -\frac{\imath}{2}\left(-\imath + \sqrt{3}\right)$, and 
     $\left(\omega_4^{m}\right)_{1 \leq m \leq 4} = \left(\imath, -1, -\imath, 1\right)$ 
     (see the computations given in the reference \citep{SUMMARYNBREF-STUB}). 
}: 
\begin{align*} 
\tagonce\label{eqn_HybridDiagCoeffHPGFs_BinomCoeff_Examples-exps_ExpGFs_v2} 
\widehat{E}_a(z) & \defequals \sum_{n \geq 0} \frac{z^n}{(an)!} = 
     \frac{1}{a}\left( 
     e^{z^{1/a}} + e^{\omega_a \cdot z^{1/a}} + 
     e^{\omega_a^2 \cdot z^{1/a}} + \cdots + 
     e^{\omega_a^{a-1} \cdot z^{1/a}} 
     \right),\ 
     a > 1. 
\end{align*} 
The next particular special cases of these 
diagonal--coefficient generating functions 
corresponding to the binomial coefficient sequence variants from 
\eqref{eqn_HybridDiagCoeffHPGFs_BinomCoeff_Examples-exps_v1} of 
Section \ref{subsubSection_Intro_Examples_Fact-RelatedSeqs_GenByTheConvFns} 
are then given through the following coefficient extraction identities 
provided by \eqref{eqn_kGenHProdGFs_RationalDiagonalGF_Idents-stmts_v1} 
\citeOEIS{A166351,A066802}: 
\begin{align*} 
\frac{(6n)!}{(3n)!} & = 
     \frac{6^{6n} 
     \bcancel{\Pochhammer{1}{n}} 
     \bcancel{\Pochhammer{\frac{2}{6}}{n}} 
     \bcancel{\Pochhammer{\frac{3}{6}}{n}} \times 
     \Pochhammer{\frac{1}{6}}{n} 
     \Pochhammer{\frac{3}{6}}{n} 
     \Pochhammer{\frac{5}{6}}{n} 
     }{ 
     3^{3n} 
     \bcancel{\Pochhammer{1}{n}} 
     \bcancel{\Pochhammer{\frac{1}{3}}{n}} 
     \bcancel{\Pochhammer{\frac{2}{3}}{n}} 
     \phantom{\qquad}} \\ 
     & = 
     24^{n} \times 6^n \Pochhammer{1/6}{n} \times 
     2^{n} \Pochhammer{1/2}{n} \times 6^n \Pochhammer{5/6}{n} \\ 
     & = 
     [x_2^0 x_1^0 z^n] \left( 
     \ConvGF{n}{6}{5}{\frac{24z}{x_2}} \ConvGF{n}{2}{1}{\frac{x_2}{x_1}} 
     \ConvGF{n}{6}{1}{x_1} 
     \right) \\ 
     & = 
     8^n \times \AlphaFactorial{6n-5}{6} 
     \AlphaFactorial{6n-3}{6} \AlphaFactorial{6n-1}{6} \\ 
     & = 
     [x_2^0 x_1^0 z^n] \left( 
     \ConvGF{n}{-6}{6n-5}{\frac{8z}{x_2}} 
     \ConvGF{n}{-6}{6n-3}{\frac{x_2}{x_1}} 
     \ConvGF{n}{-6}{6n-1}{x_1} 
     \right) \\ 
\binom{6n}{3n} & = 
     [x_3^0 x_2^0 x_1^0 z^n] \Biggl( 
     \ConvGF{n}{6}{5}{\frac{8z}{x_3}} \ConvGF{n}{6}{3}{\frac{x_3}{x_2}} 
     \ConvGF{n}{6}{1}{\frac{x_2}{x_1}} \times \\ 
     & \phantom{= [x_3^0 x_2^0 x_1^0 z^n] \Biggl( \quad} \times 
     \undersetbrace{\widehat{E}_3(x_1) = E_{3,1}(x_1)}{
     \frac{1}{3}\left( 
     e^{x_1^{1/3}} + 2 e^{-\frac{x_1^{1/3}}{2}} 
     \cos\left(\frac{\sqrt{3} \cdot x_1^{1/3}}{2}\right) 
     \right) 
     } 
     \Biggr) \\ 
     & = 
     [x_3^0 x_2^0 x_1^0 z^n] \Biggl( 
     \ConvGF{n}{-6}{6n-5}{\frac{8z}{x_3}} 
     \ConvGF{n}{-6}{6n-3}{\frac{x_3}{x_2}} \times \\ 
     & \phantom{= [x_3^0 x_2^0 x_1^0 z^n] \Biggl( \quad} \times 
     \ConvGF{n}{-6}{6n-1}{\frac{x_2}{x_1}} \times 
     \undersetbrace{\widehat{E}_3(x_1)}{
     \frac{1}{3}\left( 
     e^{x_1^{1/3}} + 2 e^{-\frac{x_1^{1/3}}{2}} 
     \cos\left(\frac{\sqrt{3} \cdot x_1^{1/3}}{2}\right) 
     \right) 
     } 
     \Biggr). 
\end{align*} 
Similarly, the following related sequence cases forming particular 
expansions of these binomial coefficient variants are generated by 
\begin{align*} 
\binom{8n}{4n} & = 
     \frac{2^{16n}}{(4n)!} \times 
     \mathsmaller{\Pochhammer{\frac{1}{8}}{n} 
     \Pochhammer{\frac{3}{8}}{n} \Pochhammer{\frac{5}{8}}{n} 
     \Pochhammer{\frac{7}{8}}{n}} \\ 
     & = 
     [x_1^0 x_2^0 x_3^0 x_4^0 z^n] \Biggl( 
     \ConvGF{n}{8}{7}{\frac{16z}{x_4}} \ConvGF{n}{8}{5}{\frac{x_4}{x_3}} 
     \ConvGF{n}{8}{3}{\frac{x_3}{x_2}} \times \\ 
     & \phantom{= [x_1^0 x_2^0 x_3^0 x_4^0 z^n] \Biggl( \quad} \times 
     \ConvGF{n}{8}{1}{\frac{x_2}{x_1}} \times 
     \undersetbrace{\widehat{E}_4(x_1) = E_{4,1}(x_1)}{
     \frac{1}{2}\left( 
     \cos\left(x_1^{1/4}\right) + \cosh\left(x_1^{1/4}\right) 
     \right) 
     } 
     \Biggr) \\ 
     & = 
     \frac{2^{4n}}{(4n)!} \times 
     \AlphaFactorial{8n-7}{8} \AlphaFactorial{8n-5}{8} 
     \AlphaFactorial{8n-3}{8} \AlphaFactorial{8n-3}{8} \\ 
     & = 
     [x_1^0 x_2^0 x_3^0 x_4^0 z^n] \Biggl( 
     \ConvGF{n}{-8}{8n-7}{\frac{16z}{x_4}} 
     \ConvGF{n}{-8}{8n-5}{\frac{x_4}{x_3}} \times \\ 
     & \phantom{= [x_1^0 x_2^0 x_3^0 x_4^0 z^n] \Biggl( \quad} \times 
     \ConvGF{n}{-8}{8n-3}{\frac{x_3}{x_2}} 
     \ConvGF{n}{-8}{8n-1}{\frac{x_2}{x_1}} \times 
     \widehat{E}_4(x_1) 
     \Biggr). 
\end{align*} 

\sublabel{Generating the subfactorial function (sequence of derangements)} 
%% 
Another pair of convergent--based generating function identities 
enumerating the sequence of 
\emph{subfactorials}, $\left(!n\right)_{n \geq 1}$, 
or \emph{derangements}, $\left(n\?\right)_{n \geq 1}$, 
are expanded for $n \geq 1$ as follows 
(see Example \ref{remark_FactSumIdents_SubfactorialSums_ConvIdents} and the 
related examples cited in 
Section \ref{subsubSection_Apps_Example_SumsFactFn_Seqs}) 
\citep[\S 5.3]{GKP} 
\citep[\cf \S 8.4]{NISTHB} \citeOEIS{A000166}: 
\begin{align*} 
\tagtext{Subfactorial Function} 
!n & \defequals 
     %\frac{n!}{0!} - \frac{n!}{1!} + \cdots + (-1)^{n} \frac{n!}{n!} = %\\ 
     %& \phantom{:} = 
     n! \times \sum_{i=0}^{n} \frac{(-1)^{i}}{i!} 
     \quad \seqmapsto{A000166} 
     \left(0, 1, 2, 9, 44, 265, 1854, 14833, \ldots \right) \\ 
     & \phantom{:} = 
     [z^n x^0] \left( 
     \frac{e^{-x}}{(1-x)} \times \ConvGF{n}{-1}{n}{\frac{z}{x}} 
     \right) \\ 
     & \phantom{:} = 
     [x^0 z^n] \left( 
     \frac{e^{-x}}{(1-x)} \times \ConvGF{n}{1}{1}{\frac{z}{x}} 
     \right). 
\end{align*} 

%% 
\begin{remark}[Laplace--Borel Transformations of Formal Power Series] 
\label{remark_Formal_Laplace-Borel_Transforms} 
%% 
The sequence of subfactorials is enumerated through the 
previous equations as the diagonals of generating function products 
where the rational convergent functions, $\ConvGF{n}{\alpha}{R}{z}$, 
generate the sequence multiplier of $n!$ 
corresponding to the (\emph{formal}) \emph{Laplace--Borel transform}, 
$\mathcal{L}(f(t); z)$, defined by the integral transforms in the next 
equations 
\citep[\cf \S 2.2]{FLAJOLET80B} 
\citep[\S B.4]{ACOMB-BOOK} \citep[p.\ 566]{GKP}, 
applied termwise to the power series given by the 
exponential generating function, $f(x) \defequals e^{-x} / (1-x)$, 
for this sequence. 
\begin{align*} 
\tagtext{Laplace-Borel Transformation Integrals} 
\mathcal{L}(\widehat{F}; z) & = 
     \phantom{\frac{1}{2\pi}} \int_0^{\infty} e^{-t} \widehat{F}(tz) dt \\ 
\mathcal{L}^{-1}(\widetilde{F}; z) & = 
     \frac{1}{2\pi} \int_0^{2\pi} 
     \widetilde{F}\left(-z e^{-\imath s}\right) 
     e^{-e^{\imath s}} ds. 
\end{align*} 
%% 
The necessary condition for primality from the example given in 
\eqref{eqn_POddPrime_NecessaryCond_HNumGF_example-stmt_v1} of the introduction 
is constructed by employing a similar technique with the 
Stirling number generating functions given by the following equation 
when $p \defequals 1$ \citep[\S 7.4]{GKP}: 
\begin{align*} 
\sum_{n \geq 0} \gkpSI{n+1}{p+1} \frac{z^{n}}{n!} & = 
     \mathlarger{\frac{\partial}{\partial z}} \left[ 
     \frac{\Log\left(\frac{1}{1-z}\right)^{p+1}}{(p+1)!} 
     \right] = 
     \frac{(-1)^{p}}{p!} \cdot \frac{\Log(1-z)^{p}}{(1-z)}. 
\end{align*} 
The applications cited in 
Section \ref{subsubSection_Apps_Example_SumsFactFn_Seqs} and 
Section \ref{subsubSection_Apps_Example_SumsOfPowers_Seqs} 
in this article below 
employ this particular generating function technique to enumerate the 
factorial function multipliers provided by 
these rational convergent functions in several particular cases of 
sequences involving finite sums over factorial functions, 
sums of powers sequences, and new forms of approximate 
generating functions for the binomial coefficients and 
sequences of binomials. 
%% 
\RemarkQED 
\end{remark} 

\subsection{Examples: Expanding arithmetic progressions of the 
            single factorial function} 
\label{subsubSection_Apps_ArithmeticProgs_of_the_SgFactFns} 

One application suggested by the results in the previous section 
provides $a$--fold reductions of the 
$h$--order series approximations otherwise required to 
exactly enumerate 
arithmetic progressions of the single factorial function according to the 
next result. 
\begin{align} 
\label{eqn_AKPlusB_FactFn_Conv_Ident-stmt_v1} 
\left(an+r\right)! & = 
     [z^{an+r}] \ConvGF{h}{-1}{an+r}{z},\ 
     \forall n \geq 1, a \in \mathbb{Z}^{+}, 0 \leq r < a, 
     \forall h \geq an+r 
\end{align} 
The arithmetic progression sequences of the single factorial function 
formed in the 
particular special cases when $a \defequals 2, 3$ 
(and then for particular cases of $a \defequals 4,5$) 
are expanded in the 
examples cited below illustrate the utility to these convergent--based 
formal generating function approximations. 

\begin{prop}[Factorial Function Multiplication Formulas]
%% 
The statement of 
\emph{Gauss's multiplication formula} for the gamma function yields the 
following decompositions of the single factorial functions, $(an+r)!$, 
into a finite product over $a$ of the integer--valued multiple factorial 
sequences defined by \eqref{eqn_nAlpha_Multifact_variant_rdef} for 
$n \geq 1$ and whenever $a \geq 2$ and $0 \leq r < a$ 
are fixed natural numbers 
\citep[\S5.5(iii)]{NISTHB} \citep[\S 2]{ATLASOFFUNCTIONS} 
\citep{WOLFRAMFNSSITE-INTRO-FACTBINOMS}: 
\StartGroupingSubEquations 
\begin{align} 
(an+r)! & = 
     \undersetbrace{ \mathsmaller{(an+r)! = 
     \prod\limits_{i=0}^{a-1} \AlphaFactorial{an+r-i}{a} = 
     \prod\limits_{i=0}^{a-1} \pn{n}{-a}{an+r-i} }}{ 
     \AlphaFactorial{an+r}{a} \times \AlphaFactorial{an+r-1}{a} 
     \times \cdots \times 
     \AlphaFactorial{an+r-a+1}{a} 
     } \\ 
(an+r)! & = 
     \undersetbrace{ \mathsmaller{(an+r)! = 
     \prod\limits_{i=1}^{a} a^{n} \times \Pochhammer{\frac{r+i}{a}}{n} = 
     \prod\limits_{i=1}^{a} \pn{n}{a}{r+i} } }{ 
     r! \cdot a^{an} \Pochhammer{\frac{1+r}{a}}{n} 
     \Pochhammer{\frac{2+r}{a}}{n} \times \cdots \times 
     \Pochhammer{\frac{a-1+r}{a}}{n} 
     \Pochhammer{\frac{a+r}{a}}{n} 
     } \\ 
\notag 
     & = 
     r! \cdot a^{an} \Pochhammer{1+\frac{r}{a}}{n} 
     \Pochhammer{1+\frac{r-1}{a}}{n} \times \cdots \times 
     \Pochhammer{1+\frac{r-a+1}{a}}{n},\ 
     \forall a,n \in \mathbb{Z}^{+}, r \geq 0. 
\end{align} 
\EndGroupingSubEquations 
%% 
\end{prop} 
%%%% 
\begin{proof} 
%% 
The first identity corresponds to the expansions of the 
single factorial function by a product of $\alpha$ distinct 
$\alpha$--factorial functions for any fixed integer $\alpha \geq 2$ in the 
following forms: 
\begin{align*} 
n! & = n!! \cdot (n-1)!! = n!!! \cdot (n-1)!!! \cdot (n-2)!!! 
     = \prod_{i=0}^{\alpha-1} \AlphaFactorial{n-i}{\alpha},\ 
     \alpha \in \mathbb{Z}^{+}. 
\end{align*} 
The expansions of the last two identities stated above also follow from the 
known \emph{multiplication formula} for the Pochhammer symbol 
expanded by the next equation \citep{WOLFRAMFNSSITE-INTRO-FACTBINOMS} 
for any fixed integers $a \geq 1$, $r \geq 0$, and where 
$(an+r)! = \Pochhammer{1}{an+r}$ by 
Lemma \ref{lemma_GenConvFn_EnumIdents_pnAlphaRSeq_idents_combined_v1}. 
\begin{align*} 
\tagtext{Pochhammer Symbol Multiplication Formula} 
\Pochhammer{x}{an+r} & = 
     \Pochhammer{x}{r} \times a^{an} \times 
     \prod_{j=0}^{a-1} \Pochhammer{\frac{x+j+r}{a}}{n}, 
\end{align*} 
The two identities involving the corresponding products of the 
sequences from \eqref{eqn_GenFact_product_form} 
provided by the braced formulas follow similarly from the lemma, and 
lead to several direct expansions of the convergent--function--based 
product sequences identities expanded in the next examples. 
%% 
\end{proof} 

\subsubsection{Expansions of arithmetic progression sequences involving the 
               double factorial function ($\mathbf{a \defequals 2}$)} 
In the particular cases where $a \defequals 2$ (with $r \defequals 0, 1$), 
we obtain the following forms of the corresponding alternate expansions of 
\eqref{eqn_AKPlusB_FactFn_Conv_Ident-stmt_v1} 
enumerated by the diagonal coefficients of the next 
convergent--based product generating functions for all $n \geq 1$ 
\citep[\cf \S 2]{ATLASOFFUNCTIONS} 
\citeOEIS{A010050,A009445}: 
\begin{align*} 
\tagtext{Double Factorial Function Expansions} 
(2n)! 
      & = 2^{n} n! \times (2n-1)!! \\ 
      & = [z^n] [x^0] \left( 
          \ConvGF{n}{-1}{n}{\frac{2z}{x}} 
          \ConvGF{n}{-2}{2n-1}{x} 
          \right) \\ 
      & = 2^{n} n! \times 2^{n} \Pochhammer{1/2}{n} \\ 
      & = [x^0 z^n] \left( 
          \ConvGF{n}{1}{1}{\frac{2z}{x}} \ConvGF{n}{2}{1}{x} 
          \right) \\ 
(2n+1)! 
      & = 2^{n} n! \times (2n+1)!! \\ 
      & = [z^n] [x^0] \left( 
          \ConvGF{n}{-1}{n}{\frac{2z}{x}} 
          \ConvGF{n}{-2}{2n+1}{x} 
          \right) \\ 
      & = 2^{n} n! \times 2^{n} \Pochhammer{3/2}{n} \\ 
      & = [x^1 z^n] \left( 
          \ConvGF{n}{1}{1}{\frac{2z}{x}} \ConvGF{n}{2}{1}{x} 
          \right) \\ 
      & = [x^0 z^n] \left( 
          \ConvGF{n}{1}{1}{\frac{2z}{x}} \ConvGF{n}{2}{3}{x} 
          \right). 
\end{align*} 

\subsubsection{Expansions of arithmetic progression sequences involving the 
               triple factorial function ($\mathbf{a \defequals 3}$)} 
%Similarly, 
When $a \defequals 3$ we similarly obtain the next few 
alternate expansions generating the triple factorial products 
for the arithmetic progression sequences in 
\eqref{eqn_AKPlusB_FactFn_Conv_Ident-stmt_v1} 
stated in the following equations for any $n \geq 2$ 
by extending the constructions of the identities for the 
expansions of the double factorial products in the previous equations 
\citep[\S 2]{ATLASOFFUNCTIONS} 
\citeOEIS{A100732,A100089,A100043}: 
\begin{align*} 
\tagtext{Triple Factorial Function Expansions} 
(3n)! & = (3n)!!! \times (3n-1)!!! \times (3n-2)!!! \\ 
      & = [z^n] [x_2^0 x_1^0] \left( 
          \ConvGF{n}{-1}{n}{\frac{3z}{x_2}} 
          \ConvGF{n}{-3}{3n-1}{\frac{x_2}{x_1}} 
          \ConvGF{n}{-3}{3n-2}{x_1} 
          \right) \\ 
      & = 3^n n! \times 3^n \Pochhammer{2/3}{n} \times 
          3^n \Pochhammer{1/3}{n} \\ 
      & = [x_1^{0} x_2^{0} z^{n}] \left( 
          \ConvGF{n}{1}{1}{\frac{3z}{x_2}} 
          \ConvGF{n}{3}{1}{\frac{x_2}{x_1}} 
          \ConvGF{n}{3}{2}{x_1} 
          \right) \\ 
(3n+1)! & = (3n)!!! \times (3n-1)!!! \times (3(n+1)-2)!!! \\ 
        & = [z^n] [x_2^0 x_1^{-1}] \left( 
          \ConvGF{n}{-1}{n}{\frac{3z}{x_2}} 
          \ConvGF{n}{-3}{3n-1}{\frac{x_2}{x_1}} 
          \ConvGF{n}{-3}{3n+1}{x_1} 
          \right) \\ 
      & = 3^n n! \times 3^n \Pochhammer{2/3}{n} \times 
          3^n \Pochhammer{4/3}{n} \\ 
      & = [x_1^{0} x_2^{0} z^{n}] \left( 
          \ConvGF{n}{1}{1}{\frac{3z}{x_2}} 
          \ConvGF{n}{3}{4}{\frac{x_2}{x_1}} 
          \ConvGF{n}{3}{2}{x_1} 
          \right) \\ 
(3n+2)! & = (3n)!!! \times (3(n+1)-1)!!! \times (3(n+1)-2)!!! \\ 
        & = [z^n] [x_2^{-1} x_1^0] \left( 
          \ConvGF{n}{-1}{n}{\frac{3z}{x_2}} 
          \ConvGF{n}{-3}{3n+2}{\frac{x_2}{x_1}} 
          \ConvGF{n}{-3}{3n+1}{x_1} 
          \right) \\ 
      & = 2 \times 3^n n! \times 3^n \Pochhammer{5/3}{n} \times 
          3^n \Pochhammer{4/3}{n} \\ 
      & = [x_1^{1} x_2^{0} z^{n}] \left( 
          \ConvGF{n}{1}{1}{\frac{3z}{x_2}} 
          \ConvGF{n}{3}{4}{\frac{x_2}{x_1}} 
          \ConvGF{n}{3}{2}{x_1} 
          \right). 
\end{align*} 

\subsubsection{Other special cases involving the 
               quadruple and quintuple factorial functions 
               ($\mathbf{a \defequals 4,5}$)} 
The additional forms of the diagonal--coefficient generating functions 
corresponding to the special cases of the sequences in 
\eqref{eqn_AKPlusB_FactFn_Conv_Ident-stmt_v1} where 
$(a, r) \defequals (4, 2)$ and $(a, r) \defequals (5, 3)$, respectively 
involving the \emph{quadruple} and \emph{quintuple factorial} functions 
are also cited in the next equations to further illustrate the procedure 
outlined by the previous two example cases. 
\begin{align*} 
(4n+2)! & = (4n)!!!! \times (4n-1)!!!! \times (4(n+1)-2)!!!! \times 
            (4(n+1)-3)!!!! \\ 
      & = [z^n] [x_3^{0} x_2^{-1} x_1^0] \Biggl( 
          \ConvGF{n}{-1}{n}{\frac{4z}{x_3}} 
          \ConvGF{n}{-4}{4n-1}{\frac{x_3}{x_2}} \times \\ 
      & \phantom{= [z^n] [x_3^{0} x_2^0 x_1^0] \Biggl(} \times 
          \ConvGF{n}{-4}{4n+2}{\frac{x_2}{x_1}} 
          \ConvGF{n}{-4}{4n+1}{x_1} 
          \Biggr) \\ 
      & = 
      2 \times 4^{4n} \times 
      \Pochhammer{1}{n} \Pochhammer{3/4}{n} 
      \Pochhammer{3/2}{n} \Pochhammer{5/4}{n} \\ 
      & = [x_1^0 x_2^1 x_3^0 z^n] \Biggl( 
          \ConvGF{n}{1}{1}{\frac{4z}{x_3}} 
          \ConvGF{n}{4}{3}{\frac{x_3}{x_2}} 
          \ConvGF{n}{4}{2}{\frac{x_2}{x_1}} \times \\ 
      & \phantom{= [x_1^0 x_2^1 x_3^0 z^n] \Biggl(} \times 
          \ConvGF{n}{4}{1}{x_1} 
          \Biggr),\ n \geq 2 \\ 
(5n+3)! & = (5n)!_{(5)} \times (5n-1)!_{(5)} \times (5(n+1)-2)!_{(5)} \times 
            (5(n+1)-3)!_{(5)} \times (5(n+1)-4)!_{(5)} \\ 
        & = [z^n] [x_4^0 x_3^{-1} x_2^0 x_1^0] \Biggl( 
            \ConvGF{n}{-1}{n}{\frac{5z}{x_4}} 
            \ConvGF{n}{-5}{5n-1}{\frac{x_4}{x_3}} \times \\ 
        & \phantom{= [z^n] [x_4^0 x_3^{-1} x_2^0 x_1^0] \Biggl( } \times 
            \ConvGF{n}{-5}{5n+3}{\frac{x_3}{x_2}} 
            \ConvGF{n}{-5}{5n+2}{\frac{x_2}{x_1}} \times \\ 
        & \phantom{= [z^n] [x_4^0 x_3^{-1} x_2^0 x_1^0] \Biggl( } \times 
            \ConvGF{n}{-5}{5n+1}{x_1} 
            \Biggr) \\ 
      & = 
      6 \times 5^{5n} \times 
      \Pochhammer{1}{n} \Pochhammer{4/5}{n} \Pochhammer{8/5}{n} 
      \Pochhammer{7/5}{n} \Pochhammer{6/5}{n} \\ 
      & = [x_1^0 x_2^0 x_3^1 x_4^0 z^n] \Biggl( 
          \ConvGF{n}{1}{1}{\frac{5z}{x_4}} 
          \ConvGF{n}{5}{4}{\frac{x_4}{x_3}} 
          \ConvGF{n}{5}{3}{\frac{x_3}{x_2}} \times \\ 
      & \phantom{= [x_1^0 x_2^0 x_3^1 x_4^0 z^n] \Biggl(} \times 
          \ConvGF{n}{5}{2}{\frac{x_2}{x_1}} 
          \ConvGF{n}{5}{1}{x_1} 
          \Biggr),\ n \geq 2. 
\end{align*} 
%% 
Convergent--based generating function identities enumerating 
specific expansions corresponding to other cases of 
\eqref{eqn_AKPlusB_FactFn_Conv_Ident-stmt_v1} 
when $a \defequals 4,5$ are given in the reference 
\citep{SUMMARYNBREF-STUB}. 
%% 
\begin{remark} 
%% 
The truncated power series approximations generating the 
single factorial functions 
formulated in the last few special case examples expanded in this section 
are also compared to the known results for 
extracting arithmetic progressions 
from any formal, ordinary power series generating function of an 
arbitrary sequence 
through the primitive $a^{th}$ roots of unity, 
$\omega_a \defequals \exp\left(2\pi\imath / a\right)$, 
stated in the references 
\citep[\S 1.2.9]{TAOCPV1} \citep[Ex.\ 1.26, p.\ 84]{ADVCOMB}, and 
in the special cases of the 
exponential series generating functions defined in 
\eqref{eqn_HybridDiagCoeffHPGFs_BinomCoeff_Examples-exps_ExpGFs_v2} of the 
previous section. 
%% 
\RemarkQED 
\end{remark} 

\subsection{Examples: Generalized superfactorial function products and 
            relations to the Barnes $G$--function} 

The \emph{superfactorial function}, $S_1(n)$, 
also denoted by $S_{1,0}(n)$ in the notation of 
\eqref{eqn_SAlphadn_GenSuperFactFnSeqs_product_based_def_v1} below, 
is defined for integers $n \geq 1$ 
by the factorial products \citeOEIS{A000178} 
\begin{align*} 
\tagtext{Superfactorial Function} 
S_1(n) & \defequals \prod_{k \leq n} k! %+ \Iverson{n = 0} && 
     \quad \seqmapsto{A000178} 
     \left(1, 2, 12, 288, 34560, 24883200, \ldots \right). 
\end{align*} 
These superfactorial functions are given in terms of the 
\emph{Barnes G--function}, $G(z)$, for $z \in \mathbb{Z}^{+}$ through the 
relation $S_1(n) = G(n+2)$. 
The Barnes G--function, $G(z)$, corresponds to a so--termed 
\quotetext{\emph{double gamma function}} satisfying a 
functional equation of the following form 
for natural numbers $n \geq 1$ 
\citep[\S 5.17]{NISTHB} \citep{CONTRIB-THEORY-BARNESGFN}: 
\begin{equation*} 
\tagtext{Barnes G-Function} 
G(n+2) = \Gamma(n+1) G(n+1) + \Iverson{n = 1}. 
\end{equation*} 
We can similarly expand the superfactorial function, $S_1(n)$, by 
unfolding the factorial products in the previous definition recursively 
according to the formulas given in the next equation. 
\begin{align*} 
S_1(n) & = 
     n! \cdot (n-1)! \times \cdots \times (n-k+1)! \cdot S_1(n-k),\ 
     0 \leq k < n 
\end{align*} 
%% 
The product sequences over the single factorial functions 
formed by the last equations then 
lead to another application of the 
diagonal--coefficient product generating functions 
involving the rational convergent 
functions that enumerate the functions, $(n-k)!$, when $n-k \geq 1$. 

\subsubsection{Generating the ordinary superfactorial function products} 

In particular, 
these particular cases of the 
diagonal coefficient, Hadamard--product--like sequences involving the 
single factorial function are generated as the coefficients 
\begin{align*} 
S_1(n) = 
     \left([z^{n}] \ConvGF{n}{-1}{n}{z}\right) & \times 
     \left([z^{n}] z \cdot \ConvGF{n}{-1}{n-1}{z}\right) 
     \times \\ 
     & \times 
     \left([z^{n}] z^2 \cdot \ConvGF{n}{-1}{n-2}{z}\right) 
     \times \cdots \times \\ 
     & \times 
     \left([z^{n}] z^{n} \cdot \ConvGF{n}{-1}{1}{z}\right) \\ 
S_1(n) = 
     \left([z^{n}] \ConvGF{n}{1}{1}{z}\right) 
     \phantom{_{+1}-} & \times 
     \left([z^{n}] z \cdot \ConvGF{n}{1}{1}{z}\right) 
     \times \\ 
     & \times 
     \left([z^{n}] z^2 \cdot \ConvGF{n}{1}{1}{z}\right) 
     \times \cdots \times 
     \left([z^{n}] z^{n} \cdot \ConvGF{n}{1}{1}{z}\right). 
\end{align*} 
Stated more precisely, the 
superfactorial sequence is generated by the following 
finite, rational products of the generalized convergent functions 
for any $n \geq 2$: 
\begin{align*} 
\tagonce\label{eqn_SuperFactFn_S1n_RationalHP-DiagGF_ProductIdents-stmts_v1} 
S_1(n) 
     & = 
     [x_1^{-1} x_2^{-1} \cdots x_{n-1}^{-1} x_n^{n}] 
     \Biggl( 
     \prod_{i=0}^{n-2} 
     \ConvGF{n}{-1}{n-i}{\frac{x_{n-i}}{x_{n-i-1}}} \times 
     \ConvGF{n}{-1}{1}{x_1} 
     \Biggr) \\ 
     & = 
     [x_1^{-1} x_2^{-1} \cdots x_{n-1}^{-1} x_n^{n}] 
     \Biggl( 
     \prod_{i=0}^{n-2} 
     \ConvGF{n}{1}{1}{\frac{x_{n-i}}{x_{n-i-1}}} \times 
     \ConvGF{n}{1}{1}{x_1} 
     \Biggr). 
\end{align*} 
%% 

\subsubsection{Generating generalized superfactorial product sequences} 
%% 
Let the more general superfactorial functions, $S_{\alpha,d}(n)$, 
forming the analogs products of the integer--valued 
multiple, $\alpha$--factorial function cases from 
\eqref{eqn_nAlpha_Multifact_variant_rdef} 
correspond to the expansions defined by the next equation. 
\begin{align*} 
\tagonce\label{eqn_SAlphadn_GenSuperFactFnSeqs_product_based_def_v1} 
S_{\alpha,d}(n) & \defequals 
     \prod_{j=1}^{n} (\alpha j - d)!_{(\alpha)},\ 
     n \geq 1, \alpha \in \mathbb{Z}^{+}, 0 \leq d < \alpha 
\end{align*} 
%% 
Observe that the corollary of 
Lemma \ref{lemma_GenConvFn_EnumIdents_pnAlphaRSeq_idents_combined_v1} 
cited in \eqref{eqn_AlphaFactFn_anm1_SpCase_SeqIdents-stmts_v0} 
implies that 
whenever $n \geq 1$, and for any fixed $\alpha \in \mathbb{Z}^{+}$, 
we immediately obtain the next identity 
corresponding to the so--termed 
\quotetext{ordinary} case of these 
superfactorial functions, $S_1(n) = S_{1,0}(n)$, 
in the notation for these sequences defined above. 
\begin{align*} 
S_1(n) = \alpha^{-\binom{n+1}{2}} 
     \prod_{j=1}^{n} (\alpha j)!_{(\alpha)} = 
     \alpha^{-\binom{n+1}{2}} S_{\alpha,0}(n),\ 
     \forall \alpha \in \mathbb{Z}^{+},\ n \geq 1. 
\end{align*} 
%% 
For other cases of the parameter $d > 0$, the 
generalized superfactorial function products defined by 
\eqref{eqn_SAlphadn_GenSuperFactFnSeqs_product_based_def_v1} 
are enumerated in a similar fashion to the previous 
constructions of the convergent--based generating function 
identities expanded by 
\eqref{eqn_SuperFactFn_S1n_RationalHP-DiagGF_ProductIdents-stmts_v1} 
in the following forms for $n \geq 1$, $\alpha \in \mathbb{Z}^{+}$, 
and any fixed $0 \leq d < \alpha$: 
\begin{align*} 
S_{\alpha,d}(n) 
     & = 
     [x_1^{-1} x_2^{-1} \cdots x_{n-1}^{-1} x_n^{n}] 
     \Biggl( 
     \prod_{i=0}^{n-2} 
     \ConvGF{n}{-\alpha}{\alpha(n-i)-d}{\frac{x_{n-i}}{x_{n-i-1}}} \times 
     \ConvGF{n}{-\alpha}{\alpha-d}{x_1} 
     \Biggr) \\ 
\tagonce\label{eqn_SuperFactFn_S1n_RationalHP-DiagGF_ProductIdents-stmts_v2} 
     & = 
     [x_1^{-1} x_2^{-1} \cdots x_{n-1}^{-1} x_n^{n}] 
     \Biggl( 
     \prod_{i=0}^{n-2} 
     \ConvGF{n}{\alpha}{\alpha-d}{\frac{x_{n-i}}{x_{n-i-1}}} \times 
     \ConvGF{n}{\alpha}{\alpha-d}{x_1} 
     \Biggr). 
\end{align*} 

\subsubsection{Special cases of the 
               generalized superfactorial products and their relations to the 
               Barnes G--function at rational $z$} 
%% 
The special case sequences formed by the double factorial products, 
$S_{2,1}(n)$, and the quadruple factorial products, $S_{4,2}(n)$, 
are simplified by \Mm{} to obtain the next 
closed--form expressions given by 
\begin{align*} 
S_{2,1}(n) & \defequals \prod_{j=1}^{n} (2j-1)!! = 
     \frac{A^{3/2}}{2^{1/24} e^{1/8} \pi^{1/4}} \cdot 
     \frac{2^{n(n+1)/2}}{\pi^{n/2}} \times 
     G\left(n + \frac{3}{2}\right) \\ 
S_{4,2}(n) & \defequals \prod_{j=1}^{n} (4j-2)!!!! = 
     \frac{A^{3/2}}{2^{1/24} e^{1/8} \pi^{1/4}} \cdot 
     \frac{4^{n(n+1)/2}}{\pi^{n/2}} \times 
     G\left(n + \frac{3}{2}\right), 
\end{align*} 
where $A \approx 1.2824271$ denotes \emph{Glaisher's constant} 
\citep[\S 5.17]{NISTHB}, and where the particular constant multiples 
in the previous equation correspond to the special case values, 
$\Gamma(1/2) = \sqrt{\pi}$ and 
$G(3/2) = A^{-3/2} 2^{1/24} e^{1/8} \pi^{1/4}$ 
\citep{CONTRIB-THEORY-BARNESGFN}. 
%% 
In addition, 
since the sequences defined by 
\eqref{eqn_SAlphadn_GenSuperFactFnSeqs_product_based_def_v1} 
are also expanded as the products 
\begin{align*} 
S_{\alpha,d}(n) & = 
     \prod_{j=1}^{n} 
     \undersetbrace{= \alpha^{j} \times 
     \frac{\Gamma\left(j+1-\frac{d}{\alpha}\right)}{ 
     \Gamma\left(1-\frac{d}{\alpha}\right)}}{
     \left( 
     \alpha^{j} \times \Pochhammer{1 - \frac{d}{\alpha}}{j} 
     \right) 
     } = 
     \alpha^{\binom{n+1}{2}} G(n+2) \times 
     \prod_{j=1}^{n} \binom{j-\frac{d}{\alpha}}{j}, 
\end{align*} 
further computations with \Mm{} yield the next 
few representative special cases of these generalized superfactorial 
functions when $\alpha \defequals 3, 4, 5$
\citep[\cf \S 2]{CONTRIB-THEORY-BARNESGFN}: 
\begin{align*} 
\tagtext{Special Case Products} 
S_{3,1}(n) \defequals 
     \prod_{j=1}^{n} (3j-1)!!! & = 
     3^{n(n-1)/2} \left( \frac{2 \cdot G\left(\frac{5}{3}\right)}{ 
     G\left(\frac{8}{3}\right)}\right)^{n} 
     \times \frac{G\left(n+\frac{5}{3}\right)}{G\left(\frac{5}{3}\right)} \\ 
S_{4,1}(n) \defequals  
\prod_{j=1}^{n} (4j-1)!!!! & = 
     4^{n(n-1)/2} \left( \frac{3 \cdot G\left(\frac{7}{4}\right)}{ 
     G\left(\frac{11}{4}\right)}\right)^{n} 
     \times \frac{G\left(n+\frac{7}{4}\right)}{G\left(\frac{7}{4}\right)} \\ 
S_{5,1}(n) \defequals 
\prod_{j=1}^{n} \AlphaFactorial{5j-1}{5} & = 
     5^{n(n-1)/2} \left( \frac{4 \cdot G\left(\frac{9}{5}\right)}{ 
     G\left(\frac{14}{5}\right)}\right)^{n} 
     \times \frac{G\left(n+\frac{9}{5}\right)}{G\left(\frac{9}{5}\right)} \\ 
S_{5,2}(n) \defequals 
\prod_{j=1}^{n} \AlphaFactorial{5j-2}{5} & = 
     5^{n(n-1)/2} \left( \frac{3 \cdot G\left(\frac{8}{5}\right)}{ 
     G\left(\frac{13}{5}\right)}\right)^{n} 
     \times \frac{G\left(n+\frac{8}{5}\right)}{G\left(\frac{8}{5}\right)}. 
\end{align*} 
We are then led to conjecture inductively, without proof 
given in this example, that these sequences satisfy the form of the 
next equation involving the Barnes G--function over the 
rational--valued inputs prescribed according to the 
next formula for $n \geq 1$. 
\begin{align*} 
\tagtext{Generalized Superfactorial Function Identity} 
S_{\alpha,d}(n) 
     & = 
     \frac{\alpha^{\binom{n}{2}} (\alpha - d)^{n}}{ 
     \Gamma\left(2 - \frac{d}{\alpha}\right)^{n}} \times 
     \frac{G\left(n + 2 - \frac{d}{\alpha}\right)}{ 
     G\left(2 - \frac{d}{\alpha}\right)}. 
\end{align*} 

\begin{remark}[Generating Rational--Valued Cases of the Barnes G--Function] 
%% 
The identities for the $\alpha$--factorial function products given in the 
previous examples suggest further avenues to 
enumerating other particular forms of the Barnes G--function formed by these 
generalized integer--parameter product sequence cases. 
These functions are generated by extending the constructions of the 
rational generating function methods outlined in this section 
\citep{CONTRIB-THEORY-BARNESGFN,ON-HYPGEOMFNS-PHKSYMBOL}, 
which then suggest additional identities for the Barnes G-functions, 
$G(z+2)$, over rational--valued $z > 0$ 
involving the special function zeros already defined by 
Section \ref{subSection_Intro_GenConvFn_Defs_and_Properties} and in 
Section \ref{subSection_Congruences_for_Series_ModuloIntegers_p}. 
The convergent--based generating function identities stated in the 
previous equations also suggest further applications to 
enumerating specific new identities corresponding to the 
special case constant formulas expanded in the reference 
\citep[\S 2]{CONTRIB-THEORY-BARNESGFN}. 
%% 
\RemarkQED 
\end{remark} 

\begin{remark}[Expansions of Hyperfactorial Function Products]  
%% 
The generalized superfactorial sequences defined by 
\eqref{eqn_SAlphadn_GenSuperFactFnSeqs_product_based_def_v1} in the 
previous example are also related to the \emph{hyperfactorial function}, 
$H_1(n)$, defined for $n \geq 1$ by the products \citeOEIS{A002109} 
\begin{align*} 
\tagtext{Hyperfactorial Function Products} 
H_1(n) & \defequals 
     \prod_{1 \leq j \leq n} j^j 
     %= \frac{(n!)^{n+1}}{S_1(n)} 
     \quad \seqmapsto{A002109} 
     \left(1, 4, 108, 27648, 86400000, \ldots \right), 
\end{align*} 
%% 
The exercises in the reference state additional known formulas 
establishing relations between these 
expansions of the hyperfactorial function defined above, and 
products of the binomial coefficients, including the following identities 
\citep[\S 5; Ex.\ 5.13, p.\ 527]{GKP} 
\citeOEIS{A001142}: 
\begin{align*} 
\tagtext{Binomial Coefficient Products} 
B_1(n) & \defequals 
     \prod_{k=0}^{n} \binom{n}{k} = 
     \frac{(n!)^{n+1}}{S_1(n)^{2}} = 
     \frac{H_1(n)}{S_1(n)} = 
     \frac{H_1(n)^{2}}{(n!)^{n+1}}. 
\end{align*} 
Statements of congruence properties and other relations 
connecting these sequences are also considered in the references 
\citep{GENWTHM-DBLHYPERSUPER-FACTFNS,
       CONTRIB-THEORY-BARNESGFN,ON-HYPGEOMFNS-PHKSYMBOL}. 
%% 
\RemarkQED 
\end{remark} 

\subsection{Examples: Enumerating sequences involving 
            sums of factorial--related functions, 
            sums of factorial powers, and more challenging 
            combinatorial sums involving factorial functions} 
\label{subsubSection_Apps_Example_SumsFactFn_Seqs} 

The coefficients of the convergent--based generating function 
constructions for the factorial product sequences 
given in the previous section are compared to the next several identities 
expanding the corresponding sequences of finite sums involving 
factorial functions 
\citeOEIS{A003422,A005165,A033312,A001044,A104344,A061062}
\citep[\cf \S 3; Ex.\ 3.30 p.\ 168]{ADVCOMB}\footnotemod{ 
     Refer to the 
     \href{http://mathworld.wolfram.com/FactorialSums.html}{MathWorld} 
     site for references to definitions of several other factorial--related 
     finite sums and series. 
}\footnotemod[Generalized Factorial Sum Identities]{ 
     A generalization of the second identity given in 
     \eqref{eqn_FactorialSumIdents_examples_afn_sf2n_sf3n-stmts_v1} 
     due to Gould is stated in the reference as \citep[p.\ 168]{ADVCOMB} 
     \begin{align*} 
     \sum_{k=0}^{n} \binom{x}{k}^{p} \left(\frac{k!}{x^{k+1}}\right)^{p} 
          \left( (x-k)^{p} - x^{p} \right) & = 
          \binom{x}{n+1}^{p} \left(\frac{(n+1)!}{x^{n+1}}\right)^{p} - 1. 
     \end{align*} 
}: 
\begin{align*} 
\tagtext{Left Factorials} 
L!n & \defequals \sum_{k=0}^{n-1} k! = 
     [z^n] \left( 
     \frac{z}{(1-z)} \cdot \ConvGF{n}{1}{1}{z} 
     \right) \\ 
\tagtext{Alternating Factorials} 
\af(n) & \defequals \sum_{k=1}^{n} (-1)^{n-k} \cdot k! = 
     [z^n] \left( 
     \frac{1}{(1+z)} \cdot \left(\ConvGF{n}{1}{1}{z} - 1\right) 
     \right) \\ 
\tagonce\label{eqn_FactorialSumIdents_examples_afn_sf2n_sf3n-stmts_v1} 
\Sf_2(n) & \defequals \sum_{k=1}^{n} k \cdot k! = 
       (n+1)! - 1 \\ 
     & \phantom{:} = 
     [x^0 z^n] \left( 
     \frac{1}{(1-z)} \frac{x}{(1-x)^2} 
     \ConvGF{n}{1}{1}{\frac{z}{x}} 
     \right) \\ 
\tagtext{Sums of Single Factorial Powers} 
\Sf_3(n) & \defequals \sum_{k=1}^{n} (k!)^{2} \\ 
     & \phantom{:} = 
     [x^0 z^n] \left( 
     \frac{1}{(1-z)} \times 
     \left( 
     \ConvGF{n}{1}{1}{x} \ConvGF{n}{1}{1}{\frac{z}{x}} - 1 
     \right) 
     \right) \\ 
\Sf_4(n) & \defequals \sum_{k=0}^{n} (k!)^{3} \\ 
     & \phantom{:} = 
     [x_1^0 x_2^0 z^n] \left( 
     \frac{1}{(1-z)} \times 
     \ConvGF{n}{1}{1}{\frac{z}{x_2}} \ConvGF{n}{1}{1}{\frac{x_2}{x_1}} 
     \ConvGF{n}{1}{1}{x_1} + 1 
     \right). 
\end{align*} 

\subsubsection{Generating sums of double and triple factorial powers} 
The expansion of the second to last sum, denoted $\Sf_3(n)$ in 
\eqref{eqn_FactorialSumIdents_examples_afn_sf2n_sf3n-stmts_v1}, is 
generalized to form the following variants of sums over the squares of the 
$\alpha$--factorial functions, $n!!$ and $n!!!$, 
through the generating function identities given in 
\eqref{eqn_MultFactFn_ConvSeq_def_v2} of the introduction 
\citeOEIS{A184877}: 
\begin{align*} 
\tagtext{Sums of Double Factorial Squares} 
\Sf_{3,2}(n) & \defequals \sum_{k=0}^{n} (k!!)^{2} \\ 
\notag 
     & \phantom{:} = 
     [x^0 z^n] \Biggl( 
     \frac{1}{(1-z)} \times \biggl( 
     \ConvGF{n}{2}{2}{x} \ConvGF{n}{2}{2}{\frac{z^2}{x}} \\ 
     & \phantom{\defequals [x^0 z^n] \Biggl(\frac{1}{(1-z)} \times 
                \biggl( \quad} + 
     z \cdot \ConvGF{n}{2}{3}{x} \ConvGF{n}{2}{3}{\frac{z^2}{x}} 
     \biggr) 
     \Biggr) \\ 
\notag 
     & \phantom{:} = 
     [x^0 z^n] \Biggl( 
     \frac{1}{(1-z)} \times \biggl( 
     \ConvGF{n}{2}{2}{x} \ConvGF{n}{2}{2}{\frac{z^2}{x}} \\ 
     & \phantom{\phantom{:} = [x^0 z^n] \Biggl(\frac{1}{(1-z)} \times 
                \biggl( \quad } + 
     z^{-1} \cdot \ConvGF{n}{2}{1}{x} \ConvGF{n}{2}{1}{\frac{z^2}{x}} - 1 
     \biggr) 
     \Biggr) \\ 
\tagtext{Sums of Triple Factorial Squares} 
\Sf_{3,3}(n) & \defequals \sum_{k=0}^{n} (k!!!)^{2} \\ 
\notag 
     & \phantom{:} = 
     [x^0 z^n] \Biggl( 
     \frac{1}{(1-z)} \times \biggl( 
     \ConvGF{n}{3}{3}{x} \ConvGF{n}{3}{3}{\frac{z^3}{x}} \\ 
     & \phantom{\defequals [x^0 z^n] 
       \Biggl( \frac{1}{(1-z)} \times \biggl( \quad} + 
     z^{-1} \cdot \ConvGF{n}{3}{2}{x} \ConvGF{n}{3}{2}{\frac{z^3}{x}} \\ 
     & \phantom{\defequals [x^0 z^n] 
       \Biggl( \frac{1}{(1-z)} \times \biggl( \quad} + 
     z^{-2} \cdot \ConvGF{n}{3}{1}{x} \ConvGF{n}{3}{1}{\frac{z^3}{x}} - 
     1 - \frac{1}{z} 
     \biggr) 
     \Biggr). 
\end{align*} 
The next form of the cube--factorial--power 
sequences, $\Sf_4(n)$, defined in 
\eqref{eqn_FactorialSumIdents_examples_afn_sf2n_sf3n-stmts_v1} 
corresponding to the next sums taken over powers of the 
double factorial function are similarly generated by\footnotemod{ 
     The sums over factorial function power sequences defined by the 
     functions $\Sf_{3,2}(n)$, $\Sf_{3,3}(n)$, and $\Sf_{4,2}(n)$ 
     do not appear to have corresponding entries in the 
     current \OEISRef{OEIS} database \citep{OEIS}. 
} 
\begin{align*} 
\tagtext{Sums of Double Factorial Cubes} 
\Sf_{4,2}(n) & \defequals \sum_{k=0}^{n} (k!!)^{3} \\ 
     & \phantom{:} = 
     [x_1^0 x_2^0 z^n] \Biggl( 
     \frac{1}{(1-z)} \times \biggl( 
     \ConvGF{n}{2}{2}{\frac{z^2}{x_2}} \ConvGF{n}{2}{2}{\frac{x_2}{x_1}} 
     \ConvGF{n}{2}{2}{x_1} \\ 
     & \phantom{\defequals [x_1^0 x_2^0 z^n] \Biggl( 
                \biggl( \quad } + 
     z^{-1} \cdot 
     \ConvGF{n}{2}{1}{\frac{z^2}{x_2}} \ConvGF{n}{2}{1}{\frac{x_2}{x_1}} 
     \ConvGF{n}{2}{1}{x_1} - \frac{1}{z} 
     \biggr) + 2 
     \Biggr). 
\end{align*} 

\subsubsection{Another convergent--based generating function identity} 
%% 
The second variant of the factorial sums, denote $\Sf_2(n)$ in 
\eqref{eqn_FactorialSumIdents_examples_afn_sf2n_sf3n-stmts_v1}, is 
enumerated through an alternate approach provided by the 
more interesting summation identities cited in the reference 
\citep[\S 3, p.\ 168]{ADVCOMB}. 
In particular, we have another pair of identities generating these sums 
expanded as 
\begin{align*} 
(n+1)! - 1 & = (n+1)! \times \sum_{k=0}^{n} \frac{k}{(k+1)!} \\ 
           & = [z^n x^0] \left( 
     \left(\frac{1}{x \cdot (1-x)} - \frac{e^{x}}{x}\right) \times 
     \ConvGF{n+2}{-1}{n+1}{\frac{z}{x}} 
     \right) \\ 
           & = [x^0 z^{n+1}] \left( 
     \left(\frac{1}{(1-x)} - e^{x}\right) \times 
     \ConvGF{n+1}{1}{1}{\frac{z}{x}} 
     \right). 
\end{align*} 
The convergent--based generating function identities enumerating the 
sequences stated next in 
Example \ref{example_SumsInvolving_DblFactFns-exps_examples_v1} and 
Example \ref{remark_FactSumIdents_SubfactorialSums_ConvIdents} 
below provide additional examples of the termwise formal 
Laplace--Borel--like transform provided by coefficient extractions 
involving these rational convergent functions outlined by 
Remark \ref{remark_Formal_Laplace-Borel_Transforms}. 

\subsubsection{Enumerating more challenging combinatorial sums involving 
               double factorials} 

\begin{example} 
\label{example_SumsInvolving_DblFactFns-exps_examples_v1} 
%% 
Since we know that $(2k-1)!! = [z^k] \ConvGF{n}{2}{1}{z}$ 
for all $0 \leq k < n$, the terms of the next 
modified product sequences are generated through the 
following related forms obtained from the formal series 
expansions of the convergent generating functions: 
\begin{align*} 
\frac{(k+1)}{k!} \cdot (2k-1)!! & = 
     [x^0] [z^k] \left( 
     \ConvGF{k}{-1}{2k-1}{\frac{z}{x}} \cdot (x+1) e^{x} 
     \right) \\ 
     & = 
     [x^0] [z^k] \left( 
     \ConvGF{k}{2}{1}{\frac{z}{x}} \cdot (x+1) e^{x} 
     \right). 
\end{align*} 
The convergent--based expansions of the next \quotetext{round number} identity 
generating the double factorial function given cited in the reference 
are then easily obtained from the 
previous equations in the following forms 
\citep[\S 4.3]{DBLFACTFN-COMBIDENTS-SURVEY}: 
\begin{align*} 
(2n-1)!! & = 
     \sum_{k=1}^{n} \frac{(n-1)!}{(k-1)!} \cdot k \cdot (2k-3)!! \\ 
     & = 
     (n-1)! \times [x_2^{n}] [x_1^0] \left( 
     \frac{x_2}{(1-x_2)} \times \ConvGF{n}{2}{1}{\frac{x_2}{x_1}} \times 
     (x_1+1) e^{x_1} 
     \right) \\ 
     & = 
     [x_1^0 x_2^0 x_3^{n-1}] \left( 
     \ConvGF{n}{1}{1}{\frac{x_3}{x_2}} 
     \ConvGF{n}{2}{1}{\frac{x_2}{x_1}} \times 
     \frac{(x_1+1)}{(1-x_2)} \cdot e^{x_1} 
     \right). 
\end{align*} 
%% 
Related challenges are posed in the statements of several other 
finite sum identities involving the double factorial function cited in the 
references \citep{MAA-FUN-WITH-DBLFACT,DBLFACTFN-COMBIDENTS-SURVEY}. 
%% 
\ExampleQED 
%% 
\end{example} 

\subsubsection{Other examples of convergent--based generating function 
               identities enumerating the subfactorial function} 
The first convergent--based generating function expansions approximating the 
formal ordinary power series over the 
subfactorial sequence given in 
Section \ref{subsubSection_remark_HybridDiagonalHPGFs} 
are expanded as \citeOEIS{A000166} 
\begin{align*} 
\tagtext{Subfactorial OGF Identities} 
!n & = 
     n! \times \sum_{i=0}^{n} \frac{(-1)^{i}}{i!} = 
     [x^0 z^n] \left( 
     \frac{e^{-x}}{(1-x)} \times \ConvGF{n}{1}{1}{\frac{z}{x}} 
     \right) \\ 
   & = 
     \sum_{k=0}^{n} \binom{n}{k} (-1)^{n-k} k! = 
     [z^n x^n] \left( 
     \frac{(x+z)^{n}}{(1+z)} \times \ConvGF{n}{1}{1}{x} 
     \right). 
\end{align*} 
The constructions of the convergent--based formal power series for the 
ordinary generating functions of the 
subfactorial function, $!n$, outlined in the previous section 
are extended to enumerate a few other special case 
identities for this sequence related to the sums from 
Example \ref{example_SumsInvolving_DblFactFns-exps_examples_v1}. 

\begin{example} 
\label{remark_FactSumIdents_SubfactorialSums_ConvIdents}
%% 
The following pair of 
alternate, factorial--function--like auxiliary recurrence relations 
from the references \citep[\S 5.3 -- \S 5.4]{GKP} \citep[\S 4.2]{ADVCOMB} 
exactly defining the subfactorial function when $n \geq 2$ 
correspond to the respective expansions of the 
generating functions in previous equation involving the 
first--order partial derivatives of the convergent functions, 
$\ConvGF{n}{\alpha}{R}{t}$, with respect to $t$ 
\citep[\cf \S 7.2]{GKP} \citep[\cf \S 2.2]{GFOLOGY}: 
\begin{align*} 
\tagtext{Factorial-Related Recurrence Relations} 
!n & = (n-1) \left( !(n-1) + !(n-2) \right) \\ 
     & = (n-1) \times !(n-1) + (n-2) \times !(n-2) + !(n-2) \\ 
     & = 
     [x_1^0 z^n] \left( 
     \frac{(z^2+z^3) \cdot e^{-x_1}}{x_1 \cdot (1-x_1)} \times 
     \Conv_n^{(1)}\left(1, 1; \frac{z}{x_1}\right) + 
     \frac{z^2 \cdot e^{-x_1}}{(1-x_1)} \times 
     \ConvGF{n}{1}{1}{\frac{z}{x_1}} + 1 
     \right) \\ 
!n & = 
     n \times !(n-1) + (-1)^{n} \\ 
     & = 
     (n-1) \times !(n-1) + !(n-1) + (-1)^{n} \\ 
     & = 
     [x_1^0 z^n] \left( 
     \frac{z^2 \cdot e^{-x_1}}{x_1 \cdot (1-x_1)} \times 
     \Conv_n^{(1)}\left(1, 1; \frac{z}{x_1}\right) + 
     \frac{z \cdot e^{-x_1}}{(1-x_1)} \times 
     \ConvGF{n}{1}{1}{\frac{z}{x_1}} + 
     \frac{1}{(1+z)} 
     \right). 
\end{align*} 
The next sums provide another summation--based recursive formula for the 
subfactorial function derived from the known exponential generating function, 
$\widehat{D}_{n\tiny{\?}}(z) = e^{-z} \cdot (1-z)^{-1}$, for this sequence 
\citep[\S 5.4]{GKP} \citep[\S 4.2]{ADVCOMB}. 
\begin{align*} 
\tagtext{Basic Subfactorial Recurrence} 
!n & = 
     n! - \sum_{i=1}^{n} \binom{n}{i} !(n-i) \\ 
     & = 
     n! \times \left(1 - \sum_{i=1}^{n} \frac{1}{i!} \cdot 
     \frac{!(n-i)}{(n-i)!}\right) \\ 
     & = 
     n! \times \left(1 - [x_1^0 x_2^0 x_3^n] \left( 
     (e^{x_3}-1) \ConvGF{n}{1}{1}{\frac{x_3}{x_2 x_1}} 
     \frac{e^{x_2-x_1}}{(1-x_1)} \right) 
     \right) \\ 
     & = 
     [x_x^0 x_2^0 x_3^0 z^n] \left( 
     \ConvGF{n}{1}{1}{\frac{z}{x_3}} \left( \frac{1}{(1-x_3)} - 
     \ConvGF{n}{1}{1}{\frac{x_3}{x_2 x_1}} 
     \frac{e^{x_2-x_1} \cdot (e^{x_3}-1)}{(1-x_1)}
     \right) 
     \right). 
\end{align*} 
The rational convergent--based expansions that generate the last 
equation immediately above then correspond to the effect of 
performing a termwise Laplace--Borel transformation approximating the 
complete integral transform, $\mathcal{L}(\widehat{D}_{n\tiny{\?}}(t); z)$, 
defined by 
Remark \ref{remark_Formal_Laplace-Borel_Transforms}, 
which is related to the regularized sums involving the 
incomplete gamma function given in the examples from 
Section \ref{subSection_Intro_Examples_DivergentCFracOGFs} 
of the introduction. 
%% 
\ExampleQED 
\end{example} 

\subsection{Examples: Generating sums of powers of natural numbers, 
            binomial coefficient sums, and sequences of binomials} 
\label{subsubSection_Apps_Example_SumsOfPowers_Seqs} 

\subsubsection{Generating variants of sums of powers sequences} 

As a starting point for the next generating function identities 
that provide expansions of the sums of powers 
sequences defined by 
\eqref{eqn_defs_and_ConvFnExps_of_the_GenSUmsOfPowersSeqs} 
in this section below, 
let $p \geq 2$ be fixed, and 
suppose that $m \in \mathbb{Z}^{+}$. 
The convergent--based generating function 
series over the integer powers, $m^{p}$, 
are generated by an application of the binomial theorem to 
form the next sums: 
\StartGroupingSubEquations 
\begin{align} 
\label{eqn_mPowPm1_IntPows_BinomCoeffGF_exps-stmts_v1} 
m^{p} - 1 & = 
     (p-1)! \cdot \left( 
     p \times \sum_{k=0}^{p-1} \frac{(m-1)^{p-k}}{k! (p-k)!} 
     \right) \\ 
\label{eqn_mPowPm1_IntPows_BinomCoeffGF_exps-stmts_v2} 
m^{p} - 1 & = 
     [z^{p-1}] [x^0] \Biggl( 
     \ConvGF{p}{-1}{p-1}{\frac{z}{x}} \times (m e^{mx} - e^{x}) 
     \Biggr). 
\end{align} 
\EndGroupingSubEquations 
Next, consider the generating function expansions 
enumerating the finite sums of the $p^{th}$ power sequences in 
\eqref{eqn_mPowPm1_IntPows_BinomCoeffGF_exps-stmts_v1} summed over 
$0 \leq m \leq n$ as follows \citep[\cf \S 7.6]{GKP}: 
\begin{align*} 
\tagonce\label{eqn_Spn_Conv_and_BNumGF_ident_ex-stmt_v2} 
\widetilde{B}_u(w, x) & \defequals 
     \sum_{n \geq 0} \left( 
     \sum_{m=0}^{n} (me^{mx}-e^{x}) u^{m} 
     \right) w^{n} \\ 
     & \phantom{:} = 
     \sum_{n \geq 0} \left( 
     \left(\frac{1}{1-u} + \frac{n e^{nx}}{u e^{x} - 1} - 
     \frac{e^{nx}}{(u e^{x} - 1)^2}\right) e^{x} u^{n+1} + 
     \left(\frac{u}{(u e^{x} - 1)^2} - \frac{1}{1-u}\right) e^{x} 
     \right) w^{n} \\ 
     & \phantom{:} = 
     -\frac{u^2w^2 e^{3x} - 2 uw e^{2x} + (u^2w^2 -uw + 1) e^{x}}{ 
     (1-w) (1-uw) (uw e^{x} - 1)^2} \\ 
     & \phantom{:} = 
     \frac{e^{x}}{(1-w)(1-uw)} - 
     \frac{1}{(1-w) (e^{x} uw-1)^{2}} + 
     \frac{1}{(1-w) (e^{x} uw - 1)} \\ 
\widetilde{B}_{a,b,u}(w, x) & \defequals 
     \sum_{n \geq 0} \left( 
     \sum_{m=0}^{n} ((am+b) e^{(am+b) x}-e^{x}) u^{m} 
     \right) w^{n} \\ 
     & \phantom{:} = 
     \frac{e^{x} - be^{bx} + 
     \left((b-a)e^{(a+b)x} - 2 e^{(a+1)x} + be^{bx}\right) uw + 
     \left((a-b)e^{(a+b)x} + e^{(2a+1)x} \right) u^{2} w^{2}}{ 
     (1-w) (1-uw) (uw e^{ax} - 1)^2} \\ 
     & \phantom{:} = 
     \frac{be^{bx} + (a-b) e^{(a+b) x} uw}{(1-w) (uw e^{ax}-1)^{2}} - 
     \frac{e^{x} - 2 e^{(a+1) x} uw + e^{(2a+1) x} u^2 w^2}{ 
     (1-w) (1-uw) (uw e^{ax}-1)^{2}}. 
\end{align*} 
We then obtain the next cases of the 
convergent--based generating function identities exactly 
enumerating the corresponding first variants of the 
sums of powers sequences obtained from 
\eqref{eqn_mPowPm1_IntPows_BinomCoeffGF_exps-stmts_v1} 
stated in the following forms \citep[\S 6.5, \S 7.6]{GKP}: 
\StartGroupingSubEquations 
\label{eqn_defs_and_ConvFnExps_of_the_GenSUmsOfPowersSeqs} 
\begin{align*} 
\tagtext{Sums of Powers Sequences} 
S_p(n) & \defequals 
     \sum_{0 \leq m < n} m^{p} \\ 
\tagonce\label{eqn_Spn_Conv_and_BNumGF_ident_ex-stmt_v1} 
     & \phantom{:} = 
     n + 
     [w^{n-1}] [z^{p-1} x^0] \Biggl( 
     \ConvGF{p}{-1}{p-1}{\frac{z}{x}} \widetilde{B}_1(w, x) 
     \Biggr) \\ 
     & \phantom{:} = 
     n + 
     [w^{n-1}] [x^0 z^{p-1}] \Biggl( 
     \ConvGF{p}{1}{1}{\frac{z}{x}} \widetilde{B}_1(w, x) 
     \Biggr). 
\end{align*} 
A somewhat related set of results for 
variations of more general cases of the 
power sums expanded above is expanded similarly 
for $p \geq 1$, fixed scalars $a,b \neq 0$, and 
any non--zero indeterminate $u$ according to the 
next convergent function identities given by 
\begin{align*} 
\tagtext{Generalized Sums of Powers Sequences} 
S_{p}(u, n) & \defequals 
     \sum_{0 \leq m < n} m^{p} u^{m} \\ 
     & \phantom{:} = 
     \frac{u^{n} - 1}{u-1} + 
     [w^{n-1}] [z^{p-1} x^{0}] \Biggl( 
     \ConvGF{p}{-1}{p-1}{\frac{z}{x}} 
     \widetilde{B}_u(w, x)
     \Biggr) \\ 
\tagonce\label{eqn_Spn_Conv_and_BNumGF_ident_ex-stmt_v2} 
     & \phantom{:} = 
     \frac{u^{n} - 1}{u-1} + 
     [w^{n-1}] [x^0 z^{p-1}] \Biggl( 
     \ConvGF{p}{1}{1}{\frac{z}{x}} 
     \widetilde{B}_u(w, x)
     \Biggr) \\ 
S_{p}(a, b; u, n) & \defequals 
     \sum_{0 \leq m < n} (am+b)^{p} u^{m} \\ 
     & \phantom{:} = 
     \frac{u^{n} - 1}{u-1} + 
     [w^{n-1}] [z^{p-1} x^{0}] \Biggl( 
     \ConvGF{p}{-1}{p-1}{\frac{z}{x}} 
     \widetilde{B}_{a,b,u}(w, x)
     \Biggr) \\ 
     & \phantom{:} = 
     \frac{u^{n} - 1}{u-1} + 
     [w^{n-1}] [x^0 z^{p-1}] \Biggl( 
     \ConvGF{p}{1}{1}{\frac{z}{x}} 
     \widetilde{B}_{a,b,u}(w, x)
     \Biggr). 
\end{align*} 
\EndGroupingSubEquations 


\subsubsection{Comparisons to other formulas and known generating functions 
               for special cases of the sums of powers sequences} 

\begin{remark}[Relations to the Bernoulli and Euler Polynomials] 
%% 

For fixed $n \geq 0$, integers $a,b$, and some $u \neq 0$, 
exponential generating functions for the 
generalized sums, $S_p(a,b; u, n+1)$, with respect to $p$ 
are given by the following sums \citep[\S 7.6]{GKP}: 
\begin{align*} 
%S_p(a,b; u, n+1) & = [z^p] \left( 
%     \sum_{0 \leq k \leq n} \frac{u^{k}}{1-(ak+b) z} 
%     \right) \\ 
\frac{S_p(a,b; u, n+1)}{p!} & = [z^p] \left( 
     \sum_{0 \leq k \leq n} e^{(ak+b) z} u^{k}
     \right) = 
     [z^p] \left( 
     e^{bz} \times \frac{e^{a(n+1) z} u^{n+1} -1}{u e^{az} - 1} 
     \right). 
\end{align*} 
The bivariate, two--variable exponential generating functions, 
$\widetilde{B}_{u}(w, x)$ and $\widetilde{B}_{a,b,u}(w, x)$, 
involved in enumerating the respective sequences in each of 
\eqref{eqn_Spn_Conv_and_BNumGF_ident_ex-stmt_v1} and 
\eqref{eqn_Spn_Conv_and_BNumGF_ident_ex-stmt_v2} are related to the 
generating functions for the 
\emph{Bernoulli} and \emph{Euler polynomials}, $B_n(x)$ and $E_n(x)$, 
defined in the references when the parameter $u \defmapsto \pm 1$ 
\citep[\S 24.2]{NISTHB} \citep[\S 4.2.2, \S 4.2.3]{UC}. 
%% 
For $u \defequals \pm 1$, the sums defined by the left--hand--sides of the 
previous two equations also correspond to special cases of the following 
known identities involving these polynomial sequences 
\citep[\S 24.4(iii)]{NISTHB}: 
\begin{align*} 
\tagtext{Sums of Powers Formulas} 
\sum_{m=0}^{n} (am+b)^{p} & = 
     \frac{a^{p}}{p+1} \left( 
     B_{p+1}\left(n+1+\frac{b}{a}\right) - 
     B_{p+1}\left(\frac{b}{a}\right) 
     \right) \\ 
\sum_{m=0}^{n} (-1)^{m} (am+b)^{p} & = 
     \frac{a^{p}}{2} \left( 
     (-1)^{n} \cdot E_{p}\left(n+1+\frac{b}{a}\right) + 
     E_{p}\left(\frac{b}{a}\right) 
     \right). 
\end{align*} 
The results in the previous equations are also compared to the 
forms of other well--known sequence generating functions 
involving the \emph{Bernoulli numbers}, $B_n$, the 
\emph{first--order Eulerian numbers}, $\gkpEI{n}{m}$, and the 
\emph{Stirling numbers of the second kind}, $\gkpSII{n}{k}$, 
in the next few cases of the established identities for these 
sequences expanded in 
Remark \ref{remark_SumsOfPowers_CompsToOtherKnownSeqGFs} 
\citep[\cf \S 6]{GKP} \citeOEIS{A027641,A027642,A008292,A008277}. 
%% 
\RemarkQED 
\end{remark} 

\begin{remark}[Comparisons to Other Formulas and Special Generating Functions] 
\label{remark_SumsOfPowers_CompsToOtherKnownSeqGFs}
%% 
The sequences enumerated by 
\eqref{eqn_Spn_Conv_and_BNumGF_ident_ex-stmt_v1} are first compared 
to the following known expansions that exactly generate these 
finite sums over $n \geq 0$ and $p \geq 1$ 
\citep[\S 24.4(iii), \S 24.2]{NISTHB} \citep[\S 6.5, \S 7.4]{GKP}
\footnotemod[Two--Variable EGFs for the First-Order Eulerian Numbers]{ 
     Two bivariate \quotetext{super} generating function for the 
     \emph{first--order Eulerian numbers}, $\gkpEI{n}{m}$, 
     are given by the following equations 
     where $\gkpEI{n}{m} = \gkpEI{n}{n-1-m}$ for all 
     $n \geq 1$ and $0 \leq m < n$ by the row--wise 
     symmetry in the triangle 
     \citep[\S 7.4, \S 6.2]{GKP} \citep[\S 26.14(ii)]{NISTHB}: 
     \begin{align*} 
     \tagtext{First-Order Eulerian Number EGFs} 
     \sum_{m,n \geq 0} \gkpEI{n}{m} \frac{w^{m} z^{n}}{n!} & = 
          \frac{1-w}{e^{(w-1) z} - w} \\ 
     \sum_{m,n \geq 0} \gkpEI{m+n+1}{m} \frac{w^{m} z^{n}}{(m+n+1)!} & = 
          \frac{e^{w} - e^{z}}{w e^{z} - z e^{w}}. 
     \end{align*} 
}: 
\begin{align*} 
\tagtext{Relations to the Bernoulli Numbers} 
S_p(n+1) & = 
     \frac{B_{p+1}(n+1) - B_{p+1}(0)}{p+1} \\ 
     & = 
     \sum_{s=0}^{p} \binom{p+1}{s} \frac{B_s \cdot (n+1)^{p+1-s}}{(p+1)} \\ 
\tagtext{Expansions by the Stirling Numbers} 
S_p(n+1) & = 
     \sum_{j=0}^{p} \gkpSII{p}{j} \frac{\FFactII{(n+1)}{j+1}}{(j+1)} \\ 
     & = 
     \sum_{0 \leq j,k \leq p+1} \gkpSII{p}{j} \gkpSI{j+1}{k} 
     \frac{(-1)^{j+1-k} (n+1)^{k}}{j+1} \\ 
\tagtext{Relations to Special Generating Functions} 
S_p(n+1) & = 
     [z^{n}] \left( 
     \sum_{j=0}^{p} \gkpSII{p}{j} \frac{z^{j} \cdot j!}{(1-z)^{j+2}} 
     \right) \\ 
     & = 
     [z^{n}] \left( 
     \sum_{i \geq 0} \gkpEI{p}{i} \frac{z^{i+1}}{(1-z)^{p+2}} 
     \right) \\ 
     & = 
     p! \cdot [w^{n} z^{p}] \left( 
     \frac{w \cdot e^{z}}{(1-w) (1 - w e^{z})} 
     \right). 
\end{align*} 
Similarly, the generalized forms of the sums generated by 
\eqref{eqn_Spn_Conv_and_BNumGF_ident_ex-stmt_v2} are related to the 
more well--known combinatorial sequence identities expanded as follows 
\citep[\S 26.8]{NISTHB} \citep[\S 7.4]{GKP}: 
\begin{align*} 
S_p(u, n+1) & = 
     \sum_{j=0}^{p} \gkpSII{p}{j} u^{j} \times 
     \frac{\partial^{(j)}}{{\partial u}^{(j)}} 
     \Biggl( 
     \frac{1}{1-u} - \frac{u^{n+1}}{1-u} 
     \Biggr) \\ 
   & = 
     [w^{n}] \left( 
     \sum_{j=0}^{p} \gkpSII{p}{j} \frac{(uw)^{j} \cdot j!}{(1-w) (1-uw)^{j+1}} 
     \right) \\ 
   & = 
     p! \cdot [w^{n} z^{p}] \left( 
     \frac{uw \cdot e^{z}}{(1-w) (1 - uw e^{z})} 
     \right). 
\end{align*} 
As in the examples of termwise applications of the formal 
Laplace--Borel transforms noted above, 
the role of the parameter $p$ corresponding to the forms of the 
special sequence triangles in the identities given above is 
phrased through the implicit dependence of the convergent functions on the 
fixed $p \geq 1$ in each of 
\eqref{eqn_mPowPm1_IntPows_BinomCoeffGF_exps-stmts_v2}, 
\eqref{eqn_Spn_Conv_and_BNumGF_ident_ex-stmt_v1}, and 
\eqref{eqn_Spn_Conv_and_BNumGF_ident_ex-stmt_v2}. 
%% 
\RemarkQED 
\end{remark} 

\subsubsection{Semi--rational generating function constructions 
               enumerating sequences of binomials} 
%% 
A second motivating example highlighting the procedure outlined 
in the examples above 
expands the binomial power sequences, $2^{p} - 1$ for $p \geq 1$, through an 
extension of the first result given in 
\eqref{eqn_mPowPm1_IntPows_BinomCoeffGF_exps-stmts_v1} when $m \defequals 2$. 
The finite sums for the integer powers provided by the binomial theorem in 
these cases correspond to removing, or selectively peeling off, the 
$r$ uppermost--indexed terms from the 
first sum for subsequent choices of the $p \geq r \geq 1$ in the 
following forms \citep[\cf \S 2.2, \S 2.4]{PRIMEREC}: 
\begin{align*} 
m^{p} - 1 & = \sum_{i=0}^{r} \binom{p}{p+1-i} (m-1)^{i} \\ 
     & \phantom{=\ } + 
     (p-r-1)! \cdot \left( 
     p(p-1) \cdots (p-r) \times \sum_{k=0}^{p-r-1} 
     \frac{(m-1)^{k+1}}{(k+1)! (p-1-k)!} 
     \right). 
\end{align*} 
The generating function identities phrased in terms of 
\eqref{eqn_mPowPm1_IntPows_BinomCoeffGF_exps-stmts_v1} 
in the previous examples 
are then modified slightly according to this equation for the 
next few special cases of $r \geq 1$. 
%% 
For example, these sums are employed to obtain the next analogs 
forms of the convergent--based generating function 
expansions generalizing the result in 
\eqref{eqn_mPowPm1_IntPows_BinomCoeffGF_exps-stmts_v2} above 
\citeOEIS{A000225}. 
\begin{align*} 
2^{p} - 1 = 
     [z^{p-2}] [x^0] \Biggl( & 
     \frac{1}{(1-z)} + 
     \left(4 e^{2x} - 2 e^{x}\right) \times 
     \ConvGF{p}{-1}{p-2}{\frac{z}{x}} 
     \Biggr),\ p \geq 2 \\ 
     = 
     [z^{p-3}] [x^0] \Biggl( & 
     \frac{4-3z}{(1-z)^2} + 
     \left(8 e^{2x} - e^{x} \cdot (x+5) \right) \times 
     \ConvGF{p}{-1}{p-3}{\frac{z}{x}} 
     \Biggr),\ p \geq 3 \\ 
     = 
     [z^{p-4}] [x^0] \Biggl( & 
     \frac{11 - 17 z + 7 z^2}{(1-z)^3} \\ 
     & + 
     \left(16 e^{2x} - \frac{e^{x}}{2} \cdot (x^2 + 10x + 24) \right) \times 
     \ConvGF{p}{-1}{p-4}{\frac{z}{x}} 
     \Biggr),\ p \geq 4 \\ 
     = [z^{p-5}] [x^0] \Biggl( & 
     \frac{26 - 62z + 52 z^2 - 15 z^3}{(1-z)^4} \\ 
     & + 
     \frac{x^5 e^{x}}{6} (192 e^x - (x^3 + 18 x^2 + 96 x + 162)) \times 
     \ConvGF{p}{-1}{p-5}{\frac{z}{x}} 
     \Biggr),\ p \geq 5. %\\ 
     %= 
     %[z^{p-6}] [x^0] \Biggl( & 
     %\frac{57 - 186 z + 238 z^2 - 139 z^3 + 31 z^4}{(1-z)^5} + 
     %64 x^6 e^{2x} \times \ConvGF{p}{-1}{p-6}{\frac{z}{x}} \\ 
     %& - 
     %\frac{x^6 e^{x}}{24} (x^4 + 28 x^3 + 264 x^2 + 1008 x + 1392) 
     %\times \ConvGF{p}{-1}{p-6}{\frac{z}{x}} 
     %\Biggr),\ p \geq 6. 
\end{align*} 
%% 
The special cases of these generating functions for the 
$p^{th}$ powers defined above are also expanded in the next more general 
forms of these convergent function identities for $p > m \geq 1$. 
\begin{align*} 
2^p - 1 
       & = [z^{p-m-1} x^0] \left( 
       \frac{\widetilde{\ell}_{m,2}(z)}{(1-z)^{m}} + 
       \left(2^{m+1} \cdot e^{2x} - 
       \frac{e^{x}}{(m-1)!} \cdot \widetilde{p}_{m,2}(x) \right) \times 
       \ConvGF{p}{-1}{p-m-1}{\frac{z}{x}} 
       \right) \\ 
       & = [x^0 z^{p-m-1}] \left( 
       \frac{\widetilde{\ell}_{m,2}(z)}{(1-z)^{m}} + 
       \left(2^{m+1} \cdot e^{2x} - 
       \frac{e^{x}}{(m-1)!} \cdot \widetilde{p}_{m,2}(x) \right) \times 
       \ConvGF{p}{1}{1}{\frac{z}{x}} 
       \right),\ 
       p > m 
\end{align*} 
The listings provided in 
\tableref{table_ConvGF_Examples_for_PthPowerSeqs} 
cite the particular special cases of the polynomials, 
$\ell_{m,2}(z)$ and $p_{m,2}(x)$, that provide the generalizations of the 
first cases expanded in the previous equations. 
%% 
The constructions of these new identities, 
including the variations for the sequences 
formed by the binomial coefficient sums for the powers, 
$2^{p} - 2$, are motivated in the context of divisibility modulo $p$ by the 
reference \citep[\S 8]{HARDYWRIGHTNUMT} \citeOEIS{A000918}. 

Further cases of the more general $p^{th}$ power sequences 
of the form $(s+1)^{p} - 1$ for any fixed $s > 0$ are enumerated 
similarly through the next formulas. 
\begin{align*} 
(s+1)^p - 1 = 
       [z^{p-m-1} x^0] \Biggl( 
       \frac{s^2 \ell_{m,s+1}(z)}{(1-sz)^{m}} + & 
       \left(-e^{x} + (s+1)^{m+1} \cdot e^{(s+1) x} - 
       \frac{s^2 e^{sx}}{(m-1)!} \cdot p_{m,s+1}(sx) \right) \times \\ 
       & \phantom{\Biggl( } \times 
       \ConvGF{p}{-1}{p-m-1}{\frac{z}{x}} 
       \Biggr),\ 
       p > m \geq 1 \\ 
       = 
       [x^0 z^{p-m-1}] \Biggl( 
       \frac{s^2 \ell_{m,s+1}(z)}{(1-sz)^{m}} + & 
       \left(-e^{x} + (s+1)^{m+1} \cdot e^{(s+1) x} - 
       \frac{s^2 e^{sx}}{(m-1)!} \cdot p_{m,s+1}(sx) \right) \times \\ 
       & \phantom{\Biggl( } \times 
       \ConvGF{p}{1}{1}{\frac{z}{x}} 
       \Biggr),\ 
       p > m \geq 1 
\end{align*} 
The second set of listings provided in 
\tableref{table_ConvGF_Examples_for_PthPowerSeqs} 
expand several additional special cases corresponding to the 
polynomial sequences, $\ell_{m,s+1}(z)$ and $p_{m,s+1}(x)$, 
required to generate the more general cases of these 
particular $p^{th}$ power sequences when $p > m \geq 1$ 
\citeOEIS{A000225,A024023,A024036,A024049}. 
Related expansions of the \emph{sequences of binomials} of the forms 
$a^{n} \pm 1$ and $a^{n} \pm b^{n}$ are considered in the references 
(see Section \ref{subsubSection_remark_OtherApps_of_WThm_and_NewCongProps_to_PrimeSubseqs}) 
\citep[\cf \S 2.2, \S 2.4]{PRIMEREC}. 

\section{Applications of new identities resulting from expansions by 
         finite difference equations} 
\label{subSection_FiniteDiffEqns_for_the_GenFactFns} 

The rationality of the convergent functions, $\ConvGF{h}{\alpha}{R}{z}$, in 
$z$ for all $h$ suggests new forms of $h$--order finite difference equations 
with respect to $n$ satisfied by the product sequences, $p_n(\alpha, R)$, 
when $\alpha$ and $R$ correspond to fixed parameters independent of the 
sequence indices $n$. 
In particular, the 
rationality of the $h^{th}$ convergent functions immediately 
implies the next results stated in 
Proposition \ref{prop_ExactFormulas_CongruencesModh_from_FiniteDiffEqns} 
below, which provides both forms of the congruence properties stated below in 
\eqref{eqn_pnAlphaR_seqs_finite_sum_reps_modulop-stmts_v1} modulo 
integers $p \geq 2$, and for the exact expansions of the 
generalized products, $p_n(\alpha, R)$, stated by the results in 
\eqref{eqn_pnAlphaR_seqs_exact_finite_sum_reps-stmts_v2} 
\citep[\S 2.3]{GFLECT} \citep[\S 7.2]{GKP}. 

%% 
When the initially indeterminate parameter, $R$, 
assumes an implicit dependence on the sequence index, $n$, the 
results phrased by the previous equations, 
somewhat counter-intuitively, do not immediately imply difference equations 
satisfied between the generalized product sequences, 
either exactly, or modulo the prescribed choices of $p \geq 2$. 
The new formulas connecting the generalized product sequences, 
$\pn{n}{\alpha}{\beta n + \gamma}$, resulting from 
\eqref{eqn_pnAlphaR_seqs_finite_sum_reps_modulop-stmts_v1} and 
\eqref{eqn_pnAlphaR_seqs_exact_finite_sum_reps-stmts_v2} 
in these cases are, however, reminiscent of the relations satisfied 
between the generalized Stirling polynomial and 
convolution polynomial sequences expanded in the references 
\citep{CVLPOLYS,MULTIFACTJIS} \citep[\cf \S 6.2]{GKP} 
(see Section \ref{subsubSection_Examples_BinomCoeff_CongCvlIdents}). 

\subsection{Exact formulas and finite sum representations for the 
            generalized product sequences modulo $h$} 
\label{subSection_FiniteDiffEqns_for_the_GenFactFns-ExactFormulas_Stmts} 

\begin{prop}[Exact Expansions and Convolution Formulas Modulo $h$] 
\label{prop_ExactFormulas_CongruencesModh_from_FiniteDiffEqns} 
%% 
Given any fixed integer $\alpha \neq 0$, prescribed $h \geq 2$, 
natural numbers $n \geq 0$, and some 
integer--valued $0 \leq t \leq h$, the 
following finite sum identities for the 
generalized product sequences, $\pn{n}{\alpha}{R}$, modulo $h$ 
are stated in terms of the coefficients, 
$C_{h,k}(\alpha, R) \defequals [z^{k}] \ConvFP{h}{\alpha}{R}{z}$, 
from \eqref{eqn_Vandermonde-like_PHSymb_exps_of_PhzCfs} and 
\eqref{eqn_Chn_formula_stmts} of 
Section \ref{subsubSection_Properties_Of_ConvFn_Phz}: 
\begin{subequations} 
\label{eqn_pnAlphaR_seqs_finite_sum_reps_modulop} 
\begin{align} 
\label{eqn_pnAlphaR_seqs_finite_sum_reps_modulop-stmts_v0} 
p_{n}(\alpha, R) & \equiv 
     \sum\limits_{k=1}^{n} 
     \binom{h}{k} 
     \mathsmaller{ 
     (-1)^{k+1} 
     p_{k}(-\alpha, R + (h-1) \alpha) p_{n-k}(\alpha, R) 
     } && \pmod{h} \\ 
\notag 
     & \phantom{\equiv\sum\quad} + 
     C_{h,n}(\alpha, R) \Iverson{h > n \geq 1} + \Iverson{n = 0} 
     && \\ 
\label{eqn_pnAlphaR_seqs_finite_sum_reps_modulop-stmts_v1} 
p_{n}(\alpha, R) & \equiv 
     \sum\limits_{k=0}^{n} \binom{h}{k} (-\alpha)^{k} 
     \pn{k}{-\alpha}{R + (h-1) \alpha} \pn{n-k}{\alpha}{R} 
     && \pmod{h} \\ 
\notag 
     & = %\equiv 
     \sum\limits_{k=0}^{n} \binom{h}{k} \alpha^{n+k} 
     \Pochhammer{1-h-R / \alpha}{k} \Pochhammer{R / \alpha}{n-k} 
     && \\ 
\label{eqn_pnAlphaR_seqs_finite_sum_reps_modulop-stmts_v2} 
\pn{n}{\alpha}{R} & \equiv 
     \sum\limits_{k=0}^{n} \binom{h}{k} \alpha^{n+(t+1)k} 
     \Pochhammer{1-h-R / \alpha}{k} \Pochhammer{R / \alpha}{n-k} 
     && \pmod{h \alpha^{t}}. 
\end{align} 
\end{subequations} 
Additionally, for any fixed $r \geq 1$, the following finite sums 
provide exact formulas for the expansions of the 
generalized product sequences, 
$\pn{n}{\alpha}{R}$, from \eqref{eqn_GenFact_product_form}: 
\begin{align*} 
\tagonce\label{eqn_pnAlphaR_seqs_exact_finite_sum_reps-stmts_v2} 
p_{n}(\alpha, R) 
     & = 
     \sum_{k=0}^{n-1} 
     \mathsmaller{ 
     \binom{n}{k+1} (-1)^{k} 
     p_{k+1}(-\alpha, R + (n-1) \alpha) p_{n-1-k}(\alpha, R) + 
     \Iverson{n = 0} 
     } \\ 
     & = 
     \sum_{k=0}^{n-1} 
     \mathsmaller{ 
     \binom{n+r}{k+1} (-1)^{k} 
     p_{k+1}(-\alpha, R + (n-1+r) \alpha) 
     p_{n-1-k}(\alpha, R) 
     } \\ 
     & \phantom{= \sum \quad} + 
     C_{n+r,n}(\alpha, R). 
\end{align*} 
%% 
\end{prop} 
%%%% 
\begin{proof}[Proof of \eqref{eqn_pnAlphaR_seqs_finite_sum_reps_modulop-stmts_v0} and \eqref{eqn_pnAlphaR_seqs_exact_finite_sum_reps-stmts_v2}] 
%% 
If we let the shorthand, 
$\pn{n,h}{\alpha}{R} \defequals [z^n] \ConvGF{h}{\alpha}{R}{z}$, 
denote the series coefficients of the $h^{th}$ convergent function, 
we notice that these terms satisfy the following congruence properties: 
\begin{align*} 
\pn{n,h}{\alpha}{R} & = 
     \pn{n}{\alpha}{R},\ \phantom{\pmod{h}} \forall h \geq n \\ 
\pn{n,h}{\alpha}{R} & \equiv 
     \pn{n}{\alpha}{R} \pmod{h},\ \forall n \geq 0. 
\end{align*} 
For fixed $\alpha \neq 0$, integers $h \geq 2$, and 
natural numbers $n, n-s \geq 0$, 
we employ the identities given in 
\eqref{eqn_CoeffsOfzk_FQhaRz_restmts_by_GenProductSeqs-v1} 
as properties of the generalized convergent denominator functions, 
$\FQ_h(\alpha, R; z)$, already noted in 
Section \ref{subsubSection_Properties_Of_ConvFn_Qhz} 
to expand the series in the following equation: 
\begin{align*} 
\undersetbrace{
     \ConvGF{h}{\alpha}{R}{z}
}{
     \mathsmaller{ 
     \left( 
     \sum\limits_{n \geq 0} \pn{n,h}{\alpha}{R} z^n 
     \right) 
     } 
} \times 
\undersetbrace{ 
     \ConvFQ{h}{\alpha}{R}{z} 
}{ 
     \mathsmaller{ 
     \left(
     1 - \sum\limits_{i=1}^{h} 
     \binom{h}{i} (-1)^{i+1} \pn{i}{-\alpha}{R+(h-1) \alpha} z^{i} 
     \right) 
     } 
} & = %\ConvFP{h}{\alpha}{R}{z}
\undersetbrace{\ConvFP{h}{\alpha}{R}{z} 
}{ 
     \sum_{n=0}^{h-1} C_{h,n}(\alpha, R) z^{n}. 
} 
\end{align*} 
Since the auxiliary terms, $C_{h,n}(\alpha, R)$, are zero--valued 
whenever $n \geq h$, the series in the previous equation gives a 
$h$--order finite difference equation for the approximate products 
corresponding to the convergent function coefficients, 
$\pn{n,h}{\alpha}{R} \equiv \pn{n}{\alpha}{R} \pmod{h}$, 
of the following forms: 
\begin{align*} 
p_{n}(\alpha, R) & \equiv 
   \begin{rcases*} 
     \begin{cases} 
     \mathsmaller{
     \sum\limits_{k=1}^{h} \binom{h}{k} (-1)^{k+1} 
     p_{k}(-\alpha, R + (h-1) \alpha) 
     \left[p_{n-k}(\alpha, R) \pmod{h}\right] 
     }, & 
     \text{if $1 < h \leq n$;} \\ 
     \mathsmaller{ 
     \sum\limits_{k=1}^{n} \binom{h}{k} (-1)^{k+1} 
     p_{k}(-\alpha, R + (h-1) \alpha) 
     \pn{n-k}{\alpha}{R} + C_{h,n}(\alpha, R)
     }, & 
     \text{if $0 \leq n < h$.} 
     \end{cases} 
   \end{rcases*} 
   \pmod{h}. 
\end{align*} 
The previous equations immediately imply the first congruence stated in 
\eqref{eqn_pnAlphaR_seqs_finite_sum_reps_modulop-stmts_v0}. 
Additionally, 
since $\pn{n}{\alpha}{R} = [z^n] \ConvGF{n+n_0}{\alpha}{R}{z}$ 
for all $n+n_0 \geq n$, we similarly obtain proofs of the results for the 
pair of exact formulas 
expanded in \eqref{eqn_pnAlphaR_seqs_exact_finite_sum_reps-stmts_v2}. 
%% 
\end{proof} 

\subsubsection{Examples: Binomial--coefficient--related congruences and 
               comparisons to known convolution identities for the 
               Pochhammer symbol and Gould polynomial sequences} 
\label{subsubSection_Examples_BinomCoeff_CongCvlIdents} 

The product sequence forms expanded in terms of the 
Pochhammer symbol, 
$\pn{n}{\alpha}{R} = \alpha^{n} \Pochhammer{R / \alpha}{n}$, and the 
Pochhammer $k$--symbol, $\pn{n}{\alpha}{x} = \Pochhammer{x}{n,\alpha}$, 
yield comparisons between various known generalized forms of 
Vandermonde's convolution identity, stated as in 
Section \ref{subsubSection_Properties_Of_ConvFn_Phz} 
above, with the new formulas stated in the previous equations 
\footnotemod[Convolution Polynomial Sequence Generating Functions Enumerating the Pochhammer $k$--Symbol]{ 
     \label{footnote_PHkSymb_CvlPolyRef} 
     For $k \neq 0$, the polynomials, 
     $f_n(x) \defequals \Pochhammer{x}{n,k}$, 
     also form a convolution family with 
     corresponding exponential generating function given by 
     $F(z)^{x} = (1-kz)^{-x/k}$ \citep{CVLPOLYS}. 
}. 
%% 
The particular identity noted in 
\eqref{eqn_pnAlphaBetanpGamma_GouldPolyExp_Ident-stmt_v1} of the 
introduction relating the generalized product sequences, 
$\pn{n-s}{-\alpha}{\beta n + \gamma}$, 
to the Gould polynomials, $G_n(x; a, b)$, 
which satisfy another convolution formula expansion given by the finite sums 
\citep[\S 4.1.4]{UC} 
\begin{align*} 
G_{n+1}(x; a, -b) & = 
     -\frac{x}{b} \times \sum_{k=0}^{n} 
     \binom{\frac{ak+a-b}{b}}{k} \FFactII{n}{k} \times G_{n-k}(x; a, -b), 
\end{align*} 
is combined with the congruence properties in 
\eqref{eqn_pnAlphaR_seqs_finite_sum_reps_modulop-stmts_v1} 
to provide the next binomial--coefficient--related expansions of 
this Sheffer sequence modulo the prescribed integers 
$p$ and $p \alpha^{t}$. 

\begin{example}[Binomial--Coefficient--Related Congruences] 
%% 
For integers $p \geq 2$, 
some $0 \leq t \leq p$, and index offsets $n, n-s \geq 1$, the 
following congruences result from the convolution identities stated in 
Proposition \ref{prop_ExactFormulas_CongruencesModh_from_FiniteDiffEqns}
above: 
\begin{align*} 
\mathsmaller{ \binom{\frac{\beta n+\gamma}{\alpha}}{n-s} (-\alpha)^{n-s} (n-s)! } 
     & \equiv 
     \mathsmaller{ 
     \sum\limits_{k=0}^{n-s} 
     \binom{p}{k} 
     \binom{-\frac{\beta n+\gamma}{\alpha}+p-1}{k} 
     \binom{\frac{\beta n+\gamma}{\alpha}}{n-s-k} \times 
     (-\alpha)^{n-s+(t+1) k} 
     } \times && \pmod{p \alpha^{t}} \\ 
     & \phantom{\equiv \sum \qquad} \times 
     \mathsmaller{ 
     k! (n-s-k)! 
     } && \\ 
     & \equiv 
     \mathsmaller{ 
     \sum\limits_{k=0}^{n-s} 
     \binom{p}{k} 
     \binom{\frac{\beta n+\gamma}{\alpha}-p+1}{k} 
     \binom{\frac{\beta n+\gamma}{\alpha}}{n-s-k} \times 
     (-1)^{k} (-\alpha)^{n-s+(t+1) k} 
     } \times && \pmod{p \alpha^{t}} \\ 
     & \phantom{\equiv \sum \qquad} \times 
     \mathsmaller{ 
     k! (n-s-k)! 
     } && \\ 
     & \equiv 
     \mathsmaller{ 
     \sum\limits_{k=0}^{n-s} 
     \binom{p}{k} 
     \binom{\frac{\beta n+\gamma}{\alpha}+p-1}{k} 
     \binom{\frac{\beta n+\gamma}{\alpha}}{n-s-k} \times 
     (-1)^{tk} (-\alpha)^{n-s+(t+1) k} 
     } \times && \pmod{p \alpha^{t}} \\ 
     & \phantom{\equiv \sum \qquad} \times 
     \mathsmaller{ 
     k! (n-s-k)! 
     } && \\ 
\mathsmaller{ 
\binom{\frac{\beta n+\gamma}{\alpha}+n-s-1}{n-s} \alpha^{n-s} (n-s)! 
} 
     & \equiv 
     \mathsmaller{ 
     \sum\limits_{k=0}^{n-s} 
     \binom{p}{k} 
     \binom{\frac{\beta n+\gamma}{\alpha}+p-1}{k} 
     \binom{\frac{\beta n+\gamma}{\alpha}+n-s-k-1}{n-s-k} \times 
     \alpha^{n-s+(t+1) k} 
     } \times && \pmod{p \alpha^{t}} \\ 
     & \phantom{\equiv \sum \qquad} \times 
     \mathsmaller{ 
     k! (n-s-k)! 
     }. && 
\end{align*} 
%% 
The previous equation is also compared to the known generalization of 
Vandermonde's identity for the coefficients of the 
\emph{generalized binomial series}, 
$\mathcal{B}_t(z) \defequals \sum_{k \geq 0} \FFactII{(tk)}{k-1} z^k / k!$, 
defining exponential generating functions for the particular 
binomial coefficient sequence variants of the forms 
\begin{align*} 
     \binom{tk+r}{k} \frac{r}{tk+r} & = 
     [z^k] \mathcal{B}_t(z)^{r} = 
     r(r+tk-1) \cdots (r+tk-(k-1)) / k! \\ 
\binom{tk+r+s}{k} & = 
     [z^k] \left( 
     \frac{\mathcal{B}_t(z)^{r+s+1}}{t+(1-t) \mathcal{B}_t(z)} 
     \right), 
     \text{ any $r,s,t \in \mathbb{R}$ \ and \ $k \in \mathbb{N}$}, 
\end{align*} 
expanded by the convolution formulas given in the 
references \citep[\S 5.4; Table 169, Table 202]{GKP} \citep{CVLPOLYS}. 
%% 
\ExampleQED 
\end{example} 

\subsubsection{Combinatorial identities for the 
               double factorial function and 
               finite sums involving the $\alpha$--factorial functions} 
\label{ssS_example_GenDblFactFnSumIdents_FiniteSumsInvolving_AlphaFactFns} 

The double factorial function, $(2n-1)!!$, satisfies a number of 
known expansions through the finite sum identities summarized as in 
\citep{MAA-FUN-WITH-DBLFACT,DBLFACTFN-COMBIDENTS-SURVEY}. 
For example, when $n \geq 1$, the 
double factorial function is generated by the 
expansion of finite sums of the form 
\citep[\S 4.1]{DBLFACTFN-COMBIDENTS-SURVEY} 
\begin{align} 
\label{eqn_2nm1_DblFactFn_round_number_idents-stmt_v1}
(2n-1)!! & = 
     \sum_{k=0}^{n-1} \binom{n}{k+1} (2k-1)!! (2n-2k-3)!!. 
\end{align} 
The particular combinatorial identity for the double factorial 
function expanded in the form of equation 
\eqref{eqn_2nm1_DblFactFn_round_number_idents-stmt_v1} above 
is remarkably similar to the statement of the first sum in 
\eqref{eqn_pnAlphaR_seqs_exact_finite_sum_reps-stmts_v2} 
satisfied by the more general product function cases, 
$\pn{n}{\alpha_0}{R_0}$, 
generating the $\alpha$--factorial functions, 
$\AlphaFactorial{\alpha n-1}{\alpha}$, when 
$(n, \alpha_0, R_0) \defmapsto (n, \alpha, \alpha-1), (n, -\alpha, \alpha n-1)$. 

\begin{example}[Combinatorial Identities and Finite Sums Involving the $\alpha$--Factorial Functions] 
\label{example_GenDblFactFnSumIdents_FiniteSumsInvolving_AlphaFactFns} 
%% 
If we assume that $\alpha \geq 2$ is integer--valued, and proceed to 
expand these cases of the $\alpha$--factorial functions 
according to the expansions from 
\eqref{eqn_AlphaFactFn_anm1_SpCase_SeqIdents-stmts_v1} and 
\eqref{eqn_pnAlphaR_seqs_exact_finite_sum_reps-stmts_v2} above, 
we see readily that \citep[\cf \S 5.5]{GKP} 
\footnotemod[Simplifications of $\alpha$--Factorial Function Identities Involving the Pochhammer Symbol]{ 
     \label{footnote_PHSymbol_AlphaFactFn_SimplIdents} 
     We note the simplification 
     $\Pochhammer{\frac{1}{\alpha}}{-(k+1)} = 
      \frac{(-\alpha)^{k+1}}{\AlphaFactorial{\alpha(k+1)-1}{\alpha}}$ 
     where the expansions of the $\alpha$--factorial functions, 
     $\AlphaFactorial{\alpha n-1}{\alpha}$, by the Pochhammer symbol 
     correspond to the results given in 
     Lemma \ref{lemma_GenConvFn_EnumIdents_pnAlphaRSeq_idents_combined_v1} 
     and in the corollaries from 
     \eqref{eqn_AlphaFactFn_anm1_SpCase_SeqIdents-stmts_v1} of 
     Section \ref{subSection_NewCongruence_Relations_Modulo_Integer_Bases}. 
     The Pochhammer symbol identities cited in the 
     reference \citep{WOLFRAMFNSSITE-INTRO-FACTBINOMS} 
     provide other related simplifications of the terms in these sums. 
} 
%% : See \"finite-diff-eqns-for-pnaR-seqs.*\": 
\begin{align*} 
(\alpha n - 1)!_{(\alpha)} = 
     \sum_{k=0}^{n-1} \binom{n-1}{k+1} (-1)^{k} 
     & \times 
     \Pochhammer{\frac{1}{\alpha}}{-(k+1)} 
     \Pochhammer{\frac{1}{\alpha}-n}{k+1} \\ 
     & \times 
     (\alpha (k+1) - 1)!_{(\alpha)} 
     (\alpha (n-k-1) - 1)!_{(\alpha)} \times \\ 
(\alpha n - 1)!_{(\alpha)} = 
     \sum_{k=0}^{n-1} \binom{n-1}{k+1} (-1)^{k} & \times 
     \binom{\frac{1}{\alpha} + k - n}{k+1} 
     \binom{\frac{1}{\alpha} - 1}{k+1}^{-1} \\ 
     & \times 
     (\alpha (k+1) - 1)!_{(\alpha)} 
     (\alpha (n-k-1) - 1)!_{(\alpha)}. 
\end{align*} 
The first sum above combined with the expansions of the 
Pochhammer symbols, $\Pochhammer{\pm x}{n}$, given in 
Lemma \ref{lemma_footnote_PHSymbol_BinomIdents} and 
footnote \ftref{footnote_PHSymbol_AlphaFactFn_SimplIdents}, and the 
form of Vandermonde's convolution stated in 
Section \ref{subsubSection_Properties_Of_ConvFn_Phz} 
also leads to the following pair of double sum identities for the 
$\alpha$--factorial functions when $\alpha, n \geq 2$ are integer--valued: 
\begin{align*} 
(\alpha n - 1)!_{(\alpha)} & = 
     \mathsmaller{
     \sum\limits_{k=0}^{n-1} \sum\limits_{i=0}^{k+1} 
     \binom{n-1}{k+1} \binom{k+1}{i} 
     (-1)^{k} \alpha^{k+1-i} 
     \AlphaFactorial{\alpha i-1}{\alpha} 
     \AlphaFactorial{\alpha (n-1-k)-1}{\alpha} 
     \Pochhammer{n-1-k}{k+1-i} 
     } \\ 
     & = 
     \mathsmaller{
     \sum\limits_{k=0}^{n-1} \sum\limits_{i=0}^{k+1} 
     \binom{n-1}{k+1} \binom{k+1}{i} \binom{n-1-i}{k+1-i} 
     (-1)^{k} \alpha^{k+1-i} 
     \AlphaFactorial{\alpha i-1}{\alpha} 
     \AlphaFactorial{\alpha (n-1-k)-1}{\alpha} 
     (k+1-i)! 
     }. 
\end{align*} 
The construction of further 
analogues for generalized variants of the finite summations and 
more well--known combinatorial identities satisfied by the 
double factorial function cases when $\alpha \defequals 2$ from the 
references is suggested as a topic for future investigation in 
Section \ref{subsubSection_FutureResTopics_GenDblFactFnSumIdents_FiniteSums}. 
%% 
\ExampleQED
\end{example} 

\subsection{Multiple summation identities and finite--degree polynomial 
            expansions of the generalized product sequences in $n$} 

We are primarily concerned with cases of generalized factorial--related 
sequences formed the products, $p_n(\alpha, R_n)$, when the 
parameter $R_n \defequals \beta n+\gamma$ depends linearly on $n$ for some 
$\beta, \gamma \in \mathbb{Q}$. 
Strictly speaking, once we evaluate the indeterminate, $R$, as a 
function of $n$ in these cases of the generalized product sequences, 
$p_n(\alpha, R)$, the corresponding generating functions over the 
coefficients enumerated by the approximate convergent function series 
no longer correspond to predictably rational functions of $z$. 
We may, however, still prefer to work with these sequences formulated as 
finite--degree polynomials in $n$ through a few useful forms of the 
next multiple sums expanded below related to the identities given in 
Section \ref{subsubSection_Properties_Of_ConvFn_Phz-AuxNumFn_Subsequences}. 

\begin{cor}[Generalized Polynomial Expansions and Multiple Sum Identities] 
\label{cor_GenFactFnSeqs_MultipleSummationIdents} 
%% 
For integers $n,s,n-s \geq 1$ and fixed 
$\alpha, \beta, \gamma \in \mathbb{Q}$, the 
following finite, multiple sum identities provide 
particular polynomial expansions in $n$ satisfied by the 
generalized factorial function cases: 
\begin{align*} 
\tagonce\label{eqn_GenFactFnSeqs_MultipleSum_Idents_exps-stmts_v1} 
p_{n-s}(\alpha, \beta n + \gamma) = 
     \sum\limits_{
          \substack{0 \leq m \leq k < n-s \\ 0 \leq r \leq p \leq n-s}
          } 
     \sum_{t=0}^{n-s-k} 
     \binom{m}{r} & \binom{n-s}{k} \binom{t}{p-r} 
     \gkpSI{k}{m} \gkpSI{n-s-k}{t} 
     \times \\ 
     & \times 
     (-1)^{p-r-1} 
     \alpha^{n-s-m-t} \beta^{r} \gamma^{m-r} 
     (\alpha + \beta)^{p-r} \times \\ 
     & \times 
     (\alpha (s+1) - \gamma)^{t-(p-r)} \times n^{p} \\ 
     & + 
     \Iverson{0 \leq n \leq s} \\ 
p_{n-s}(\alpha, \beta n + \gamma) = 
     \sum\limits_{ 
          \substack{0 \leq r \leq p \leq u \leq 3n \\ 
          0 \leq m,i \leq k < n-s} 
          } 
     \sum_{t=0}^{n-s-k} 
     \binom{m}{r} & \binom{i}{u-p} \binom{t}{p-r} 
     \gkpSI{k}{m} \gkpSI{k}{i} \gkpSI{n-s-k}{t} \times \\ 
     & \times 
     \frac{(-1)^{u-r+k+1}}{k!} \alpha^{n-s-m-t} \beta^{r} \gamma^{m-r} 
     (\alpha + \beta)^{p-r} \times \\ 
     & \times 
     (\alpha (s+1) - \gamma)^{t-(p-r)} \times s^{p-u+i} n^{u} \\ 
     & + 
     \Iverson{0 \leq n \leq s}. 
\end{align*} 
The forms of these expansions for the generalized factorial function 
sequence variants stated in 
\eqref{eqn_GenFactFnSeqs_MultipleSum_Idents_exps-stmts_v1} 
are provided by this corollary without citing the complete details to 
a somewhat tedious, and unnecessary, proof derived from the 
well--known polynomial expansions of the products, 
$p_n(\alpha, R) = \alpha^{n} \Pochhammer{R / \alpha}{n}$ by the 
Stirling number triangles. 
%% 
\end{cor} 
%%%% 
\begin{proof}[Proof Sketch] 
%% 
%% 
More concretely, 
for $n, k \geq 0$ and fixed 
$\alpha, \beta, \gamma, \rho, n_0 \in \mathbb{Q}$, the 
following particular expansions 
suffice to show enough of the detail needed to more carefully prove 
each of the multiple sum identities cited in 
\eqref{eqn_GenFactFnSeqs_MultipleSum_Idents_exps-stmts_v1} 
starting from the first statements provided in 
\eqref{eqn_pnAlphaR_seqs_exact_finite_sum_reps-stmts_v2}: 
\begin{align*} 
p_k(\alpha, \beta n + \gamma + \rho) 
     & = 
     \alpha^k \cdot \Pochhammer{\frac{\beta n + \gamma + \rho}{\alpha}}{k} \\ 
     & = 
     \sum_{m=0}^{k} \gkpSI{m}{k} \alpha^{k-m} 
     \left(\beta n + \gamma + \rho\right)^{m} \\ 
     & = 
     \sum_{p=0}^{k} \left( 
     \sum_{m=p}^{k} \gkpSI{k}{m} \binom{m}{p} \alpha^{k-m} \beta^{p} 
     \left(\gamma + \rho + \beta n_0\right)^{m-p} 
     \right) \times (n - n_0)^{p}. 
\end{align*} 
%% 
The simplified 
triple sum expansions of interest in the 
Example \ref{example_TripleSumIdents_App_to_WThm} below 
correspond to a straightforward 
simplification of the more general multiple finite quintuple $5$--sums and 
$6$--sum identities  
that exactly enumerate the functions, $p_{n-s}(\alpha, \beta n + \gamma)$, 
when $(s, \alpha, \beta, \gamma) \defequals (1, -1, 1, 0)$. 
%% 
\end{proof} 

\subsubsection{Examples of finite triple sum expansions of the 
               single factorial function by the 
               Stirling numbers of the first kind} 

One immediate consequence of 
Corollary \ref{cor_GenFactFnSeqs_MultipleSummationIdents} 
phrases the form of the next multiple sums that 
exactly generate the single factorial functions, $(n-s)!$, 
modulo any prescribed integers $p \geq 2$. 
%% 
In particular, these results lead to the following 
finite, triple sum expansions of the single factorial function 
cases implicit to the statements of both 
Wilson's theorem and Clement's theorem from the introduction 
considered as examples in the next subsection 
(see Section \ref{subsubSection_S1TripleSums_GenSPolyExps}) 
\citep[\cf \S 7]{HARDYWRIGHTNUMT} 
\footnotemod[A Formula of Riordan from the References]{ 
     The third exact triple sum identity given in 
     \eqref{eqn_SingleFactFn_TripleSum_Ident_exps-stmts_v1} is further 
     expanded through the formula of Riordan cited in the references 
     as follows \citep[p.\ 173]{ADVCOMB} 
     \citep[\cf Ex.\ 5.65, p.\ 534]{GKP}: 
     \begin{align*} 
     \tagtext{A Formula of Riordan} 
     n^{n} & = \sum\limits_{0 \leq k < n} 
          \binom{n-1}{k} (k+1)! \times n^{n-1-k} = 
          \sum_{k=0}^{n-1} \binom{n-1}{k} (n-k)! \times n^{k}. 
     \end{align*} 
}: 
\begin{align*} 
\tagonce\label{eqn_SingleFactFn_TripleSum_Ident_exps-stmts_v1} 
(n-1)! & = 
     \sum_{p=0}^{n} \left( 
     \sum_{0 \leq t \leq k < n} 
     \binom{n}{n-1-k} \gkpSI{n-1-k}{p} \gkpSI{k}{k-t} (-1)^{n-1-p} 
     \right) \times 
     (n-1)^{p} \\ 
     & = 
     \sum_{p=0}^{n} \left( 
     \sum\limits_{\substack{0 \leq k < n \\ 0 \leq t \leq n-1-k}} 
     \binom{n}{k} \gkpSI{k}{p} \gkpSI{n-1-k}{n-1-k-t} (-1)^{n-1-p} 
     \right) \times 
     (n-1)^{p} \\ 
     & = 
     \sum_{p=0}^{n} \left( 
     \sum_{0 \leq t \leq k < n} 
     \binom{n}{n-1-k} \gkpSI{n-1-k}{n-p} \gkpSI{k}{k-t} (-1)^{p+1} 
     \right) \times 
     (n-1)^{n-p}. 
\end{align*} 
%% 
A couple of the characteristic examples of these 
polynomial expansions in $n$ by the 
Stirling numbers of the first kind in 
\eqref{eqn_SingleFactFn_TripleSum_Ident_exps-stmts_v1} are are considered by 
Example \ref{example_TripleSumIdents_App_to_WThm} 
in the next section to illustrate the notable special cases of 
Wilson theorem and Clement's theorem 
modulo some as yet unspecified odd prime, $n \geq 3$. 

\subsubsection{Several related examples of finite triple sum identities 
               expanding the double factorial function} 

%% 
For comparison, the next several equations provide 
related forms of finite, triple sum identities for the 
double factorial function, $(2n-1)!!$. 
\begin{align*} 
\notag 
\tagtext{Double Factorial Triple Sums} 
(2n-1)!! 
   & = 
     \sum\limits_{1 \leq j \leq k \leq n} 
     \gkpSI{k-1}{j-1} 2^{n-j} (-1)^{n-k} \Pochhammer{1-n}{n-k} \\ 
   & = 
     \sum\limits_{\substack{1 \leq j \leq k \leq n \\ 0 \leq m \leq n-k}} 
     \gkpSI{k-1}{j-1} \gkpSI{n-k+1}{m+1} 2^{n-j} (-1)^{n-k-m} n^{m} \\ 
(2n-1)!!   
   & = 
     \sum\limits_{1 \leq j \leq k \leq n} 
     \binom{2n-k-1}{k-1} \gkpSI{k}{j} 
     (2n-2k-1)!! \\ 
   & = 
     \sum\limits_{\substack{1 \leq j \leq k \leq n \\ 0 \leq m \leq n-k}} 
     \binom{2n-k-1}{k-1} \gkpSI{k}{j} \gkpSI{n-k}{m} 2^{n-k-m} \\ 
   & = 
     \sum\limits_{\substack{1 \leq j \leq k \leq n \\ 0 \leq m \leq n-k}} 
     \binom{2n-k-1}{k-1} \gkpSI{k}{j} \FcfII{2}{n-k+1}{m+1} 
     (-1)^{n-k-m} \left(2n - 2k\right)^{m} 
\end{align*} 
The expansions of the double factorial function in the previous 
equations are obtained from the lemma in 
\eqref{eqn_AlphaFactFn_anm1_SpCase_SeqIdents-stmts_v1} 
applied to the known double sum identities involving the 
Stirling numbers of the first kind 
documented in the reference 
\citep[\S 6]{DBLFACTFN-COMBIDENTS-SURVEY}. 
%% 

\subsubsection{Expansions of related sums involving the 
               Stirling polynomials and 
               generalized Bernoulli numbers} 
\label{subsubSection_S1TripleSums_GenSPolyExps} 

%% 
For $n, x \in \mathbb{N}$, 
let the modified Stirling polynomials, $\sigma_n^{\ast}(x)$, 
corresponding to the first cases of these polynomials in 
\tableref{table_GenStirlingAlphaCvlPolys} be defined by 
\begin{align*} 
\tagtext{Modified Stirling Polynomials} 
\sigma_n^{\ast}(x) & \defequals 
     \begin{cases} 
     \frac{x! \cdot (x-n)}{(x-n)!} \cdot \sigma_n(x), & 
     \text{if $x > 0$;} \\ 
     1, & \text{if $x = 0$,} 
     \end{cases} 
\end{align*} 
where the ordinary cases of the 
\emph{Stirling convolution polynomials}, $\sigma_n(x)$, 
correspond to the expansions 
\citep[\S 6.2, Ex.\ 6.77]{GKP} 
\begin{align*} 
\sigma_n(x) & = 
     \gkpSI{x}{x-n} \frac{(x-n-1)!}{x!},\ && 
     \text{ if $x > n > 0$ } \\ 
     %= [z^n] 
     %\left(\frac{z e^{z}}{e^{z}-1}\right)^{x} \\ 
     & = 
     \frac{(-1)^{n+x-1}}{x! (n-x)!} \times 
     \sum_{k=0}^{x-1} \gkpSI{x}{x-k} \frac{B_{n-k}}{n-k},\ && 
     \text{ if $n \geq x > 0$. } 
\end{align*} 
%% 
The last of the triple sum identities given in 
\eqref{eqn_SingleFactFn_TripleSum_Ident_exps-stmts_v1} is expanded 
in the forms of the following equations where the respective 
terms with respect to each individual component sum each correspond to 
rational polynomials in $n$ prescribed by the generalized 
Stirling polynomials and Bernoulli number sequences 
defined above \citep[\cf \S 6.2, \S 7.4]{GKP} \citep{MULTIFACTJIS,CVLPOLYS}: 
\begin{align*} 
(n-1)! = 
     \sum\limits_{\substack{0 \leq t \leq k < p \leq n}} 
     \binom{n}{n-1-k} & (-1)^{p+1} 
     \sigma_t^{\ast}(k) \sigma_{p-1-k}^{\ast}(n-1-k) 
     (n-1)^{n-p} \\ 
     = 
     \sum\limits_{\substack{0 \leq t \leq k < p \leq n \\ 
                            0 \leq r \leq n-p}} 
     \binom{n}{n-1-k} & \binom{n-p}{r} \times (-1)^{p+1} 
     \sigma_t^{\ast}(k) \sigma_{p-1-k}^{\ast}(n-1-k) \times \\ 
     & \phantom{\binom{n-p}{r}} \times 
     (n-\ell_0)^{r} (\ell_0-1)^{n-p-r},\ 
     \ell_0 \neq 1. 
\end{align*} 
The sums over the Stirling numbers defined by the applications 
cited above also satisfy further expansions by the 
Stirling polynomial sequences, and by the 
generalized Bernoulli numbers, or N\"{o}rlund polynomials, 
in the immediate forms stated in 
\eqref{eqn_SingleFactFn_TripleSum_Ident_exps-stmts_v1} 
\citep{MULTIFACTJIS}. 

\subsection{Applications to 
            variants of Wilson's theorem and 
            Clement's theorem concerning the twin primes} 
\label{subsubSection_NewIdentsFromFiniteDiffEqns_ExpsOfPrime-RelatedCongr} 

\subsubsection{Expansions of parameterized congruences 
               involving the single factorial function} 

\begin{definition} 
\label{def_v2} 
%% 
We define the next parameterized congruence variants, 
denoted by $F_{\omega,n}(\xp, \xt, \xk)$, 
corresponding to the first triple sum identity expanded in 
\eqref{eqn_SingleFactFn_TripleSum_Ident_exps-stmts_v1} 
for some application--dependent, prescribed functions, 
$N_{\omega,p}(n)$ and $M_{\omega}(n)$, and where the formal variables 
$\{\xp, \xt, \xk\}$, index the terms in each 
individual sum over the respective variables, $p$, $t$, and $k$. 
\begin{align} 
\label{eqn_FwnXpXtXk_RHS_CongruenceFn_def-stmt_v1} 
F_{\omega,n}(\xp, \xt, \xk) \defequals 
     \sum\limits_{\substack{0 \leq t \leq k < n \\ 0 \leq p \leq n}} & 
     \binom{n}{n-1-k} \gkpSI{n-1-k}{p} \gkpSI{k}{k-t} \times && \\ 
\notag 
     & \times 
     (-1)^{n-1-p} \times N_{\omega,p}(n) 
     \times \{\xp^p \xt^t \xk^k\} 
     && \pmod{M_{\omega}(n)} 
\end{align} 
Notice that 
when $N_{\omega,p}(n) \defequals (n-1)^{p}$, the function 
$F_{\omega,n}(1, 1, 1)$ exactly generates the 
single factorial function, $(n-1)!$, modulo any specific choice of the 
function, $M_{\omega}(n)$, depending on $n$. 
%% 
\DefinitionQED 
\end{definition} 

\subsubsection{Applications 
               to variants of Wilson's theorem} 
\label{subsubSection_TripleSumIdents_App_to_WThm} 

\begin{example}[Wilson's Theorem] 
\label{example_TripleSumIdents_App_to_WThm} 
%% 
The next specialized forms of the parameters implicit to the 
congruence in \eqref{eqn_FwnXpXtXk_RHS_CongruenceFn_def-stmt_v1} 
of the previous definition are chosen as follows to form another 
restatement of Wilson's theorem given immediately below: 
\begin{align*} 
\tagtext{Wilson Parameter Definitions} 
\left(\omega, N_{\omega,p}(n), M_{\omega}(n)\right) \defmapsto 
     \left(\Wilson, (-1)^{p}, n\right). 
\end{align*} 
Then we see that 
\begin{align*} 
\tag{Wilson's Theorem} 
\text{ $n \geq 2$ prime } \iff 
     F_{\Wilson,n}(1, 1, 1) \equiv -1 \pmod{M_{\Wilson}(n)}. 
\end{align*} 
%% 
\sublabel{A summary of computations of the formal polynomial properties 
          from the reference} 
Numerical computations with \Mm's 
\texttt{PolynomialMod} function suggest several nice properties 
satisfied by the trivariate 
polynomial sequences, $F_{\Wilson,n}(\xp, \xt, \xk)$, 
defined by \eqref{eqn_FwnXpXtXk_RHS_CongruenceFn_def-stmt_v1} 
when $n$ is prime, particularly as formed in the 
cases taken over the following polynomial configurations of the 
three formal variables, $\xp$, $\xt$, and $\xk$: 
\[ 
\label{footnote_WThm_MathematicaPolyMod_NotedProperties} 
(\xp, \xt, \xk) \in \left\{(x, 1, 1), (1, x, 1), (1, 1, x)\right\}. 
\] 
%% 
     In particular, these computations suggest the following 
     properties satisfied by these sums for integers $n \geq 2$ 
     where the coefficients of the functions, $F_{\Wilson,n}(\xp, \xt, \xk)$, 
     are computed termwise with respect to the formal variables, 
     $\{\xp, \xt, \xk\}$, modulo each $M_{\Wilson}(n) \defmapsto n$: 
     \begin{itemize} 
     \item[\bf (1)] 
     $F_{\Wilson,n}(\xp, 1, 1) \equiv n-1 \pmod{n}$ when $n$ is prime 
     where $\deg_{\xp} \left\{ F_{\Wilson,n}(\xp, 1, 1) \pmod{n} \right\} > 0$ 
     when $n$ is composite; 
     \item[\bf (2)] 
     $F_{\Wilson,n}(1, 1, \xk) \equiv 
      (n-1) \cdot \xk^{n-1} \Iverson{\text{$n$ prime}} \pmod{n}$; and 
     \item[\bf (3)] 
     $F_{\Wilson,n}(1, \xt, 1) \equiv \sum_{i=0}^{n-2} \xt^{i} \pmod{n}$ 
     when $n$ is prime, and where 
     $\deg_{\xt} \left\{ F_{\Wilson,n}(1, 1, \xt) \pmod{n} \right\} < n-2$ 
     when $n$ is composite. 
     \item[\bf (4)] 
     For fixed $0 \leq p < n$, the outer sums in the 
     definition of \eqref{eqn_FwnXpXtXk_RHS_CongruenceFn_def-stmt_v1}, 
     each implicitly indexed by powers of the 
     formal variable $\xp$ in the 
     parameterized congruence expansions defined above, 
     yield the Stirling number terms given by the coefficients 
     \[ 
     [\xp^{p}] F_{\omega,n}(\xp, 1, 1) = 
      N_{\omega,p}(n) \times (-1)^{n-1} (p+1) \gkpSI{n}{p+1}. 
     \] 
     %% 
     Moreover, for any fixed lower index, $p \geq 1$, the 
     Stirling number terms resulting from these sums are 
     related to factorial multiples of the $r$--order harmonic number 
     sequences expanded by the properties stated in 
     Section \ref{subsubSection_remark_SNum_R-OrderHNum_SeqExpIdents_spcases_v1} below 
     when $r \in \mathbb{Z}^{+}$ \citep[\cf \S 4.3]{MULTIFACTJIS} 
     (see the congruence properties for these sequences 
     modulo any integers $n \geq 2$ expanded by the 
     rational generating function constructions enumerated in the results from 
     Section \ref{subsubSection_remark_New_Congruences_for_GenS1Triangles_and_HNumSeqs}). 
     \end{itemize} 
     The computations in the attached summary notebook file, 
     \TheSummaryNBFile, 
     provide several specific examples of the properties 
     suggested by these configurations of the 
     special congruence polynomials for these cases 
     \citep{SUMMARYNBREF-STUB}. 
%% 
\ExampleQED 
%% 
\end{example} 

\begin{prop}[Congruences for Powers Modulo Double and Triple Integer Products] 
%% 
For integers $p \geq 0$, $n \geq 1$, and any fixed $j > k \geq 1$, 
it is not difficult to prove that the 
following congruence properties hold: 
\StartGroupingSubEquations 
\label{eqn_CongruencesForPowsOfN_ModDblTripleIntProducts} 
\begin{align} 
(n-1)^{p} 
     & \equiv 
     \frac{(-1)^{p}}{k}\left(k + (1- (k+1)^{p}) \cdot n\right) 
     && \pmod{n(n+k)} \\ 
(n-1)^{p}     
     & \equiv 
     \mathsmaller{ 
     (-1)^{p} \left( 
     \frac{(n+k)(n+j)}{jk} + 
     \frac{n(n+j) (k+1)^{p}}{k(k-j)} + 
     \frac{n(n+k) (j+1)^{p}}{j(j-k)} 
     \right) 
     } 
     && \pmod{n(n+k)(n+j)}. 
\end{align} 
\EndGroupingSubEquations 
%% 
\end{prop} 
%%%% 
\begin{proof} 
%% 
First, notice that a na\"{\i}ve expansion by repeated appeals to the 
binomial theorem yields the following exact expansions of the 
fixed powers of $(n-1)^{p}$: 
{\smaller 
     \begin{align*} 
     (n-1)^{p} & = 
          (-1)^{p} + 
          \mathsmaller{ 
          \sum\limits_{s=1}^{p} \binom{p}{s} \binom{s-1}{0} 
          n \cdot (-1)^{p-s} \cdot (-k)^{s-1} + %\\ 
          %& + 
          \sum\limits_{s=1}^{p} 
          \undersetbrace{\equiv 0 \pmod{n(n+k)}}{ 
          \binom{p}{s} \binom{s-1}{1} 
          n \cdot (n+k) \times (-1)^{p-s} (-k)^{s-2} 
          } 
          } \\ 
          & + 
          \mathsmaller{ 
          \sum\limits_{s=1}^{p} \sum\limits_{r=2}^{s-1} 
          \undersetbrace{\equiv 0 \pmod{n(n+k)}}{ 
          \binom{p}{s} \binom{s-1}{r} \binom{r-1}{0} 
          n \cdot (n+k) \times 
          (-1)^{p-s} (-k)^{s-1-r} (k-j)^{r-1} 
          } 
          } \\ 
          & + 
          \mathsmaller{ 
          \sum\limits_{s=1}^{p} \sum\limits_{r=2}^{s-1} \sum\limits_{t=1}^{r-1} 
          \undersetbrace{\equiv 0 \pmod{n(n+k),n(n+k)(n+j)}}{ 
          \binom{p}{s} \binom{s-1}{r} \binom{r-1}{t} 
          n \cdot (n+k) \cdot (n+j) \times 
          (-1)^{p-s} (-k)^{s-1-r} (k-j)^{r-1-t} (n+j)^{t-1} 
          } 
          }, 
     \end{align*}}
Each of the stated congruences are then easily obtained by summing the 
non--trivial remainder terms modulo the cases of the 
integer double products, $n(n+k)$, and the 
triple products, $n(n+k)(n+j)$, respectively. 
%% 
\end{proof} 

The special case of these parameterized expansions of the 
congruence variants defined by 
\eqref{eqn_FwnXpXtXk_RHS_CongruenceFn_def-stmt_v1} 
corresponding to the classical congruence--based 
characterization of the \emph{twin primes} \citeOEIS{A001359,A001097} 
formulated in the statement of Clement's theorem is of 
particular interest in continuing the discussion from 
Section \ref{subSection_Intro_Examples}. 

\begin{example}[Clement's Theorem] 
\label{example_TripleSumIdents_App_to_CThm} 
%% 
%% 
When $k \defequals 2$ in the first congruence result given by 
\eqref{eqn_CongruencesForPowsOfN_ModDblTripleIntProducts} 
of the proposition, the 
parameters in \eqref{eqn_FwnXpXtXk_RHS_CongruenceFn_def-stmt_v1} 
are formed as the particular expansions 
\begin{align*} 
\tagtext{Clement Parameter Definitions} 
\mathsmaller{ 
\left(\omega, N_{\omega,p}(n), M_{\omega}(n)\right) \defmapsto 
     \left(\Clement, \frac{(-1)^{p}}{2}\left(2 + (1- 3^{p}) \cdot n\right), 
     n(n+2)\right). 
} 
\end{align*} 
The corresponding expansion of this alternate formulation of 
Clement's theorem initially stated as in 
Section \ref{subSection_Wthm_CThm_SpCase_Apps} 
of the introduction then results in the restatement of this 
result given in following equation \citep[\S 4.3]{PRIMEREC}
\footnotemod[Computations of the Formal Polynomial Properties in Clement's Theorem]{ 
     \label{footnote_CThm_MathematicaPolyMod_NotedProperties}    
     One other noteworthy property satisfied by the sums, 
     $F_{\Clement,n}(\xp, \xt, \xk)$, 
     modulo each prescribed $M_{\Clement}(n) \defequals n(n+2)$ 
     for the first several cases of the integers $n \geq 3$, 
     suggests that whenever $n$ is prime and $n+2$ is composite we have that 
     \begin{align*} 
     F_{\Clement,n}(1, 1, \xk) & \equiv n+4 + (n^2-4) \xk^{n-1} \pmod{n(n+2)}, 
     \end{align*} 
     where 
     $\deg_{\xk}\left\{ F_{\Clement,n}(1, 1, \xk) \pmod{n(n+2)} \right\} > 0$ 
     when $n$ is prime. 
     See the summary notebook reference \citep{SUMMARYNBREF-STUB} 
     for more detailed computations of these formal polynomial 
     congruence properties. 
}: 
\begin{align*} 
\tag{Clement's Theorem} 
\text{ $n, n+2$ prime } \iff 
     4 \cdot F_{\Clement,n}(1, 1, 1) + 4 + n\equiv 0 
     \pmod{M_{\Clement}(n)}. 
\end{align*} 
%% 
There are numerous other examples of prime--related congruences 
that are also easily adapted by extending the procedure 
for the classical cases given above. 
%% 
A couple of related approaches to congruence--based primality 
conditions for prime pairs formulated through the triple sum expansions 
phrased in 
Example \ref{example_TripleSumIdents_App_to_WThm} 
above are provided by the applications 
given in the next examples of the prime--related 
subsequences highlighted in 
Section \ref{subsubSection_Examples-remarks_RelatedCongruences}. 
%% 
\ExampleQED 
\end{example} 

\subsubsection{Wilson's theorem for prime triples and 
               expansions of the parameterized congruences for the 
               sexy prime triplets} 

%% 
A special case of the generalized congruences results for 
prime $k$--tuples obtained by 
induction from Wilson's theorem in the 
supplementary reference results \citep{SUMMARYNBREF-STUB} 
implies the next statement 
characterizing odd integer triplets, or $3$--tuples, of the form 
$(n, n+d_2, n+d_3)$, for some $n \geq 3$ and some 
prescribed, application--specific choices of the even integer--valued 
parameters, $d_3 > d_2 \geq 2$. 
\begin{align*} 
\tagtext{Wilson's Theorem for Prime Triples} 
(n, n+d_2, & n+d_3) \in \mathbb{P}^{3} \iff && \\ 
     & 
     (1 + (n-1)!)(1 + (n+d_2-1)!)(1+(n+d_3-1)!) \equiv 0 
     && \pmod{n(n+d_2)(n+d_3)} 
\end{align*} 
A partial characterization of the 
\emph{sexy prime triplets}, or prime--valued 
odd integer triples of the form, $(n, n+6, n+12)$, 
defined by convention so that $n+18$ is composite, then occurs 
whenever \citeOEIS{A046118,A046124} 
\begin{align*} 
\mathsmaller{ 
     P_{\SPTriple,1}(n) \times (n-1)! + P_{\SPTriple,2}(n) \times (n-1)!^{2} + 
     P_{\SPTriple,3}(n) \times (n-1)!^{3} 
} & \equiv -1 \pmod{n(n+6)(n+12)}, 
\end{align*} 
and where the three polynomials, $P_{\SPTriple,i}(n)$ for 
$i \defequals 1,2,3$, in the 
previous equation are expanded by the definitions given in the 
following equations: 
\begin{align*} 
P_{\SPTriple,1}(n) & \defequals 
     1 + \Pochhammer{n}{6} + \Pochhammer{n}{12} \\ 
P_{\SPTriple,2}(n) & \defequals 
     \Pochhammer{n}{6} + \Pochhammer{n}{12} + 
     \Pochhammer{n}{6} \times \Pochhammer{n}{12} \\ 
P_{\SPTriple,3}(n) & \defequals 
     \Pochhammer{n}{6} \times \Pochhammer{n}{12}. 
\end{align*} 

\begin{example}[Sexy Prime Triplets] 
\label{example_FirstSPT_Result} 
%% 
Let the congruence parameters in 
\eqref{eqn_FwnXpXtXk_RHS_CongruenceFn_def-stmt_v1} 
corresponding to the sexy prime triplet congruence expansions of the 
single factorial function powers from the previous equations be 
defined as follows: 
\begin{align*} 
\tagtext{Sexy Prime Triplet Congruence Parameters} 
 & \left(\omega, N_{\omega,p}(n), M_{\omega}(n)\right) \\ 
     & \phantom{\qquad} \defmapsto 
     \mathsmaller{ 
     \left(
     \SPTriple, \frac{(-1)^{p}}{72}\left( 
     (n+6)(n+12) - 2 n(n+12) \cdot 7^{p} + n(n+6) \cdot 13^{p} 
     \right), n(n+6)(n+12) 
     \right)
     }. 
\end{align*} 
For comparison with the parameterized congruences 
defined by the corresponding special prime triplet results 
computed in the reference \citep{SUMMARYNBREF-STUB}, 
we similarly see that the elements of an odd integer triple of the 
form $(n, n+6, n+12)$, are all prime whenever $n \geq 3$ satisfies the 
next divisibility requirement modulo the integer triple products, 
$n(n+6)(n+12)$. 
\begin{align*} 
\tagtext{Sexy Prime Triplets} 
\sum\limits_{1 \leq i \leq 3} P_{\SPTriple,i}(n) \times 
     {F_{\SPTriple,n}(1, 1, 1)}^{i} & \equiv -1 
     \pmod{M_{\SPTriple}(n)} 
\end{align*} 
The other notable special case triples of interest in the 
print references, and 
in the additional polynomial congruence cases computed in the 
supplementary reference data \citep{SUMMARYNBREF-STUB}, 
include applications to the 
prime $3$--tuples of the forms $(p+d_1, p+d_2, p+d_3)$ for 
$(d_1, d_2, d_3) \in \left\{(0,2,6), (0,4,6)\right\}$ 
\citep[\cf \S 1.4]{HARDYWRIGHTNUMT} \citep[\S 4.4]{PRIMEREC} 
\citeOEIS{A022004,A022005}. 
%% 
\ExampleQED 
\end{example} 

\subsubsection{Remarks on expansions of the Stirling number triangles by 
               the $r$--order harmonic numbers} 
\label{subsubSection_remark_SNum_R-OrderHNum_SeqExpIdents_spcases_v1} 

%% 
The divisibility of the Stirling numbers of the first kind is tied to 
well--known expansions of the triangle involving the 
generalized \emph{$r$--order harmonic numbers}, 
$H_n^{(r)} \defequals \sum_{k=1}^{n} k^{-r}$, 
for integer--order $r \geq 1$ \citep[\S 6]{GKP} 
\citep[\cf \S 5.7]{ADVCOMB} \citep[\cf \S 7--8]{HARDYWRIGHTNUMT}. 
The applications cited in the references 
provide statements of the following established 
special case identities for these coefficients 
\citep[\S 4.3]{MULTIFACTJIS} \citep[\S 6.3]{GKP}  
\citeOEIS{A001008,A002805,A007406,A007407,A007408,A007409}: 
\begin{align*} 
\tagtext{Harmonic Number Expansions of the Stirling Numbers} 
\gkpSI{n+1}{2} & = n! \cdot H_n \\ 
\gkpSI{n+1}{3} & = \frac{n!}{2}\left(H_n^2 - H_n^{(2)}\right) \\ 
\gkpSI{n+1}{4} & = 
     \frac{n!}{6}\left(H_n^3 - 3 H_n H_n^{(2)} + 2 H_n^{(3)}\right) \\ 
\tagonce\label{eqn_S1k234_HNum_exp_idents-restmts_v1} 
\gkpSI{n+1}{5} & = 
     \frac{n!}{24}\left( 
     H_n^4 - 6 H_n^{2} H_n^{(2)} + 
     3 \left(H_n^{(2)}\right)^{2} + 8 H_n H_n^{(3)} - 6 H_n^{(4)}\right). 
\end{align*} 
The reference \citep[p.\ 554, Ex.\ 6.51]{GKP} 
gives a related precise statement of the 
necessary condition on the primality of odd integers $p > 3$ 
implied by Wolstenholme's theorem from the example cited in 
Section \ref{subSection_Wthm_CThm_SpCase_Apps} 
of the introduction in the following form 
\citep[\cf \S 7.8]{HARDYWRIGHTNUMT}: 
\begin{align*} 
\tagtext{Stirling Number Variant of Wolstenholme's Theorem} 
p > 3 \text{ prime } & \implies \\ 
     & \phantom{\quad} 
     p^2 \mid \mathsmaller{\gkpSI{p}{2}},\ 
     p^2 \mid \mathsmaller{p \gkpSI{p}{3} - p^{2} \gkpSI{p}{4} + 
                           \cdots + p^{p-2} \gkpSI{p}{p}}. 
\end{align*} 
The expansions given in the remarks of 
Section \ref{subsubSection_remark_MmCompsWith_the_SigmaPkg} and in 
Section \ref{subsubSection_MoreGeneralExps_congruences_multiple_factfns} 
suggest similar expansions of congruences involving the 
$\alpha$--factorial functions through more general cases 
$r$--order harmonic number sequences, such as the sequence variants, 
$H_{n,\alpha}^{(r)}$, specifically defined in the 
next sections of the article. 
%% 

\subsection{Expansions of several new forms of 
            prime--related congruences and 
            other prime subsequence identities} 
\label{subsubSection_Examples-remarks_RelatedCongruences} 

\subsubsection{Statements of several 
               results providing finite sum expansions of the 
               single factorial function modulo fixed integers} 

\begin{remark}[Congruences for the Single Factorial Function]  
\label{remark_lemma_new_congruences_for_the_SgFactFn} 
%% 
The second cases of the generalized factorial function congruences in 
\eqref{eqn_pnAlphaR_seqs_finite_sum_reps_modulop-stmts_v1} 
are of particular utility in expanding several of the non--trivial 
results given in 
Section \ref{subsubSection_NewIdentsFromFiniteDiffEqns_ExpsOfPrime-RelatedCongr} 
below when $h - (n-s) \geq 1$. 
The results related to the double factorial functions and the 
central binomial coefficients expanded through the congruences in 
Section \ref{subsubSection-example_OtherRelatedCongruences_DblFactFns} 
employ the second cases of 
\eqref{eqn_pnAlphaR_seqs_finite_sum_reps_modulop-stmts_v1} and 
\eqref{eqn_pnAlphaR_seqs_finite_sum_reps_modulop-stmts_v2} stated in 
Proposition \ref{prop_ExactFormulas_CongruencesModh_from_FiniteDiffEqns}, 
which we do not prove explicitly above. 
The next few results stated in 
\eqref{eqn_lemma_Chn11_Chnm1nms_SgFactFnCongruences-exps_v1} and 
\eqref{eqn_lemma_Chn11_Chnm1nms_SgFactFnCongruences-exps_v2} are 
provided as lemmas needed to state many of the congruence results for the 
prime--related sequence cases given as examples in 
Section \ref{subsubSection_Examples-remarks_RelatedCongruences} and 
Section \ref{subsubSection-Examples_SomeResults_for_Prime_k-Tuples}. 
%% 
\RemarkQED 
\end{remark} 

\begin{lemma}[Expansions of Several Congruences for the Single Factorial Function] 
\label{lemma_} 
%% 
For natural numbers, $n,n-s \geq 0$, the single factorial function, $(n-s)!$, 
satisfies the following congruences 
whenever $h \geq 2$ is fixed (or when $h$ 
corresponds to some fixed function with an implicit dependence on the 
sequence index $n$): 
\StartGroupingSubEquations 
\label{eqn_lemma_Chn11_Chnm1nms_SgFactFnCongruences-exps_subeqns_ref} 
\begin{align*} 
\tagonce\label{eqn_lemma_Chn11_Chnm1nms_SgFactFnCongruences-exps_v1} 
(n-s)! 
     & \equiv 
     C_{h,n-s}(1, 1) && \pmod{h} \\ 
     & = 
     \sum_{i=0}^{n-s} \binom{h}{i} \Pochhammer{-h}{i} (n-s-i)! && \\ 
     & = 
     \sum_{i=0}^{n-s} \binom{h}{i}^{2} (-1)^{i} i! (n-s-i)! && \\ 
     & = 
     \sum_{i=0}^{n-s} \binom{h}{i} \binom{i-h-1}{i} i! (n-s-i)! && \\ 
\tagonce\label{eqn_lemma_Chn11_Chnm1nms_SgFactFnCongruences-exps_v2}  
(n-s)! 
     & \equiv 
     C_{h,n-s}(-1, n-s) && \pmod{h} \\ 
     & = 
     \sum_{i=0}^{n-s} \binom{h}{i} 
     \Pochhammer{n+1-s-h}{i} \times 
     (-1)^{n-s-i} \Pochhammer{-(n-s)}{n-s-i} && \\ 
     & = 
     \sum_{i=0}^{n-s} \binom{h}{i} \binom{n-s}{i} \binom{h-n+s-1}{i} 
     (-1)^{i} i! \times (n-s-i)!. && 
\end{align*} 
\EndGroupingSubEquations 
%% 
The right--hand--side terms, $C_{h,n-s}(\alpha, R)$, in the 
previous two equations correspond to the auxiliary convergent function 
sequences defined in 
Section \ref{subsubSection_Properties_Of_ConvFn_Phz}, and the 
corresponding multiple sum expansions stated in 
\eqref{eqn_Vandermonde-like_PHSymb_exps_of_PhzCfs} and in 
\eqref{eqn_Chn_formula_stmts}, 
as highlighted by the listings given in 
\tablerefIII{table_ConvNumFnSeqs_Chn_AlphaR_SpCaseListings}{table_ConvNumFnSeqs_Chn_AlphaR_SpCaseListings-first_subtable_pageref}. 
%% 
\end{lemma} 
%%%% 
\begin{proof} 
%% 
Since $n! = \pn{n}{-1}{n}$ and $n! = \pn{n}{1}{1}$ for all $n \geq 1$, the 
identities in \eqref{eqn_pnAlphaR_seqs_finite_sum_reps_modulop-stmts_v1} of 
Proposition \ref{prop_ExactFormulas_CongruencesModh_from_FiniteDiffEqns} 
imply the pair of congruences stated in each of 
\eqref{eqn_lemma_Chn11_Chnm1nms_SgFactFnCongruences-exps_v1} and 
\eqref{eqn_lemma_Chn11_Chnm1nms_SgFactFnCongruences-exps_v2} 
modulo any fixed, prescribed setting of the integer--valued $h \geq 2$. 
The expansions of the remaining sums follow first from 
\eqref{eqn_Vandermonde-like_PHSymb_exps_of_PhzCfs}, and then from the 
results stated in 
Lemma \ref{lemma_footnote_PHSymbol_BinomIdents} 
applied to each of the expansions of these first two sums. 
%% 
\end{proof} 

\begin{prop}[Special Cases of the Congruences for the Single Factorial Function]  
%% 
If $n,n-s,d,an+r \in \mathbb{Z}^{+}$ are selected so that 
$n+d, an+r > n-s$, the coefficient identities for the sequences, 
$\pn{n-s}{1}{1} = [z^{n-s}] \ConvFP{n+d}{1}{1}{z} \pmod{n+d, an+r}$, stated in 
\eqref{eqn_Vandermonde-like_PHSymb_exps_of_PhzCfs-stmt_v1} and 
\eqref{eqn_Chn_formula_stmts} of 
Section \ref{subsubSection_Properties_Of_ConvFn_Phz} 
provide that 
\begin{align*} 
\tagonce\label{eqn_SingFactFn_nms_first_ChnSumExps_Modnpd_Modanpr-stmts_v1} 
(n-s)! 
     & \equiv \sum_{i=0}^{n-s} \binom{n+d}{i}^{2} (-1)^{i} i! \times 
     (n-s-i)! && \pmod{n+d} \\ 
(n-s)! 
     & \equiv \sum_{i=0}^{n-s} \binom{an+r}{i} 
     \Pochhammer{-(an+r)}{i} (n-s-i)! && \pmod{an+r}. %\\ 
\end{align*} 
The results expanded through the symbolic computations with these sums 
obtained from \Mm{}'s \SigmaPkg package outlined in 
Section \ref{subsubSection_Examples-remarks_RelatedCongruences-MmCompsWith_the_SigmaPkg}
provide additional forms of the prime--related congruences 
involving the congruence cases defined by the previous several equations. 
%% 
\end{prop} 
%%%% 
\begin{proof} 
%% 
The expansions of the congruences for the single factorial function 
provided by the lemmas stated in 
\eqref{eqn_lemma_Chn11_Chnm1nms_SgFactFnCongruences-exps_v1} 
follow as immediate consequences of the results in 
Proposition \ref{prop_ExactFormulas_CongruencesModh_from_FiniteDiffEqns}. 
The previous equations then correspond to the particular cases of these 
results when $h \defmapsto n+d$ and $h \defmapsto an+r$ 
respective order of the equations stated above. 
%% 
\end{proof} 

\subsubsection{Examples: Consequences of Wilson's theorem} 

\sublabel{Expansions of variants of Wilson's theorem} 
%% 
The previous identities lead to additional examples phrasing 
congruences equivalent to the primality condition in 
Wilson's theorem involving products of the 
single factorial functions, $n!$ and $(n+1)!$, 
modulo some odd integer $p \defequals 2n+1$ of unspecified primality to be 
determined by an application of these results. 
For example, we can prove that for $n \geq 1$, 
an odd integer $p \defequals 2n+1$ is prime if and only if
\citep[\cf \S 8.9]{HARDYWRIGHTNUMT}\footnotemod{ 
     The first equation restates a result proved by Sz\'{a}nt\'{o}  
     in 2005 given on the 
     \href{http://mathworld.wolfram.com/WilsonsTheorem.html}{
     \emph{MathWorld}} site. 
} 
\begin{align*} 
\tagonce\label{eqn_MathWorld_FormsOf_WilsonsThm} 
2^{1-n} \cdot n! \cdot (n+1)! & \equiv (-1)^{\binom{n+2}{2}} && \pmod{2n+1} \\ 
\left( n! \right)^{2} & \equiv (-1)^{n+1} && \pmod{2n+1}. 
\end{align*} 
The first congruence in 
\eqref{eqn_MathWorld_FormsOf_WilsonsThm} 
yields the following additional forms of 
necessary and sufficient conditions on the primality of the odd integers, 
$p \defequals 2n+1$, resulting from Wilson's theorem: 
\begin{align*} 
\tagonce\label{eqn_MathWorld_FormsOf_WilsonsThm-Chm_prodsum_exp-stmt_v2} 
\frac{1}{2^{n-1}} \times \left( 
     \prod\limits_{s \in \{0,1\}} \sum_{i=0}^{n+s} 
     \binom{2n+1}{i}^2 (-1)^{i} i! (n+s-i)! 
     \right) & \equiv (-1)^{(n+1)(n+2) / 2} && \pmod{2n+1} \\ 
\left(\sum_{i=0}^{n} \binom{2n+1}{i}^{2} (-1)^{i} i! (n-i)!\right)^{2} 
     & \equiv (-1)^{n+1} && \pmod{2n+1}. 
\end{align*} 
\sublabel{Congruences for primes of the form $n^2+1$} 
If we further seek to determine new properties of the odd 
primes of the form $p \defequals n^2 + 1 \geq 5$, 
obtained from adaptations of the new forms given by these sums, 
the second consequence of Wilson's theorem provided in 
\eqref{eqn_MathWorld_FormsOf_WilsonsThm} above 
leads to an analogs requirement expanded in the forms of the 
next equations \citep[\S 3.4(D)]{PRIMEREC} \citeOEIS{A002496}. 
\begin{align*} 
n^2 + 1 \text{ prime } & \iff 
     \mathsmaller{ 
     \left(\sum_{i=0}^{n^2 / 2} 
     \binom{n^2+1}{i} \Pochhammer{-(n^2+1)}{i} 
     \left(\mathsmaller{\frac{1}{2} (n^2 - 2i)}\right)! 
     \right)^{2} 
     } 
     && \equiv (-1)^{n^2 / 2 + 1} && \pmod{n^2+1} \\ 
\phantom{n^2 + 1 \text{ prime }} & \iff 
     \mathsmaller{ 
     \left(\sum_{i=0}^{n^2 / 2} 
     \binom{n^2+1}{i} \binom{i-n^2-2}{i} i! 
     \left(\mathsmaller{\frac{1}{2} (n^2 - 2i)}\right)! 
     \right)^{2} 
     } 
     && \equiv (-1)^{n^2 / 2 + 1} && \pmod{n^2+1} 
\end{align*} 
For comparison with the previous two congruences, the 
first classical statement of Wilson's theorem 
stated as in the introduction is paired 
with the next expansions of the fourth and fifth multiple sums stated in 
\eqref{eqn_Chn_formula_stmts} 
to show that an odd integer $p \geq 5$ of the form 
$p \defequals n^2+1$ for some even $n \geq 2$ whenever 
\begin{align*} 
\sum\limits_{\substack{0 \leq m \leq k \leq n^2 \\ 
             0 \leq v \leq i \leq s \leq n^2} 
             } & 
     \underset{C_{h,k}(\alpha, R) \text{ where } 
          h \defmapsto n^2+1,\ k \defmapsto n^2,\ 
          \alpha \defmapsto -1,\ R \defmapsto n^2 
          \text{ in \eqref{eqn_Chn_formula_stmts-exp_v4} }}{\underline{ 
     \mathsmaller{
     \binom{n^2+1}{k} \binom{m}{s} \binom{i}{v} \binom{n^2+1+v}{v} 
     \gkpSI{k}{m} \gkpSII{s}{i} (-1)^{m+i-v} i! \Pochhammer{-n^2}{n^2-k} 
     \left(n^2 + 1\right)^{m-s}}}} 
     && \equiv -1 \pmod{n^2+1} \\ 
\sum\limits_{\substack{0 \leq i \leq n^2 \\ 
                       0 \leq m \leq k \leq n^2 \\ 
                       0 \leq t \leq s \leq n^2} 
                 } & 
     \underset{C_{h,n}(\alpha, R) \text{ where } 
          h \defmapsto n^2+1,\  n \defmapsto n^2,\ 
          \alpha \defmapsto 1,\ R \defmapsto 1 
          \text{ in \eqref{eqn_Chn_formula_stmts-exp_v5} }}{\underline{ 
     \mathsmaller{
     \binom{n^2+1}{k} \binom{m}{t} \gkpSI{k}{m} 
     \gkpSI{n^2-k}{s-t} \gkpSII{s}{i} 
     (-1)^{m+s-i} n^{2m-2t} 
     \times i! 
     }} 
     } 
     && \equiv -1 \pmod{n^2+1}. 
\end{align*} 
These congruences are straightforward to adapt to 
form related results characterizing prime subsequences of the form 
$p \defequals an^2+bn+c$ for some fixed constants 
$a,b,c \in \mathbb{Z}$ satisfying the constraints given in the 
reference at natural numbers $n \geq 1$ 
\citep[\S 2.8]{HARDYWRIGHTNUMT}. 
%% 

\sublabel{Congruences for the Wilson primes} 
%% 
The sequence of \emph{Wilson primes} denotes the subsequence odd primes $n$ 
such that the \emph{Wilson quotient}, 
$\WilsonQuotient{n} \defequals 
 \frac{\left((n-1)! + 1\right)}{n}$, is divisible by $n$, 
or equivalently the sequence of odd integers $n \geq 3$ 
with the divisibility property of the single factorial 
function, $(n-1)!$, 
modulo $n^2$ defined in the next equation 
\citep[\S 5.4]{PRIMEREC} \citep[\S 6.6]{HARDYWRIGHTNUMT} 
\citeOEIS{A007619,A007540}. 
\begin{align*} 
\tagtext{Wilson Primes} 
\WilsonPrimeSet & \defequals \left\{ 
     n \geq 3 : \text{ $n$ prime \ and \ } n^2 | (n-1)! + 1 
     \right\} 
     \quad \seqmapsto{A007540} \left(5, 13, 567, \ldots \right). 
\end{align*} 
%% 
A few additional expansions of congruences characterizing the 
Wilson primes correspond to the imposing the following additional 
equivalent requirements on the divisibility of the 
single factorial function (modulo $n$) in Wilson's theorem: 
\begin{align*} 
\tagtext{Wilson Prime Congruences} 
\underset{C_{n^2,n-1}(1, 1)\ \equiv\ (n-1)! \pmod{n^2}}{
     \underbrace{ 
     \sum_{i=0}^{n-1} \binom{n^2}{i}^{2} (-1)^{i} i! (n-1-i)! 
     } 
} 
     & \equiv -1 && \pmod{n^2} \\ 
\underset{(n-1)! \pmod{n^2}}{
     \underbrace{ 
     \sum_{i=0}^{n-1} \binom{n^2}{i} \binom{i-n^2-1}{i} i! (n-1-i)! 
     } 
} 
     & \equiv -1 && \pmod{n^2} \\ 
\underset{C_{n^2,n-1}(-1, n-1)\ \equiv\ (n-1)! \pmod{n^2}}{
     \underbrace{ 
     \sum_{i=0}^{n-1} \binom{n^2}{i} 
     \FFactII{(n^2-n)}{i} \times (-1)^{n-1-i} \FFactII{(n-1)}{n-1-i} 
     } 
} 
     & \equiv -1 && \pmod{n^2}. 
\end{align*} 
The congruences in the previous equation 
are verified numerically in the reference \citep{SUMMARYNBREF-STUB} to 
hold for the first few hundred primes, $p_n$, 
only when $p_n \in \{5, 13, 563\}$. 
%% 
The third and fourth multiple sum expansions of the coefficients, 
$C_{n^2,n-1}(-1, n-1)$, given in 
\eqref{eqn_Chn_formula_stmts} 
similarly provide that an odd integer $n > 3$ is a 
Wilson prime if and only if either of the following pair of 
congruences holds modulo the integer squares $n^2$: 
\begin{align*} 
\undersetbrace{C_{n^2,n-1}(-1, n-1) 
     \text{ in \eqref{eqn_Chn_formula_stmts-exp_v3} }}{ 
     \mathsmaller{ 
     \sum\limits_{s=0}^{n-1} \sum\limits_{i=0}^{s} \left( 
     \sum\limits_{k=0}^{n-1} \sum\limits_{m=0}^{k} 
     \binom{n^2}{k} \binom{n^2}{i} \binom{m}{s} 
     \gkpSI{k}{m} \gkpSII{s}{i} (-1)^{n-1-k} 
     \Pochhammer{1-n}{n-1-k} (-n)^{m-s} i! 
     \right)} 
     } 
     & 
     \mathsmaller{ 
     \equiv -1 \pmod{n^2} 
     } \\ 
\tag*{$\ExampleQEDSymbol$} 
     \undersetbrace{C_{n^2,n-1}(-1, n-1) 
     \text{ in \eqref{eqn_Chn_formula_stmts-exp_v4} }}{ 
\mathsmaller{ 
     \sum\limits_{s=0}^{n-1} \sum\limits_{i=0}^{s} \sum\limits_{v=0}^{i} 
     \left( 
     \sum\limits_{k=0}^{n-1} \sum\limits_{m=0}^{k} 
     \binom{n^2}{k} \binom{m}{s} \binom{i}{v} \binom{n^2+v}{v} 
     \gkpSI{k}{m} \gkpSII{s}{i} (-1)^{i-v} 
     \FFactII{(n-1)}{n-1-k} (-n)^{m-s} i! 
     \right)} 
     } 
     & 
     \mathsmaller{ 
     \equiv -1 \pmod{n^2}. 
     } 
\end{align*} 
%% 

\sublabel{Congruences for special prime pair subsequences} 
%% 
The constructions of the new results expanded above 
are combined with the known congruences established in the reference 
\citep[\S 3, \S 5]{ONWTHM-AND-POLIGNAC-CONJ} to obtain the 
alternate necessary and sufficient conditions for the 
twin prime pairs 
stated in \eqref{eqn_TwinPrime_NewExpsOfKnownCongruenceResults-stmts_v1} 
of the introduction \citeOEIS{A001359,A001097}. 
The results in the references also provide analogs 
expansions of the congruence statements 
corresponding to characterizations of the 
\emph{cousin prime} and \emph{sexy prime} pairs 
expanded in the following equations \citeOEIS{A023200,A023201}: 
\begin{align*} 
\tagtext{Cousin Prime Pairs} 
 & 2n+1, 2n+5 \text{ odd primes } && \\ 
 & \qquad \iff 
   \mathsmaller{ 
     36\left(\sum\limits_{i=0}^{n} 
     \binom{(2n+1)(2n+5)}{i}^2 (-1)^i i! (n-i)!\right)^2} & \\ 
 & \phantom{\qquad\iff} + 
   (-1)^{n} (29-14n) && \equiv 0 \pmod{(2n+1)(2n+5)} \\ 
 & \qquad \iff 
   \mathsmaller{
     96 \left(\sum\limits_{i=0}^{2n} 
     \binom{(2n+1)(2n+5)}{i}^2 (-1)^i i! (2n-i)!\right)} & \\ 
 & \phantom{\qquad\iff} + 
   46n+\color{penguinbook}{119} && \equiv 0 \pmod{(2n+1)(2n+5)} \\ 
%%%% 
 & 2n+1, 2n+7 \text{ odd primes } && \\ 
\tagtext{Sexy Prime Pairs} 
 & \qquad \iff 
   \mathsmaller{ 
     1350\left(\sum\limits_{i=0}^{n} 
     \binom{(2n+1)(2n+7)}{i}^2 (-1)^i i! (n-i)!\right)^2} & \\ 
 & \phantom{\qquad\iff} + 
   (-1)^{n} (578n+1639) && \equiv 0 \pmod{(2n+1)(2n+7)} \\ 
 & \qquad \iff 
   \mathsmaller{
     4320\left(\sum\limits_{i=0}^{2n} 
     \binom{(2n+1)(2n+7)}{i}^2 (-1)^i i! (2n-i)!\right)} & \\ 
\tagonce\label{eqn_CousinSexyPrimePairs_CongruenceStmts-exps_v1} 
 & \phantom{\qquad\iff} + 
   1438n+5039 && \equiv 0 \pmod{(2n+1)(2n+7)}. 
\end{align*} 
%% 
The multiple sum expansions of the single factorial functions in the 
congruences given in the previous two examples also yield 
similar restatements of the pair of congruences in 
\eqref{eqn_TwinPrime_NewExpsOfKnownCongruenceResults-stmts_v1} 
from Section \ref{subSection_Wthm_CThm_SpCase_Apps} 
providing that for some $n \geq 1$, the odd integers, 
$(p_1, p_2) \defequals (2n+1, 2n+3)$, are both prime 
whenever either of the following divisibility conditions hold: 
\begin{align*} 
\tagtext{Twin Prime Pairs} 
2 & \times 
     \undersetbrace{C_{(2n+1)(2n+3),n}(-1,n) 
     \text{ in \eqref{eqn_Chn_formula_stmts-exp_v3} }}{\left(
     \mathsmaller{ 
     \sum\limits_{\substack{0 \leq i \leq s \leq n \\ 
                       0 \leq m \leq k \leq n}} 
     \binom{(2n+1)(2n+3)}{i} \binom{(2n+1)(2n+3)}{k} \binom{m}{s} 
     \gkpSI{k}{m} \gkpSII{s}{i} (-1)^{s+k} i! \times 
     \FFactII{n}{n-k} (n+1)^{m-s} 
     } 
     \right)}^{2}  \\ 
     & \phantom{\quad} + 
     (-1)^{n} (10n+7) \equiv 0 \hspace{3in} \pmod{(2n+1)(2n+3)} \\ 
4 & \times 
     \undersetbrace{C_{(2n+1)(2n+3),2n}(-1,2n) 
     \text{ in \eqref{eqn_Chn_formula_stmts-exp_v4} }}{ 
     \left( 
     \mathsmaller{ 
     \sum\limits_{\substack{0 \leq v \leq i \leq s \leq 2n \\ 
                         0 \leq m \leq k \leq 2n}} 
     \binom{(2n+1)(2n+3)}{k} \binom{(2n+1)(2n+3)+v}{v} 
     \binom{i}{v} \binom{m}{s} \gkpSI{k}{m} \gkpSII{s}{i} 
     (-1)^{s-i+v+k} i! \times 
     \FFactII{(2n)}{2n-k} (2n+1)^{m-s} 
     } 
     \right)}  \\ 
     & \phantom{\quad} + 
     (2n+5) \equiv 0 \hspace{3.5in} \pmod{(2n+1)(2n+3)}. 
\end{align*} 
%% 
The treatment of the modular congruence identities involved in 
these few notable example cases in 
Example \ref{example_TripleSumIdents_App_to_WThm}, and in the 
last several examples from the remarks above, 
is by no means exhaustive, but serves to demonstrate the utility 
of this approach in formulating several new forms of 
non--trivial prime number results 
with many notable applications. 
%% 

\subsubsection{Expansions of congruences involving the 
               double factorial function} 
\label{subsubSection-example_OtherRelatedCongruences_DblFactFns} 

\sublabel{Statements of congruences for the double factorial function} 
%% 
The coefficient expansion given by the last identity in 
\eqref{eqn_pnAlphaR_seqs_finite_sum_reps_modulop-stmts_v2} 
provides the alternate forms of congruences for the double factorial functions, 
$(2n-1)!! = \pn{n}{-2}{2n-1}$ and $2^{n} \Pochhammer{1/2}{n} = \pn{n}{2}{1}$, 
modulo $2^{s} \cdot h$ stated in the next equations 
for fixed integers $h \geq 2$ and any integer $0 \leq s \leq h$ 
\footnotemod[Binomial Coefficient Identities Involving the Double Factorial Function]{ 
     For natural numbers $n \geq 0$, the central binomial coefficients 
     satisfy an expansion by the following identity 
     given in the reference \citep[\S 5.3]{GKP}: 
     \begin{align*} 
     \tagtext{Binomial Coefficient Half--Index Identities} 
     \Pochhammer{1/2}{n} & = 
          \binom{-1/2}{n} \times (-1)^{n} n! = 
          \binom{2n}{n} \times \frac{n!}{4^{n}}. %\\ 
     \end{align*} 
}: 
\begin{align*} 
(2n-1)!! & \equiv 
     \sum_{i=0}^{n} 
     \binom{h}{i} 2^{n+(s+1) i} \Pochhammer{1/2-h}{i} \Pochhammer{1/2}{n-i} 
     && \pmod{2^{s} h} \\ 
     & \equiv 
     \sum_{i=0}^{n} 
     \binom{h}{i} \binom{2n-2i}{n-i} 
     \frac{2^{n+ (s+1) i}}{4^{n-i}} \times 
     \Pochhammer{1/2-h}{i} (n-i)!
     && \pmod{2^{s} h} \\ 
     & \equiv 
     \sum_{i=0}^{n} \binom{h}{i} (-2)^{n+(s+1) i} 
     \Pochhammer{1/2+n-h}{i} \Pochhammer{1/2-n}{n-i} 
     && \pmod{2^{s} h}. 
\end{align*} 

\sublabel{Semi--polynomial congruences expanding the 
          central binomial coefficients} 
The next polynomial congruences satisfied by the central binomial coefficients 
modulo integer multiples of the individual 
polynomial powers, $n^p$, of $n$ for some fixed $p \geq 1$ 
also provide additional examples of some of the 
double--factorial--related phrasings of the expansions of 
\eqref{eqn_pnAlphaR_seqs_finite_sum_reps_modulop-stmts_v1} and 
\eqref{eqn_pnAlphaR_seqs_finite_sum_reps_modulop-stmts_v2} 
following from the noted identity given in 
\eqref{eqn_Vandermonde-like_PHSymb_exps_of_PhzCfs-stmt_v1} of 
Section \ref{subsubSection_Properties_Of_ConvFn_Phz} 
(see the computations contained in the reference \citep{SUMMARYNBREF-STUB}): 
\begin{align*} 
\tagonce\label{eqn_Wolstenholme-like_congruences_for_central_binomials} 
\binom{2n}{n} & = 
     \frac{2^{n}}{n!} \times (2n-1)!! && \\ 
     & \equiv 
     \left\lbrace 
     \undersetline{\mod{x^p} \quad \looparrowright \quad x \defmapsto n}{
     \sum_{i=0}^{n} \binom{x^p}{i} 
     2^{i} \Pochhammer{1/2 - x^p}{i} \Pochhammer{1/2}{n-i} 
     \times \frac{2^{2n}}{n!} 
     } 
     \right\rbrace 
     && \pmod{n^p} \\ 
\notag 
     & \equiv 
     \left\lbrace 
     \undersetline{\mod{x^p} \quad \looparrowright \quad x \defmapsto n}{
     \sum_{i=0}^{n} \binom{x^p}{i} \binom{2x-2i}{x-i} 
     \Pochhammer{1/2 - x^p}{i} \times 
     \frac{8^{i} \cdot (n-i)!}{n!} 
     } 
     \right\rbrace 
     && \pmod{n^p}. 
\end{align*} 
The special cases of the congruences in 
\eqref{eqn_Wolstenholme-like_congruences_for_central_binomials} 
corresponding to $p \defequals 3$ and $p \defequals 4$, respectively, 
are related to the necessary condition for the primality of 
odd integers $n > 3$ in Wolstenholme's theorem and to the sequence of 
\emph{Wolstenholme primes} defined as 
\citep[\S 2.2]{PRIMEREC} \citep[\cf \S 7]{HARDYWRIGHTNUMT} 
\citeOEIS{A088164} 
\begin{align*} 
\tagtext{Wolstenholme Primes} 
\WolstPrimeSet & \defequals \left\{ n \geq 5: 
     \text{ $n$ prime \ and \ } 
     \mathsmaller{\binom{2n}{n} \equiv 2 \pmod{n^4}} 
     \right\} \\ 
     & \phantom{:} = 
     \left(16843, 2124679, \ldots \right). 
\end{align*} 

\sublabel{An identity for the single factorial function 
          involving expansions of double factorial functions} 
%% 
As another example of the applications of these new results expanded 
through the double factorial function, 
notice that the 
following identity gives the form of another exact, finite double sum 
expansion of the single factorial function over 
convolved products of the double factorials: 
\begin{align*} 
(n-1)! & = (2n-3)!! + 
     \sum_{k=1}^{n-2} \sum_{j=k}^{n-1} 
     (-1)^{j+1} \Pochhammer{-j}{k} \Pochhammer{-(2n-k-j-2)}{j-k} 
     (2n-2j-3)!! \\ 
     & \phantom{ = (2n-3)!!} + 
     \sum_{k=1}^{n-2} \sum_{j=k+1}^{n-1} 
     (-1)^{j} \Pochhammer{-j}{k+1} \Pochhammer{-(2n-k-j-3)}{j-k-1} 
     (2n-2j-3)!! 
\end{align*} 
     This identity is straightforward to prove 
     starting from the first non--round sum given in 
     {\S 5.1} of the reference combined with second identity for the 
     component summation terms 
     in {\S 6.3} of the same article \citep{DBLFACTFN-COMBIDENTS-SURVEY}. 
A modified approach involving the congruence techniques 
outlined in either the first cases cited in 
Example \ref{example_TripleSumIdents_App_to_WThm}, 
or as suggested in the previous few example cases from the 
last subsections of the article, 
then suggests even further applications adapting the 
results for new variants of the 
established, or otherwise well--known, special case 
congruence--based identities 
expanded for the notable prime number subsequences cited above. 

\subsection{Computations of symbolic sums with Mathematica's Sigma package} 
\label{subsubSection_Examples-remarks_RelatedCongruences-MmCompsWith_the_SigmaPkg}

\subsubsection{Example I: Expansions of the first sum from the proposition} 

The working summary notebook, \TheSummaryNBFile, 
attached to the article \citep{SUMMARYNBREF-STUB} 
includes computations with \Mm's \SigmaPkg package 
that yield additional forms of the identities expanded in 
\eqref{eqn_SingFactFn_nms_first_ChnSumExps_Modnpd_Modanpr-stmts_v1}, 
\eqref{eqn_MathWorld_FormsOf_WilsonsThm}, and 
\eqref{eqn_MathWorld_FormsOf_WilsonsThm-Chm_prodsum_exp-stmt_v2}, 
for the single factorial function, $(n-s)!$, when $s \defequals 0$. 
%% 
For example, 
alternate forms of the identity for the first sum in 
\eqref{eqn_SingFactFn_nms_first_ChnSumExps_Modnpd_Modanpr-stmts_v1} are 
expanded as follows: 
\begin{align*} 
n! & \equiv 
     \sum_{i=0}^{n} \binom{n+d}{i} 
     \Pochhammer{-(n+d)}{i} (n-i)! && \pmod{n+d} \\ 
n! & \equiv 
     \sum_{i=0}^{n} \binom{n+d}{i} \binom{i-n-d-1}{i} i! (n-i)! 
     && \pmod{n+d} \\ 
n! & \equiv 
     \sum_{i=0}^{n} \binom{n+d}{i}^{2} (-1)^{i} i! (n-i)! && \pmod{n+d} \\ 
   & = 
     (-1)^{n} \Pochhammer{2d}{n} \times \left( 
     1 + d^2 \times \sum_{i=1}^{n} \binom{i+d}{i} 
     \frac{(-1)^{i} \Pochhammer{-(i+d)}{i}}{(i+d)^2 \Pochhammer{2d}{i}} 
     \right) && \\ 
\tagonce\label{eqn_SingFactFn_nms_first_ChnSumExps_Modnpd_Modanpr-stmts_v2} 
   & = 
   \undersetbrace{\defequals S_{1,d}(n)}{ 
   (-1)^{n} d^{2} \times \sum_{i=0}^{n} \binom{i+d}{d}^{2} \times 
   \frac{i! \cdot \Pochhammer{2d+i}{n-i}}{(i+d)^{2}}
   }. && 
\end{align*} 
The first special cases of the sums, $S_{1,d}(n)$, 
defined in the last equation 
are expanded in terms of the first--order harmonic numbers 
for integer--valued cases of $d \geq 1$ as follows: 
\begin{align*} 
S_{1,1}(n) & = 
     (-1)^{n} \Pochhammer{2}{n} \times H_{n+1} \\ 
     & = 
     (-1)^{n} (n+1)! \times H_{n+1} \\ 
S_{1,2}(n) & = 
     (-1)^{n} (n+2)! \times \left( 
     (n+3) H_{n+2} - 2(n+2) 
     \right) \\ 
     & = 
     \frac{3}{2} \times (-1)^{n} \Pochhammer{4}{n} \times \left( 
     H_{n} - \frac{(2n^3+8n^2+7n-1)}{(n+1)(n+2)(n+3)} 
     \right) \\ 
     %& = 
     %\mathsmaller{\gkpEII{n+2}{n}} \\ 
S_{1,3}(n) & = 
     \frac{1}{4} \times (-1)^{n} (n+4)! \times \left( 
     (n+5) H_{n+3} - 3(n+3) 
     \right) \\ 
     & = 
     \frac{10}{3} \times (-1)^{n} \Pochhammer{6}{n} \times \left( 
     H_{n} - \frac{(3n^4+24n^3+60n^2+46n-1)}{(n+1)(n+2)(n+3)(n+5)} 
     \right) \\ 
S_{1,4}(n) & = 
     \frac{1}{108} \times (-1)^{n} (n+4)! \times \left( 
     3(n+5)(n+6)(n+7) H_{n+4} - (n+4)(11n^2+118n+327)  
     \right) \\ 
     & = 
     \frac{35}{12} \times (-1)^{n} \Pochhammer{8}{n} \times \left( 
     \mathsmaller{ 
     3 H_{n} - 
     \frac{(11n^7+260n^6+2498n^5+12404n^4+33329n^3+45548n^2+24426n-108)}{ 
     (n+1)(n+2)(n+3)(n+4)(n+5)(n+6)(n+7)} 
     } 
     \right). 
\end{align*} 
When the parameter $d \defmapsto d_n$ 
in the previous expansions of the sums, $S_{1,d}(n)$, 
depends linearly, or quadratically on $n$, the harmonic--number--based 
identities expanding these congruence forms yield the forms of the 
next examples considered in the results given directly below in this 
section (see 
Section \ref{subsubSection_remark_MmCompsWith_the_SigmaPkg} below). 

\subsubsection{Example II: Expansions of the second sum from the proposition 
               (sums with a linear dependence of $h$ on $n$)} 
Further computation with \Mm's \SigmaPkg package 
similarly yields the following alternate form of the second sums in 
\eqref{eqn_SingFactFn_nms_first_ChnSumExps_Modnpd_Modanpr-stmts_v1} 
implicit to the congruence identities stated in 
\eqref{eqn_MathWorld_FormsOf_WilsonsThm} and 
\eqref{eqn_MathWorld_FormsOf_WilsonsThm-Chm_prodsum_exp-stmt_v2} above: 
\StartGroupingSubEquations 
\label{eqn_nFactMod2np1_CongruenceIdent_SigmaPkgAltSums} 
\begin{align} 
n! & \equiv 
     \mathsmaller{ 
     \sum\limits_{i=0}^{n} \binom{2n+1}{i}^2 (-1)^{i} 
     i! (n-i)! 
     } 
     \qquad \pmod{2n+1} \\ 
     & = 
     \mathsmaller{ 
     \frac{(-1)^{n} (3n+1)!}{8 \left(n!\right)^{2}} \times \left( 
     8 - \sum\limits_{i=1}^{n} \binom{2i+1}{i}^{2} 
     \frac{(i!)^{3}}{2 \cdot (3i+1)!} \left( 
     11 + \frac{20}{i} - \frac{8}{(2i+1)} + \frac{1}{(2i+1)^2} 
     \right) 
     \right) 
     } \\ 
\notag 
     & = 
     \mathsmaller{ 
     \frac{(-1)^{n} (3n+1)!}{\left(n!\right)^{2}} - 
     \frac{(-1)^{n}}{16} \times 
     \sum\limits_{i=1}^{n} \binom{2i+1}{i}^{2} 
     \frac{i! \cdot \Pochhammer{3i+2}{3n-3i}}{ 
     \Pochhammer{i+1}{n-i}^{2}} \left( 
     11 + \frac{20}{i} - \frac{8}{(2i+1)} + \frac{1}{(2i+1)^2} 
     \right) 
     } \\ 
     & = 
     \mathsmaller{ 
     \frac{(-1)^{n} (3n+1)!}{8 \left(n!\right)^{2}} \times \left( 
     8 - \sum\limits_{i=1}^{n} \binom{2i+1}{i}^{2} 
     \frac{(i!)^{3}}{(3i)!} \left( 
     \frac{10}{i} + \frac{5}{(2i+1)} - \frac{1}{(2i+1)^2} - 
     \frac{32}{(3i+1)} 
     \right) 
     \right) 
     } \\ 
\notag 
     & = 
     \mathsmaller{ 
     \frac{(-1)^{n} (3n+1)!}{\left(n!\right)^{2}} - 
     \frac{(3n+1) (-1)^{n}}{8} \times 
     \sum\limits_{i=1}^{n} \binom{2i+1}{i}^{2} 
     \frac{i! \cdot \Pochhammer{3i+1}{3n-3i}}{ 
     \Pochhammer{i+1}{n-i}^{2}} \left( 
     \frac{10}{i} + \frac{5}{(2i+1)} - \frac{1}{(2i+1)^2} - 
     \frac{32}{(3i+1)} 
     \right) 
     }. 
\end{align} 
\EndGroupingSubEquations 
%% 
The documentation for the \SigmaPkg package in the 
reference \citep[Ex.\ 3.3]{SYMB-SUM-COMB-SIGMAPKGDOCS} 
contains several identities 
related to the partial harmonic--number--related 
expansions of the single factorial function sums 
given in the next examples, and for the additional computations 
contained in the reference \citep{SUMMARYNBREF-STUB}. 

\subsubsection{Example III: Expansions of congruences for special prime pairs 
               (sums involving a quadratic dependence of $h$ on $n$)} 

%% 
The second (non--square) sums implicit to the congruences providing 
characterizations of the twin prime pairs given in 
\eqref{eqn_TwinPrime_NewExpsOfKnownCongruenceResults-stmts_v1} 
of the introduction, and of the 
cousin and sexy primes expanded in 
\eqref{eqn_CousinSexyPrimePairs_CongruenceStmts-exps_v1} 
of the previous remark, are easily generalized to form related results for 
other prime pairs \citeOEIS{A023202,A023203,A046133}. 
%% 
In particular, 
for positive integers $d \geq 1$, the special cases of these 
expansions for the prime pair sequences considered above lead to 
more general congruence--based characterizations of the odd prime pairs, 
$(2n+1, 2n+1+2d)$, in the forms of the following equation 
for some $a_d, b_d, c_d \in \mathbb{Z}$ and where the parameter 
$h_d \defequals (2n+1)(2n+1+2d) > 2n$ implicit to these sums 
depends quadratically on $n$ 
\citep[\cf \S 3, \S 5]{ONWTHM-AND-POLIGNAC-CONJ}: 
\begin{align*} 
\tagonce\label{eqn_GenPrimePairCongFn_hndabc_v1} 
2n+1, 2n+1+2d \text{ prime } & \iff 
     a_d \times 
     \underset{\mathlarger{ \defequals S_n(h_d) \equiv (2n)! \pmod{h_d}}}{\underbrace{
     \sum_{i=0}^{2n} \binom{h_d}{i}^{2} (-1)^{i} i! (2n-i)!} 
     } + 
     b_d n + c_d \equiv 0 \pmod{h_d}. 
\end{align*} 
%% 
For natural numbers $h, i, n \geq 0$, 
let the shorthand for the functions, 
$T_{h,n}$ and $H_{h,i}$, be defined as in the next equations. 
\begin{align*} 
\tagonce\label{eqn_ShorthandFnNotation_Thk_Hhi-defs_exps_v1} 
S_n(h) & \defequals 
     \sum_{i=0}^{2n} \binom{h}{i}^{2} (-1)^{i} i! (2n-i)! \\ 
T_{h,n} & \defequals 
     \prod_{j=1}^{n} \left( 
     \frac{(h-2j)^2 (h+1-2j)^2}{2 \cdot (2h+1-2j) (h-j)} 
     \right) \\ 
     & \phantom{:} = 
     4^{n} \times 
     \frac{\Pochhammer{\frac{1-h}{2}}{n}^2 \Pochhammer{1-\frac{h}{2}}{n}^2}{ 
     \Pochhammer{\frac{1}{2}-h}{n} \Pochhammer{1-h}{n}} = 
     \frac{\Pochhammer{1-h}{2n}^2}{\Pochhammer{1-2h}{2n}} \\ 
H_{h,i} & \defequals 
     \frac{h (h+1) (2h-1)}{(h-1)(h-i)} + 
     \frac{2 (h+1)^2 (2h+1)}{h (2h+1-2i)} + 
     \frac{2 (h+1)}{h (h-1) (h+1-2i)} \\ 
     & \phantom{:} = 
     \frac{(h+1) (2h+1-4i) (h-2i)}{(2h+1-2i) (h+1-2i) (h-i)} 
\end{align*} 
%% 
%% 
Computations with the \SigmaPkg package 
yield the next alternate expansion of the first sum 
defined in \eqref{eqn_ShorthandFnNotation_Thk_Hhi-defs_exps_v1} 
given by 
\begin{align*} 
S_n(h) & = \sum_{i=0}^{n} \binom{h}{2i}^{2} (2i)! \times 
     \frac{T_{h,n}}{T_{h,i}} \times H_{h,i}, 
\end{align*} 
where the ratios of the product functions in the previous equation 
are simplified by 
Lemma \ref{lemma_footnote_PHSymbol_BinomIdents} 
as follows: 
\begin{align*} 
\tagonce\label{eqn_ThiThnFnRatioTerms_PHSymbol_SimplificationIdent} 
\frac{T_{h,n}}{T_{h,i}} & = 
     \frac{\Pochhammer{1-h}{2n}^2}{\Pochhammer{1-2h}{2n}} \times 
     \frac{\Pochhammer{1-2h}{2i}}{\Pochhammer{1-h}{2i}^2} = 
     \frac{\Pochhammer{1-h+2i}{2n-2i}^2}{\Pochhammer{1-2h+2i}{2n-2i}},\ 
     n \geq i. 
\end{align*} 
The forms of the generalized sums, $S_n(h)$, obtained 
from the special case identity above 
using the \SigmaPkg software package routines 
are then expanded by the harmonic--number--related sums over the 
originally fixed indeterminate parameter $h$ in the following forms 
\footnotemod[Alternate Binomial Coefficient Expansions in the Prime Pair Congruence Sums]{ 
      \label{footnote_AltBinomCoeffExps_in_SigmaPrimePairSums} 
      Note that the binomial coefficient identity, 
      $\binom{\binom{k}{2}}{2} = 3 \binom{k+1}{4}$, 
      given in the exercises section of the reference 
      \citep[p.\ 535, Ex.\ 5.67]{GKP}, 
      suggests simplifications of the sums, $S_n(h)$, 
      when $h$ denotes some fixed, implicit application--dependent 
      quadratic function of $n$ obtained by 
      first expanding the inner terms, $\binom{h}{i}$, 
      as a (finite) linear combination of binomial coefficient terms whose 
      upper index corresponds to a linear function of $n$ 
      \citep{SUMMARYNBREF-STUB}. 
      The \SigmaPkg package is able to obtain alternate forms of these 
      pre--processed finite sums defining the functions, 
      $S_n(\beta n+\gamma)$, for scalar--valued $\beta,\gamma$ 
      that generalize the last two expansions provided above in 
      \eqref{eqn_nFactMod2np1_CongruenceIdent_SigmaPkgAltSums}. 
}: 
\begin{align*} 
S_n(h) & = 
     \mathsmaller{ 
     \sum\limits_{i=0}^{n} \binom{h}{2i}^{2} 
     \frac{(2i)! \times 
     \Pochhammer{1-h+2i}{2n-2i}^2}{\Pochhammer{1-2h+2i}{2n-2i}} 
     \times \left( 
     \frac{h (h+1) (2h-1)}{(h-1)(h-i)} + 
     \frac{2 (h+1)^2 (2h+1)}{h (2h+1-2i)} + 
     \frac{2 (h+1)}{h (h-1) (h+1-2i)} 
     \right) 
     } \\ 
%S_n(h) 
     & = 
     \mathsmaller{ 
     \sum\limits_{i=0}^{n} \binom{h}{2n-2i}^{2} 
     (2n-2i)! \times 
     \frac{\Pochhammer{1-h+2n-2i}{2i}^2}{\Pochhammer{1-2h+2n-2i}{2i}} 
     \times \left( 
     \frac{(h+1) (2h+1-4(n-i)) (h-2(n-i))}{(2h+1-2(n-i)) (h+1-2(n-i)) (h-n+i)}
     \right) 
     } 
\end{align*} 
The first sum on the right--hand--side of 
\eqref{eqn_ShorthandFnNotation_Thk_Hhi-defs_exps_v1} 
denotes the special prime pair congruence expansions for the 
twin, cousin, and sexy prime pairs already defined by the 
examples cited in the last sections corresponding to the 
respective forms of \eqref{eqn_GenPrimePairCongFn_hndabc_v1} where 
\[\left(d, a_d, b_d, c_d\right)_{d=1}^{3} \defequals \left\{
 (1, 4, 2, 5), (2, 96, 48, {\color{penguinbook}{119}}), 
 (3, 4320, 1438, 5039) \right\}, 
\] 
and where $h \defmapsto (2n+1)(2n+1+2d)$, 
computed in the reference \citep{SUMMARYNBREF-STUB}. 

\subsubsection{Remarks: Some ideas towards a 
               more general, unified new 
               approach aimed at classifying 
               prime subsequences through conditions on the expansions of 
               harmonic--number--related identities 
               suggested by the previous results} 
\label{subsubSection_remark_MmCompsWith_the_SigmaPkg}

%% 
The sums in the previous several equations provide approaches to these 
congruences for the prime--related subsequences cited as 
applications in the examples above through expansions by 
generalized Stirling number triangles, Stirling and 
Bernoulli polynomials, and the 
corresponding sequences of generalized $r$--order harmonic numbers 
noted in the identities given in 
Section \ref{subsubSection_remark_SNum_R-OrderHNum_SeqExpIdents_spcases_v1} and 
Section \ref{subsubSection_MoreGeneralExps_congruences_multiple_factfns}. 
The expansions of these congruence sums suggest a 
more general approach to the factorial--related sequence expansions 
implicit to many of the examples of the prime--related congruences 
cited as applications so far by generalized harmonic number sequences. 
In particular, 
the expansions of the multiple factorial functions 
underlying the various forms of these congruences 
suggest analogues to known Wolstenholme--prime--like identities, 
results involving expansions of Ap\'{e}ry--like congruence forms, and 
other related necessary conditions involving 
generalized harmonic number sequences for the 
primality of odd integer pairs and other cases of the 
prime--related subsequences. 

\subsection{Applications of Wilson's theorem in 
            other famous and notable special case prime subsequences} 
\label{subsubSection-Examples_SomeResults_for_Prime_k-Tuples} 

\subsubsection{Examples: The Factorial primes and the 
               Fermat prime subsequences} 
\label{subsubSection_example_PrimeSubsequences_ImmediateAppsOfWThm} 

%% 
The sequences of \emph{factorial primes} 
of the form $p \defequals n! \pm 1$ for some $n \geq 1$ 
satisfy congruences of the 
following form modulo $n! \pm 1$ given by the expansions of 
\eqref{eqn_Vandermonde-like_PHSymb_exps_of_PhzCfs-stmt_v1} and 
\eqref{eqn_Chn_formula_stmts} from 
Section \ref{subsubSection_Properties_Of_ConvFn_Phz} 
\citep[\cf \S 2.2]{PRIMEREC} \citeOEIS{A002981,A002982}: 
\begin{align*} 
\tagtext{Factorial Prime Congruences} 
n! + 1 \text{ prime } 
     & \iff && \\ 
     & \phantom{\iff} 
     \underset{\mathlarger{(n!)! \equiv C_{n!+1,n!}(1, 1) \pmod{n!+1}}}{ 
     \underbrace{ 
     \sum_{i=0}^{n!} \binom{n!+1}{i}^{2} (-1)^{i} i! (n!-i)!} 
     } 
     & \equiv -1 && \pmod{n!+1} \\ 
n! - 1 \text{ prime } 
     & \iff 
     \underset{\mathlarger{(n!-2)! \equiv C_{n!-1,n!-2}(1, 1) \pmod{n!-1}}}{ 
     \underbrace{ 
     \sum_{i=0}^{n!-2} \binom{n!-1}{i}^{2} (-1)^{i} i! (n!-2-i)!} 
     } 
     & \equiv -1 && \pmod{n!-1}. 
\end{align*} 
The \emph{Fermat numbers}, $F_n$, 
generating the subsequence of \emph{Fermat primes} of the form 
$p \defequals 2^{m}+1$ where $m = 2^{n}$ for some $n \geq 0$ 
similarly satisfy the next congruences expanded through the 
identities for the single and double factorial functions expanded above 
\citep[\S 2.6]{PRIMEREC} \citep[\S 2.5]{HARDYWRIGHTNUMT} 
\citeOEIS{A000215,A019434}: 
\begin{align*} 
\tagtext{Fermat Prime Congruences} 
F_n \defequals 2^{2^n}+1 \text{ prime } 
     & \iff 
     2^{2^n}+1 \mid \left(2^{2^n}\right)! + 1 & && \\ 
     & \iff 
     2^{2^n}+1 \mid 2^{2^{2^n-1}} 
     \sum_{i=0}^{2^{2^n}} \binom{2^{2^n}+1}{i}^2 (-1)^{i} 
     i! \left(2^{2^n}-i\right)! +1 & && \\ 
     & \iff 
     2^{2^n}+1 \mid 2^{2^{2^n-1}} 
     \left(2^{2^{n}-1}\right)! \left(2^{2^n}-1\right)!! + 1 & && \\ 
     & \iff 
     2^{2^n}+1 \mid 2^{\frac{3}{4} \cdot 2^{2^n}} 
     \left(2^{2^{n}-2}\right)! \left(2^{2^n-1}-1\right)!! 
     \left(2^{2^n}-1\right)!! + 1 & && \\ 
     & \iff 
     2^{2^n}+1 \mid 2^{\frac{7}{8} \cdot 2^{2^n}} 
     \left(2^{2^{n}-3}\right)! 
     \left(2^{2^n-2}-1\right)!! 
     \left(2^{2^n-1}-1\right)!! 
     \left(2^{2^n}-1\right)!! + 1. & && 
\end{align*} 
For integers $h \geq 2$ and $r \geq 1$ such that $2^{r} \mid h$, the 
expansions of the congruences in the previous several equations correspond to 
forming the products of the single and double factorial functions 
modulo $h+1$ from the previous examples to require that 
\begin{align*} 
2^{\left(1-2^{-r}\right) \cdot h} \times \left(\frac{h}{2^{r}}\right)! 
     \left(\frac{h}{2^{r-1}} -1\right)!! 
     \left(\frac{h}{2^{r-2}} -1\right)!! \times \cdots \times 
     \left(\frac{h}{2} -1\right)!! 
     \left(h -1\right)!! & \equiv -1 \pmod{h+1}. 
\end{align*} 
The \emph{generalized Fermat numbers}, 
$F_n(\alpha) \defequals \alpha^{2^n}+1$, and the corresponding 
\emph{generalized Fermat prime} subsequences 
when $\alpha \defequals 2, 4, 6$ suggest generalizations of the 
approach to the results in the previous equations through the 
procedure to the multiple, $\alpha$--factorial function expansions outlined in 
Section \ref{subsubSection_MoreGeneralExps_congruences_multiple_factfns} 
that generalized the procedure to expanding the congruences above for the 
Fermat primes when $\alpha \defequals 2$. 
%% 

\subsubsection{Examples of generalized congruences 
               involving mixed expansions of the 
               single and double factorial functions} 

\begin{example}[Mersenne Primes] 
\label{example_PrimeSubsequences_ImmediateAppsOfWThm_v2} 
%% 
The \emph{Mersenne primes} correspond to prime pairs of the 
form $(p, M_p)$ for $p$ prime and where 
$M_n \defequals 2^{n}-1$ is a \emph{Mersenne number} for some 
(prime) integer $n \geq 2$ 
\citep[\S 2.7]{PRIMEREC} \citep[\S 2.5; \S 6.15]{HARDYWRIGHTNUMT} 
\citep[\cf \S 4.3, \S 4.8]{GKP} 
\citeOEIS{A001348,A000668,A000043}. 
The requirements in Wilson's theorem for the primality of both 
$p$ and $M_p$ provide elementary proofs of the following equivalent 
necessary and sufficient conditions for the primality of the 
prime pairs of these forms: 
\begin{align*} 
\tagtext{Mersenne Prime Congruences} 
\left(p, 2^{p}-1\right) \in \mathbb{P}^{2} 
     & \iff && \\ 
     & \phantom{\iff\ } 
     p(2^p-1) \mid (p-1)! (2^p-2)! + (p-1)! + (2^p-2)! + 1 && \\ 
     & \iff 
     p(2^p-1) \mid \bigl( 
     C_{p(2^p-1),p-1}(-1, p-1) C_{p(2^p-1),2^p-2}(-1, 2^p-2) && \\ 
     & \phantom{\iff p(2^p-1) \mid \bigl( \quad } + 
     C_{p(2^p-1),p-1}(-1, p-1) + C_{p(2^p-1),2^p-2}(-1, 2^p-2) + 1 
     \bigr) && \\ 
     & \iff 
     p(2^p-1) \mid \bigl( 
     C_{p(2^p-1),p-1}(1, 1) C_{p(2^p-1),2^p-2}(1, 1) && \\ 
     & \phantom{\iff p(2^p-1) \mid \bigl( \quad } + 
     C_{p(2^p-1),p-1}(1, 1) + C_{p(2^p-1),2^p-2}(1, 1) + 1 
     \bigr) && \\ 
     & \iff 
     p(2^p-1) \mid 2^{2^{p-1}-1} (p-1)! (2^{p-1}-1)! (2^{p}-3)!! + 
     (p-1)! + (2^p-2)! + 1. && 
\end{align*} 
The congruences on the right--hand--sides of the previous equations 
are then expanded by the results in 
\eqref{eqn_Vandermonde-like_PHSymb_exps_of_PhzCfs-stmt_v1} and 
\eqref{eqn_Chn_formula_stmts} from 
Section \ref{subsubSection_Properties_Of_ConvFn_Phz} 
through the second cases of the more general 
product function congruences stated in 
\eqref{eqn_pnAlphaR_seqs_finite_sum_reps_modulop-stmts_v1} above. 
%% 
\ExampleQED 
\end{example} 

\begin{example}[Sophie Germain Primes] 
%% 
Wilson's theorem similarly implies the next related 
congruence--based characterizations of the 
\emph{Sophie Germain primes} corresponding to the 
prime pairs of the form $(p, 2p+1)$ 
expanded through the results given above when $p, p-1, 2p < p(2p+1)$ 
\citep[\S 5.2]{PRIMEREC} \citeOEIS{A005384}. 
\begin{align*} 
\tagtext{Sophie Germain Prime Congruences} 
\left(p, 2p+1\right) \in \mathbb{P}^{2} & & \\      
     & \iff 
     (p-1)! (2p)! + (p-1)! + (2p)! & \equiv -1 & \pmod{p(2p+1)} \\ 
     & \iff 
     2^p p! (p-1)! (2p-1)!! + (p-1)! + (2p)! & \equiv -1 
     & \pmod{p(2p+1)} 
\end{align*} 
The expansions of the generalized forms of the Sophie Germain primes 
noted in the reference \citep[\S 5.2]{PRIMEREC} 
also provide applications of the 
multiple, $\alpha$--factorial function identities cited in 
Section \ref{subsubSection_MoreGeneralExps_congruences_multiple_factfns} 
of the article below 
which result from expansions of the arithmetic progressions 
of the single factorial functions given in 
Section \ref{subsubSection_Apps_ArithmeticProgs_of_the_SgFactFns}. 
%% 
\ExampleQED 
\end{example} 

\subsubsection{Remarks on related expansions of congruences involving 
               some of the more 
               varied structures implicit to 
               other special case prime subsequences} 
\label{subsubSection_remark_OtherApps_of_WThm_and_NewCongProps_to_PrimeSubseqs} 

%% 
The integer congruences obtained from Wilson's theorem for the 
particular special sequence cases noted in 
Section \ref{subsubSection_example_PrimeSubsequences_ImmediateAppsOfWThm} 
are easily generalized to give constructions over the forms other 
prime subsequences including the following: 
\begin{enumerate} 
     \renewcommand{\itemsep}{-1mm} 

\item 
The \quotetext{\emph{Pierpont primes}} 
of the form $p \defequals 2^{u} 3^{v} + 1$ 
for some $u, v \in \mathbb{N}$ 
\citeOEIS{A005109}; 

\item 
The subsequences of primes of the form $p \defequals n 2^{n} \pm 1$ 
\citeOEIS{A002234,A080075}; 

\item 
The \quotetext{\emph{Wagstuft primes}} 
corresponding to prime pairs of the form 
$\left(p, \frac{1}{3}(2^{p} + 1)\right) \in \mathbb{P}^{2}$ 
\citeOEIS{A000978,A123176}; and 

\item 
The generalized cases of the multifactorial prime sequences tabulated 
as in the reference \citep[Table 6, \S 2.2]{PRIMEREC} 
consisting of prime elements of the form 
$p \defequals \MultiFactorial{n}{\alpha} \pm 1$ for a fixed 
integer--valued $\alpha \geq 2$ and some $n \geq 1$. 

\end{enumerate} 
%% 
Notice that most of the factorial function expansions involved in the 
results formulated by the previous few examples do not 
immediately imply corresponding congruences obtained from 
\eqref{eqn_pnAlphaR_seqs_finite_sum_reps_modulop} 
satisfied by the \emph{Wieferich prime} sequence defined by 
\citep[\S 5.3]{PRIMEREC} \citeOEIS{A001220} 
\begin{align*} 
\tagtext{Wieferich Primes} 
\WieferichPrimeSet & \defequals 
     \left\{ n \geq 2: \text{ $n$ prime \ and \ } 
     2^{n-1} \equiv 1 \pmod{n^2} 
     \right\} \\ 
     & \phantom{\defequals} \quad 
     \seqmapsto{A001220} \left(1093, 3511, \ldots \right), 
\end{align*} 
nor results for the variations of the sequences of 
binomials enumerated by the rational convergent--function--based 
generating function identities over the 
binomial coefficient sums constructed in 
Section \ref{subsubSection_Apps_Example_SumsOfPowers_Seqs} 
modulo prime powers $p^m$ for $m \geq 2$. 

%% 
However, indirect expansions of the sequences of binomials, 
$2^{n-1}$ and $2^{n-1}-1$, by the 
Stirling numbers of the second kind through 
Lemma \ref{lemma_footnote_PHSymbol_BinomIdents}, yield the following 
divisibility requirements characterizing the sequence of 
Wieferich primes defined above, where $\Pochhammer{2}{n} = (n+1)!$: 
\begin{align*} 
2^{n-1} \phantom{-1} 
     & = 
     \sum_{k=0}^{n-1} \gkpSII{n-1}{k} (-1)^{n-1-k} \Pochhammer{2}{k} && \\ 
     & \equiv 
     \sum_{k=0}^{n-1} \sum_{i=0}^{k+1} 
     \gkpSII{n-1}{k} \binom{n^2}{i}^{2} (-1)^{n-1-k-i} i! (k+1-i)! && 
     \pmod{n^2} \\ 
2^{n-1}-1 
     & = 
     2^{n-2} + 2^{n-3} + \cdots + 2 + 1 && \\ 
     & = 
     \sum_{\substack{ 0 \leq j \leq i \leq n-2}} 
     \gkpSII{i}{j} (-1)^{i-j} \Pochhammer{2}{j} && \\ 
     & \equiv 
     \sum_{\substack{ 0 \leq m \leq j \leq i \leq n-2}} 
     \gkpSII{i}{j} \binom{n^2}{m} (-1)^{i-j+m} 
     \FFactII{(n^2+1)}{m} \Pochhammer{2}{j-m} && 
     \pmod{n^2}. 
\end{align*} 
The constructions of the corresponding congruences 
for the sequences of binomials, $a^{n-1} - 1 \pmod{n^2}$, are 
obtained by a similar procedure \citep[\cf \S 5.3; Table 45]{PRIMEREC}. 
Additional expansions of related congruences for the terms 
$3^{t} + 1 \pmod{2t+1}$ for primes $2t+1 \in \mathbb{P}$, 
for example, in the particular forms of 
\begin{align*} 
3^{t} + 1 
     & \equiv 
     \phantom{4 \times} 
     \sum_{j=0}^{t} \sum_{i=0}^{j} 
     \gkpSII{t}{j} \binom{2t+1}{m} (-1)^{t-j+m} 
     \FFactII{(2t+3)}{m} \Pochhammer{3}{j-m} + 1 && 
     \pmod{2t+1} \\ 
3^{t} + 1     
     & \equiv 
     4 \times \sum_{m=0}^{t-1} \sum_{j=0}^{m} \sum_{i=0}^{j} 
     \gkpSII{m}{j} \binom{2t+1}{i} (-1)^{t-1-j+i} 
     \FFactII{(2t+3)}{i} \Pochhammer{3}{j-i} && 
     \pmod{2t+1}, 
\end{align*} 
where $\Pochhammer{3}{j} = \frac{1}{2} (i+2)!$, 
lead to double and triple sums providing the necessary and sufficient 
condition for the primality of the Fermat primes, $F_k$, 
from the reference \citep[\S 6.14]{HARDYWRIGHTNUMT} 
when $t \defmapsto 2^{2^{k}-1}$ for some integer $k \geq 1$. 


\subsection{More general expansions of the new congruence results by 
            multiple factorial functions and 
            generalized harmonic number sequences} 
\label{subsubSection_MoreGeneralExps_congruences_multiple_factfns} 

\subsubsection{Expansions of factorial--related congruence identities by 
               $\alpha$--factorial functions} 

%% 
Additional identities formed as cases of other known 
prime--related congruences involving both the 
single and double factorial functions suggest even further 
applications of the properties of the $\alpha$--factorial 
functions phrased for the more general sequences cases by 
\eqref{eqn_pnAlphaR_seqs_finite_sum_reps_modulop} from above. 
If the form the odd primes $p \defequals an+r$ is known 
in the previous congruences cited in the examples from the previous 
subsections, the 
expansions by the arithmetic progressions of the 
single factorial function given in 
Section \ref{subSection_DiagonalGFSequences_Apps} 
suggest applications of the generalized Stirling number triangles 
in \eqref{eqn_Fa_rdef} to interpreting the new forms of these 
congruence results. 

\subsubsection{Example: Expansions of congruences for 
               generalized cases of Sophie Germain prime pairs} 

%% 
Generalizations of the congruences for the 
Sophie Germain primes given in 
Example \ref{example_PrimeSubsequences_ImmediateAppsOfWThm_v2} 
of the previous section provide the following analogs results for 
prime pairs of the form $(p, 2kp+1)$ when $k \defequals 2, 3$ 
\citep[\S 5.2]{PRIMEREC}: 
\begin{align*} 
\left(p, 4p+1\right) \in \mathbb{P}^{2} 
     & \iff 
     (p-1)! (4p)! + (p-1)! + (4p)! \equiv -1 \pmod{p(4p+1)} \\ 
     & \iff 
     \mathsmaller{ 
     p(4p+1) \mid (p-1)! \left(1 + 
     \AlphaFactorial{4p}{4} \AlphaFactorial{4p-1}{4} 
     \AlphaFactorial{4p-2}{4} \AlphaFactorial{4p-3}{4} 
     \right) 
     } \\ 
     & \phantom{\iff p(4p+1) \mid \quad} + 
     \mathsmaller{ 
     (4p)! + 1 
     } \\ 
     & \iff 
     \mathsmaller{ 
     p(4p+1) \mid (p-1)! \left(1 + 
     4^{4p} \Pochhammer{1}{p} \Pochhammer{\frac{1}{4}}{p} 
     \Pochhammer{\frac{1}{2}}{p} \Pochhammer{\frac{3}{4}}{p} 
     \right) + 
     (4p)! + 1 
     } \\ 
\left(p, 6p+1\right) \in \mathbb{P}^{2} 
     & \iff 
     (p-1)! (6p)! + (p-1)! + (6p)! \equiv -1 \pmod{p(6p+1)} \\ 
     & \iff 
     \mathsmaller{ 
     p(6p+1) \mid C_{p(6p+1),p-1}(-1, p-1) \left(1 + 
     \prod\limits_{i=0}^{5} C_{p(6p+1),p}(-6, 6p-i) \right) 
     } \\ 
     & \phantom{\iff p(6p+1) \mid \quad} + 
     \mathsmaller{ 
     C_{p(6p+1),6p}(-1, 6p) + 1 
     } \\ 
     & \iff 
     \mathsmaller{ 
     p(6p+1) \mid C_{p(6p+1),p-1}(-1, p-1) \left(1 + 
     \prod\limits_{i=1}^{6} 6^{p} \times C_{p(6p+1),p}(6, i) \right) 
     } \\ 
     & \phantom{\iff p(6p+1) \mid \quad} + 
     \mathsmaller{ 
     C_{p(6p+1),6p}(-1, 6p) + 1 
     }. 
\end{align*} 

\subsubsection{Expansions of the $\alpha$--factorial triangles by 
               generalized harmonic numbers} 

The noted relations of the divisibility of the 
Stirling numbers of the first kind to the 
$r$--order harmonic number sequences expanded by the special cases from 
\eqref{eqn_S1k234_HNum_exp_idents-restmts_v1} 
are generalized to the $\alpha$--factorial function coefficient 
cases through the following forms of the 
exponential generating functions given in 
\eqref{eqn_FcfIIAlphanp1mp1_two-variable_EGFwz-footnote_v1} of 
Section \ref{subSection_GenAlphaFactorialTriangle_exps} 
\citep[\cf \S 3.3]{MULTIFACTJIS}: 
\begin{align*} 
\tagonce\label{eqn_FcfIIAlphanp1mp1_two-variable_EGFwz} 
\sum_{n \geq 0} \FcfII{\alpha}{n+1}{m+1} \frac{z^n}{n!} & = 
          \frac{(1- \alpha z)^{-1 / \alpha}}{m! \cdot \alpha^{m}} \times 
          \Log\left(\frac{1}{1-\alpha z}\right)^{m}. 
\end{align*} 
The special cases of these coefficients generated by the 
previous equation when $m \defequals 1,2$ are then expanded by the 
sums involving the $r$--order harmonic number sequences in the 
following equations: 
\begin{align*} 
\tagonce\label{eqn_FcfAlphaGenCoeffs_HNumExpIdents-stmts_v1} 
\FcfII{\alpha}{n+1}{2} \frac{1}{n!} & = 
     \alpha^{n-1} \times \sum_{k=0}^{n} 
     \binom{1-\frac{1}{\alpha}}{k} (-1)^{k} H_{n-k} \\ 
     %& = 
     %\frac{\alpha^{n} \cdot \Pochhammer{\frac{1}{\alpha}}{n}}{n!} \times 
     %\sum_{1 \leq k \leq n} \frac{1}{(\alpha k+1-\alpha)} \\ 
\FcfII{\alpha}{n+1}{3} \frac{1}{n!} & = 
     \frac{\alpha^{n-2}}{2} \times \sum_{k=0}^{n} 
     \binom{1-\frac{1}{\alpha}}{k} (-1)^{k} \left( 
     H_{n-k}^2 - H_{n-k}^{(2)} 
     \right). 
     %& = 
     %\frac{\alpha^{n} \cdot \Pochhammer{\frac{1}{\alpha}}{n}}{2 \cdot n!} 
     %\times \left( 
     %\left(\sum_{1 \leq k \leq n} \frac{1}{(\alpha k+1-\alpha)}\right)^2 - 
     %\sum_{1 \leq k \leq n} \frac{1}{(\alpha k+1-\alpha)^{2}} 
     %\right). 
\end{align*} 
Identities providing expansions of the 
generalized $\alpha$--factorial triangles in \eqref{eqn_Fa_rdef} 
at other specific cases of the lower indices $m \geq 3$ 
that involve the slightly generalized cases of the 
$r$--order harmonic number sequences, $H_{\alpha,n}^{(r)}$, 
defined by the next equation 
are expanded through related constructions. 
\begin{align*} 
\tagtext{Generalized Harmonic Number Definitions} 
H_{\alpha,n}^{(r)} & \defequals 
     \sum_{k=1}^{n} \frac{1}{(\alpha k+1-\alpha)^{r}},\ 
     n \geq 1, \alpha, r > 0 
\end{align*} 
For comparison with the Stirling number identities noted in 
\eqref{eqn_S1k234_HNum_exp_idents-restmts_v1} above, the 
first few cases of the coefficient identities in 
\eqref{eqn_FcfAlphaGenCoeffs_HNumExpIdents-stmts_v1} 
are expanded explicitly by these more general integer--order 
harmonic number sequence cases in the following equations: 
\begin{align*} 
\FcfII{\alpha}{n+1}{2} \frac{1}{n!} & = 
     \alpha^{n} \binom{n+\frac{1-\alpha}{\alpha}}{n} \times 
     H_{n,\alpha}^{(1)} \\ 
\FcfII{\alpha}{n+1}{3} \frac{1}{n!} & = 
     \frac{\alpha^{n}}{2} \binom{n+\frac{1-\alpha}{\alpha}}{n} \times \left( 
     \left(H_{n,\alpha}^{(1)}\right)^{2} - H_{n,\alpha}^{(2)} 
     \right) \\ 
\FcfII{\alpha}{n+1}{4} \frac{1}{n!} & = 
     \frac{\alpha^{n}}{6} \binom{n+\frac{1-\alpha}{\alpha}}{n} \times \left( 
     \left(H_{n,\alpha}^{(1)}\right)^{3} - 
     3 H_{n,\alpha}^{(1)} H_{n,\alpha}^{(2)} + 
     2 H_{n,\alpha}^{(3)} 
     \right). 
\end{align*} 
%% 
When $\alpha \defequals 2$, we have a relation between the 
sequences, $H_{n,\alpha}^{(r)}$, and the $r$--order harmonic numbers of the 
form $H_{2,n}^{(r)} = H_{2n}^{(r)} - 2^{-r} H_n^{(r)}$, which yields 
particular coefficient expansions for the double factorial 
functions involved in stating several of the 
congruence results from the examples given above. 
The expansions of the prime--related congruences involving the 
double factorial function cited in 
Section \ref{subsubSection-example_OtherRelatedCongruences_DblFactFns} 
above also suggest additional applications to finding 
integer congruence properties and necessary conditions 
involving these harmonic number sequences related to other more general 
forms of these expansions for prime pairs and prime--related subsequences 
(see 
Section \ref{subsubSection_remark_MmCompsWith_the_SigmaPkg}). 

\section{Conclusions} 
\label{Section_ConcludingRemarks} 

\subsection{Concluding remarks} 

We have defined several new forms of 
ordinary power series approximations to the 
typically divergent ordinary generating functions of 
generalized multiple, or $\alpha$--factorial, function sequences. 
The generalized forms of these convergent functions 
provide partial truncated approximations to the sequences formally 
enumerated by these divergent power series. 
The exponential generating functions for the 
special case product sequences, $p_n(\alpha, s-1)$, 
are studied in the reference \citep[\S 5]{MULTIFACTJIS}. 
The exponential generating functions that enumerate the 
cases corresponding to the more general factorial--like sequences, 
$p_n(\alpha, \beta n + \gamma)$, are less obvious in form. 
We have also suggested a number of new, alternate 
approaches to enumerating the factorial function sequences that arise 
in applications, including 
classical identities involving the single and double factorial functions, and 
in the forms of several other noteworthy special cases. 

The key ingredient to the short proof given in 
Section \ref{Section_Proofs_of_the_GenCFracReps} 
employs known characterizations of the 
Pochhammer symbols, $\Pochhammer{x}{n}$, by 
generalized Stirling number triangles as 
polynomial expansions in the indeterminate, $x$, 
each with predictably small finite--integral--degree at any fixed $n$. 
The more combinatorial proof in the spirit of Flajolet's articles 
suggested by the discussions in 
Section \ref{subSection_GenCFrac_Reps_for_GenFactFns} 
may lead to further interesting interpretations of the 
$\alpha$--factorial functions, $(s-1)!_{(\alpha)}$, 
which motivate the investigations of the coefficient-wise 
symbolic polynomial expansions of the functions first considered in the 
article \citep{MULTIFACTJIS}. 
A separate proof of the expansions of these new continued fractions 
formulated in terms of the 
generalized $\alpha$--factorial function coefficients defined by 
\eqref{eqn_Fa_rdef}, and by their strikingly Stirling--number--like 
combinatorial properties motivated in the introduction, 
is notably missing from this article. 

The rationality of these convergent functions for all $h$ 
suggests new insight to generating the numeric sequences of interest, 
including several specific new congruence properties, derivations of 
finite difference equations that hold for these exact sequences 
modulo any integers $p$, and perhaps more interestingly, 
exact expansions of the classical single and double factorial functions 
by the special zeros of the generalized Laguerre polynomials and 
confluent hypergeometric functions. 
The techniques behind the specific identities given here 
are easily generalized and extended to further specific applications. 
The particular examples cited within this article 
are intended as suggestions at new 
starting points to tackling the expansions that arise in 
many other practical situations, both implicitly and explicitly 
involving the generalized variants of the factorial--function--like 
product sequences, $p_n(\alpha, R)$. 

\subsection{Topics for future research suggested by the article} 

\subsubsection{Generalizations of known finite sum identities involving the 
               double factorial function} 
\label{subsubSection_FutureResTopics_GenDblFactFnSumIdents_FiniteSums} 

The construction of further 
analogues for generalized variants of the finite summations and 
more well--known combinatorial identities satisfied by the 
double factorial function cases when $\alpha \defequals 2$ from the 
references is suggested as a topic for future investigation in 
Section \ref{subsubSection_FutureResTopics_GenDblFactFnSumIdents_FiniteSums}. 
The identities for the more general $\alpha$--factorial function cases 
stated in 
Example \label{example_GenDblFactFnSumIdents_FiniteSumsInvolving_AlphaFactFns} of 
Section \ref{ssS_example_GenDblFactFnSumIdents_FiniteSumsInvolving_AlphaFactFns} 
suggest one possible approach to generalizing the 
known identities summarized in the references 
\citep{MAA-FUN-WITH-DBLFACT,DBLFACTFN-COMBIDENTS-SURVEY} for the 
next few particularly interesting special cases corresponding to the 
triple and quadruple factorial function cases, $n!!!$ and $n!!!!$, 
respectively. 

\subsubsection{Generalizations of the Stirling number 
               congruence results in 
               Section \ref{subsubSection_remark_New_Congruences_for_GenS1Triangles_and_HNumSeqs}} 
\label{subsubSection_FutureResTopics_GenSNumCongResults} 

     The approach to the rational convergent functions in $z$ by the 
     finite difference equations outlined in the applications of 
     Section \ref{subSection_FiniteDiffEqns_for_the_GenFactFns} 
     provides related expansions of the series coefficients of the 
     $h^{th}$ convergents, $\ConvGF{h}{\alpha}{R}{z}$, which are also 
     rational in $R$ for all $h \geq 1$ when $\alpha \neq 0$ is fixed. 
     The corresponding properties of the coefficients of these 
     generalized convergent functions in 
     \eqref{eqn_S1FcfAlpha_GenConvFn_Coeffs_and_CongruencesResultStmts} 
     with respect to formal power series expansions in $R$ 
     as functions of $\alpha$ and $z$ are not explored in 
     depth within this article. 

\subsubsection{S--fractions for the Catalan numbers and 
               central binomial coefficients} 
\label{subsubSection_footnote_CatalanNumber_S-Fraction_Apps} 

     The rational convergents to the S--fraction series for the 
     ordinary generating function of the 
     Catalan numbers, $C_n = \binom{2n}{n} \frac{1}{(n+1)}$, 
     defined by 
     Section \ref{subSection_Intro_Examples_DivergentCFracOGFs} and 
     in the references suggests alternate 
     continued--fraction--based approaches to the 
     congruences satisfied by these coefficients modulo the 
     integer powers $n^{p}$ utilizing the methods employed to stated the 
     particular examples above 
     \citep[\S 5.5]{GFLECT} \citep[Prop.\ 5]{FLAJOLET80B} 
     \citep[\cf \S 5.3]{GKP}. 
     The summary notebook document contains specific examples 
     related to these continued fraction series for the 
     Catalan numbers and central binomial coefficients 
     \citep{SUMMARYNBREF-STUB} \citeOEIS{A000108,A000984}. 

\subsubsection{Remarks and notes on other identities 
               contained in the summary and 
               computational documentation references} 
\label{subsubSection_FutureResTopics_Rmks_in_SummaryNB} 

The summary notebook document prepared with this 
manuscript contains additional remarks and examples related to the 
results in the article. 
For example, several specific expansions of the 
binomial coefficients, $\binom{(2n+1)(2n+2d+1)}{i}$, at 
upper index inputs varying quadratically on $n$ suggested by 
footnote \ftref{footnote_AltBinomCoeffExps_in_SigmaPrimePairSums} 
on page \pageref{footnote_AltBinomCoeffExps_in_SigmaPrimePairSums} 
are computed as a starting point for simplifying the terms in these 
sums in the reference \citep{SUMMARYNBREF-STUB}. 
Additional notes providing documentation and more detailed 
computational examples will be added to 
updated versions of the summary reference. 

\section{Acknowledgments} 
\label{Section_Acks} 

The original work on the article is an extension of the 
forms of the generalized factorial function variants considered in my article 
published in the \emph{Journal of Integer Sequences} (2010). 
The research on the continued fraction representations for the 
generalized factorial functions considered in this article 
began as the topic for my final project in the 
\emph{Introduction to Mathematical Research} course at the 
University of Illinois at Urbana--Champaign 
around the time of the first publication. 
Thanks especially to Professor Bruce Reznick at the 
University of Illinois at Urbana--Champaign, and 
also to Professor Jimmy Mc Laughlin, 
for their helpful input on revising previous drafts of the article. 

\renewcommand{\refname}{References} 
\setlength{\bibsep}{0.05in} 

%\bibliographystyle{jis}  
%\bibliography{multifact-cfracs-updated} 

\begin{thebibliography}{10}

\bibitem{CONTRIB-THEORY-BARNESGFN}
V.~S. Adamchik, Contributions to the theory of the {B}arnes function, 2003, 
\url{http://arxiv.org/abs/math/0308086v1}. 
%{\bf abs/0308086v1} (2003).

\bibitem{GENWTHM-DBLHYPERSUPER-FACTFNS}
C.~Aebi and G.~Cairns, Generalizations of {W}ilson's theorem for
  double--, hyper--, sub--, and superfactorials, {\em The American Mathematical
  Monthly} {\bf 122}(5) (2015), 433--443.

\bibitem{PROPS-ZEROS-CHYPFNS80}
S.~Ahmed, Properties of the zeros of confluent hypergeometric functions, {\em
  Journal of Approximation Theory} (34) (1980), 335--347.

\bibitem{DBLFACTFN-COMBIDENTS-SURVEY}
D.~Callan, A combinatorial survey of combinatorial identities for the 
double factorial, 2009, \url{http://arxiv.org/abs/0906.1317}. 
%{\bf abs/0906.1317} (2009).

\bibitem{CLEMENTPRIMES}
P.~A. Clement, Congruences for sets of primes, {\em Amer. Math. Monthly} {\bf
  56}(1) (1949), 23--25.

\bibitem{ADVCOMB}
L.~Comtet, {\em Advanced Combinatorics: The Art of Finite and Infinite
  Expressions}, D. Reidel Publishing Company, 1974.

\bibitem{ON-HYPGEOMFNS-PHKSYMBOL}
R.~Diaz and E.~Pariguan, 
On hypergeometric functions and $k$--{P}ochhammer symbol, 2005, 
\url{http://arxiv.org/abs/math/0405596v2}. 
%{\bf abs/0405596v2} (2005).

\bibitem{ACOMB-BOOK}
P.~Flajolet and R.~Sedgewick, {\em Analytic Combinatorics}, Cambridge
  University Press, 2009 (Third printing 2010).

\bibitem{FLAJOLET80B}
P.~Flajolet, Combinatorial aspects of continued fractions, {\em Discrete
  Mathematics} {\bf 32} (1980), 125--161.

\bibitem{FLAJOLET82}
P.~Flajolet, On congruences and continued fractions for some classical
  combinatorial quantities, {\em Discrete Mathematics} {\bf 41}(2) (1982),
  145--153.

\bibitem{LGWORKS-ASYMP-SPFNZEROS2008}
W.~Gautschi and C.~Giordano, {L}uigi {G}atteschi's work on asymptotics of
  special functions and their zeros, {\em Numer Algor}  (2008).

\bibitem{MAA-FUN-WITH-DBLFACT}
H.~Gould and J.~Quaintance, Double fun with double factorials, {\em
  Mathematics Magazine} {\bf 85}(3) (2012), 177--192.

\bibitem{GKP}
R.~L. Graham, D.~E. Knuth, and O.~Patashnik, {\em Concrete Mathematics: A
  Foundation for Computer Science}, Addison-Wesley, 1994.

\bibitem{HARDYWRIGHTNUMT}
G.~H. Hardy and E.~M. Wright, {\em An Introduction to the Theory of
  Numbers}, Oxford University Press, 2008 (Sixth Edition).

\bibitem{CVLPOLYS}
D.~E. Knuth, Convolution polynomials, {\em The Mathematica J.} {\bf 2}(4)
  (1992), 67--78.

\bibitem{TAOCPV1}
D.~E. Knuth, {\em The Art of Computer Programming: Fundamental Algorithms},
  Vol.~1, Addison-Wesley, 1997.

\bibitem{GFLECT}
S.~K. Lando, {\em Lectures on Generating Functions}, American Mathematical
  Society, 2002.

\bibitem{ONWTHM-AND-POLIGNAC-CONJ}
C. Lin and L. Zhipeng, On {W}ilson's theorem and {P}olignac conjecture, {\em
  Math. Medley} {\bf 6}(1) (2005).

\bibitem{NISTHB}
F.~W.~J. Olver, D.~W. Lozier, R.~F. Boisvert, and C.~W. Clark,
  editors, {\em {NIST} Handbook of Mathematical Functions}, Cambridge
  University Press, 2010.

\bibitem{PRIMEREC}
P.~Ribenboim, {\em The New Book of Prime Number Records}, Springer, 1996.

\bibitem{UC}
S.~Roman, {\em The Umbral Calculus}, Dover, 1984.

\bibitem{MULTIFACTJIS}
M.~D. Schmidt, Generalized $j$--factorial functions, polynomials, and
  applications, {\em J. Integer Seq.} {\bf 13}(10.6.7) (2010).

\bibitem{SUMMARYNBREF-STUB}
M.~D. Schmidt, {\em Mathematica summary notebook, 
supplementary reference, and computational documentation}, 2015, 
Source code and detailed computational documentation to appear online. 
Temporary access to the summary notebook file for review of the article 
is available online at the {\em Google Drive} link: 
\href{\TheSummaryNBFileGoogleDriveLink}{\texttt{summary-abbrv-working-2015.10.30-v1.nb}}. 

\bibitem{SYMB-SUM-COMB-SIGMAPKGDOCS}
C.~Schneider, Symbolic summation assists combinatorics, {\em Sem.~Lothar.
  Combin.} {\bf 56} (2007), 1--36.

\bibitem{OEIS}
N.~J.~A. Sloane, The {O}nline {E}ncyclopedia of {I}nteger {S}equences, 2010, 
\url{http://oeis.org}. 

\bibitem{ATLASOFFUNCTIONS}
J.~Spanier and K.~B. Oldham, {\em An Atlas of Functions}, Taylor \&
  Francis / Hemisphere, Bristol, PA, USA, 1987.

\bibitem{ECV2}
R.~P. Stanley, {\em Enumerative Combinatorics}, Vol.~2, Cambridge, 1999.

\bibitem{GFOLOGY}
H.~S. Wilf, {\em Generatingfunctionology}, Academic Press, 1994.

\bibitem{WOLFRAMFNSSITE-INTRO-FACTBINOMS}
{W}olfram Functions~Site, 
{\em Introduction to the factorials and binomials}, 2015, 
\url{http://funtions.wolfram.com/GammaBetaErf/Pochhammer/introductions/FactorialBinomials/ShowAll.html}.

\end{thebibliography}

\bigskip\hrule\bigskip 
\noindent 
%% 05-XX: Combinatorics: 
%% 05Axx : Enumerative combinatorics: 
%% 05A10: Factorials, binomial coefficients, combinatorial functions; 
%% 05A15: Exact enumeration problems, generating functionsl; 
%% 11-XX: Number theory: 
%% 11Axx: Elementary number theory: 
%% 11A41: Primes; 
%% 11A51: Factorization; primality; 
%% 11A55: Continued fractions; 
%% 11Bxx: Sequences and sets: 
%% 11B37: Recurrences; 
%% 11B50: Sequences (mod $m$); 
%% 11B65: Binomial coefficients; factorials; $q$-identities; 
%% 11B73: Bell and Stirling numbers; 
%% 11B83: Special sequences and polynomials; 
%% 11Yxx: Computational number theory: 
%% 11Y11: Primality; 
%% 11Y55: Calculation of integer sequences; 
%% 11Y65: Continued fraction calculations; 
%% 11Y99: None of the above, but in this section; 
\textit{2010 Mathematics Subject Classification}: 
Primary 05A10; Secondary 05A15, 11A55, 11Y55, 11Y65, 11B65. \\ 
%% 
\textit{Keywords}: 
Continued fraction, J-fraction, S-fraction, 
Pochhammer symbol, factorial function, 
multifactorial, multiple factorial, 
double factorial, superfactorial, 
rising factorial, Pochhammer k-symbol, 
Barnes G-function, hyperfactorial, subfactorial, triple factorial, 
generalized Stirling number, Stirling number of the first kind, 
Wilson's theorem, Clement's theorem, 
confluent hypergeometric function, Laguerre polynomial; 
ordinary generating function, diagonal generating function, 
Hadamard product, divergent ordinary generating function, 
formal Laplace-Borel transform, 
binomial coefficient congruence, Stirling number congruence. 

\bigskip\hrule\bigskip 

\noindent 
(Concerned with sequences
%\seqnum{A000043}, \seqnum{A000108}, \seqnum{A000142}, \seqnum{A000165}, 
%\seqnum{A000166}, \seqnum{A000178}, \seqnum{A000215}, \seqnum{A000225}, 
%\seqnum{A000407}, \seqnum{A000668}, \seqnum{A000918}, \seqnum{A000978}, 
%\seqnum{A000984}, \seqnum{A001008}, \seqnum{A001044}, \seqnum{A001097}, 
%\seqnum{A001147}, \seqnum{A001220}, \seqnum{A001359}, \seqnum{A001448}, 
%\seqnum{A002109}, \seqnum{A002144}, \seqnum{A002234}, \seqnum{A002496}, 
%\seqnum{A002805}, \seqnum{A002981}, \seqnum{A002982}, \seqnum{A003422}, 
%\seqnum{A005109}, \seqnum{A005165}, \seqnum{A005384}, \seqnum{A006512}, 
%\seqnum{A006882}, \seqnum{A007406}, \seqnum{A007407}, \seqnum{A007408}, 
%\seqnum{A007409}, \seqnum{A007540}, \seqnum{A007559}, \seqnum{A007619}, 
%\seqnum{A007661}, \seqnum{A007662}, \seqnum{A007696}, \seqnum{A008275}, 
%\seqnum{A008277}, \seqnum{A008292}, \seqnum{A008554}, \seqnum{A009120}, 
%\seqnum{A009445}, \seqnum{A010050}, \seqnum{A019434}, \seqnum{A023200}, 
%\seqnum{A023201}, \seqnum{A023202}, \seqnum{A023203}, \seqnum{A024023}, 
%\seqnum{A024036}, \seqnum{A024049}, \seqnum{A027641}, \seqnum{A027642}, 
%\seqnum{A032031}, \seqnum{A033312}, \seqnum{A034176}, \seqnum{A046133}, 
%\seqnum{A047053}, \seqnum{A061062}, \seqnum{A066802}, \seqnum{A077800}, 
%\seqnum{A078303}, \seqnum{A080075}, \seqnum{A085157}, \seqnum{A085158}, 
%\seqnum{A087755}, \seqnum{A088164}, \seqnum{A094638}, \seqnum{A100043}, 
%\seqnum{A100089}, \seqnum{A100732}, \seqnum{A104344}, \seqnum{A123176}, 
%\seqnum{A130534}, \seqnum{A157250}, \seqnum{A166351}, 
%and \seqnum{A184877}.
%\seqnum{A000043}, \seqnum{A000108}, \seqnum{A000142}, \seqnum{A000165}, 
%\seqnum{A000166}, \seqnum{A000178}, \seqnum{A000215}, \seqnum{A000225}, 
%\seqnum{A000407}, \seqnum{A000668}, \seqnum{A000918}, \seqnum{A000978}, 
%\seqnum{A000984}, \seqnum{A001008}, \seqnum{A001044}, \seqnum{A001097}, 
%\seqnum{A001147}, \seqnum{A001220}, \seqnum{A001359}, \seqnum{A001448}, 
%\seqnum{A002109}, \seqnum{A002144}, \seqnum{A002234}, \seqnum{A002496}, 
%\seqnum{A002805}, \seqnum{A002981}, \seqnum{A002982}, \seqnum{A003422}, 
%\seqnum{A005109}, \seqnum{A005165}, \seqnum{A005384}, \seqnum{A006512}, 
%\seqnum{A006882}, \seqnum{A007406}, \seqnum{A007407}, \seqnum{A007408}, 
%\seqnum{A007409}, \seqnum{A007540}, \seqnum{A007559}, \seqnum{A007619}, 
%\seqnum{A007661}, \seqnum{A007662}, \seqnum{A007696}, \seqnum{A008275}, 
%\seqnum{A008277}, \seqnum{A008292}, \seqnum{A008554}, \seqnum{A009120}, 
%\seqnum{A009445}, \seqnum{A010050}, \seqnum{A019434}, \seqnum{A022004}, 
%\seqnum{A022005}, \seqnum{A023200}, \seqnum{A023201}, \seqnum{A023202}, 
%\seqnum{A023203}, \seqnum{A024023}, \seqnum{A024036}, \seqnum{A024049}, 
%\seqnum{A027641}, \seqnum{A027642}, \seqnum{A032031}, \seqnum{A033312}, 
%\seqnum{A034176}, \seqnum{A046118}, \seqnum{A046124}, \seqnum{A046133}, 
%\seqnum{A047053}, \seqnum{A061062}, \seqnum{A066802}, \seqnum{A077800}, 
%\seqnum{A078303}, \seqnum{A080075}, \seqnum{A085157}, \seqnum{A085158}, 
%\seqnum{A087755}, \seqnum{A088164}, \seqnum{A094638}, \seqnum{A100043}, 
%\seqnum{A100089}, \seqnum{A100732}, \seqnum{A104344}, \seqnum{A123176}, 
%\seqnum{A130534}, \seqnum{A157250}, \seqnum{A166351}, and \seqnum{A184877}.
\seqnum{A000043}, \seqnum{A000108}, \seqnum{A000142}, \seqnum{A000165}, 
\seqnum{A000166}, \seqnum{A000178}, \seqnum{A000215}, \seqnum{A000225}, 
\seqnum{A000407}, \seqnum{A000668}, \seqnum{A000918}, \seqnum{A000978},
\seqnum{A000984}, \seqnum{A001008}, \seqnum{A001044}, \seqnum{A001097}, 
\seqnum{A001142}, 
\seqnum{A001147}, \seqnum{A001220}, \seqnum{A001348}, \seqnum{A001359}, 
\seqnum{A001448}, \seqnum{A002109}, \seqnum{A002144}, \seqnum{A000215}, 
\seqnum{A002234}, \seqnum{A002496}, \seqnum{A002805}, \seqnum{A002981}, 
\seqnum{A002982}, \seqnum{A003422}, \seqnum{A005109}, \seqnum{A005165}, 
\seqnum{A005384}, \seqnum{A006512}, \seqnum{A006882}, \seqnum{A007406}, 
\seqnum{A007407}, \seqnum{A007408}, \seqnum{A007409}, \seqnum{A007540}, 
\seqnum{A007559}, \seqnum{A007619}, \seqnum{A007661}, \seqnum{A007662}, 
\seqnum{A007696}, \seqnum{A008275}, \seqnum{A008277}, \seqnum{A008292}, 
\seqnum{A008544}, \seqnum{A008554}, \seqnum{A009120}, \seqnum{A009445}, 
\seqnum{A010050}, \seqnum{A019434}, \seqnum{A022004}, \seqnum{A022005}, 
\seqnum{A023200}, \seqnum{A023201}, \seqnum{A023202}, \seqnum{A023203}, 
\seqnum{A024023}, \seqnum{A024036}, \seqnum{A024049}, \seqnum{A027641}, 
\seqnum{A027642}, \seqnum{A032031}, \seqnum{A033312}, \seqnum{A034176}, 
\seqnum{A046118}, \seqnum{A046124}, \seqnum{A046133}, \seqnum{A047053}, 
\seqnum{A061062}, \seqnum{A066802}, \seqnum{A077800}, \seqnum{A078303}, 
\seqnum{A080075}, \seqnum{A085157}, \seqnum{A085158}, \seqnum{A087755}, 
\seqnum{A088164}, \seqnum{A094638}, \seqnum{A100043}, \seqnum{A100089}, 
\seqnum{A100732}, \seqnum{A104344}, \seqnum{A123176}, \seqnum{A130534}, 
\seqnum{A157250}, \seqnum{A166351}, and \seqnum{A184877}.
) 

\newpage 

\setcounter{section}{0} 
\renewcommand{\thesection}{\Alph{section}} 
\section{Appendix: Tables referenced within the article} 
\label{Section_appendix_StartOfTableData} 
\label{page_StartOfTableData} 

\begin{table}[ht] 

\smaller 
\centering 

\begin{subtable}{\subtablewidth} 
\centering 

%\begin{tabular}{|c||lllllll|} \hline\hline 
%\trianglenk{n}{k} & 0 & 1 & 2 & 3 & 4 & 5 & 6 \\ \hline 
%0 & 1 &       &       &     &     &    &    \\ 
%1 & 0 & 1     &       &     &     &    &    \\ 
%2 & 0 & 1     & 1     &     &     &    &    \\ 
%3 & 0 & 2     & 3     & 1   &     &    &    \\ 
%4 & 0 & 6     & 11    & 6   & 1   &    &    \\ 
%5 & 0 & 24    & 50    & 35  & 10  & 1  &    \\ 
%6 & 0 & 120   & 274   & 225 & 85  & 15 & 1  \\ \hline\hline 
%\end{tabular} 
%\subcaption{%The Single Factorial Function Triangle of 
%            The Stirling numbers of the first kind, 
%            $\FcfII{1}{n}{k} = \gkpSI{n}{k}$} 
%
%\subtableskip 

\begin{tabular}{|c||lllllllll|} \hline\hline 
\trianglenk{n}{k} 
  & 0 & 1      & 2      & 3      & 4     & 5     & 6   & 7  & 8 \\ \hline 
0 & 1 &        &        &        &       &       &     &    &   \\ 
1 & 0 & 1      &        &        &       &       &     &    &   \\ 
2 & 0 & 1      & 1      &        &       &       &     &    &   \\ 
3 & 0 & 3      & 4      & 1      &       &       &     &    &   \\ 
4 & 0 & 15     & 23     & 9      & 1     &       &     &    &   \\ 
5 & 0 & 105    & 176    & 86     & 16    & 1     &     &    &   \\ 
6 & 0 & 945    & 1689   & 950    & 230   & 25    & 1   &    &   \\ 
7 & 0 & 10395  & 19524  & 12139  & 3480  & 505   & 36  & 1  &   \\ 
8 & 0 & 135135 & 264207 & 177331 & 57379 & 10045 & 973 & 49 & 1 \\ 
\hline\hline 
\end{tabular} 

\subcaption{The double factorial function triangle, $\FcfII{2}{n}{k}$} 
\subtableskip 

\begin{tabular}{|c||lllllllll|} \hline\hline 
\trianglenk{n}{k} 
  & 0 & 1       & 2       & 3      & 4      & 5     & 6    & 7  & 8 \\ \hline 
0 & 1 &         &         &        &        &       &      &    &   \\ 
1 & 0 & 1       &         &        &        &       &      &    &   \\ 
2 & 0 & 1       & 1       &        &        &       &      &    &   \\ 
3 & 0 & 4       & 5       & 1      &        &       &      &    &   \\ 
4 & 0 & 28      & 39      & 12     & 1      &       &      &    &   \\ 
5 & 0 & 280     & 418     & 159    & 22     & 1     &      &    &   \\ 
6 & 0 & 3640    & 5714    & 2485   & 445    & 35    & 1    &    &   \\ 
7 & 0 & 58240   & 95064   & 45474  & 9605   & 1005  & 51   & 1  &   \\ 
8 & 0 & 1106560 & 1864456 & 959070 & 227969 & 28700 & 1974 & 70 & 1 \\ 
\hline\hline 
\end{tabular} 

\subcaption{The triple factorial function triangle, $\FcfII{3}{n}{k}$} 
\subtableskip 

\begin{tabular}{|c||lllllllll|} \hline\hline 
\trianglenk{n}{k} 
  & 0 & 1       & 2       & 3       & 4      & 5     & 6    & 7  & 8 \\ \hline 
0 & 1 &         &         &         &        &       &      &    &   \\ 
1 & 0 & 1       &         &         &        &       &      &    &   \\ 
2 & 0 & 1       & 1       &         &        &       &      &    &   \\ 
3 & 0 & 5       & 6       & 1       &        &       &      &    &   \\ 
4 & 0 & 45      & 59      & 15      & 1      &       &      &    &   \\ 
5 & 0 & 585     & 812     & 254     & 28     & 1     &      &    &   \\ 
6 & 0 & 9945    & 14389   & 5130    & 730    & 45    & 1    &    &   \\ 
7 & 0 & 208845  & 312114  & 122119  & 20460  & 1675  & 66   & 1  &   \\ 
8 & 0 & 5221125 & 8011695 & 3365089 & 633619 & 62335 & 3325 & 91 & 1 \\ 
\hline\hline 
\end{tabular} 

\subcaption{The quadruple factorial function triangle, $\FcfII{4}{n}{k}$} 

\end{subtable} 

\caption{Special cases of the $\alpha$--factorial coefficient triangles} 
\label{table_FcfAlphankCoeffs} 

\end{table} 

\begin{table}[h] 
\centering 

\smaller 

\begin{subtable}{\subtablewidth} 
\centering 

\begin{tabular}{|c|l|} \hline 
\hline\tabletopstrut 
$n$ & $\sigma_n(x)$ \\ \hline 
0 & $\frac{1}{x}$ \\ 
1 & $\frac{1}{2}$ \\ 
2 & $\frac{1}{24} (3 x-1)$ \\ 
3 & $\frac{1}{48} x (x-1)$ \\ 
4 & $\frac{1}{5760}\left(15 x^3-30 x^2+5 x+2\right)$ \\ 
5 & $\frac{1}{11520} x (x-1) \left(3 x^2-7 x-2\right)$ \\ 
6 & $\frac{1}{2903040}\left(63 x^5-315 x^4+315 x^3+91 x^2-42 x-16\right)$ \\ 
7 & $\frac{1}{5806080} x (x-1) \left(9 x^4-54 x^3+51 x^2+58 x+16\right)$ \\ 
8 & $\frac{1}{1393459200}\left( 
     135 x^7-1260 x^6+3150 x^5-840 x^4-2345 x^3-540 x^2+404 x+144\right)$ \\ 
    \hline\hline 
\end{tabular} 

\caption{The Stirling polynomials, 
                     $\sigma_n(x) \defequals \gkpSI{x}{x-n} \frac{(x-n-1)!}{x!}$} 

\subtableskip 

\begin{tabular}{|c|l|} \hline 
\hline\tabletopstrut 
$n$ & $(2n)! \times x \sigma_n^{(\alpha)}(x)$ \\ \hline 
0 & $1$ \\ 
1 & $\alpha  x-2 (\alpha -1)$ \\ 
2 & $3 \alpha ^2 x^2-\alpha  (13 \alpha -12) x + 
     12 (\alpha -1)^2$ \\ 
3 & $15 \alpha ^3 x^3-15 \alpha ^2 (7 \alpha -6) x^2+30 \alpha 
     (7 \alpha -6) (\alpha -1) x - 
     120 (\alpha -1)^3$ \\ 
4 & $105 \alpha ^4 x^4-210 \alpha ^3 (5 \alpha -4) x^3+ 
     35 \alpha ^2 \left(97 \alpha ^2-168 \alpha +72\right) x^2$ \\ 
  & $\quad - 
     14 \alpha  \left(299 \alpha ^3-840 \alpha ^2+780 \alpha -240\right) x + 
     1680 (\alpha -1)^4$ \\ 
5 & $945 \alpha ^5 x^5-3150 \alpha ^4 (4 \alpha -3) x^4+ 
     1575 \alpha ^3 \left(37 \alpha ^2-60 \alpha +24\right) x^3$ \\ 
  & $\quad - 
     630 \alpha ^2 \left(184 \alpha ^3-485 \alpha ^2+420 \alpha - 
     120\right) x^2$ \\ 
  & $\quad + 
     1260 \alpha  \left(79 \alpha ^3-220 \alpha ^2+200 \alpha -60\right) 
     (\alpha -1) x - 30240 (\alpha -1)^5$ \\ 
6 & $10395 \alpha ^6 x^6-10395 \alpha ^5 (17 \alpha -12) x^5+ 
     51975 \alpha ^4 (3 \alpha -2) (7 \alpha -6) x^4$ \\ 
  & $\quad - 
     1155 \alpha ^3 \left(2687\alpha ^3-6660 \alpha ^2+5400 \alpha - 
     1440\right) x^3$ \\ 
  & $\quad + 
     6930 \alpha ^2 \left(617 \alpha ^4-2208 \alpha ^3+ 
     2910 \alpha ^2-1680 \alpha +360\right) x^2$ \\ 
  & $\quad - 
      1320 \alpha  \left(2081 \alpha ^5-9954 \alpha ^4+18837 
      \alpha ^3-17640 \alpha ^2+8190 \alpha -1512\right) x$ \\ 
  & $\quad + 
     665280 (\alpha -1)^6$ \\ \hline\hline 
%
$n$ & $x \sigma_n^{(\alpha)}(x)$ \\ \hline 
%0 & $1$ \\ 
1 & $1 + \frac{1}{2} \alpha  (x-2)$ \\ 
2 & $\frac{1}{2} + \frac{1}{2} \alpha  (x-2) + 
     \frac{1}{24} \alpha ^2 (x-3) (3 x-4)$ \\ 
3 & $\frac{1}{6} + 
     \frac{1}{4} \alpha  (x-2)+ 
     \frac{1}{24} \alpha ^2 (x-3) (3 x-4) + 
     \frac{1}{48} \alpha ^3 (x-4) (x-2) (x-1)$ \\ 
4 & $\frac{1}{24} + 
     \frac{1}{12} \alpha  (x-2) + 
     \frac{1}{48} \alpha ^2 (x-3) (3 x-4)+
     \frac{1}{48} \alpha ^3 (x-4) (x-2) (x-1)$ \\ 
  & $\quad + 
     \frac{1}{5760}\alpha ^4 (x-5) \left(15 x^3-75 x^2+110 x-48\right)$ \\ 
5 & $\frac{1}{120} + 
     \frac{1}{48} \alpha  (x-2) + 
     \frac{1}{144} \alpha ^2 (x-3) (3 x-4)+
     \frac{1}{96} \alpha ^3 (x-4) (x-2) (x-1)$ \\ 
  & $\quad + 
     \frac{1}{5760}\alpha ^4 (x-5) \left(15 x^3-75 x^2+110 x-48\right)$ \\ 
  & $\quad + 
     \frac{1}{11520}\alpha ^5 (x-6) (x-2) (x-1) \left(3 x^2-13 x+8\right)$ \\ 
6 & $\frac{1}{720} + 
     \frac{1}{240} \alpha  (x-2) + 
     \frac{1}{576} \alpha ^2 (x-3) (3 x-4)+
     \frac{1}{288} \alpha ^3 (x-4) (x-2) (x-1)$ \\ 
  & $\quad + 
     \frac{1}{11520}\alpha ^4 (x-5) \left(15 x^3-75 x^2+110 x-48\right)$ \\ 
  & $\quad + 
     \frac{1}{11520}\alpha ^5 (x-6) (x-2) (x-1) \left(3 x^2-13 x+8\right)$ \\ 
  & $\quad + 
     \frac{1}{2903040}\left( 
     63 x^6-1071 x^5+6615 x^4-18809 x^3+25914 x^2-16648 x+ 4032\right)$ \\ 
     \hline\hline 
\end{tabular} 

\caption{Factored forms of the generalized $\alpha$--factorial polynomials, 
         $\sigma_n^{(\alpha)}(x)$} 

\end{subtable} 

\caption{The generalized Stirling and $\alpha$--factorial 
         polynomial sequences} 
\label{table_GenStirlingAlphaCvlPolys} 

\end{table} 

\begin{table}[h] 

\smaller 
\centering 

\begin{subtable}{\subtablewidth} 
\centering 

\begin{tabular}{|c|l|} \hline 
\hline\tabletopstrut  
$h$ & $\ConvFP{h}{\alpha}{R}{z}$ \tablebottomstrut \\ \hline 
%%%%%%%%% 
%0 & $0$ \\ 
1 & $1$ \\ 
2 & $1 - (2\alpha+R)z$ \\ 
3 & $1 - (6\alpha+2R)z + (6\alpha^2+4\alpha R+R^2)z^2$ \\ 
4 & $1 - (12\alpha+3R)z + (36\alpha^2+19\alpha R+3R^2)z^2 - 
           (24\alpha^3 + 18\alpha^2R + 7\alpha R^2 + R^3) z^3$ \\ 
5 & $1 - (20 \alpha + 4 R) z + (120 \alpha^2 + 51 \alpha R + 6 R^2) z^2 - 
     (240 \alpha^3 + 158 \alpha^2 R + 42 \alpha R^2 + 4 R^3) z^3$ \\ 
  & $\phantom{1} + 
     (120 \alpha^4 + 96 \alpha^3 R + 46 \alpha^2 R^2 + 11 \alpha R^3 + R^4) z^4$ \\ 
6 & $1 - (30 \alpha + 5 R) z + (300 \alpha^2 + 106 \alpha R + 10 R^2) z^2$ \\ 
  & $\phantom{1} - 
     (1200 \alpha^3 + 668 \alpha^2 R + 138 \alpha R^2 + 10 R^3) z^3$ \\ 
   & $\phantom{1} - 
     (1800 \alpha^4 + 1356 \alpha^3 R + 469 \alpha^2 R^2 + 78 \alpha R^3 + 5 R^4) z^4$ \\ 
   & $\phantom{1} + 
     (720 \alpha^5 + 600 \alpha^4 R + 326 \alpha^3 R^2 + 101 \alpha^2 R^3 + 16 \alpha R^4 + R^5) z^5$ \\ \hline 
\hline 
\end{tabular} 
%% 
\caption{The convergent numerator functions, $\ConvFP{h}{\alpha}{R}{z}$} 

\subtableskip 

\begin{tabular}{|c|l|} \hline 
\hline\tabletopstrut 
$h$ & $\ConvFQ{h}{\alpha}{R}{z}$ \\ \hline 
%%%%%%%% 
0 & $1$ \\ 
1 & $1-Rz$ \\ 
2 & $1 - 2 (\alpha + R) z + R (\alpha + R) z^2$ \\ 
3 & $1 - 3 (2 \alpha + R) z + 3 (\alpha + R) (2 \alpha + R) z^2 - 
      R (\alpha + R) (2 \alpha + R) z^3$ \\ 
4 & $1 - 4 (3 \alpha + R) z + 6 (2 \alpha + R) (3 \alpha + R) z^2 - 
       4 (\alpha + R) (2 \alpha + R) (3 \alpha + R) z^3$ \\ 
  & $\phantom{1 } + 
     R (\alpha + R) (2 \alpha + R) (3 \alpha + R) z^4$ \\ 
5 & $1 - 5 (4 \alpha + R) z + 10 (3 \alpha + R) (4 \alpha + R) z^2 - 
     10 (2 \alpha + R) (3 \alpha + R) (4 \alpha + R) z^3$ \\ 
  & $\phantom{1 } + 
     5 (\alpha + R) (2 \alpha + R) (3 \alpha + R) (4 \alpha + R) z^4 - 
     R (\alpha + R) (2 \alpha + R) (3 \alpha + R) (4 \alpha + R) z^5$ \\ 
6 & $1 - 6 (5 \alpha + R) z + 15 (4 \alpha + R) (5 \alpha + R) z^2 - 
     20 (3 \alpha + R) (4 \alpha + R) (5 \alpha + R) z^3$ \\ 
  & $\phantom{1 } + 
     15 (2 \alpha + R) (3 \alpha + R) (4 \alpha + R) (5 \alpha + R) z^4$ \\ 
  & $\phantom{1 } - 
     6 (\alpha + R) (2 \alpha + R) (3 \alpha + R) (4 \alpha + R) 
     (5 \alpha + R) z^5$ \\ 
  & $\phantom{1 } + 
     R (\alpha + R) (2 \alpha + R) (3 \alpha + R) (4 \alpha + R) 
     (5 \alpha + R) z^6$ \\ \hline 
\hline 
\end{tabular} 
%% 

\caption{The convergent denominator functions, $\ConvFQ{h}{\alpha}{R}{z}$} 

\subtableskip 

\begin{tabular}{|c|l|l|} \hline 
\hline\tabletopstrut  
$h$ & $\ConvFP{h}{1}{1}{z}$ & $\ConvFQ{h}{1}{1}{z}$ 
\tablebottomstrut \\ \hline 
%%%%%%%%% 
%0 & $0$ & $1$ \\ 
1 & $1$ & $1-z$ \\ 
2 & $1-3 z$ & $1-4 z+2 z^2$ \\ 
3 & $1-8 z+11 z^2$ & $1-9 z+18 z^2-6 z^3$ \\ 
4 & $1-15 z+58 z^2-50 z^3$ & $1-16 z+72 z^2-96 z^3+24 z^4$ \\ 
5 & $1-24 z+177 z^2-444 z^3+274 z^4$ & 
    $1-25 z+200 z^2-600 z^3+600 z^4-120 z^5$ \\
6 & $1-35 z+416 z^2-2016 z^3$ & $1-36 z+450 z^2-2400 z^3+5400 z^4$ \\ 
  & $\quad + 3708 z^4-1764 z^5$ & $\quad - 4320 z^5+720 z^6$ \\ 
\hline\hline 
\end{tabular} 
%% 
\caption{The convergent generating functions, 
         $\ConvGF{h}{1}{1}{z} \defequals 
          \ConvFP{h}{1}{1}{z} / \ConvFQ{h}{1}{1}{z}$, 
         enumerating the single factorial function, $n!$, 
         for all $0 \leq n \leq h$ and $h \geq 1$} 

\subtableskip 

\begin{tabular}{|c|l|l|} \hline 
\hline\tabletopstrut  
$h$ & $\ConvFP{h}{2}{1}{z}$ & $\ConvFQ{h}{2}{1}{z}$ 
\tablebottomstrut \\ \hline 
%%%%%%%%% 
1 & $1$ & $1-z$ \\ 
2 & $1-5 z$ & $1-6 z+3 z^2$ \\ 
3 & $1-14 z+33 z^2$ & $1-15 z+45 z^2-15 z^3$ \\ 
4 & $1-27 z+185 z^2-279 z^3$ & $1-28 z+210 z^2-420 z^3+105 z^4$ \\ 
5 & $1-44 z+588 z^2-2640 z^3+2895 z^4$ & 
    $1-45 z+630 z^2-3150 z^3+4725 z^4-945 z^5$ \\ 
6 & $1-65 z+1422 z^2-12558 z^3$ & 
    $1-66 z+1485 z^2-13860 z^3+51975 z^4$ \\
  & $\quad + 41685 z^4-35685 z^5$ & $\quad - 62370 z^5+10395 z^6$ \\ 
\hline\hline 
\end{tabular} 
%% 
\caption{The convergent generating functions, 
                     $\ConvGF{h}{2}{1}{z} \defequals 
                      \ConvFP{h}{2}{1}{z} / \ConvFQ{h}{2}{1}{z}$, 
                     enumerating the double factorial function, 
                     $(2n-1)!! = 2^{n} \times \Pochhammer{\frac{1}{2}}{n}$, 
                     for all $0 \leq n \leq h$ and $h \geq 1$} 

\end{subtable} 

\caption{The generalized convergent numerator and 
         denominator function sequences} 
\label{table_SpCase_Listings_Of_Phz_ConvFn} 
\label{table_SpCase_Listings_Of_Qhz_ConvFn} 
\label{table_SpCase_Listings_Of_PhzQhz_ConvFn} 

\end{table} 

\begin{table}[h] 
\smaller\centering 

\begin{subtable}{\subtablewidth} 
\centering 

\begin{tabular}{|c|l|} \hline 
\hline\tabletopstrut 
$h$ & $z^{h-1} \cdot \ConvFP{h}{\alpha}{R}{z^{-1}}$ \\ \hline 
1 & $1$ \\ 
2 & $-(2\alpha+R) + z$ \\ 
3 & $6 \alpha ^2+\alpha  (4 R-6 z)+(R-z)^2$ \\ 
4 & $-24 \alpha ^3-18 \alpha ^2 (R-2 z)-\alpha  (7 R-12 z) (R-z)-(R-z)^3$ \\ 
5 & $120 \alpha ^4+2 \alpha ^2 \left(23 R^2-79 R z+60 z^2\right) + 
     48 \alpha ^3 (2 R-5 z)+\alpha  (11 R-20 z) (R-z)^2+(R-z)^4$ \\ 
6 & $-720 \alpha ^5-2 \alpha ^3 \left(163 R^2-678 R z+600 z^2\right) - 
     \alpha ^2 \left(101 R^2-368 R z+300 z^2\right) (R-z)$ \\ 
   & $\phantom{-720 \alpha ^5} - 
      600 \alpha ^4 (R-3 z)-2 \alpha  (8R-15 z) (R-z)^3-(R-z)^5$ \\ 
7  & $5040 \alpha ^6+36 \alpha ^4 \left(71 R^2-347 R z+350 z^2\right) + 
      \alpha ^2 \left(197 R^2-740 R z+630 z^2\right) (R-z)^2$ \\ 
   & $\phantom{5040 \alpha ^6} + 
      \alpha ^3 \left(932 R^3-5102 R^2z+8322 R z^2-4200 z^3\right) + 
      2160 \alpha ^5 (2 R-7 z)$ \\ 
   & $\phantom{5040 \alpha ^6} + 
      2 \alpha  (11 R-21 z) (R-z)^4 + (R-z)^6$ \\ 
8  & $-40320 \alpha ^7-36 \alpha ^5 \left(617 R^2-3466 R z+3920 z^2\right) - 
      \alpha ^2 \left(351 R^2-1342 R z+1176 z^2\right) (R-z)^3$ \\ 
   & $\phantom{-40320 \alpha ^7} + 
      \alpha ^4 \left(-9080 R^3+57286 R^2 z-105144 R z^2+58800 z^3\right)$ \\ 
   & $\phantom{-40320 \alpha ^7} - 
       \alpha ^3 \left(2311 R^3-13040 R^2 z+22210 R z^2-11760 z^3\right) 
       (R-z)-35280 \alpha ^6 (R-4 z)$ \\ 
   & $\phantom{-40320 \alpha ^7} - 
      \alpha (29 R-56 z) (R-z)^5-(R-z)^7$ \\ \hline 
\hline 
\end{tabular} 
\subcaption{The reflected numerator polynomials, 
          $\widetilde{\FP}_h(\alpha, R; z) \defequals z^{h-1} \cdot 
          \ConvFP{h}{\alpha}{R}{z^{-1}}$} 

\subtableskip 

\begin{tabular}{|c|l|} \hline 
\hline\tabletopstrut 
$h$ & $z^{h-1} \cdot \ConvFP{h}{\alpha}{z-w}{z^{-1}}$ \\ \hline 
2 & $w-2 \alpha$ \\ 
3 & $6 \alpha ^2+w^2-4 \alpha  w-2 \alpha  z$ \\ 
4 & $-24 \alpha ^3+w^3-7 \alpha  w^2+w \left(18 \alpha ^2- 
    5 \alpha  z\right)+18 \alpha ^2 z$ \\ 
5 & $120 \alpha ^4+w^4-11 \alpha  w^3+ 
     w^2 \left(46 \alpha ^2-9 \alpha  z\right)+ 
     w \left(66 \alpha ^2 z-96 \alpha ^3\right)+8 \alpha ^2 z^2- 
     144 \alpha ^3 z$ \\ 
6 & $-720 \alpha ^5+w^5-16 \alpha  w^4+ 
     w^3 \left(101 \alpha ^2-14 \alpha  z\right)+ 
     w^2 \left(166 \alpha ^2 z-326 \alpha ^3\right)$ \\ 
  & $\qquad + 
     w \left(600 \alpha ^4+33 \alpha ^2 z^2-704 \alpha ^3 z\right)- 
     170 \alpha ^3 z^2+1200 \alpha ^4 z$ \\ 
7 & $5040 \alpha ^6+w^6-22 \alpha  w^5+ 
     w^4 \left(197 \alpha ^2-20 \alpha  z\right)+ 
     w^3 \left(346 \alpha ^2 z-932 \alpha ^3\right)$ \\ 
  & $\qquad + 
     w^2 \left(2556 \alpha ^4+87 \alpha ^2 z^2-2306 \alpha ^3 z\right)$ \\ 
  & $\qquad + 
     w \left(-4320 \alpha ^5-914 \alpha ^3 z^2+7380 \alpha ^4 z\right)- 
     48 \alpha ^3 z^3+2664 \alpha ^4 z^2-10800 \alpha ^5 z$ \\ 
8 & $-40320 \alpha ^7+w^7-29 \alpha  w^6+ 
     w^5 \left(351 \alpha ^2-27 \alpha  z\right)+ 
     w^4 \left(640 \alpha ^2 z-2311 \alpha ^3\right)$ \\ 
  & $\qquad + 
     w^3 \left(9080 \alpha ^4+185 \alpha ^2 z^2-6107 \alpha ^3 z\right)+ 
     w^2 \left(-22212 \alpha ^5-3063 \alpha ^3 z^2+30046 \alpha ^4 z\right)$ \\ 
  & $\qquad + 
     w \left(35280 \alpha ^6-279 \alpha ^3 z^3+17812 \alpha ^4 z^2- 
     80352 \alpha ^5 z\right)$ \\ 
  & $\qquad + 
     1862 \alpha ^4 z^3-38556 \alpha ^5 z^2+105840 \alpha ^6 z$ \\ \hline 
\hline 
$h$ & $z^{h-1} \cdot \ConvFP{h}{-\alpha}{z-w}{z^{-1}}$ \\ \hline 
3 & $6 \alpha ^2+w^2+\alpha  (4 w+2 z)$ \\ 
4 & $24 \alpha ^3+w^3+\alpha  \left(7 w^2+5 w z\right)+ 
     \alpha ^2 (18 w+18 z)$ \\ 
5 & $120 \alpha ^4+w^4+\alpha ^2 \left(46 w^2+66 w z+8 z^2\right)+ 
     \alpha  \left(11 w^3+9 w^2 z\right)+\alpha ^3 (96 w+144 z)$ \\ 
6 & $720 \alpha ^5+w^5+\alpha ^3 \left(326 w^2+704 w z+170 z^2\right)+ 
     \alpha \left(16 w^4+14 w^3 z\right)$ \\ 
  & $\qquad + 
     \alpha ^2 \left(101 w^3+166 w^2 z+33 w z^2\right)+ 
     \alpha ^4 (600 w+1200 z)$ \\ 
7 & $5040 \alpha ^6+w^6+ 
     \alpha ^4 \left(2556 w^2+7380 w z+2664 z^2\right)+ 
     \alpha  \left(22 w^5+20 w^4 z\right)$ \\ 
  & $\qquad + 
     \alpha ^3 \left(932 w^3+2306 w^2 z+914 w z^2+48 z^3\right)+ 
     \alpha ^2 \left(197 w^4+346 w^3 z+87 w^2 z^2\right)$ \\ 
  & $\qquad + 
     \alpha ^5 (4320 w+10800 z)$ \\ 
8 & $40320 \alpha ^7+w^7+ 
     \alpha ^5 \left(22212 w^2+80352 w z+38556 z^2\right)+ 
     \alpha  \left(29 w^6+27 w^5 z\right)$ \\ 
  & $\qquad + 
     \alpha ^4 \left(9080 w^3+30046 w^2 z+17812 w z^2+1862 z^3\right)+ 
     \alpha ^2 \left(351 w^5+640 w^4 z+185 w^3 z^2\right)$ \\ 
  & $\qquad + 
     \alpha ^3 \left(2311 w^4+6107 w^3 z+3063 w^2 z^2+279 w z^3\right)+ 
     \alpha ^6 (35280 w+105840 z)$ \\ \hline\hline
\end{tabular} 
\subcaption{Modified forms of the reflected numerator polynomials, 
          $\widetilde{\FP}_h(\pm \alpha, z-w; z)$} 

\end{subtable} 

%% 
\caption{The reflected convergent numerator function sequences} 
\label{table_RelfectedConvNumPolySeqs_sp_cases} 
\end{table} 

\addtocounter{table}{1}
\setcounter{subtable}{0} 

\begin{sidewaystable}[h] 
\centering 

\smaller

\begin{tabular}{|l|l|lcc|lcc|lcc|lcc|} \hline 
%% Alpha := 1: 
\hline\tabletopstrut 
$n$ & $\MultiFactorial{n}{1}$ & 
$\widetilde{R}_2^{(1)}(n)$ & $\pod{2}$ & $\pod{2}$ &
$\widetilde{R}_3^{(1)}(n)$ & $\pod{3}$ & $\pod{3}$ &
$\widetilde{R}_4^{(1)}(n)$ & $\pod{4}$ & $\pod{4}$ &
$\widetilde{R}_5^{(1)}(n)$ & $\pod{5}$ & $\pod{5}$ \\ \hline 
 0 & 1 & 1 & 1 & 1 & 1 & 1 & 1 & 1 & 1 & 1 & 1 & 1 & 1 \\
 1 & 1 & 1 & 1 & 1 & 1 & 1 & 1 & 1 & 1 & 1 & 1 & 1 & 1 \\
 2 & 2 & 2 & 0 & 0 & 2 & 2 & 2 & 2 & 2 & 2 & 2 & 2 & 2 \\
 3 & 6 & 6 & 0 & 0 & 6 & 0 & 0 & 6 & 2 & 2 & 6 & 1 & 1 \\
 4 & 24 & 0 & 0 & 0 & 24 & 0 & 0 & 24 & 0 & 0 & 24 & 4 & 4 \\
 5 & 120 & -560 & 0 & 0 & 120 & 0 & 0 & 120 & 0 & 0 & 120 & 0 & 0 \\
 6 & 720 & -15000 & 0 & 0 & 1440 & 0 & 0 & 720 & 0 & 0 & 720 & 0 & 0 \\
 7 & 5040 & -355320 & 0 & 0 & 44100 & 0 & 0 & 5040 & 0 & 0 & 5040 & 0 & 0 \\
 8 & 40320 & -8605184 & 0 & 0 & 1568448 & 0 & 0 & 0 & 0 & 0 & 40320 & 0 & 0 \\
 9 & 362880 & -220557312 & 0 & 0 & 54676944 & 0 & 0 & -3193344 & 0 & 0 & 362880 & 0 & 0 \\
 10 & 3628800 & -6037169760 & 0 & 0 & 1896099840 & 0 & 0 & -206599680 & 0 & 0 & 7257600 & 0 & 0 \\
 11 & 39916800 & -176606100000 & 0 & 0 & 66812223060 & 0 & 0 & -10648281600 & 0 & 0 & 512265600 & 0 & 0 \\
 12 & 479001600 & -5507542216704 & 0 & 0 & 2422878480000 & 0 & 0 & -509993003520 & 0 & 0 & 39734323200 & 0 & 0 \\
\hline\hline
\end{tabular} 

\captionof{subtable}{Congruences for the single factorial function, 
                     $n! = \MultiFactorial{n}{1}$, 
                     modulo $h$ for $h \defequals 2,3,4,5$. 
        }  

\end{sidewaystable} 

\begin{sidewaystable} 
\centering 
\smaller 

\begin{tabular}{|l|l|lcc|lcc|lcc|lcc|} \hline 
\hline\tabletopstrut 
%% Alpha := 2: 
$n$ & $\MultiFactorial{n}{2}$ & 
$\widetilde{R}_2^{(2)}(n)$ & $\pod{2}$ & $\pod{4}$ &
$\widetilde{R}_3^{(2)}(n)$ & $\pod{3}$ & $\pod{6}$ &
$\widetilde{R}_4^{(2)}(n)$ & $\pod{4}$ & $\pod{8}$ &
$\widetilde{R}_5^{(2)}(n)$ & $\pod{5}$ & $\pod{10}$ \\ \hline 
 0 & 1 & 1 & 1 & 1 & 1 & 1 & 1 & 1 & 1 & 1 & 1 & 1 & 1 \\
 1 & 1 & 1 & 1 & 1 & 1 & 1 & 1 & 1 & 1 & 1 & 1 & 1 & 1 \\
 2 & 2 & 2 & 0 & 2 & 2 & 2 & 2 & 2 & 2 & 2 & 2 & 2 & 2 \\
 3 & 3 & 3 & 1 & 3 & 3 & 0 & 3 & 3 & 3 & 3 & 3 & 3 & 3 \\
 4 & 8 & 8 & 0 & 0 & 8 & 2 & 2 & 8 & 0 & 0 & 8 & 3 & 8 \\
 5 & 15 & 15 & 1 & 3 & 15 & 0 & 3 & 15 & 3 & 7 & 15 & 0 & 5 \\
 6 & 48 & 48 & 0 & 0 & 48 & 0 & 0 & 48 & 0 & 0 & 48 & 3 & 8 \\
 7 & 105 & -175 & 1 & 1 & 105 & 0 & 3 & 105 & 1 & 1 & 105 & 0 & 5 \\
 8 & 384 & 0 & 0 & 0 & 384 & 0 & 0 & 384 & 0 & 0 & 384 & 4 & 4 \\
 9 & 945 & -13671 & 1 & 1 & 945 & 0 & 3 & 945 & 1 & 1 & 945 & 0 & 5 \\
 10 & 3840 & -17920 & 0 & 0 & 3840 & 0 & 0 & 3840 & 0 & 0 & 3840 & 0 & 0 \\
 11 & 10395 & -633501 & 1 & 3 & 43659 & 0 & 3 & 10395 & 3 & 3 & 10395 & 0 & 5 \\
 12 & 46080 & -960000 & 0 & 0 & 92160 & 0 & 0 & 46080 & 0 & 0 & 46080 & 0 & 0 \\
 13 & 135135 & -28498041 & 1 & 3 & 3532815 & 0 & 3 & 135135 & 3 & 7 & 135135 & 0 & 5 \\
 14 & 645120 & -45480960 & 0 & 0 & 5644800 & 0 & 0 & 645120 & 0 & 0 & 645120 & 0 & 0 \\
 15 & 2027025 & -1343937855 & 1 & 1 & 257161905 & 0 & 3 & -5386095 & 1 & 1 & 2027025 & 0 & 5 \\
 16 & 10321920 & -2202927104 & 0 & 0 & 401522688 & 0 & 0 & 0 & 0 & 0 & 10321920 & 0 & 0 \\
 17 & 34459425 & -67747539375 & 1 & 1 & 17642360385 & 0 & 3 & -1211768415 & 1 & 1 & 34459425 & 0 & 5 \\
 18 & 185794560 & -112925343744 & 0 & 0 & 27994595328 & 0 & 0 & -1634992128 & 0 & 0 & 185794560 & 0 & 0 \\
 19 & 654729075 & -3664567145437 & 1 & 3 & 1200706189875 & 0 & 3 & -141536175885 & 3 & 3 & 3315215475 & 0 & 5 \\
 20 & 3715891200 & -6182061834240 & 0 & 0 & 1941606236160 & 0 & 0 & -211558072320 & 0 & 0 & 7431782400 & 0 & 0 \\
 21 & 13749310575 & -212363430514977 & 1 & 3 & 83236453970607 & 0 & 3 & -14054409745425 & 3 & 7 & 679112772975 & 0 & 5 \\
\hline\hline
\end{tabular} 

\captionof{subtable}{Congruences for the double factorial function, 
                     $n!! = \MultiFactorial{n}{2}$, 
                     modulo $h$ (and $2h$) for $h \defequals 2,3,4,5$. 
        Supplementary listings containing computational data for the 
        congruences, $n!! \equiv R_h^{(2)}(n) \pmod{2^i h}$, 
        for $2 \leq i \leq h \leq 5$ are tabulated in the 
        summary notebook reference. 
        } 

\end{sidewaystable} 

\begin{sidewaystable} 
\centering 
\smaller 

\begin{tabular}{|l|l|lcc|lcc|lcc|lcc|} \hline 
\hline\tabletopstrut 
%% Alpha := 3: 
$n$ & $\MultiFactorial{n}{3}$ & 
$\widetilde{R}_2^{(3)}(n)$ & $\pod{2}$ & $\pod{6}$ &
$\widetilde{R}_3^{(3)}(n)$ & $\pod{3}$ & $\pod{9}$ &
$\widetilde{R}_4^{(3)}(n)$ & $\pod{4}$ & $\pod{12}$ &
$\widetilde{R}_5^{(3)}(n)$ & $\pod{5}$ & $\pod{15}$ \\ \hline 
 0 & 1 & 1 & 1 & 1 & 1 & 1 & 1 & 1 & 1 & 1 & 1 & 1 & 1 \\
 1 & 1 & 1 & 1 & 1 & 1 & 1 & 1 & 1 & 1 & 1 & 1 & 1 & 1 \\
 2 & 2 & 2 & 0 & 2 & 2 & 2 & 2 & 2 & 2 & 2 & 2 & 2 & 2 \\
 3 & 3 & 3 & 1 & 3 & 3 & 0 & 3 & 3 & 3 & 3 & 3 & 3 & 3 \\
 4 & 4 & 4 & 0 & 4 & 4 & 1 & 4 & 4 & 0 & 4 & 4 & 4 & 4 \\
 5 & 10 & 10 & 0 & 4 & 10 & 1 & 1 & 10 & 2 & 10 & 10 & 0 & 10 \\
 6 & 18 & 18 & 0 & 0 & 18 & 0 & 0 & 18 & 2 & 6 & 18 & 3 & 3 \\
 7 & 28 & 28 & 0 & 4 & 28 & 1 & 1 & 28 & 0 & 4 & 28 & 3 & 13 \\
 8 & 80 & 80 & 0 & 2 & 80 & 2 & 8 & 80 & 0 & 8 & 80 & 0 & 5 \\
 9 & 162 & 162 & 0 & 0 & 162 & 0 & 0 & 162 & 2 & 6 & 162 & 2 & 12 \\
 10 & 280 & -980 & 0 & 4 & 280 & 1 & 1 & 280 & 0 & 4 & 280 & 0 & 10 \\
 11 & 880 & -704 & 0 & 4 & 880 & 1 & 7 & 880 & 0 & 4 & 880 & 0 & 10 \\
 12 & 1944 & 0 & 0 & 0 & 1944 & 0 & 0 & 1944 & 0 & 0 & 1944 & 4 & 9 \\
 13 & 3640 & -92300 & 0 & 4 & 3640 & 1 & 4 & 3640 & 0 & 4 & 3640 & 0 & 10 \\
 14 & 12320 & -115192 & 0 & 2 & 12320 & 2 & 8 & 12320 & 0 & 8 & 12320 & 0 & 5 \\
 15 & 29160 & -136080 & 0 & 0 & 29160 & 0 & 0 & 29160 & 0 & 0 & 29160 & 0 & 0 \\
 16 & 58240 & -6186752 & 0 & 4 & 395200 & 1 & 1 & 58240 & 0 & 4 & 58240 & 0 & 10 \\
 17 & 209440 & -8349992 & 0 & 4 & 633556 & 1 & 1 & 209440 & 0 & 4 & 209440 & 0 & 10 \\
 18 & 524880 & -10935000 & 0 & 0 & 1049760 & 0 & 0 & 524880 & 0 & 0 & 524880 & 0 & 0 \\
 19 & 1106560 & -411766784 & 0 & 4 & 51684256 & 1 & 1 & 1106560 & 0 & 4 & 1106560 & 0 & 10 \\
 20 & 4188800 & -572266240 & 0 & 2 & 70505120 & 2 & 2 & 4188800 & 0 & 8 & 4188800 & 0 & 5 \\
 21 & 11022480 & -777084840 & 0 & 0 & 96446700 & 0 & 0 & 11022480 & 0 & 0 & 11022480 & 0 & 0 \\
 22 & 24344320 & -28922921456 & 0 & 4 & 5645314048 & 1 & 4 & -144674816 & 0 & 4 & 24344320 & 0 & 10 \\
 23 & 96342400 & -40807520000 & 0 & 4 & 7668245080 & 1 & 1 & -116486720 & 0 & 4 & 96342400 & 0 & 10 \\
 24 & 264539520 & -56458612224 & 0 & 0 & 10290587328 & 0 & 0 & 0 & 0 & 0 & 264539520 & 0 & 0 \\
 25 & 608608000 & -2177450514800 & 0 & 4 & 577086766300 & 1 & 1 & -41321139200 & 0 & 4 & 608608000 & 0 & 10 \\
 26 & 2504902400 & -3101148709984 & 0 & 2 & 793943072000 & 2 & 8 & -52040160640 & 0 & 8 & 2504902400 & 0 & 5 \\
 27 & 7142567040 & -4341229572096 & 0 & 0 & 1076206288752 & 0 & 0 & -62854589952 & 0 & 0 & 7142567040 & 0 & 0 \\
 28 & 17041024000 & -176120000000000 & 0 & 4 & 58548072721600 & 1 & 1 & -7074936915200 & 0 & 4 & 153556480000 & 0 & 10 \\
% 29 & 72642169600 & -252523474362848 & 0 & 4 & 81563818763284 & 1 & 7 & -9373697840000 & 0 & 4 & 244577694400 & 0 & 10 \\
% 30 & 214277011200 & -356488837158240 & 0 & 0 & 111962799452160 & 0 & 0 & -12199504504320 & 0 & 0 & 428554022400 & 0 & 0 \\
% 31 & 528271744000 & -15270649810304000 & 0 & 4 & 6057047518375000 & 1 & 4 & -1042665722067200 & 0 & 4 & 51047522680000 & 0 & 10 \\
% 32 & 2324549427200 & -22013932954304512 & 0 & 2 & 8526639390663680 & 2 & 8 & -1412281656008704 & 0 & 8 & 67700504268800 & 0 & 5 \\
\hline\hline
\end{tabular} 

\captionof{subtable}{Congruences for the triple factorial function, 
                     $n!!! = \MultiFactorial{n}{3}$, 
                     modulo $h$ (and $3h$) for $h \defequals 2,3,4,5$. 
        Supplementary listings containing computational data for the 
        congruences, $n!!! \equiv R_h^{(3)}(n) \pmod{3^i h}$, 
        for $2 \leq i \leq h \leq 5$ are tabulated in the 
        summary notebook reference. 
     } 

\end{sidewaystable} 

\begin{sidewaystable} 
\centering 
\smaller 

\begin{tabular}{|l|l|lcc|lcc|lcc|lcc|} \hline 
\hline\tabletopstrut 
%% Alpha := 4: 
$n$ & $\MultiFactorial{n}{4}$ & 
$\widetilde{R}_2^{(4)}(n)$ & $\pod{2}$ & $\pod{8}$ &
$\widetilde{R}_3^{(4)}(n)$ & $\pod{3}$ & $\pod{12}$ &
$\widetilde{R}_4^{(4)}(n)$ & $\pod{4}$ & $\pod{16}$ &
$\widetilde{R}_5^{(4)}(n)$ & $\pod{5}$ & $\pod{20}$ \\ \hline 
 0 & 1 & 1 & 1 & 1 & 1 & 1 & 1 & 1 & 1 & 1 & 1 & 1 & 1 \\
 1 & 1 & 1 & 1 & 1 & 1 & 1 & 1 & 1 & 1 & 1 & 1 & 1 & 1 \\
 2 & 2 & 2 & 0 & 2 & 2 & 2 & 2 & 2 & 2 & 2 & 2 & 2 & 2 \\
 3 & 3 & 3 & 1 & 3 & 3 & 0 & 3 & 3 & 3 & 3 & 3 & 3 & 3 \\
 4 & 4 & 4 & 0 & 4 & 4 & 1 & 4 & 4 & 0 & 4 & 4 & 4 & 4 \\
 5 & 5 & 5 & 1 & 5 & 5 & 2 & 5 & 5 & 1 & 5 & 5 & 0 & 5 \\
 6 & 12 & 12 & 0 & 4 & 12 & 0 & 0 & 12 & 0 & 12 & 12 & 2 & 12 \\
 7 & 21 & 21 & 1 & 5 & 21 & 0 & 9 & 21 & 1 & 5 & 21 & 1 & 1 \\
 8 & 32 & 32 & 0 & 0 & 32 & 2 & 8 & 32 & 0 & 0 & 32 & 2 & 12 \\
 9 & 45 & 45 & 1 & 5 & 45 & 0 & 9 & 45 & 1 & 13 & 45 & 0 & 5 \\
 10 & 120 & 120 & 0 & 0 & 120 & 0 & 0 & 120 & 0 & 8 & 120 & 0 & 0 \\
 11 & 231 & 231 & 1 & 7 & 231 & 0 & 3 & 231 & 3 & 7 & 231 & 1 & 11 \\
 12 & 384 & 384 & 0 & 0 & 384 & 0 & 0 & 384 & 0 & 0 & 384 & 4 & 4 \\
 13 & 585 & -3159 & 1 & 1 & 585 & 0 & 9 & 585 & 1 & 9 & 585 & 0 & 5 \\
 14 & 1680 & -2800 & 0 & 0 & 1680 & 0 & 0 & 1680 & 0 & 0 & 1680 & 0 & 0 \\
 15 & 3465 & -1815 & 1 & 1 & 3465 & 0 & 9 & 3465 & 1 & 9 & 3465 & 0 & 5 \\
 16 & 6144 & 0 & 0 & 0 & 6144 & 0 & 0 & 6144 & 0 & 0 & 6144 & 4 & 4 \\
 17 & 9945 & -364871 & 1 & 1 & 9945 & 0 & 9 & 9945 & 1 & 9 & 9945 & 0 & 5 \\
 18 & 30240 & -437472 & 0 & 0 & 30240 & 0 & 0 & 30240 & 0 & 0 & 30240 & 0 & 0 \\
 19 & 65835 & -508725 & 1 & 3 & 65835 & 0 & 3 & 65835 & 3 & 11 & 65835 & 0 & 15 \\
 20 & 122880 & -573440 & 0 & 0 & 122880 & 0 & 0 & 122880 & 0 & 0 & 122880 & 0 & 0 \\
 21 & 208845 & -32086803 & 1 & 5 & 1990989 & 0 & 9 & 208845 & 1 & 13 & 208845 & 0 & 5 \\
 22 & 665280 & -40544064 & 0 & 0 & 2794176 & 0 & 0 & 665280 & 0 & 0 & 665280 & 0 & 0 \\
 23 & 1514205 & -50324483 & 1 & 5 & 4031325 & 0 & 9 & 1514205 & 1 & 13 & 1514205 & 0 & 5 \\
 24 & 2949120 & -61440000 & 0 & 0 & 5898240 & 0 & 0 & 2949120 & 0 & 0 & 2949120 & 0 & 0 \\
 25 & 5221125 & -2829930075 & 1 & 5 & 358222725 & 0 & 9 & 5221125 & 1 & 5 & 5221125 & 0 & 5 \\
 26 & 17297280 & -3647749248 & 0 & 0 & 452200320 & 0 & 0 & 17297280 & 0 & 0 & 17297280 & 0 & 0 \\
 27 & 40883535 & -4637561553 & 1 & 7 & 570989007 & 0 & 3 & 40883535 & 3 & 15 & 40883535 & 0 & 15 \\
 28 & 82575360 & -5821562880 & 0 & 0 & 722534400 & 0 & 0 & 82575360 & 0 & 0 & 82575360 & 0 & 0 \\
 29 & 151412625 & -264205859375 & 1 & 1 & 52114215825 & 0 & 9 & -1438808175 & 1 & 1 & 151412625 & 0 & 5 \\
 30 & 518918400 & -344048090880 & 0 & 0 & 65833447680 & 0 & 0 & -1378840320 & 0 & 0 & 518918400 & 0 & 0 \\
 31 & 1267389585 & -442855631151 & 1 & 1 & 82524474513 & 0 & 9 & -979895151 & 1 & 1 & 1267389585 & 0 & 5 \\
 32 & 2642411520 & -563949338624 & 0 & 0 & 102789808128 & 0 & 0 & 0 & 0 & 0 & 2642411520 & 0 & 0 \\
 33 & 4996616625 & -26469713463567 & 1 & 1 & 7078405640625 & 0 & 9 & -516689348175 & 1 & 1 & 4996616625 & 0 & 5 \\
 34 & 17643225600 & -34686740160000 & 0 & 0 & 9032888517120 & 0 & 0 & -620425428480 & 0 & 0 & 17643225600 & 0 & 0 \\
% 35 & 44358635475 & -44995462312045 & 1 & 3 & 11424400266195 & 0 & 3 & -729771817005 & 3 & 3 & 44358635475 & 0 & 15 \\
% 36 & 95126814720 & -57817775996928 & 0 & 0 & 14333232807936 & 0 & 0 & -837115969536 & 0 & 0 & 95126814720 & 0 & 0 \\
% 37 & 184874815125 & -2850677559242667 & 1 & 5 & 954595493854101 & 0 & 9 & -116777000896875 & 1 & 5 & 2469180223125 & 0 & 5 \\
% 38 & 670442572800 & -3752516756927488 & 0 & 0 & 1229523138432000 & 0 & 0 & -144933044106240 & 0 & 0 & 3394780646400 & 0 & 0 \\
% 39 & 1729986783525 & -4894808126446875 & 1 & 5 & 1569832979268645 & 0 & 9 & -178089844010715 & 1 & 5 & 4958980114725 & 0 & 5 \\
% 40 & 3805072588800 & -6330431318261760 & 0 & 0 & 1988204785827840 & 0 & 0 & -216635466055680 & 0 & 0 & 7610145177600 & 0 & 0 \\
% 41 & 7579867420125 & -329188839860939619 & 1 & 5 & 131352101280569565 & 0 & 9 & -22831347886801443 & 1 & 13 & 1126998293788125 & 0 & 5 \\
% 42 & 28158588057600 & -434920305694672896 & 0 & 0 & 170468257731803136 & 0 & 0 & -28783431158630400 & 0 & 0 & 1390822959052800 & 0 & 0 \\
\hline\hline 
\end{tabular} 

\captionof{subtable}{Congruences for the 
                     quadruple factorial ($4$--factorial) function, 
                     $n!!!! = \MultiFactorial{n}{4}$, 
                     modulo $h$ (and $4h$) for $h \defequals 2,3,4,5$. 
        Supplementary listings containing computational data for the 
        congruences, $n!!!! \equiv R_h^{(4)}(n) \pmod{4^i h}$, 
        for $2 \leq i \leq h \leq 5$ are tabulated in the 
        summary notebook reference. 
     } 

\addtocounter{table}{-1} 
\bigskip\hrule
\captionof{table}{The $\alpha$--factorial functions modulo 
                  $h$ (and $h\alpha$) 
                  for $h \defequals 2,3,4,5$ defined by the special case 
                  expansions from 
                  Section \ref{subsubSection_Examples_NewCongruences} of the 
                  introduction and in 
                  Section \ref{subSection_NewCongruence_Relations_Modulo_Integer_Bases} 
                  where 
                  $\widetilde{R}_p^{(\alpha)}(n) \defequals 
                   \left[z^{\lfloor (n+\alpha-1) / \alpha \rfloor}\right] 
                   \ConvGF{p}{-\alpha}{n}{z}$. 
                 }
\label{table_AlphaFactFns_Modulo245_spcase_examples} 

\end{sidewaystable} 

\begin{table}[h] 
\centering 

\smaller 

\begin{subtable}{\textwidth} 
\centering 

\begin{tabular}{|c|l|l|} \hline 
\hline\tabletopstrut 
$m$ & $\widetilde{\ell}_{m,2}(z)$ & $\widetilde{p}_{m,2}(x)$ \\ \hline 
1 & $1$ & $2$ \\ 
2 & $4 - 3z$ & $x+5$ \\ 
3 & $11 - 17z + 7z^2$ & $ x^2+10x+24$ \\ 
4 & $26 - 62z + 52z^2  - 15z^3$ & $x^3+18x^2+96x+192$ \\ 
5 & $57 - 186z + 238z^2 - 139z^3 + 31z^4$ & 
    $x^4+28x^3+264x^2+1008x+1392$ \\ %\hline 
6 & $120 - 501z + 868z^2 - 769z^3 + 346z^4 - 63z^5$ & 
    $x^5 + 40x^4+580x^3 + 3840x^2+11880x + 14520$ \\ \hline 
\hline 
\end{tabular} 
\subcaption{Generating the $p^{th}$ power sequences, $2^{p}-1$} 

\subtableskip 

\begin{tabular}{|c|l|l|} \hline 
\hline\tabletopstrut 
$m$ & $\ell_{m,2}(z)$ & $p_{m,2}(x)$ \\ \hline 
1 & $1$ & $1$ \\ 
2 & $4 - 3z$ & $x+4$ \\ 
3 & $11 - 17z + 7z^2$ & $x^2+10x+22$ \\ 
4 & $26 - 62z + 52z^2  - 15z^3$ & 
    $x^3+18x^2+96x+156$ \\ 
5 & $57 - 186z + 238z^2 - 139z^3 + 31z^4$ & 
    $x^4+28x^3+264x^2+1008x+1368$ \\ \hline 
%%%% 
\hline\tabletopstrut 
$m$ & $\ell_{m,3}(z)$ & $p_{m,3}(x)$ \\ \hline 
1 & $1$ & $1$ \\ 
2 & $5 - 8z$ & $x+5$ \\ 
3 & $18 - 60z + 52 z^2$ & $x^2+12x+36$ \\ 
4 & $58 - 300 z + 532 z^2  - 320 z^3$ & $x^3+21x^2+144x+348$ \\ 
5 & $179 - 1268 z + 3436 z^2 - 4192 z^3 + 1936 z^4$ & 
    $x^4+32x^3+372x^2+1968x+4296$ \\ \hline 
%6 & $543 - 4908 z + 17996 z^2 - 33312 z^3 + 31056 z^4 - 11648 z^5$ & 
%    $x^5 + 45x^4+780x^3 + 6780x^2+31320x + 65160$ \\ \hline 
%%%%% 
\hline\tabletopstrut 
$m$ & $\ell_{m,4}(z)$ & $p_{m,4}(x)$ \\ \hline 
1 & $1$ & $1$ \\ 
2 & $6 - 15z$ & $x+6$ \\ 
3 & $27 - 141z + 189 z^2$ & $x^2+14x+54$ \\ 
4 & $112 - 906 z + 2484 z^2  - 2295 z^3$ & 
    $x^3+24x^2+204x+672$ \\ 
5 & $453 - 4998 z + 20898 z^2 - 39123 z^3 + 27621 z^4$ & 
    $x^4+36x^3+504x^2+3504x+10872$ \\ \hline 
%6 & $1818 - 25473 z + 143748 z^2 - 407673 z^3 + 580446 z^4 - 331695 z^5$ & 
%    $x^5 + 50x^4+1020x^3 + 11280x^2+71880x + 218160$ \\ \hline 
%%%%% 
\hline\tabletopstrut 
$m$ & $\ell_{m,5}(z)$ & $p_{m,5}(x)$ \\ \hline 
1 & $1$ & $1$ \\ 
2 & $7 - 24z$ & $x+7$ \\ 
3 & $38 - 272z + 496 z^2$ & 
    $x^2+16x+76$ \\ 
4 & $194 - 2144 z + 7984 z^2  - 9984 z^3$ & 
    $x^3+27x^2+276x+1164$ \\ 
5 & $975 - 14640 z + 82960 z^2 - 209920 z^3 + 199936 z^4$ & 
    $x^4+40x^3+660x^2+5760x+23400$ \\ \hline 
\hline 
\end{tabular} 
\subcaption{Generating the $p^{th}$ power sequences of binomials, 
     $2^{p}-1$, 
     $3^{p}-1$, $4^{p}-1$, and $5^{p}-1$} %($p > m \geq 1$)} 

\subtableskip 

\begin{tabular}{|c|l|l|} \hline 
\hline\tabletopstrut 
$m$ & $\ell_{m,s+1}(z)$ & $p_{m,s+1}(x)$ \\ \hline 
1 & $1$ & $1$ \\ 
2 & $3+s (1-2 z)-s^2 z$ & $3+s (1+x)$ \\ 
3 & $6+s^4 z^2-4 s (-1+2 z)+s^3 z (-2+3 z)$ & 
    $12+8 s (1+x)+s^2 (2+2 x+x^2)$ \\ 
  & $\quad + 
     s^2 (1-7 z+3 z^2)$ & \\ 
4 & $10-s^6 z^3-10 s (-1+2 z)-s^5 z^2 (-3+4 z)$ & 
    $60+60 s (1+x)+15 s^2 (2+2 x+x^2)$ \\ 
  & $\quad + 
     5 s^2 (1-5 z+3 z^2) - 
     s^4 z (3-13 z+6 z^2)$ & 
    $\quad + 
     s^3 (6+6 x+3 x^2+x^3)$ \\ 
  & $\quad  + 
     s^3 (1-14 z+21 z^2-4 z^3)$ & \\ 
5 & $15+s^8 z^4-20 s (-1+2 z)+s^7 z^3 (-4+5 z)$ & 
    $360+480 s (1+x) + 180 s^2 (2+2 x+x^2)$ \\ 
  & $\quad + 
     5 s^2 (3-13 z+9 z^2)+s^6 z^2 (6-21 z+10 z^2)$ & 
    $\quad + 
     24 s^3 (6+6 x+3 x^2+x^3)$ \\ 
  & $\quad - 
     3 s^3 (-2+18 z-27 z^2+8 z^3)$ & 
    $\quad + 
     s^4 (24+24 x+12 x^2+4 x^3+x^4)$ \\ 
  & $\quad + 
     s^5 z (-4+33 z-44 z^2+10 z^3)$ & \\ 
  & $\quad + 
     s^4 (1-23 z+73 z^2-46 z^3+5 z^4)$ & \\ \hline 
\hline 
\end{tabular} 
\subcaption{Generating the $p^{th}$ power sequences, $(s+1)^{p}-1$} 

\end{subtable} 

\caption{Convergent--based generating function identities for the 
         binomial $p^{th}$ power sequences 
         enumerated by the examples in 
         Section \ref{subsubSection_Apps_Example_SumsOfPowers_Seqs}} 
\label{table_ConvGF_Examples_for_PthPowerSeqs} 

\end{table} 

\addtocounter{table}{1}
\setcounter{subtable}{0} 

\begin{sidewaystable}[h] 

\centering 
\smaller 

%\begin{tabular}{|c|l|llllll|} 
%\hline\tabletopstrut 
%$n$ & $m_{n,h}$ & 
%$p_{n,0}(h)$ & $p_{n,1}(h)$ & $p_{n,2}(h)$ & $p_{n,3}(h)$ & 
%$p_{n,4}(h)$ & $p_{n,5}(h)$ \\ \hline 
%0 & $1$ & 1 & & & & & \\ 
%1 & $h-1$ & $h$ & $1$ & & & & \\ 
%2 & $h-2$ & $h (h-1)^2$ & $h(h-2)$ & $h-1$ & & & \\ 
%3 & $(h-2)(h-3)$ & $h (h-1)^2 (h-2)$ & $h(h-1)(h-3)$ & $h(h-3)$ & 
%    $h-1$ & & \\ 
%4 & $(h-3)(h-4)$ & $h (h-1)^2 (h-2)^2 (h-3)$ & $h(h-1) (h-2)^2 (h-4)$ & 
%    $h(h-1)(h-3)(h-4)$ & $h(h-2)(h-4)$ & $(h-1)(h-2)$ & \\ 
%5 & $(h-3)(h-4)(h-5)$ & 
%    $h (h-1)^2 (h-2)^2 (h-3)(h-4)$ & 
%    $h(h-1) (h-2)^2 (h-3)(h-5)$ & 
%    $h(h-1)(h-2)(h-4)(h-5)$ & 
%    $h(h-1)(h-2)(h-4)(h-5)$ & 
%    $h(h-2)(h-5)$ & $(h-1)(h-2)$ \\ 
%\hline 
%\end{tabular} 

\begin{tabular}{|c|lll|} 
\hline\tabletopstrut 
$n$ & %$m_{n,h}$ & 
$p_{n,0}(h)$ & $p_{n,1}(h)$ & $p_{n,2}(h)$ \\ \hline 
0 & %$1$ & 
    1 & 0 & 0 \\ 
1 & %$h-1$ & 
    $h$ & $1$ & 0 \\ 
2 & %$h-2$ & 
    $h (h-1)^2$ & $h(h-2)$ & $h-1$ \\ 
3 & %$(h-2)(h-3)$ & 
    $h (h-1)^2 (h-2)$ & $h(h-1)(h-3)$ & $h(h-3)$ \\ 
4 & %$(h-3)(h-4)$ & 
    $h (h-1)^2 (h-2)^2 (h-3)$ & $h(h-1) (h-2)^2 (h-4)$ & 
    $h(h-1)(h-3)(h-4)$ \\ 
5 & %$(h-3)(h-4)(h-5)$ & 
    $h (h-1)^2 (h-2)^2 (h-3)(h-4)$ & 
    $h(h-1) (h-2)^2 (h-3)(h-5)$ & 
    $h(h-1)(h-2)(h-4)(h-5)$ \\ 
6 & %$(h-4)(h-5)(h-6)$ & 
    $h (h-1)^2 (h-2)^2 (h-3)^2 (h-4)(h-5)$ & 
    $h(h-1) (h-2)^2 (h-3)^2 (h-4)(h-6)$ & 
    $h(h-1)(h-2) (h-3)^2 (h-5)(h-6)$ \\ 
7 & $h (h-1)^2 (h-2)^2 (h-3)^2 (h-4)(h-5)(h-6)$ & 
    $h(h-1) (h-2)^2 (h-3)^2 (h-4)(h-5)(h-7)$ & 
    $h(h-1)(h-2) (h-3)^2 (h-4)(h-6)(h-7)$ \\ 
\hline\hline  
$n$ & $p_{n,3}(h)$ & $p_{n,4}(h)$ & $p_{n,5}(h)$ \\ \hline 
0 & 0 & 0 & 0 \\ 
1 & 0 & 0 & 0 \\ 
2 & 0 & 0 & 0 \\ 
3 & $h-1$ & 0 & 0 \\ 
4 & $h(h-2)(h-4)$ & $(h-1)(h-2)$ & 0 \\ 
5 & $h(h-1)(h-2)(h-4)(h-5)$ & 
    $h(h-2)(h-5)$ & $(h-1)(h-2)$ \\ 
6 & $h(h-1)(h-2)(h-4)(h-5)(h-6)$ & 
    $h(h-1)(h-3)(h-5)(h-6)$ & 
    $h(h-2)(h-3)(h-6)$ \\ 
7 & $h(h-1)(h-2)(h-3)(h-5)(h-6)(h-7)$ & 
    $h(h-1)(h-2)(h-5)(h-6)(h-7)$ & 
    $h(h-1)(h-3)(h-6)(h-7)$ \\ 
\hline\hline 
$n$ & $p_{n,6}(h)$ & $p_{n,7}(h)$ & $m_{n,h}$ \\ \hline 
0 & 0 & 0 & 1 \\ 
1 & 0 & 0 & $h-1$ \\ 
2 & 0 & 0 & $h-2$ \\ 
3 & 0 & 0 & $(h-2)(h-3)$ \\ 
4 & 0 & 0 & $(h-3)(h-4)$ \\ 
5 & 0 & 0 & $(h-3)(h-4)(h-5)$ \\ 
6 & $(h-1)(h-2)(h-3)$ & 0 & $(h-4)(h-5)(h-6)$ \\ 
7 & $h(h-2)(h-3)(h-7)$ & $(h-1)(h-2)(h-3)$ & $(h-4)(h-5)(h-6)(h-7)$ \\ 
\hline\hline 
\end{tabular} 

\captionof{subtable}{The auxiliary numerator subsequences, 
         $C_{h,n}(\alpha, R) \defmapsto 
          \frac{(-\alpha)^{n} m_{n,h}}{n!} \times 
          \sum_{i=0}^{n} \binom{n}{i} p_{n,i}(h) \Pochhammer{R / \alpha}{i}$, 
          expanded by the finite--degree polynomial sequence terms 
          defined by the Stirling number sums in 
          \eqref{eqn_Chn_formula_stmts-exp_v5} of 
          Section \ref{subsubSection_Properties_Of_ConvFn_Phz-AuxNumFn_Subsequences}. 
          } 
\label{table_ConvNumFnSeqs_Chn_AlphaR_SpCaseListings-first_subtable_pageref} 

\end{sidewaystable} 

\addtocounter{table}{-1} 

\begin{table}[h] 
\centering 
\smaller 

\begin{subtable}{\textwidth} 
\centering 

\addtocounter{subtable}{1} 

\begin{tabular}{|c|l|l|} \hline 
\hline\tabletopstrut 
$n$ & $m_{h}$ & $(-1)^{n} n! \cdot m_h^{-1} \cdot C_{h,n}(\alpha, R)$ \\ \hline 
0 & $1$ & $1$ \\ 
1 & $1$ & $-(h-1) (R+ h\alpha)$ \\ 
2 & $(h-2)$ & 
    $\alpha  \left(2 h^2-3 h-1\right) R+\alpha ^2 (h-1)^2 h+(h-1) R^2$ \\ 
3 & $(h-3)$ & 
    $3 \alpha  (h-2) \left(h^2-2 h-1\right) R^2+ 
     \alpha ^2 (h-2) \left(3 h^3-9 h^2+2 h-2\right) R$ \\ 
  & & 
    $\qquad + 
     \alpha ^3 (h-2)^2 (h-1)^2 h+(h-2) (h-1) R^3$ \\ 
4 & $(h-3)(h-4)$ & 
    $\alpha ^2 \left(6 h^4-36 h^3+53 h^2-9 h+22\right) R^2+2 \alpha ^3 \
     \left(2 h^5-15 h^4+36 h^3-36 h^2+19 h+6\right) R$ \\ 
  & & 
    $\qquad + 
     \alpha ^4 (h-3) \
     (h-2)^2 (h-1)^2 h+(h-2) (h-1) R^4+2 \alpha  (h-3) (h-2) (2 h+1) R^3$ \\ 
5 & $(h-3)(h-4)(h-5)$ & 
    $5 \alpha  (h-2) \left(h^2-3 h-2\right) R^4+5 \alpha ^2 \left(2 h^4-14 h^3+23 h^2-h+14\right) R^3$ \\ 
  & & 
    $\qquad + 
     5 \alpha ^3 \left(2 h^5-18 h^4+49 h^3-49 h^2+40 h+20\right) R^2$ \\ 
  & & 
    $\qquad + 
     \alpha ^4 \left(5 h^6-55 h^5+215 h^4-395 h^3+374 h^2-72 h+48\right) R$ \\ 
  & & 
    $\qquad + 
     \alpha ^5 (h-4) (h-3) (h-2)^2 (h-1)^2 h+(h-2) (h-1) R^5$ \\ 
6 & $(h-4)(h-5)(h-6)$ & 
    $3 \alpha  (h-3) (h-2) \left(2 h^2-7 h-5\right) R^5+5 \alpha ^2 (h-3) \left(3 h^4-24 h^3+44 h^2+3 h+34\right) R^4$ \\ 
  & & 
    $\qquad + 
     5 \alpha ^3 \left(4 h^6-54 h^5+256 h^4-519 h^3+520 h^2-357 h-270\right) R^3$ \\ 
  & & 
    $\qquad + 
     \alpha ^4 \left(15 h^7-240 h^6+1455 h^5-4335 h^4+7114 h^3-6129 h^2+764 h-1644\right) R^2$ \\ 
  & & 
    $\qquad + 
     \alpha ^5 \left(6 h^8-111 h^7+826 h^6-3246 h^5+7378 h^4-9603 h^3+6478 h^2-2448 h-720\right) R$ \\ 
  & & 
    $\qquad + 
     \alpha ^6 (h-5) (h-4) (h-3)^2 (h-2)^2 (h-1)^2 h+(h-3) (h-2) (h-1) R^6$ \\ \hline 
\hline\tabletopstrut 
$n$ & $m_{h}$ & $(-1)^{n} n! \cdot m_h^{-1} \cdot C_{h,n}(\alpha, R)$ \\ \hline 
0 & $1$ & $1$ \\ 
1 & $1$ & $\alpha  h^2+(R-\alpha ) h-R$ \\ 
2 & $(h-2)$ & 
$\alpha ^2 h^3+2 \alpha  h^2 (R-\alpha )+h \left(\alpha ^2+R^2-3 \alpha  R\right)-R (\alpha +R)$ \\ 
3 & $(h-3)$ & $\alpha ^3 h^5+3 \alpha ^2 h^4 (R-2 \alpha )+ 
     \alpha  h^3 \left(13 \alpha ^2+3 R^2-15 \alpha  R\right)$ \\ 
  & & 
    $\qquad + 
     h^2 \left(-12 \alpha ^3+R^3-12 \alpha  R^2+20 \alpha ^2 R\right)+ 
     h \left(4 \alpha ^3-3 R^3+9 \alpha  R^2-6 \alpha ^2 R\right)$ \\ 
  & & 
    $\qquad + 
     2 R (\alpha +R) (2 \alpha +R)$ \\ 
4 & $(h-3)(h-4)$ & 
    $\alpha ^4 h^6+\alpha ^3 h^5 (4 R-9 \alpha )+ 
     \alpha ^2 h^4 \left(31 \alpha ^2+6 R^2-30 \alpha  R\right)$ \\ 
  & & 
    $\qquad + 
     \alpha  h^3 \left(-51 \alpha ^3+4 R^3-36 \alpha  R^2+72 \alpha ^2 R\right)$ \\ 
  & & 
    $\qquad + 
     h^2 \left(40 \alpha ^4+R^4-18 \alpha  R^3+53 \alpha ^2 R^2- 
     72 \alpha ^3 R\right)$ \\ 
  & & 
    $\qquad + 
     h \left(-12 \alpha ^4-3 R^4+14 \alpha  R^3-9 \alpha ^2 R^2+38 \alpha ^3 R\right)+2 R (\alpha +R) (2 \alpha +R) (3 \alpha +R)$ \\ 
5 & $(h-3)(h-4)(h-5)$ & 
    $\alpha ^5 h^7+\alpha ^4 h^6 (5 R-13 \alpha )+ 
     \alpha ^3 h^5 \left(67 \alpha ^2+10 R^2-55 \alpha  R\right)$ \\ 
  & & 
    $\qquad + 
     5 \alpha ^2 h^4 \left(-35 \alpha ^3+2 R^3-18 \alpha  R^2+ 
     43 \alpha ^2 R\right)$ \\ 
  & & 
    $\qquad + 
     \alpha h^3 \left(244 \alpha ^4+5 R^4- 
     70 \alpha  R^3+245 \alpha ^2 R^2-395 \alpha ^3 R\right)$ \\ 
  & & 
    $\qquad + 
     h^2 \left(-172 \alpha ^5+R^5-25 \alpha  R^4+115 \alpha ^2 R^3- 
     245 \alpha ^3 R^2+374 \alpha ^4 R\right)$ \\ 
 & & 
    $\qquad + 
     h \left(48 \alpha ^5- 
     3 R^5+20 \alpha  R^4-5 \alpha ^2 R^3+200 \alpha ^3 R^2- 
     72 \alpha ^4 R\right)$ \\ 
 & & 
    $\qquad + 
     2 R (\alpha +R) (2 \alpha +R) (3 \alpha +R) (4 \alpha +R)$ \\ \hline 
\end{tabular} 

\subcaption{Alternate factored forms of the 
            convergent numerator function subsequences, 
            $C_{h,n}(\alpha, R)\defequals [z^n] \ConvFP{h}{\alpha}{R}{z}$, 
            gathered with respect to powers of $R$ 
            (or $R / \alpha$ scaled by $\alpha^{n}$) and $h$.} 

\begin{tabular}{|c|l|} 
\hline\tabletopstrut 
$n$ & $n! \cdot C_{h,n}(\alpha, R)$ \\ \hline 
%0 & $1$ \\ 
%1 & $-(h-1) (R+ h\alpha)$ \\ 
2 & $\alpha ^2 h^4+2 \alpha  h^3 (R-2 \alpha )+h^2 \left(5 \alpha ^2+R^2-7 
     \alpha  R\right)-h (3 R-2 \alpha ) (R-\alpha )+2 R (\alpha +R)$ \\ 
3 & $-\alpha ^3 h^6-3 \alpha ^2 h^5 (R-3 \alpha )-\alpha  h^4 \left(31 
     \alpha ^2+3 R^2-24 \alpha  R\right)+h^3 \left(51 \alpha ^3-R^3+21 
     \alpha  R^2-65 \alpha ^2 R\right)$ \\ 
  & $\qquad + 
     h^2 \left(-40 \alpha ^3+6 R^3-45 
     \alpha  R^2+66 \alpha ^2 R\right)-h (R-\alpha ) \left(12 \alpha ^2+11 
     R^2-10 \alpha R\right)+6 R (\alpha +R) (2 \alpha +R)$ \\ 
4 & $\alpha ^4 h^8+4 \alpha ^3 h^7 (R-4 \alpha )+2 \alpha ^2 h^6 
     \left(53 \alpha ^2+3 R^2-29 \alpha  R\right)+2 \alpha  h^5 
     \left(-188 \alpha ^3+2 R^3-39 \alpha  R^2+165 \alpha ^2 R\right)$ \\ 
  & $\qquad + 
     h^4 \left(769 \alpha ^4+R^4-46 \alpha R^3+377 \alpha ^2 R^2-936 
     \alpha ^3 R\right)-2 h^3 \left(452 \alpha ^4+5 R^4-94 \alpha  R^3+406 
     \alpha ^2 R^2-703 \alpha ^3 R\right)$ \\ 
  & $\qquad + 
     h^2 \left(564 \alpha ^4+35 R^4-302 \alpha R^3+721 \alpha ^2 R^2- 
     1118 \alpha ^3 R\right)-2 h (R-\alpha ) \left(-72 \alpha ^3+25 R^3-17 
     \alpha  R^2+114 \alpha ^2 R\right)$ \\ 
  & $\qquad + 
     24 R (\alpha +R) (2 \alpha +R) (3 \alpha +R)$ \\ 
5 & $-\alpha ^5 h^{10}-5 \alpha ^4 h^9 (R-5 \alpha )-5 \alpha ^3 h^8 
     \left(54 \alpha ^2+2 R^2-23 \alpha  R\right)-10 \alpha ^2 h^7 
     \left(-165 \alpha ^3+R^3-21 \alpha  R^2+111 \alpha ^2 R\right)$ \\ 
  & $\qquad - 
     \alpha  h^6 \left(6273 \alpha ^4+5 R^4-190 \alpha  R^3+1795 
     \alpha ^2 R^2-5860 \alpha ^3 R\right)$ \\ 
  & $\qquad + 
     h^5 \left(15345 \alpha ^5-R^5+85 \alpha  R^4-1425 \alpha ^2 R^3+ 
     8015 \alpha ^3 R^2-18519 \alpha ^4 R\right)$ \\ 
  & $\qquad + 
     5 h^4 \left(-4816 \alpha ^5+3 R^5-111 \alpha  R^4+1055 \alpha ^2 R^3- 
     4011 \alpha ^3 R^2+7205 \alpha ^4 R\right)$ \\ 
  & $\qquad - 
     5 h^3 \left(-4660 \alpha ^5+17 R^5-339 \alpha  R^4+1947 \alpha ^2 R^3- 
     5703 \alpha ^3 R^2+8438 \alpha ^4 R\right)$ \\ 
  & $\qquad + 
     h^2 \left(-12576 \alpha ^5+225 R^5-2200 \alpha  R^4+ 
     7975 \alpha ^2 R^3-22900 \alpha ^3 R^2+26400 \alpha ^4 R\right)$ \\ 
  & $\qquad - 
     2 h (R-\alpha ) \left(1440 \alpha ^4+137 R^4+7 \alpha  R^3+1802 \alpha ^2 R^2-1848 \alpha ^3 R\right)+120 R (\alpha +R) (2 \alpha +R) (3 \alpha +R) (4 \alpha +R)$ \\ \hline 
\hline 
\end{tabular} 

\end{subtable} 

\caption{The auxiliary convergent numerator function subsequences, 
         $C_{h,n}(\alpha, R) \defequals [z^n] \ConvFP{h}{\alpha}{R}{z}$,  
         defined in Section \ref{subsubSection_Properties_Of_ConvFn_Phz}.}
\label{table_ConvNumFnSeqs_Chn_AlphaR_SpCaseListings} 

\end{table} 

\setcounter{subtable}{0} 

\begin{table}[h] 
\centering 

\smaller 

\begin{subtable}{\textwidth} 
\centering 

\begin{tabular}{|c|l|} \hline 
\hline\tabletopstrut 
$k$ & $(-1)^{h-k} z^{-(h-k)} \cdot R_{h,h-k}(\alpha; z)$ \\ \hline 
%0 & $0$ \\ 
1 & $1$ \\ 
2 & $-\frac{1}{2} \alpha \left(h^2-h+2\right) z+h-1$ \\ 
3 & $\frac{1}{2} (h-2) (h-1) -\frac{1}{2} \alpha (h-2) 
     \left(h^2+3\right) z+\frac{1}{24} \alpha ^2 
     \left(3 h^4-10 h^3+21 h^2-14 h+24\right) z^2$ \\ 
4 & $-\frac{1}{4} \alpha  (h-3) (h-2) \left(h^2+h+4\right) z + 
     \frac{1}{24} \alpha ^2 (h-3) \left(3 h^4-4 h^3+19 h^2-2 h+56\right) z^2$ \\ 
  & $\phantom{-\frac{1}{4}} - 
     \frac{1}{48} \alpha ^3\left(h^6-7 h^5+23 h^4-37 h^3+48 h^2-28 h+48 
     \right) z^3$ \\ 
  & $\phantom{-\frac{1}{4}} + 
      \frac{1}{6} (h-3) (h-2) (h-1)$ \\ 
5 & $-\frac{1}{12} \alpha  (h-4) (h-3) (h-2) \left(h^2+2 h+5\right) z + 
     \frac{1}{48} \alpha ^2 (h-4) (h-3) \left(3 h^4+2 h^3+23 h^2+16 h+100 
     \right) z^2$ \\ 
  & $\phantom{-\frac{1}{12}} - 
     \frac{1}{48} \alpha ^3 (h-4) 
     \left(h^6-4 h^5+14 h^4-16 h^3+61 h^2-12 h+180\right) z^3$ \\ 
  & $\phantom{-\frac{1}{12}} + 
     \frac{\alpha ^4}{5760} \left(15 h^8-180 h^7+950 h^6-2688 h^5+4775
      h^4-5340 h^3+5780 h^2-3312 h+5760\right) z^4$ \\ 
  & $\phantom{-\frac{1}{12}} + 
     \frac{1}{24} (h-4) (h-3) (h-2) (h-1)$ \\ \hline 
\hline\tabletopstrut 
$k$ & $k! (-1)^{h-k} z^{-(h-k)} \cdot R_{h,h-k}(\alpha; z)$ \\ \hline 
%0 & $0$ \\ 
1 & $1$ \\ 
2 & $-\alpha h^2 z+h (\alpha  z+2)-2 (\alpha  z+1)$ \\ 
3 & $\frac{3}{4} \alpha ^2 h^4 z^2- 
     \frac{1}{2} \alpha  h^3 z (5 \alpha  z+6)+ 
     \frac{3}{4} h^2 \left(7 \alpha ^2 z^2+8 \alpha  z+4\right)$ \\ 
  & $\qquad + 
     \frac{1}{2} h \left(-7 \alpha ^2 z^2-18 \alpha z- 
     18\right)+6 \left(\alpha ^2 z^2+3 \alpha  z+1\right)$ \\ 
4 & $-\frac{1}{2} \alpha ^3 h^6 z^3 + 
     \frac{1}{2} \alpha ^2 h^5 z^2 (7 \alpha  z+6) - 
     \frac{1}{2} \alpha  h^4 z \left(23 \alpha ^2 z^2+26 \alpha  z+12\right)$ \\ 
  & $\qquad + 
     \frac{1}{2} h^3 \left(37 \alpha ^3 z^3+62 \alpha ^2 z^2+ 
     48 \alpha  z+8\right)$ \\ 
  & $\qquad + 
     h^2 \left(-24 \alpha ^3 z^3-59 \alpha ^2 z^2-30 \alpha  z-24\right)$ \\ 
  & $\qquad + 
     2 h \left(7 \alpha ^3 z^3+31 \alpha ^2 z^2+42 \alpha  z+ 
     22\right)-24 \left(\alpha ^3 z^3+7 \alpha ^2 z^2+6 \alpha  z+1\right)$ \\ 
5 & $\frac{5}{16} \alpha ^4 h^8 z^4- 
     \frac{5}{4} \alpha ^3 h^7 z^3 (3 \alpha  z+2)+ 
     \frac{5}{24} \alpha ^2 h^6 z^2 \left(95 \alpha ^2 z^2+96 \alpha z+ 
     36\right)$ \\ 
  & $\qquad - 
     \frac{1}{2} \alpha  h^5 z \left(112 \alpha ^3 z^3+150 \alpha ^2 z^2+ 
     95 \alpha  z+20\right)$ \\ 
  & $\qquad + 
     \frac{5}{48} h^4 \left(955 \alpha ^4 z^4+1728 \alpha ^3 z^3+ 
     1080 \alpha ^2 z^2+672 \alpha  z+48\right)$ \\ 
  & $\qquad - 
     \frac{5}{4} h^3 \left(89 \alpha ^4 z^4+250 \alpha ^3 z^3+ 
     242 \alpha ^2 z^2+104 \alpha  z+40\right)$ \\ 
  & $\qquad + 
     \frac{5}{12} h^2 \left(289 \alpha ^4 z^4+1536 \alpha ^3 z^3+ 
     1584 \alpha ^2 z^2+408 \alpha  z+420\right)$ \\ 
  & $\qquad + 
     h \left(-69 \alpha ^4 z^4-570 \alpha ^3 z^3-1270 \alpha ^2 z^2- 
     820 \alpha  z-250\right)$ \\ 
  & $\qquad + 
     120 \left(\alpha ^4 z^4+15 \alpha ^3 z^3+25 \alpha ^2 z^2+ 
     10 \alpha  z+1\right)$ \\ \hline 
\hline 
\end{tabular} 

\end{subtable} 

\caption{The auxiliary convergent numerator function subsequences, 
         $R_{h,k}(\alpha; z) \defequals [R^k] \ConvFP{h}{\alpha}{R}{z}$, 
         defined by 
         Section \ref{subsubSection_Properties_Of_ConvFn_Phz-AuxNumFn_Subsequences}.}
\label{table_ConvNumFnSeqs_Rhk_Alphaz_SpCaseListings} 

\end{table} 

\end{document}

%% End Content;
