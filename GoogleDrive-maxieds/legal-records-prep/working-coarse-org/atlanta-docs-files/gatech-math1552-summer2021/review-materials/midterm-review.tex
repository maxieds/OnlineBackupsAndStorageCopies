\documentclass[12pt]{article}
\usepackage[utf8]{inputenc}
\usepackage{amsmath}
\usepackage{amssymb}
\usepackage{amsthm}
\usepackage{fullpage}
\setlength{\parskip}{1em}
\pagestyle{empty}

\begin{document}

\vskip 1in
\begin{center}
     {\LARGE{\bf Math 1552: Integral Calculus \\[0.2ex] 
                Review Problems for the Midterm Exam \\[1ex] 
                Summer 2021
     }}
\end{center} 
     
     \bigskip 
     \noindent
     \emph{\textbf{***NOTE FROM THE INSTRUCTORS: ***} \\ 
           While the list of sections above does 
           not include every integration technique, please note that students are expected 
           to also understand how to integrate with u-substitutions, by parts, and using 
           trig identities, as these techniques may be needed in order to evaluate 
           integrals from the above listed sections.  You will find some review problems 
           from previous sections incorporated in the problems below.}

\vskip 0.2in

\bigskip\hrule\bigskip

\section{Review for Sections: 4.8, 5.1-5.6, 8.2-8.3} 

\subsection*{Formula Recap}

\noindent 1. Complete each of the following formulas.

\noindent (a) The general Riemann Sum is found using the formula:
\vskip 1 cm
\noindent (b) Some helpful summation formulas are:
$$\sum_{i=1}^n c =$$
$$\sum_{i=1}^n i = $$
$$\sum_{i=1}^n i^2 = $$
\medskip
\noindent (c) Properties of the definite integral:
$$\int_a^a f(x) dx = $$
$$\int_b^a f(x) dx = $$
$$\int_a^b c f(x) dx = $$
\medskip
\noindent (d) State the Fundamental Theorem of Calculus:
\vskip 2 cm
\noindent (e) Using the FTC:
$${{d}\over{dx}} \left [ \int_{a(x)}^{b(x)} f(t) dt \right ] = $$
\vskip 2 cm
\noindent (f) If $F$ is an antiderivative of $f$, that means:
\vskip 1 cm
\noindent (g) If $F$ is an antiderivative of $f$, then:
$$\int f(g(x)) g^{\prime}(x) dx = $$
\medskip
$$\int_a^b f(g(x)) g^{\prime}(x) dx = $$
\medskip
\noindent (h) To find the area between two curves, use the following 
steps:
\vskip 2.5 cm
\noindent (h) Evaluate an integral using {\it integration by 
parts\/} if:
\vskip 1 cm
\noindent To choose the value of $u$, use the rule:
$\_\_\_\_\_\_\_\_\_\_\_\_\_$.
\bigskip
\noindent (i) To evaluate integrals with powers or products of trig 
functions, use the following trig identities to try to obtain a 
$u$-substitution: 
\vskip 4 cm
\noindent 2. Fill in the integration formulas below:

$$\int x^n dx, \quad (n \neq -1) =$$
$$\int \sin (ax) dx = $$
$$\int \cos (ax) dx = $$
$$\int \sec^2 (ax) dx = $$
$$\int \sec (ax) \tan (ax) dx = $$
$$\int \csc (ax) \cot (ax) dx = $$
$$\int \csc^2 (ax) dx = $$
$$\int {{1}\over{1+(ax)^2}} dx = $$
$$\int {{1}\over{\sqrt{1-(ax)^2}}} dx = $$
$$\int {{1}\over{x}} dx = $$
$$\int e^{ax} dx = $$
$$\int b^{ax} dx = $$
$$\int \tan x dx = $$
$$\int \sec x dx = $$
$$\int \csc x dx = $$
$$\int \cot x dx = $$
\vfil\eject

\subsection*{Problems Similar to the Studio Worksheets}

\noindent 1. True or False?

\noindent (a) If $F$ and $G$ are both antiderivatives of $f$, then $F=G$.

\noindent (b) The antiderivative of $\sec^2 (3x)$ is ${{1}\over{3}} \tan 
(3x)$.

\noindent (c) The indefinite integral of a function $f$ is the collection of 
all antiderivatives of $f$.

\noindent (d) We know how to find the antiderivative of $\cos(x^2)$, and it is 
$\sin (x^2)$.

\noindent (e) To find the upper sum $U_f$ of a function $f$ on $[a,b]$, 
after partitioning the interval into $n$ pieces, evaluate $f$ at the right-hand 
endpoint of each subinterval.

\noindent (f) When the interval $[a,b]$ is partitioned into $n$ pieces, there are 
exactly $n$ endpoints.

\noindent (g) A partition of the interval $[a,b]$ does not need to be evenly spaced 
in order to calculate a Riemann Sum.

\noindent (h) If $f$ is positive and continuous on $[a,b]$, and $A$ is the 
actual area bounded by $f$, $x=a$, $x=b$, and the $x$-axis, then $L_f < A < 
U_f$.

\noindent (i) We always set $x_i^*$ to be the right-hand endpoint of the $i^{th}$ 
interval.

\noindent (j) $\sum_{i=1}^{n} i^2 = \left ( {{n(n+1)}\over{2}} \right )^2.$

\noindent (k) If $f(x) \ge 0$ on $[a,b]$, then $\int_a^b f(x) \; dx$ represents the 
total area bounded by $f$, $x=a$, $x=b$, and the $x$-axis.

\noindent (l) If $f$ is a continuous function, then the function $F(x) = \int_a^x 
f(t) dt$ is an anti-derivative of $f$.

\noindent (m) If $F$ is an anti-derivative of $f$, then $\int_a^b f(x) dx$ 
represents the slope of the secant line of F(x) on the interval $[a,b]$.

\noindent (n) ${{d}\over{dx}} \left [ \int_a^b f(t) dt \right ] = f(x)$.

\noindent (o) Given that $f$ is continuous on $[a,b]$ and $F^{\prime}(x) = f(x)$, 
then $F(b)-F(a)$ represents the 
net area bounded by the graph of $y=f(x)$, the lines $x=a$, $x=b$, and the 
$x$-axis.

\noindent (p) $\int f(x) g(x) \; dx = \left ( \int f(x) \; dx \right ) \cdot \left 
( \int g(x) \; dx \right )$

\noindent (q) To evaluate $\int \sin^{-1} (x) dx$ by parts, choose $u=\sin^{-1} 
(x)$ and $dv = dx$.

\noindent (r) To evaluate $\int x \ln (x) \; dx$ by parts, choose $u=x$ and $dv = 
\ln (x) \; dx$.

\noindent (s) To evaluate $\int \cot(x) \; dx$, integrate by substitution choosing 
$u=\sin (x)$.

\vskip 1in

\noindent 2. Evaluate the following indefinite integrals.

\noindent (a) $\int \left ( \sqrt{x} - {{1}\over{x}} \right )^2 dx$

\noindent (b) $\int \left [ 4^{-2x} + e^{-5x}  \right ] dx$

\noindent (c)  $\int \left ( {{e^{\sqrt{2}} +  
x^{\sqrt{2}}}\over{\sqrt{x}}} \right ) dx$

\noindent (d) $\int \left ( {{2}\over{3x}} - 
{{1}\over{\sqrt{4-x^2}}} \right ) dx$

\bigskip
\noindent 3. A particle travels with a velocity given by $v(t) = 
-{{1}\over{3}} t^2 + 4t +2$, where position is measured in meters and time in 
seconds.  
 
\noindent (a) Find the acceleration of the particle when $t=1$ second.

\noindent (b) If the initial position is 4 m, find the position of the 
particle at $t=1$ second.

\bigskip
\noindent 4.  ({\it Applying the Riemann Sum\/})
You are driving when all of a sudden, you see traffic stopped in front of 
you.  You slam the brakes to come to a stop.  While your brakes are 
applied, the velocity of the car is measured, and you obtain the 
following measurements:
\medskip
Time since applying breaks (sec) \hskip 0.2 cm 0 \hskip 0.4 cm 1 \hskip 
0.4 cm 2 \hskip 0.4 cm 3 \hskip 0.4 cm 4 \hskip 0.4 cm 5

Velocity of car (in ft/sec) \hskip 1.2 cm 88 \hskip 0.2 cm 60 \hskip 0.2 
cm 40 \hskip 0.2 cm 25 \hskip 0.3 cm 10 \hskip 0.35 cm 0

\medskip
\noindent (a) Plot the points on a curve of velocity vs. time.

\noindent (b) Using the points given, determine upper and lower bounds 
for the total distance traveled before the car came to a stop.

\bigskip
\noindent 5.  Estimate the area under the graph of $f(x) = 10 - x^2
$ between the lines $x=-3$ and $x=2$ using $n=5$ equally spaced
subintervals, by finding:

\noindent (a) the upper sum, $U_f$.

\noindent (b) the lower sum, $L_f$.

\medskip
\noindent 6.  ({\it Applying the Definite Integral\/})
A marketing company is trying a new campaign.  The campaign lasts for 
three weeks, and during this time, the company finds that it gains 
customers as a function of time according to the formula:
$$C(t) = 5t - t^2,$$
where $t$ is time in weeks and the number of customers is given in 
thousands.

\noindent Using the general form of the definite integral,
$$\int_a^b f(x) dx = \lim_{n \to \infty} {{b-a}\over{n}} \sum_{i=1}^{n} 
f(x_i^*),$$
calculate the {\bf average} number of customers gained during the 
three-week campaign.

\medskip
\noindent 7.  Explain why the following property is true:
$$|\int_a^b f(x) dx | \leq \int_a^b |f(x)| dx.$$
Can you find an example where the inequality is strict?
\medskip
\noindent 8.  Using the general form of the definite integral,
$\int_a^b f(x) dx = \lim_{n \to \infty} \sum_{i=1}^{n} 
f(x_i^*) \Delta x$, evaluate:
$$\int_2^4 (x-1)^2 dx.$$
\medskip
\noindent 9.  Evaluate $\int_{0}^{2} |x-1| dx$ using integral properties 
from class.  (HINT: draw a picture, and use geometry!)

\medskip
\noindent 10.  Suppose that $f(x)$ is an even function such 
that $\int_0^2 f(x)dx = 5$ and
$\int_0^3 f(x)dx = 8$. Find the value of $\int_{-2}^3 f(x)dx$.
\medskip
\noindent 11.  Evaluate the integrals:

\noindent (a) $\int_{1}^{2} {{3x-5}\over{x^3}} dx$.

\noindent (b) $\int_2^5 (2-x)(x-5) dx$.

\noindent (c) $\int_{\pi}^{{{7 \pi}\over{2}}} {{1 + \cos(2t)}\over{2}} dt$.

\bigskip
\noindent 12.  Find $F^{\prime}(2)$ for the function
$$F(x) = \int_{{8}\over{x}}^{x^2} \left ( {{t}\over{1 - \sqrt{t}}} 
\right ) dt.$$
\bigskip
\noindent 13.  (a) Given the function $f$ below, evaluate $\int_1^9 f(x) dx$.
$$f(x) = \left \{ \begin{matrix} x^2 + 4, & x < 4 \cr \sqrt{x} - x, & x \ge 4 \end{matrix} \right\}.$$ 

\noindent (b) Would you get the same answer to part (a) if you evaluated 
$F(9)-F(1)$?  What does this tell you about the FTC and continuity?

\bigskip
\noindent 14.  (a) Evaluate the expressions:
$$\int_0^1 x(1+x) \; dx, \quad \left ( \int_0^1 x \; dx \cdot \int_0^1 
(1+x) \; dx 
\right )$$

\noindent (b) Looking at your answer in part (a), what, if anything, can 
you say in general about $\int (f(x) \cdot g(x)) dx$?

\bigskip
\noindent 15.  For each integral below, determine if we can evaluate the 
integral using the method of $u$-substitution.  If the answer is "yes", 
evaluate the integral.

\noindent (a) $\int {{1}\over{x^2}} \sec \left ( {{1}\over{x}} \right ) 
\tan \left ( {{1}\over{x}} \right ) dx$

\noindent (b) $\int x \csc^2 (x) dx$

\noindent (c) $\int {{\sin 3x - \cos 3x}\over{\sin 3x + \cos 3x}} dx$

\noindent (d) $\int \tan (x^2) dx $

\bigskip
\noindent 16.  Evaluate the following integrals using the method of 
substitution.

\noindent (a) $\int {{1}\over{\ln(x^x)}} dx$

\noindent (b) $\int {{e^{2x}}\over{\sqrt{4-3e^{2x}}}} dx$

\noindent (c) $\int {{dx}\over{\sqrt{4 - (x+3)^2}}}$

\noindent 17.  Suppose that $y=f(x)$ and $y=g(x)$ are both continuous 
functions on the interval $[a,b]$.  Determine if each statement below is always 
true or sometimes false.

\noindent (a) Suppose that $f(c) > g(c)$ for some number $c \in (a,b)$.  Then the 
area bounded by $f$, $g$, $x=a$, and $x=b$ can be found by evaluating the integral 
$\int_a^b (f(x) - g(x) ) \; dx$. 

\noindent (b) If $\int_a^b (f(x) - g(x)) \; dx$ evaluates to -5, then the area 
bounded by $f$, $g$, $x=a$, and $x=b$ is 5.

\noindent (c) If $f(x) > g(x)$ for every $x \in [a,b]$, then
$\int_a^b |f(x) - g(x) | \; dx = \int_a^b (f(x) - g(x)) \; dx$.

\noindent 18.  Find the area bounded by the region
between the curves $f(x)=x^3+2x^2$ and 
$g(x)=x^2+2x$.

\medskip
\noindent 19.  Find the area bounded by the region enclosed by the three 
curves $y=x^3$, $y=-x$, and $y=-1$.

\medskip
\noindent 20.  Find the area bounded by the curves $y=\cos x$
and $y=\sin(2x)$ on the interval $\left [ 0, {{\pi}\over{2}} \right ]$.

\medskip
\noindent 21.  Find the area of the triangle with vertices at the points 
(0,1), (3,4), and (4,2).  USE CALCULUS.

\medskip
\noindent 22.  For each function below: (i) determine which method to use to 
evaluate the function (formula, u-substitution, or integration by parts, 
and (ii) evaluate the integral.

\medskip
\noindent (a) $\int_1^e {{\sqrt{\ln x}}\over{x}} \; dx$

\noindent (b) $\int (\ln x)^2 \; dx$

\noindent (c) $\int x^2 e^{x^3} \; dx$

\noindent (d)  $\int x^3 e^{x^2} \; dx$

\noindent (e) $\int 4^{-x} \; dx$

\noindent (f)  $\int x^2 \cdot 4^x \; dx$

\bigskip
\noindent 23. Determine if each integral below can be evaluated using a method we 
have 
learned so far (formula, u-substitution, integration by parts, or trig identities).  
If so, evaluate the integral.  If not, explain why it cannot be evaluated.

\medskip
\noindent (a)  $\int x^5 \ln (x) \; dx$

\noindent (b)  $\int \sin^5(2x) \cos^3 (2x) \; dx$

\noindent (c)  $\int \cos^2 (3x) \; dx$

\noindent (d)  $\int \tan (x) \ln [\cos (x)] \; dx$

\noindent (e)  $\int \sin \left (x^2 \right ) \; dx$

\noindent (f)  $\int \tan^4(x) dx$

\noindent (g)  $\int e^{2x} \sin(3x) dx$

\vskip 1in

\subsection*{Additional Midterm Review Problems}

\noindent 24.  True or false?

\noindent (a) When evaluating a {\bf definite} integral using $u$-subsitution, 
different choices of $u$ may lead to different final answers.

\noindent (b) Integration by Parts is a Product Rule in integral form.

\noindent (c) The goal of integration by parts is to go from an integral $\int 
f'(x) g'(x) dx$ that we can't evaluate to an integral $\int f(x) g(x) 
dx$ that we can evaluate.

\noindent (d) Definite integrals can not be evaluated by Integration by Parts.

\noindent (e) If $f$ is a continuous, increasing function, then the right-hand 
Riemann sum method always overestimates the definite integral.

\noindent (f)  Let $f$ be a continuous function and $av(f)$ be the 
average of $f$. Then $av(f) \cdot (b-a) =\int_{a}^{b} f(x)dx$.

\noindent (g) When finding the area between the curves $y=x^3-x$ 
and $y=x^2+x$ it suffices to find the value of the definite integral $\int_{-1}^2 
\left [ (x^3 - x) - (x^2+x) \right ] \; dx$, and then take the absolute value of 
this value to get the right answer.

\noindent (h) To find the area between the curves $y=x^3-x$ and 
$y=x^2+x$ , first set the equations equal and solve to find the intersection points 
$x=-1$ and $x=2$,  plug in a test-point into the equations or graph the curves to 
determine {\bf top} and {\bf bot}, and then evaluate $\int_{-1}^2 
({\bf top}) - ({\bf bot}) \; dx$.

\noindent (i) If $\int_0^1 f(x) \; dx = 9$ and $f(x) \ge 0$, then 
$\int_0^1 \sqrt{f(x)} \; dx = 3$.

\bigskip 
\noindent 25.  Evaluate the following integrals.

\noindent (a) $\int_{0}^{{{\pi}\over{4}}} \sec ^2 (t) e^{1+\tan (t)} \; dt$

\noindent (b) $\int \sin^3(x)\cos^3(x) \; dx$

\noindent (c) $\int {{1}\over{\sqrt{4-9w^2}}} \; dw$

\noindent (d) $\int x \sin (x) \cos (x) \; dx$

\noindent (e) $\int \sec^4(x) \; dx$

\noindent (f) $\int \ln (x+1) \; dx$

\noindent (g) $\int \left(1+\sqrt{x}\right)^{12} dx$ (HINT: Do not expand out the integrand.)

\bigskip
\noindent 26. Suppose: $f(1)=2, f(4)=7, f'(1)=5, f'(4)=3$ and $f''(x)$ is 
continuous. Find the value of:
$$\int _1^4 x f''(x) \; dx.$$
\bigskip
\noindent 27.   Consider the following limit
$$\lim_{n\to\infty} \sum_{i=1}^{n} \cos \left (\pi\cdot 
{{i}\over{n}} \right )\cdot{{\pi}\over{2n}}.$$

\noindent (a) Express the limit as a definite integral.

\noindent (b)  Compute the definite integral from part (a).

\bigskip
\noindent 28.  Let $f(x) = 3x+4$.

\noindent (a) Estimate the area of the region between the graph of $f$, the lines 
$x=-1$ and $x=2$,  and the 
$x$-axis using a upper sum with three rectangles of equal width.

\noindent (b) Find the actual area in part (a) by taking the limit of a general 
Riemann Sum
using $n$ equally spaced subintervals, and taking $x^{*}_i$ as the right-hand 
endpoint of each interval.

\bigskip
\noindent 29. Find the area bounded by the curves $y=\cos^2(x)$ and $y=-\sin^2(x)$, 
and the 
lines $x=0$ and $x=\pi$. (Hint: draw a picture in GeoGebra - an online 
graphing tool.)

\bigskip
\noindent 30. Find the area bounded by the curves $y=-x^2+6x$ and $y=x^2-2x-24$. 
(Hint: sketch the curves or make a sign chart.)

\bigskip
\noindent 31. Find $F^{\prime}(4)$ if $$F(x) = 
\int_{{x^2}\over{4}}^{x^2} \ln (\sqrt{t}) \; dt.$$

\bigskip
\noindent 32. What value of $b > -1$ maximizes the integral:
$$\int_{-1}^b x^2(7-x) dx?$$

\bigskip
\noindent 33.  Find a number $c$ so that $f(c)$ is equal to the average 
value of the function $f(x) = 1 + x$ on the interval $[-1,3]$.  
Graphically, what does that mean?

\vskip 1in

\subsection*{Answers to Selected Problems}

\noindent 1.  (b), (c), (g), (k), (l), (o), (q), (s) are true

\noindent 2.  (a) ${{1}\over{2}} x^2 - 4 \sqrt{x} - {{1}\over{x}} + C$

\noindent (b) $-{{1}\over{2 \ln 4}} 4^{-2x} -{{1}\over{5}} e^{-5x} + C$

\noindent (c)  $2 e^{\sqrt{2}} \sqrt{x} + {{1}\over{\sqrt{2}+1/2}} 
x^{\sqrt{2}+1/2} + C$

\noindent (d) ${{2}\over{3}} \ln |x| - \sin^{-1} \left ( {{x}\over{2}} 
\right ) + C$

\noindent 3.  (a) ${{10}\over{3}}$ $m/s^2$, (b) $7 {{8}\over{9}}$ m

\noindent 4. (b) Upper:  223 ft, Lower: 135 ft

\noindent 5. (a) 44 (b) 31

\noindent 6.  4,500 customers

\noindent 7.  Consider the difference between NET and TOTAL area.

\noindent 8.  ${{26}\over{3}}$ 

\noindent 9.  1 

\noindent 10. 13

\noindent 11. (a) $-{{3}\over{8}}$;  (b) ${{9}\over{2}}$;  (c) ${{5 \pi}\over{4}}$

\noindent 12. -24

\noindent 13. (a) ${{79}\over{6}}$; (b) you cannot use the FTC as stated when $f$ 
is discontinuous somewhere on the interval $[a,b]$

\noindent 14.  (a) ${{5}\over{6}}$ and ${{3}\over{4}}$; no general rule

\noindent 15.  (a) $- \sec \left ( {{1}\over{x}} \right ) + C$, (c) 
$-{{1}\over{3}} \ln |\sin 3x + \cos 3x| + C$

\noindent 16.  (a) $\ln |\ln x| + C$, (b) $-{{1}\over{3}} \sqrt{4-3e^{2x}} 
+ C$, (c) $\sin^{-1} \left ( {{x+3}\over{2}} \right ) + C$

\noindent 17. (c) is true 

\noindent 18. ${{37}\over{12}}$ 

\noindent 19. ${{5}\over{4}}$ 

\noindent 20. ${{1}\over{2}}$ 

\noindent 21. 5. 4.5

\noindent 22. (a) ${{2}\over{3}}$

\noindent (b) $x (\ln x)^2 - 2x \ln x + 2x + C$ 

\noindent (c) ${{1}\over{3}} e^{x^3} + C$

\noindent (d) ${{x^2 e^{x^2}}\over{2}} - {{e^{x^2}}\over{2}} + C$

\noindent (e) $-{{1}\over{\ln 4}} 4^{-x} + C$

\noindent (f) ${{1}\over{\ln 4}} x^2 \cdot 4^x - {{2}\over{(\ln 
4)^2}} x \cdot 4^x + {{2}\over{(\ln 4)^3}} 4^x + C$

\noindent 23. (a) ${{x^6 \ln x}\over{6}} - {{x^6}\over{36}} + C$

\noindent (b) ${{1}\over{12}} \sin^6 (2x) -  {{1}\over{16}} \sin^8 (2x) + C$

\noindent (c) ${{1}\over{2}} x + {{1}\over{12}} \sin (6x) + C$

\noindent (d) $-{{1}\over{2}} (\ln [\cos(x)])^2 + C$

\noindent (e) Cannot be evaluated

\noindent (f) ${{1}\over{3}} \tan^3 (x) - \tan (x) + x + C$

\noindent (g) ${{2}\over{13}} e^{2x} \sin(3x) - {{3}\over{13}} e^{2x} \cos(3x) +
C$

\noindent 24.  (e), (f) are true

\noindent 25.  (a)  $e^2-e$

\noindent (b)  ${{1}\over{6}}\cos^6(x)-{{1}\over{4}}\cos^4(x)+C$

\noindent (c)  ${{1}\over{3}} \arcsin \left ( {{3w}\over{2}} \right ) +C$

\noindent (d)  ${{x}\over{2}} \sin^2 x - {{1}\over{4}}x + {{1}\over{8}}\sin 2x +C$

\noindent (e)  $\tan(x)+{{\tan^3(x)}\over{3}}+C$

\noindent (f) $(x+1) \ln (x+1) - (x+1) + C$

\noindent (g) $\frac{1}{7}(1+\sqrt{x})^{14} - \frac{2}{13}(1+\sqrt{x})^{13} + C$

\noindent 26.  2

\noindent 27.  (a) $\int_{0}^{{\pi}\over{2}} \cos(2x) dx$, (b) 0

\noindent 28.  (a) 21, (b) 16.5

\noindent 29.  $\pi$

\noindent 30.  ${{512}\over{3}}$ or approximately $170.67$

\noindent 31.  $14 \ln 2$

\noindent 32.  $b=7$

\noindent 33.  $c=1$

\section{Review for Sections: 4.5, 8.4-8.5, 8.8}

\subsection*{Content Recap}

\noindent (a) To 
apply L'Hopital's rule, the limit must have 
the indeterminate form 
$\_\_\_\_\_\_\_\_\_\_\_\_\_$ or $\_\_\_\_\_\_\_\_\_\_\_\_\_$.

\bigskip
\noindent (b) An integral $\int_a^b f(x) dx$ is {\it improper\/} if at 
least one of the 
limits of integration is \hfil\break $\_\_\_\_\_\_\_\_\_\_\_\_\_$, or if 
there is a
$\_\_\_\_\_\_\_\_\_\_\_\_\_$ $\_\_\_\_\_\_\_\_\_\_\_\_\_$ on the interval 
$[a,b]$.

\bigskip
\noindent (c) If we would evaluate an integral using {\it trig 
substitution\/}, the 
integral should contain an expression of one of these forms:
$\_\_\_\_\_\_\_\_\_\_\_\_\_$, $\_\_\_\_\_\_\_\_\_\_\_\_\_$, or 
$\_\_\_\_\_\_\_\_\_\_\_\_\_$.

\bigskip
\noindent Write out the trig substitution you would use for each form listed 
above.

\vskip 2 cm
\noindent (d) To use the method of {\it partial fractions\/}, we must first 
factor the denominator completely into $\_\_\_\_\_\_\_\_\_\_\_\_\_$ or 
$\_\_\_\_\_\_\_\_\_\_\_\_\_$ $\_\_\_\_\_\_\_\_\_\_\_\_\_$ terms. 

\bigskip
\noindent In the partial fraction decomposition, if the term in the 
denominator is raised to the $k$th power, then we have 
$\_\_\_\_\_\_\_\_\_\_\_\_\_$ partial fractions. 

\bigskip
\noindent For each linear term, the numerator of the partial fraction will 
be $\_\_\_\_\_\_\_\_\_\_\_\_\_$.

\bigskip
\noindent For each irreducible quadratic term, the numerator will be 
$\_\_\_\_\_\_\_\_\_\_\_\_\_$.

\subsection*{Studio Worksheet Type Problems}

\noindent 2.  Determine if the following statements below are always true 
or sometimes false.

\noindent (a) If an integral contains the term $a^2+x^2$, we should use 
the substitution $x=a \sec \theta$.

\smallskip
\noindent (b) The expression $\tan \left ( \sin^{-1} (x) \right )$ cannot 
be simplified.

\smallskip
\noindent (c) When using a trig substitution with a term of the form 
$a^2-x^2$, we could use either $x=a \sin \theta$ or $x= a 
\cos \theta$ and obtain equivalent answers (that may differ only by a 
constant).

\smallskip
\noindent (d) If we use the trig substitution $x = \sin \theta$, then 
it is possible that $\sqrt{1-x^2} = -\cos \theta$. 

\smallskip
\noindent (e) The partial fraction decomposition of ${{x}\over{(x+3)^2}}$ 
is ${{A}\over{x+3}} + {{B}\over{(x+3)^2}}$.

\smallskip
\noindent (f) $\int {{dx}\over{(x+3)^2}} = \ln (x+3)^2 + C.$

\smallskip
\noindent (g) The integral $\int {{x}\over{x^2-9}} dx$ could be best 
evaluated using the method of partial fractions.

\smallskip
\noindent (h) The integral $\int {{dx}\over{x (x^4+1)}}$ cannot be 
evaluated using the method of partial fractions. 

\smallskip
\noindent (i) $\lim_{x \to \infty} xe^x$ has the indeterminate form 
$\infty^{\infty}$.

\smallskip
\noindent (j) $\lim_{x \to 0^+} (\cos x)^{{1}\over{x}}$ has the indeterminate 
form $1^{\infty}$.

\smallskip
\noindent (k) $\lim_{x \to \infty} \left ( 1 + {{1}\over{x}} \right )^{2x} = 
2e$.

\smallskip
\noindent (l) When evaluating a limit using L'Hopital's rule, we first need to 
find $\left ( {{f}\over{g}} \right )^{\prime}$. 

\smallskip
\noindent (m) If $f$ has a vertical asymptote at $x=a$, then $\int_a^b f(x) dx 
= \lim_{c \to a^+} \int_c^b f(x) dx$.

\smallskip
\noindent (n) $\int_{-1}^1 {{1}\over{x}} dx = 0.$

\smallskip
\noindent (o) Saying that an improper integral converges means that the 
integral must evaluate to a finite number.

\smallskip
\noindent (p) Indefinite integrals can be improper.

\bigskip
\noindent 3. Evaluate the following integrals using any method we have 
learned so far: \hfil\break
$u$-substitutions, integration by parts, integrating trig functions,
trigonometric substitutions, or partial fractions.

\noindent (a) $\int {{x^2}\over{(x^2+4)^{3/2}}} dx$

\noindent (b)  $\int (x^2+1) e^{2x} dx$

\noindent (c) $\int {{\sqrt{1-x^2}}\over{x^4}} dx$

\noindent (d)  $\int {{dx}\over{e^x \sqrt{e^{2x} - 9}}}$

\noindent (e)  $\int \sin^2(x) \cos^2(x) dx$

\noindent (f) $\int {{x+3}\over{(x-1)(x^2-4x+4)}} dx$

\noindent (g) $\int x^5 \ln (x) dx$

\noindent (h) $\int {{x+4}\over{x^3+x}} dx$

\noindent (i) $\int \sqrt{25-x^2} dx$

\noindent (j) $\int {{x-1}\over{(x+1)^3}} dx$

\noindent (k) $\int {{x+2}\over{x+1}} dx$

\noindent (l) $\int {{x+1}\over{x^2(x-1)}} dx$

\medskip
\noindent 4.  Evaluate the following limits using L'Hopital's Rule.

\noindent (a) $\lim_{x \to 0^+} [x (\ln(x))^2]$

\noindent (b) $\lim_{x \to \infty} (x + e^x)^{2/x}$

\noindent (c) $\lim_{x \to {{\pi}\over{2}}} \left [ {{\ln (\sin x)}\over{(\pi - 
2x)^2}} \right ]$

\bigskip
\noindent 5.  Evaluate the improper integrals if they converge, or show 
that the integral diverges.

\noindent (a) $\int_0^3 {{x}\over{(x^2-1)^{2/3}}} dx$

\noindent (b) $\int_0^{\infty} x^2 e^{-2x}dx$

\noindent (c) $\int_1^4 {{dx}\over{x^2-5x+6}}$

\subsection*{Additional Midterm Review Problems}

\noindent 10. Determine if each statement below is always true or sometimes 
false.

\noindent (a) $\lim_{x\to 2} {{x-2}\over{x^2+x-6}}$ is of an 
indeterminate form.

\smallskip
\noindent (b) $\lim_{x\to \infty} \left ( 1-{{1}\over{x}} \right )^x=e$

\smallskip
\noindent (c) The integral $\int x^3 \sqrt{1-x^2} \; dx$ can be evaluated by 
trigonometric substitution by setting $x=\sin x$.

\smallskip
\noindent (d) $\sin(\cos^{-1}(x)) = \tan (x)$.

\smallskip
\noindent (e) For the rational expression ${{x}\over{(x+10)(x-10)^2}}$ , the 
partial fraction decomposition is of the form ${{A}\over{x+10}} + 
{{B}\over{(x-10)^2}}$.

\smallskip
\noindent (f) For the rational expression ${{2x+3}\over{x^2 (x+2)^2}}$, the 
partial fraction decomposition is of the form 
${{A}\over{x^2}}+{{B}\over{x+2}}+{{C}\over{(x+2)^2}}$.

\smallskip
\noindent (m) The integral $\int_{-1}^1 {{1}\over{x^2}} \; dx$ can be evaluated 
using the Fundamental Theorem of Calculus.

\noindent 11.  Evaluate each integral using any method we have learned.

\noindent (a) $\int {{2x+1}\over{x^2-7x+12}} \; dx$

\noindent (b) $\int {{8 \; dx}\over{x^2 \sqrt{4-x^2}}}$

\noindent (c) $\int {{8 \; dx}\over{(4x^2+1)^2}}$

\noindent (d) $\int {{1}\over{(x+1)(x^2+1)}}\; dx$

\bigskip
\noindent 12.  Use L'Hopital's rule to evaluate the following limits.

\noindent (a) $\lim_{x\to\infty}(\ln x)^{{1}\over{x^2+1}}$

\noindent (b) $\lim_{x\to 0^+}(\ln x)^x$

\noindent (c) $\lim_{x \to 0} \left [ {{1}\over{x}} - \cot x 
\right ]$

\noindent (d) $\lim_{x \to \infty} \left [ \cos \left ( {{1}\over{x}} \right ) 
\right ]^x$

\medskip
\noindent 13.  Find values of $a$ and $b$ so that
$$\lim_{x \to 0} {{\cos (ax) - b}\over{2x^2}} = -4.$$

\subsection*{Answers to Selected Questions} 

\noindent 2.  (c), (e), (h), (j), (m), (o), (s), (u), (w), (x), (y) are true

\noindent 3. (a) $\ln | {{\sqrt{x^2+4}}\over{2}} + {{x}\over{2}} | - 
{{x}\over{\sqrt{x^2+4}}} + C$
\hskip 0.2 cm
(b) ${{1}\over{2}} (x^2+1) e^{2x} - {{1}\over{2}}
x e^{2x} + {{1}\over{4}} e^{2x} + C$

\smallskip
\noindent (c) $-{{1}\over{3}} \cdot {{(1-x^2)^{3/2}}\over{x^3}} + C$
\hskip 0.1 cm (d) ${{\sqrt{e^{2x}-9}}\over{9e^x}} + C$
\hskip 0.1 cm
(e) ${{x}\over{8}} - {{1}\over{32}} \sin(4x) + C$

\noindent (f)  $4 \ln \left | {{x-1}\over{x-2}} \right | - 
{{5}\over{x-2}} + 
C$ (partial fractions)

\smallskip
\noindent (g) ${{x^6 \ln x}\over{6}} - {{x^6}\over{36}} + C$ (by parts)

\smallskip
\noindent (h)  $4 \ln |x| - 2 \ln (x^2+1) + \tan^{-1} (x) + C$ (partial 
fractions)

\noindent (i) ${{25}\over{2}} \sin^{-1} \left ( {{x}\over{5}} \right ) + 
{{x \sqrt{25-x^2}}\over{2}} + C$ (trig sub)

\noindent (j)  $-{{1}\over{x+1}} + {{1}\over{(x+1)^2}} + C$ \hskip 0.5 cm
(k)  $x + \ln|x+1| + C$ 

\noindent (l)  $-2 \ln|x| + {{1}\over{x}} + 2 \ln|x-1| + C$

\noindent 4. (a) 0, (b) $e^2$, (c) $-{{1}\over{8}}$ 

\noindent 5. (a) ${{9}\over{2}}$, (b) ${{1}\over{4}}$, (c) diverges

\noindent 10.  (a), (c), (i), (j), (l) are true

\noindent 11. (a) $-7 \ln|x-3|+9\ln |x-4|+C$,  (b) ${{-2\sqrt{4-x^2}}\over{x}}+ 
C$ 

\noindent (c) $2 \tan ^{-1}(2x)+{{4x}\over{4x^2+1}}+C$, (d) 
${{1}\over{2}}\ln|x+1|+{{1}\over{2}}\arctan x-{{1}\over{4}}\ln|x^2+1|+C$

\noindent 12. (a) 1, (b) 1, (c) 0, (d) 1

\noindent 13.  $a = \pm 4$, $b=1$

\end{document}




