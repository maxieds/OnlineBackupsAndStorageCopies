\documentclass[12pt]{article}

\usepackage[utf8]{inputenc}
\usepackage{amsmath}
\usepackage{amssymb}
\usepackage{amsthm}
\usepackage{fullpage}

\usepackage[total={7in, 10in},tmargin=0.75in,headsep=8pt]{geometry}

\setlength{\parskip}{1em}
\pagestyle{empty}

\begin{document}

\vskip 1in
\begin{center}
     {\LARGE{\bf Math 1552: Integral Calculus \\[0.2ex] 
                Review Problems for the Final Exam \\[1ex] 
                Summer 2021
     }}
\end{center} 
     
     \bigskip 
     \noindent
     \emph{\textbf{***PLEASE NOTE: ***} \\ 
           In preparing for the final exam, you should review the 
           following problems from our past study guides, and the additional 
           problems on volumes listed below.
           You will also find some review problems 
           from previous sections incorporated in the problems below. 
           While the list of sections above does 
           not include every integration technique, please note that students are expected 
           to also understand all of the techniques we have seen in the class, and how to 
           combine them in a single problem. This means, for example, that you may have to 
           apply a substitution followed by another method in an integration problem, or apply 
           L'Hopital's rule to evaluate an improper integral, as we have seen examples of 
           throughout the course.
     }


\vskip 0.2in

\bigskip\hrule\bigskip

\section{Review of Sections 8.4--8.5, 4.5, 8.8, 10.1--10.2}

\subsection*{Content Recap}

\noindent (a) To 
apply L'Hopital's rule, the limit must have 
the indeterminate form 
$\_\_\_\_\_\_\_\_\_\_\_\_\_$ or $\_\_\_\_\_\_\_\_\_\_\_\_\_$.

\bigskip
\noindent (b) An integral $\int_a^b f(x) dx$ is {\it improper\/} if at 
least one of the 
limits of integration is \hfil\break $\_\_\_\_\_\_\_\_\_\_\_\_\_$, or if 
there is a
$\_\_\_\_\_\_\_\_\_\_\_\_\_$ $\_\_\_\_\_\_\_\_\_\_\_\_\_$ on the interval 
$[a,b]$.

\bigskip
\noindent (c) If we would evaluate an integral using {\it trig 
substitution\/}, the 
integral should contain an expression of one of these forms:
$\_\_\_\_\_\_\_\_\_\_\_\_\_$, $\_\_\_\_\_\_\_\_\_\_\_\_\_$, or 
$\_\_\_\_\_\_\_\_\_\_\_\_\_$.

\bigskip
\noindent Write out the trig substitution you would use for each form listed 
above.

\vskip 2 cm
\noindent (d) To use the method of {\it partial fractions\/}, we must first 
factor the denominator completely into $\_\_\_\_\_\_\_\_\_\_\_\_\_$ or 
$\_\_\_\_\_\_\_\_\_\_\_\_\_$ $\_\_\_\_\_\_\_\_\_\_\_\_\_$ terms. 

\bigskip
\noindent In the partial fraction decomposition, if the term in the 
denominator is raised to the $k$th power, then we have 
$\_\_\_\_\_\_\_\_\_\_\_\_\_$ partial fractions. 

\bigskip
\noindent For each linear term, the numerator of the partial fraction will 
be $\_\_\_\_\_\_\_\_\_\_\_\_\_$.

\bigskip
\noindent For each irreducible quadratic term, the numerator will be 
$\_\_\_\_\_\_\_\_\_\_\_\_\_$.

\bigskip
\noindent (e) Define the least upper bound and greatest lower bound of a 
sequence.

\vskip 2 cm
\noindent (f) What does it mean for a sequence to be monotonic?

\vskip 2 cm
\noindent (g) A sequence {\bf converges} if $\_\_\_\_\_\_\_\_\_\_\_\_\_$ and 
{\bf diverges} if $\_\_\_\_\_\_\_\_\_\_\_\_\_$.

\bigskip
\noindent (h) If a sequence is $\_\_\_\_\_\_\_\_\_\_\_\_\_$ and 
$\_\_\_\_\_\_\_\_\_\_\_\_\_$, then it converges.

\bigskip
\noindent (i) A geometric series has the general form 
$\_\_\_\_\_\_\_\_\_\_\_\_\_$.  \hfil\break
The series converges when 
$\_\_\_\_\_\_\_\_\_\_\_\_\_$ and diverges when 
$\_\_\_\_\_\_\_\_\_\_\_\_\_$.

\medskip
\noindent (j) The harmonic series  has the general 
form $\_\_\_\_\_\_\_\_\_\_\_\_\_$, and it always $\_\_\_\_\_\_\_\_\_\_\_\_\_$!

\medskip 
\noindent (k) To find the sum of a telescoping series, we should first break it 
into  $\_\_\_\_\_\_\_\_\_\_\_\_\_$ $\_\_\_\_\_\_\_\_\_\_\_\_\_$. 

\bigskip
\noindent (l) The series $\sum a_n$ diverges if the limit is NOT equal to 
$\_\_\_\_\_\_\_\_\_\_\_\_\_$.

\subsection*{Problems Similar to the Studio Worksheets}

\noindent 2.  Determine if the following statements below are always true 
or sometimes false.

\noindent (a) If an integral contains the term $a^2+x^2$, we should use 
the substitution $x=a \sec \theta$.

\smallskip
\noindent (b) The expression $\tan \left ( \sin^{-1} (x) \right )$ cannot 
be simplified.

\smallskip
\noindent (c) When using a trig substitution with a term of the form 
$a^2-x^2$, we could use either $x=a \sin \theta$ or $x= a 
\cos \theta$ and obtain equivalent answers (that may differ only by a 
constant).

\smallskip
\noindent (d) If we use the trig substitution $x = \sin \theta$, then 
it is possible that $\sqrt{1-x^2} = -\cos \theta$. 

\smallskip
\noindent (e) The partial fraction decomposition of ${{x}\over{(x+3)^2}}$ 
is ${{A}\over{x+3}} + {{B}\over{(x+3)^2}}$.

\smallskip
\noindent (f) $\int {{dx}\over{(x+3)^2}} = \ln (x+3)^2 + C.$

\smallskip
\noindent (g) The integral $\int {{x}\over{x^2-9}} dx$ could be best 
evaluated using the method of partial fractions.

\smallskip
\noindent (h) The integral $\int {{dx}\over{x (x^4+1)}}$ cannot be 
evaluated using the method of partial fractions. 

\smallskip
\noindent (i) $\lim_{x \to \infty} xe^x$ has the indeterminate form 
$\infty^{\infty}$.

\smallskip
\noindent (j) $\lim_{x \to 0^+} (\cos x)^{{1}\over{x}}$ has the indeterminate 
form $1^{\infty}$.

\smallskip
\noindent (k) $\lim_{x \to \infty} \left ( 1 + {{1}\over{x}} \right )^{2x} = 
2e$.

\smallskip
\noindent (l) When evaluating a limit using L'Hopital's rule, we first need to 
find $\left ( {{f}\over{g}} \right )^{\prime}$. 

\smallskip
\noindent (m) If $f$ has a vertical asymptote at $x=a$, then $\int_a^b f(x) dx 
= \lim_{c \to a^+} \int_c^b f(x) dx$.

\smallskip
\noindent (n) $\int_{-1}^1 {{1}\over{x}} dx = 0.$

\smallskip
\noindent (o) Saying that an improper integral converges means that the 
integral must evaluate to a finite number.

\smallskip
\noindent (p) Indefinite integrals can be improper.

\smallskip
\noindent (q) If $\{a_n\}$ is bounded, then it converges.

\smallskip
\noindent (r) If $\{a_n\}$ converges, then it is monotonic.

\smallskip
\noindent (s) An unbounded sequence diverges.

\smallskip
\noindent (t) If $\{a_n\}$ diverges, then $\lim_{n \to \infty} a_n = \infty$.

\smallskip
\noindent (u) The sequence $\{ {{1}\over{n}} \}$ converges to 0.

\smallskip
\noindent (v) If $\lim_{n \to \infty} a_n =0$, then the series $\sum a_n$ 
converges.

\smallskip
\noindent (w) If $\lim_{n \to \infty} a_n =0$, then the sequence $\{a_n\}$ 
converges.

\smallskip
\noindent (x) The series $\sum_{n=0}^{\infty} {{1}\over{n+1}}$ diverges.

\smallskip
\noindent (y) The sum of two divergent series also diverges.

\smallskip
\noindent (z) The series $\sum_{n=1}^{\infty} r^n$ converges to 
${{1}\over{1-r}}$ if $|r|<1$.

\bigskip
\noindent 3. Evaluate the following integrals using any method we have 
learned so far: \hfil\break
$u$-substitutions, integration by parts, integrating trig functions,
trigonometric substitutions, or partial fractions.

\noindent (a) $\int {{x^2}\over{(x^2+4)^{3/2}}} dx$

\noindent (b)  $\int (x^2+1) e^{2x} dx$

\noindent (c) $\int {{\sqrt{1-x^2}}\over{x^4}} dx$

\noindent (d)  $\int {{dx}\over{e^x \sqrt{e^{2x} - 9}}}$

\noindent (e)  $\int \sin^2(x) \cos^2(x) dx$

\noindent (f) $\int {{x+3}\over{(x-1)(x^2-4x+4)}} dx$

\noindent (g) $\int x^5 \ln (x) dx$

\noindent (h) $\int {{x+4}\over{x^3+x}} dx$

\noindent (i) $\int \sqrt{25-x^2} dx$

\noindent (j) $\int {{x-1}\over{(x+1)^3}} dx$

\noindent (k) $\int {{x+2}\over{x+1}} dx$

\noindent (l) $\int {{x+1}\over{x^2(x-1)}} dx$

\medskip
\noindent 4.  Evaluate the following limits using L'Hopital's Rule.

\noindent (a) $\lim_{x \to 0^+} [x (\ln(x))^2]$

\noindent (b) $\lim_{x \to \infty} (x + e^x)^{2/x}$

\noindent (c) $\lim_{x \to {{\pi}\over{2}}} \left [ {{\ln (\sin x)}\over{(\pi - 
2x)^2}} \right ]$

\bigskip
\noindent 5.  Evaluate the improper integrals if they converge, or show 
that the integral diverges.

\noindent (a) $\int_0^3 {{x}\over{(x^2-1)^{2/3}}} dx$

\noindent (b) $\int_0^{\infty} x^2 e^{-2x}dx$

\noindent (c) $\int_1^4 {{dx}\over{x^2-5x+6}}$

\medskip
\noindent 6.  For each sequence below, find the l.u.b. and g.l.b., and 
determine if the sequence is monotonic.

\noindent (a) $\{ \sin (n \pi) \}$

\noindent (b) $\left \{ (-1)^{n+1} {{1}\over{5^n}} \right \}$

\noindent (c) $\left \{ {{n+1}\over{n}} \right \}$

\medskip
\noindent 7.  Determine whether or not each sequence converges.  If so, find 
the limit.

\noindent (a) $\left \{ {{2n^2}\over{\sqrt{9n^4+1}}} \right \}$ 

\noindent (b) $\left \{ \left ( 1 - {{1}\over{8n}} \right )^n \right \}$ 

\noindent (c) $\left \{ {{n!}\over{e^n}} \right \}$

\noindent (d) $\left \{ \left ( {{n}\over{n+5}} \right )^n \right \}$

\medskip
\noindent 8.  Use series to write the repeating decimal 0.31313131... as a 
rational number.

\medskip
\noindent 9. Find the sum of each convergent series below, or explain why 
the series diverges.

\noindent (a) $\sum_{k=7}^{\infty} {{1}\over{(k-3)(k+1)}}$ 

\smallskip
\noindent (b) $\sum_{k=0}^{\infty} (-1)^k$

\smallskip
\noindent (c) $\sum_{k=2}^{\infty} {{2^k + 1}\over{3^{k+1}}}$

\smallskip
\noindent (d) $\sum_{k=1}^{\infty} {{5k^2+8}\over{7k^2+6k+1}}$

\smallskip
\noindent (e) $\sum_{n=0}^{\infty} \left ( {{2}\over{3^n}} + {{(-1)^n}\over{6^n}} 
\right )$

\subsection*{Additional Review Problems for These Sections}

\noindent 10. Determine if each statement below is always true or sometimes 
false.

\noindent (a) $\lim_{x\to 2} {{x-2}\over{x^2+x-6}}$ is of an 
indeterminate form.

\smallskip
\noindent (b) $\lim_{x\to \infty} \left ( 1-{{1}\over{x}} \right )^x=e$

\smallskip
\noindent (c) The integral $\int x^3 \sqrt{1-x^2} \; dx$ can be evaluated by 
trigonometric substitution by setting $x=\sin x$.

\smallskip
\noindent (d) $\sin(\cos^{-1}(x)) = \tan (x)$.

\smallskip
\noindent (e) For the rational expression ${{x}\over{(x+10)(x-10)^2}}$ , the 
partial fraction decomposition is of the form ${{A}\over{x+10}} + 
{{B}\over{(x-10)^2}}$.

\smallskip
\noindent (f) For the rational expression ${{2x+3}\over{x^2 (x+2)^2}}$, the 
partial fraction decomposition is of the form 
${{A}\over{x^2}}+{{B}\over{x+2}}+{{C}\over{(x+2)^2}}$.

\smallskip
\noindent (g) If a sequence $\{a_n\}$ converges to a finite number $L$, then 
either the least upper bound or the greatest lower bound for $\{a_n\}$ is equal 
to $L$. 

\smallskip
\noindent (h) If a sequence $\{a_n\}$ has both an upper bound and a lower 
bound, then $\{a_n\}$ converges.

\smallskip
\noindent (i) If a sequence $\{a_n\}$ does not have an upper bound, then 
$\{a_n\}$ diverges.

\smallskip
\noindent (j) The sum of two convergent geometric series is also convergent.

\smallskip
\noindent (k) The difference of two divergent series is also divergent.

\smallskip
\noindent (l) If $\sum a_n$ converges, then $\lim_{n \to \infty} a_n = 0$.

\smallskip
\noindent (m) The integral $\int_{-1}^1 {{1}\over{x^2}} \; dx$ can be evaluated 
using the Fundamental Theorem of Calculus.

\smallskip
\noindent (n) The integral $\int_1^{\infty} {{1}\over{x^p}} \; dx$ 
converges when $p\geq 1$. 

\bigskip
\noindent 11.  Evaluate each integral using any method we have learned.

\noindent (a) $\int {{2x+1}\over{x^2-7x+12}} \; dx$

\noindent (b) $\int {{8 \; dx}\over{x^2 \sqrt{4-x^2}}}$

\noindent (c) $\int {{8 \; dx}\over{(4x^2+1)^2}}$

\noindent (d) $\int {{1}\over{(x+1)(x^2+1)}}\; dx$
\bigskip
\noindent 12.  Use L'Hopital's rule to evaluate the following limits.

\noindent (a) $\lim_{x\to\infty}(\ln x)^{{1}\over{x^2+1}}$

\noindent (b) $\lim_{x\to 0^+}(\ln x)^x$

\noindent (c) $\lim_{x \to 0} \left [ {{1}\over{x}} - \cot x 
\right ]$

\noindent (d) $\lim_{x \to \infty} \left [ \cos \left ( {{1}\over{x}} \right ) 
\right ]^x$

\medskip
\noindent 13.  Find values of $a$ and $b$ so that
$$\lim_{x \to 0} {{\cos (ax) - b}\over{2x^2}} = -4.$$

\medskip
\noindent 14.  Determine whether the sequences converge or diverge. Find the 
limit of each convergent sequence.

\noindent (a) $\left\{ \left (1-{{1}\over{n^2}} \right )^n\right\} $

\noindent (b) $\left\{ (10n)^{{1}\over{n}} \right\}$

\noindent (c) $\left\{ {{2n+(-1)^n}\over{4+3n}} \right\}$

\medskip
\noindent 15.  Find a formula for the $n$th term of the sequence. Then, 
determine whether the sequences converge or diverge. Find the limit of each 
convergent sequence.

\noindent (a) $\left\{ 1, -1, 1, -1 ,1,-1, \ldots\right\} $

\noindent (b) $\left\{ \sqrt{5} - \sqrt{4},\sqrt{6} - \sqrt{5},\sqrt{7} - 
\sqrt{6},\sqrt{8} - \sqrt{7},\sqrt{9} - \sqrt{8},\sqrt{10} - 
\sqrt{9},\ldots\right\}$

\noindent (c) $\left\{ 
\sin\left({{\sqrt{2}}\over{5}}\right),\sin\left({{\sqrt{3}}\over{10}}\right),
\sin\left({{\sqrt{4}}\over{17}}\right),\sin\left({{\sqrt{5}}\over{26}}\right),
\sin\left({{\sqrt{6}}\over{37}}\right),\sin\left({{\sqrt{7}}\over{50}}\right),\ldots 
\right\}$

\medskip
\noindent 16.  Determine if each integral below converges or diverges, and evaluate 
the convergent integrals.

\noindent (a) $\int_1^{\infty} {{\ln (x)}\over{x^2}} \; dx$

\noindent (b) $\int_7^{\infty} {{dx}\over{x^2-x}}$

\noindent (c) $\int_6^{\infty} {{3t^2}\over{\sqrt{t^3-8}}} \; dt$

\noindent (d) $\int_2^6 {{3t^2}\over{\sqrt{t^3-8}}} \; dt$

\noindent (e) $\int_2^{\infty} {{3t^2}\over{\sqrt{t^3-8}}} \; dt$

\medskip
\noindent 17.  Determine if each infinite series converges or diverges.  If it 
converges, find the sum.

\noindent (a) $\sum_{n=0}^{\infty} e^{-3n}$

\smallskip
\noindent (b) $\sum_{n=1}^{\infty} {{4^n + 5^n}\over{9^{n-1}}}$

\smallskip
\noindent (c) $\sum_{n=0}^{\infty} \ln \left ( {{n+5}\over{n+6}} \right )$

\smallskip
\noindent (d) $\sum_{n=0}^{\infty} \cos (5 \pi n)$

\smallskip
\noindent (e) $\sum_{n=1}^{\infty} \left ( {{1}\over{\sqrt{n+3}}} - 
{{1}\over{\sqrt{n+5}}} \right )$

\subsection*{Answers}

\noindent 2.  (c), (e), (h), (j), (m), (o), (s), (u), (w), (x), (y) are true

\noindent 3. (a) $\ln | {{\sqrt{x^2+4}}\over{2}} + {{x}\over{2}} | - 
{{x}\over{\sqrt{x^2+4}}} + C$
\hskip 0.2 cm
(b) ${{1}\over{2}} (x^2+1) e^{2x} - {{1}\over{2}}
x e^{2x} + {{1}\over{4}} e^{2x} + C$

\smallskip
\noindent (c) $-{{1}\over{3}} \cdot {{(1-x^2)^{3/2}}\over{x^3}} + C$
\hskip 0.1 cm (d) ${{\sqrt{e^{2x}-9}}\over{9e^x}} + C$
\hskip 0.1 cm
(e) ${{x}\over{8}} - {{1}\over{32}} \sin(4x) + C$

\noindent (f)  $4 \ln \left | {{x-1}\over{x-2}} \right | - 
{{5}\over{x-2}} + 
C$ (partial fractions)

\smallskip
\noindent (g) ${{x^6 \ln x}\over{6}} - {{x^6}\over{36}} + C$ (by parts)

\smallskip
\noindent (h)  $4 \ln |x| - 2 \ln (x^2+1) + \tan^{-1} (x) + C$ (partial 
fractions)

\noindent (i) ${{25}\over{2}} \sin^{-1} \left ( {{x}\over{5}} \right ) + 
{{x \sqrt{25-x^2}}\over{2}} + C$ (trig sub)

\noindent (j)  $-{{1}\over{x+1}} + {{1}\over{(x+1)^2}} + C$ \hskip 0.5 cm
(k)  $x + \ln|x+1| + C$ 

\noindent (l)  $-2 \ln|x| + {{1}\over{x}} + 2 \ln|x-1| + C$

\noindent 4. (a) 0, (b) $e^2$, (c) $-{{1}\over{8}}$ 

\noindent 5. (a) ${{9}\over{2}}$, (b) ${{1}\over{4}}$, (c) diverges

\noindent 6.  (a) l.u.b.=g.l.b.=0 \hskip 0.2 cm (b) l.u.b.=${{1}\over{5}}$ and 
g.l.b.=$-{{1}\over{25}}$ \hfil\break
(c) l.u.b.=2 and g.l.b.=1

\noindent 7.  (a) ${{2}\over{3}}$ \hskip 0.2 cm (b) $e^{-1/8}$ \hskip 0.2 cm 
(c) diverges \hskip 0.2 cm (d) ${{1}\over{e^5}}$

\noindent 8.  ${{31}\over{99}}$

\noindent 9.  (a) $\approx 0.1899$ \hskip 0.2 cm (b) diverges \hskip 0.2 cm
(c) ${{1}\over{2}}$ \hskip 0.2 cm
(d) diverges \hskip 0.2 cm (e) $3 {{6}\over{7}}$

\noindent 10.  (a), (c), (i), (j), (l) are true

\noindent 11. (a) $-7 \ln|x-3|+9\ln |x-4|+C$,  (b) ${{-2\sqrt{4-x^2}}\over{x}}+ 
C$ 

\noindent (c) $2 \tan ^{-1}(2x)+{{4x}\over{4x^2+1}}+C$, (d) 
${{1}\over{2}}\ln|x+1|+{{1}\over{2}}\arctan x-{{1}\over{4}}\ln|x^2+1|+C$

\noindent 12. (a) 1, (b) 1, (c) 0, (d) 1

\noindent 13.  $a = \pm 4$, $b=1$

\noindent 14.  (a) 1, (b) 1, (c) ${{2}\over{3}}$

\noindent 15.  (a) $\left\{ (-1)^{n+1} \right\} $ and diverges, (b) 
$\left\{ \sqrt{n+4}-\sqrt{n+3} \right\}$ and converges to $0$

\noindent (c) $\left\{ \sin\left({{\sqrt{n+1}}\over{1+(n+1)^2}}\right) 
\right\}$ and converges to $0$. 

\noindent 16.  (a) 1; (b) $\ln \left ( {{7}\over{6}} \right )$; (c) 
diverges; (d) $8\sqrt{13}$; (e) diverges

\noindent 17. (a) ${{e^3}\over{e^3-1}}$, (b) ${{369}\over{20}}$, (c) diverges, (d) 
diverges, (e) ${{1}\over{2}} + {{1}\over{\sqrt{5}}}$

\section{Review of Sections 10.3--10.9}

\subsection*{Content Recap}

\noindent 1.  Terminology review: complete the following statements.

\noindent (a) A geometric series has the general form 
$\_\_\_\_\_\_\_\_\_\_\_\_\_$.  \hfil\break
The series converges when 
$\_\_\_\_\_\_\_\_\_\_\_\_\_$ and diverges when 
$\_\_\_\_\_\_\_\_\_\_\_\_\_$.

\medskip
\noindent (b) A p-series has the general form
$\_\_\_\_\_\_\_\_\_\_\_\_\_$.  The series converges when
$\_\_\_\_\_\_\_\_\_\_\_\_\_$ and diverges when
$\_\_\_\_\_\_\_\_\_\_\_\_\_$.  To show these results, we can use the
$\_\_\_\_\_\_\_\_\_\_\_\_\_$ test.

\medskip
\noindent (c) The harmonic series has the form $\_\_\_\_\_\_\_\_\_\_\_\_\_$, and it $\_\_\_\_\_\_\_\_\_\_\_\_\_$. 

\medskip
\noindent (d) If you want to show a series converges, compare it to a 
$\_\_\_\_\_\_\_\_\_\_\_\_\_$ series that also converges.  If you want to 
show a series diverges, compare it to a $\_\_\_\_\_\_\_\_\_\_\_\_\_$ 
series that also diverges.

\medskip
\noindent (e) If the direct comparison test does not have the correct 
inequality, you can instead use the $\_\_\_\_\_\_\_\_\_\_\_\_\_$ 
$\_\_\_\_\_\_\_\_\_\_\_\_\_$ test.  In this test, if the limit is a 
$\_\_\_\_\_\_\_\_\_\_\_\_\_$ number (not equal to 
$\_\_\_\_\_\_\_\_\_\_\_\_\_$), then both series converge or both 
series diverge.

\medskip
\noindent (f)  In the ratio and root tests, the series will 
$\_\_\_\_\_\_\_\_\_\_\_\_\_$ if the limit is less than 1 and 
$\_\_\_\_\_\_\_\_\_\_\_\_\_$ if the limit is greater than 1.  If the limit 
equals 1, then the test is $\_\_\_\_\_\_\_\_\_\_\_\_\_$.

\medskip
\noindent (g) If $\sum_k a_k$ is an alternating series, then it converges $\_\_\_\_\_\_\_\_\_\_\_\_\_$ if $\sum_k |a_k|$ 
converges.  It converges $\_\_\_\_\_\_\_\_\_\_\_\_\_$ if $\sum_k |a_k|$ diverges and (i) the limit of the terms is 
$\_\_\_\_\_\_\_\_\_\_\_\_\_$ and (ii) the sequence of terms is $\_\_\_\_\_\_\_\_\_\_\_\_\_$.

\medskip
\noindent (h) If an alternating series converges, we can estimate the sum 
by adding the first $n$ terms.  Stopping after $n$ terms will give us an 
error at most equal to the magnitute of the $\_\_\_\_\_\_\_\_\_\_\_\_\_$ 
term in the sequence.

\medskip
\noindent (i) If $\lim_{n \to \infty} a_n = 0$, then what, if anything, do 
we know about 
the series $\sum_n a_n$?  $\_\_\_\_\_\_\_\_\_\_\_\_\_$

\medskip
\noindent (j) A power series has the general form: 
$\_\_\_\_\_\_\_\_\_\_\_\_\_$.  To find the radius of convergence $R$, use 
either the  $\_\_\_\_\_\_\_\_\_\_\_\_\_$ or $\_\_\_\_\_\_\_\_\_\_\_\_\_$ 
test.  The series converges $\_\_\_\_\_\_\_\_\_\_\_\_\_$ when $|x-c|<R$.  
To find the interval of convergence, don't forget to check the 
$\_\_\_\_\_\_\_\_\_\_\_\_\_$. 

\medskip
\noindent (k) A Taylor polynomial has the general form: 
$\_\_\_\_\_\_\_\_\_\_\_\_\_$.  The Taylor polynomial is the 
$n^{th}$ $\_\_\_\_\_\_\_\_\_\_\_\_\_$ $\_\_\_\_\_\_\_\_\_\_\_\_\_$ of the Taylor series with general form: 
$\_\_\_\_\_\_\_\_\_\_\_\_\_$.

\medskip 
\noindent (l) The Taylor remainder theorem says that $|R_n| \le 
{{M}\over{(n+1)!}} |x-a|^{n+1}$, where $M$ represents 
the maximum value of the $\_\_\_\_\_\_\_\_\_\_\_\_\_$ derivative of $f$ on the interval between $x$ and $a$.  The 
remainder term decreases when $n$ $\_\_\_\_\_\_\_\_\_\_\_\_\_$ or when $x$ is $\_\_\_\_\_\_\_\_\_\_\_\_\_$ to $a$.

\medskip
\noindent (m) A MacLaurin Series is a Taylor series centered at 
$\_\_\_\_\_\_\_\_\_\_\_\_\_$.

\medskip 
\noindent (n) Complete the formulas for the common MacLaurin series.
$$e^x = \sum_{k=0}^{\infty}$$
$$\ln (1+x) = \sum_{k=0}^{\infty}$$
$$\sin (x) =  \sum_{k=0}^{\infty}$$
$$\cos (x) = \sum_{k=0}^{\infty}$$
$${{1}\over{1-x}} = \sum_{k=0}^{\infty}$$

\medskip
\noindent (o) Fill in the formulas for the derivatives and 
anti-derivatives of a power series.
$${{d}\over{dx}} \left [ \sum_{k=0}^{\infty} a_k x^k \right ]= $$
$$\int_0^x \left [ \sum_{k=0}^{\infty} a_k t^k \right ] \; dt = $$

\subsection*{Problems Similar to the Studio Worksheets}

\noindent 2.  Determine if each of the following statements is always true or sometimes false.

\noindent (a) $\sum_{k=1}^{\infty} {{1}\over{\sqrt{k^3+1}}}$ is a 
$p$-series with $p={{3}\over{2}}$.

\noindent (b) $\sum_{k=2}^{\infty} {{1}\over{k (\ln k)^p}}$ converges when 
$p > 1$.

\noindent (c) To show a series $\sum_k a_k$ converges by the Basic 
Comparison Test, we should find a smaller series $\sum_k b_k$ that also 
converges. 

\noindent (d) A limit of 0 or $\infty$ from the Limit Comparison Test may 
not give us a conclusive answer as to whether our series converges or 
diverges. 

\noindent (e) To determine whether $\sum_{k=3}^{\infty} 
{{k}\over{k^3-10}}$ converges or diverges, use the Basic Comparison 
Test with $\sum_{k=3}^{\infty} {{1}\over{k^2}}$.

\noindent (f) If $\lim_{n \to \infty} {{a_n}\over{a_{n+1}}}< 1$, then the 
series $\sum_k a_k$ converges.

\noindent (g) We should use the root test if all of the terms are raised 
to 
the $k^{th}$ power.

\noindent (h) We can use the root test to show that the $p$-series $\sum_k 
{{1}\over{\sqrt{k}}}$ diverges.

\noindent (i) The ratio test would be inconclusive for the series $\sum_k 
{{k}\over{k^3+1}}$.

\noindent (j) $(2k)!=2k!$

\noindent (k) If an alternating series converges absolutely, then it also 
converges conditionally.

\noindent (l) If $\sum_k |a_k|$ converges, then the alternating series 
$\sum_k a_k$ also converges.

\noindent (m) If $\sum_k a_k$ is an alternating series and $\{ |a_k| \}$ 
is a decreasing sequence, then $\sum_k a_k$ 
converges.

\noindent (n) If $\sum_k a_k$ is an alternating series and $\sum_k |a_k|$ 
diverges, then $\sum_k a_k$ cannot converge 
absolutely.

\noindent (o) If $\sum_k a_k$ is an alternating series and $\lim_{k \to 
\infty} |a_k| \neq 0$, then $\sum_k a_k$ diverges.

\bigskip
\noindent 3. Determine whether the following series converge or 
diverge.  Justify your answers using any of the tests we have 
discussed in class.  Make sure that you (1) name the test and 
state the conditions needed for the test you are using, (2) show work for 
the test that requires some math, and (3) state a conclusion that 
explains why the test shows convergence or divergence.

\noindent (a)  $\sum_{k=1}^{\infty} {{e^k}\over{4+e^{2k}}}$

\smallskip
\noindent (b) $\sum_{k=1}^{\infty} \left ( 1 - {{3}\over{k}} \right )^k$

\smallskip
\noindent (c)  $\sum_{k=1}^{\infty} k \tan \left ( {{1}\over{k}} \right 
)$

\smallskip
\noindent (d) $\sum_{k=2}^{\infty} {{1}\over{k (\ln k)^3}}$

\smallskip
\noindent (e) $\sum_{k=1}^{\infty} {{3^{2k}}\over{8^k - 3}}$

\smallskip
\noindent (f) $\sum_{k=1}^{\infty} {{k+2}\over{\sqrt{k^5 + 4}}}$

\smallskip
\noindent (g) $\sum_{k=1}^{\infty} {{k+3}\over{\sqrt{k^2+1}}}$

\smallskip
\noindent (h) ${{1}\over{1 \cdot 3}} + {{1}\over{3 \cdot 5}} + 
{{1}\over{5 \cdot 7}} + ...$

\smallskip
\noindent (i) $\sum_{k=1}^{\infty} {{\ln k}\over{k^4}}$

\smallskip
\noindent (j) $\sum_{k=1}^{\infty} {{(2k)^k}\over{k!}}$

\smallskip
\noindent (k) $\sum_{k=1}^{\infty} \left ( {{k}\over{k+1}} \right 
)^{2k^2}$

\smallskip
\noindent (l) $\sum_{n=1}^{\infty} {{1 \cdot 3 \cdot 5 \cdot ... \cdot 
(2n-1)}\over{4^n 2^n n!}}$

\medskip
\noindent 4.   Suppose $r > 0$.  Find the values of $r$, if any, 
for which $\sum_{k=1}^{\infty} {{r^k}\over{k^r}}$ converges.

\medskip
\noindent 5.  Determine whether the following alternating series
converge absolutely, converge conditionally, or diverge.  Justify your
answers using the tests we discussed in class.

\noindent (a)  $\sum_{k=2}^{\infty} (-1)^{k+1} 
{{3k}\over{\sqrt{k^3+4}}}$

\smallskip
\noindent (b)  $\sum_{k=2}^{\infty} (-1)^k {{k}\over{k^4-1}}$

\smallskip
\noindent (c)  $\sum_{k=0}^{\infty} (-1)^{k} {{k}\over{5^k+2^k}}$

\smallskip
\noindent (d)  $\sum_{k=2}^{\infty} (-1)^k {{1}\over{k \ln k \sqrt{\ln 
\ln k}}}$

\medskip
\noindent 6.  Find the radius and interval of convergence of the following 
power series:

\noindent (a)  $\sum_{k=2}^{\infty} \left ( {{k}\over{k-1}} \right ) 
{{(x+2)^k}\over{2^k}}$
\smallskip
\noindent (b) $\sum_{n=1}^{\infty} {{(3x+2)^n}\over{\sqrt{n}}}$

\medskip
\noindent 7.  Find a power series representation for the 
function $f(x)={{x}\over{4+x^4}}$.  For what values of $x$ does the series 
converge?

\medskip
\noindent 8.  Find the third degree Taylor polynomial of the function
$f(x)=\tan^{-1}(x)$ in powers of $x-1$.

\medskip
\noindent 9.  Use a Taylor polynomial to estimate the value of $\sqrt{e}$ 
with an error of at most 0.01.  HINT: Choose $a=0$ and use the fact that 
$e<3$.

\medskip
\noindent 10.  For what values of $x$ can we replace $\cos x$ with $1 - 
{{x^2}\over{2!}} + {{x^4}\over{4!}}$ within an error range of no more 
that 0.001? 

\medskip
\noindent 11.  Use the MacLaurin series for $f(x)={{1}\over{1-x}}$ 
to find a power series representation of the function 
$$g(x)={{x}\over{(1-x)^3}}.$$
HINT: You will need to differentiate.

\medskip
\noindent 12.  Find $f^{(7)}(0)$ for the function $f(x)=x \sin (x^2)$.

\medskip
\noindent 13.  Find a power series (i.e., MacLaurin series) representation 
for the following functions.
When is your series valid? \hfil\break
\noindent (a)  $f(x)={{3x}\over{2+4x}}$ 

\medskip
\noindent (b)  $g(x)=xe^{-x}$ 

\medskip
\noindent 14.  Find a MacLaurin series for the function $f(x) = \tan^{-1} 
x$.

\medskip
\noindent 15.  Find the sum of the series:
$${{\pi}\over{2}} - {{\pi^3}\over{8 \cdot 3!}} + {{\pi^5}\over{32 \cdot 
5!}}
+ ... + (-1)^n {{\pi^{2n+1}}\over{2^{2n+1} (2n+1)!}} + ...$$

\medskip
\noindent 16.  Use a MacLaurin series to estimate $\int_0^1 e^{-x^2} dx$
within an error of no more than 0.01.

\subsection*{Additional Review Problems on These Sections}

\noindent 17.  Let $\{a_n\}$ and $\{b_n\}$ be a sequences of non-negative 
terms.  Are the following statements {\it always} true or sometimes false?

\noindent (a) If $\lim_{n \to \infty} {a_n}=L$, then the series $\sum_n a_n 
= L$.

\noindent (b) If $\lim_{n \to \infty} a_n = 0$, then $\{a_n\}$ converges 
to 0.

\noindent (c) If $\lim_{n \to \infty} a_n = 0$, then $\sum_n a_n$ 
converges.

\noindent (d) If $\sum_n a_n$ converges, then $\lim_{n \to \infty} a_n = 
0$.

\noindent (e) If $\sum_n a_n$ diverges, then $\lim_{n \to \infty} a_n \neq 
0$.

\noindent (f) If $\lim_{n \to \infty} a_n \neq 0$, then $\sum_n a_n$ 
diverges.

\noindent (g) If $\lim_{n \to \infty} a_n \neq 0$, then $\{a_n\}$ 
diverges.

\noindent (h) If $\int_1^{\infty} f(x) dx = L$, where $0 < L < \infty$, 
then $\sum_n f(n) = L$.

\noindent (i) If $\sum_n b_n$ converges and $a_n > b_n$ for all $n \ge 1$, 
then $\sum_n a_n$ also converges.

\noindent (j) If $\sum_n b_n$ diverges and $a_n > b_n$ for all $n \ge 1$,
then $\sum_n a_n$ also diverges.

\noindent (k) If $\lim_{n \to \infty} {{a_n}\over{b_n}} = 0$ and $\sum_n 
b_n$ diverges, then $\sum_n a_n$ also diverges. 

\medskip
\noindent 18.  Determine if each statement below is always true or 
sometimes false.

\noindent (a) If $p>1$ the alternating $p$-series $\sum_{n=1}^{\infty} 
{{(-1)^{n-1}}\over{n^p}}$ converges conditionally. 

\noindent (b) The alternating harmonic series $\sum_{n=1}^{\infty} 
(-1)^{n+1} {{1}\over{n}}$ is conditionally convergent.

\noindent (c) If $\sum_{n=1}^\infty a_n$ is a convergent series 
of nonnegative numbers, then $\sum_{n=1}^\infty {{a_n}\over{n}}$ 
also converges.

\noindent (d) If $\sum_{n=1}^{\infty} a_n$ is a convergent series of nonnegative numbers, 
then $\sum_{n=1}^{\infty} \sin(a_n)$ also converges.

\noindent (e) The series $\sum_{n=2}^{\infty} {{1}\over{n(\ln n)^p}}$ 
converges for any $p>1$.

\noindent (f) The series $\sum_{n=2}^{\infty} {{1}\over{n(\ln n)^p}}$ 
diverges for any $p\leq 1$.

\noindent (g) Let $\{a_n\}$ be a positive sequence and $L = 
\lim_{n\to \infty} {{a_{n+1}}\over{a_n}}$ be a finite number.
If a series $\sum_{n=1}^{\infty} a_n$ converges, then $L < 1$.

\noindent (h) Let $\{a_n\}$ be a positive sequence and $L = 
\lim_{n\to \infty}\left(a_n\right)^{{1}\over{n}}$ be a finite number.
If a series $\sum_{n=1}^{\infty} a_n$ diverges, then $L \geq 
1$.

\noindent (i) If the radius of convergence of a power series is $0$, then 
the power series diverges everywhere.

\noindent (j) If a power series $\sum_{n=0}^\infty a_nx^n$ converges in 
$(-1,1)$, then its radius of convergence is $1$.

\noindent (k) Every Taylor series is a power series.

\noindent (l) The fifth degree Taylor polynomial for $\cos (x)$ about 
$x=0$ is $1 - {{x^2}\over{2}} + {{x^4}\over{4!}}$.

\noindent (m) For any Taylor polynomial, the error in the approximation is 
no more than the magnitude of the $(n+1)^{st}$ term.

\medskip
\noindent 19.  Let  $f(x) = \int_0^x t \sin(t^3) dt$.  Use a MacLaurin 
series to find $f^{(11)}(0)$.

\noindent 20.  (a) Estimate $\cos \left ( {{\pi}\over{12}} \right )$ using 
a fourth-degree Taylor polynomial.

\smallskip
\noindent (b) Find a MacLaurin series for the function $g(x)=\int_0^x 
{{\sin(t/2)}\over{2t}} dt$.

\medskip
\noindent 21.  Determine if the alternating series converges absolutely or 
converges conditionally.  Justify your answer fully by: (1) name the test 
and
state the conditions needed for the test you are using, (2) show work for
the test that requires some math, and (3) state a conclusion that
explains why the test shows convergence or divergence.

\noindent (a) $\sum_{n=1}^{\infty} (-1)^n {{1}\over{\sqrt{n}}}$

\medskip
\noindent (b) $\sum_{n=1}^{\infty} (-5)^{-n}$

\medskip
\noindent 22. Determine whether the given series converges or diverges. 
Make sure that you (1) name the test and state the conditions needed for 
the test you are using,
(2) show work for the test that requires some math, and (3) state a 
conclusion 
that explains why the test shows convergence or divergence.

\noindent (a) $\sum_{n=2}^\infty {{n+1}\over{\sqrt{n^3-2}}}$

\medskip
\noindent (b) $\sum_{n=1}^\infty \ln\left(1+{{1}\over{n^2}}\right)$
(Hint: Limit Comparison with $\sum_{n=1}^\infty {{1}\over{n^2}}$.)

\medskip
\noindent (c) $\sum_{n=3}^{\infty} {{10}\over {n\, \ln n\, \ln(\ln n)}}$

\medskip
\noindent (d) $\sum_{n=1}^{\infty} ne^{-n}$

\medskip
\noindent (e) $\sum_{n=1}^{\infty} {{n!(n+1)!}\over{(3n)!}}$

\medskip
\noindent (f) $\sum_{n=1}^{\infty}\left( {{\sin n}\over{1+n}}\right)^n$

\medskip
\noindent (g) $\sum_{n=1}^{\infty} {{2^{n+1}}\over{2\cdot 3^{n-1}}}$

\medskip
\noindent (h) $\sum_{n=1}^{\infty} \left( 1-{{1}\over{n}}\right)^{n^{2}}$

\medskip
\noindent 23.  Find the radius and interval of convergence of each power 
series below.

\noindent (a) $\sum_{n=1}^\infty {{(-1)^n(2x-4)^n}\over{n^2}}$

\medskip
\noindent (b) $\sum_{n=1}^\infty \left (1-{{1}\over{n}} \right )^{n}x^n$

\subsection*{Answers}

\noindent 2. (b), (d), (g), (i), (l), (n), and (o) are true

\noindent 3. (a), (d), (f), (h), (i), (k), and (l) converge

\noindent 4. converges 
when $0 < r < 1$

\noindent 5. (a) and (d) converge conditionally, (b) and (c) converge 
absolutely

\noindent 6.  (a) $R=2$, $I.C. = (-4,0)$, (b) $R={{1}\over{3}}$, $
I.C. = \left [ -1, -{{1}\over{3}} \right )$

\noindent 7.  $\sum_{k=0}^{\infty} {{(-1)^kx^{4k+1}}\over{4^{k+1}}}$, $|x| < 4^{1/4}$

\noindent 8. ${{\pi}\over{4}} + {{1}\over{2}} (x-1) - {{1}\over{4}} 
(x-1)^2 + {{1}\over{12}} (x-1)^3$

\noindent 9. 1.6458 

\noindent 10. $x \in (-0.9467, 0.9467)$

\noindent 11.  ${{1}\over{2}}\sum_{k=2}^{\infty} k(k-1)x^{k-1}$

\noindent 12.  -840

\noindent 13. (a) $3 \sum_{k=0}^{\infty} (-1)^k 2^{k-1} x^{k+1}$, $|x| <
{{1}\over{2}}$ \hskip 1 cm (b) $\sum_{k=0}^{\infty} {{(-1)^k 
x^{k+1}}\over{k!}}$, $x \in \Re$


\noindent 14. $\sum_{k=0}^{\infty} (-1)^k \cdot {{x^{2k+1}}\over{2k+1}}$,
$|x| < 1$ 

\noindent 15. 1 

\noindent 16. $\approx 0.743$

\noindent 17.  Statements (b), (d), (f), and (j) are true.

\noindent 18.  (b), (c), (d), (e), (f), (h), (k), and (l) are true

\noindent 19.  $-{{10!}\over{6}}$

\noindent 20.  (a) 0.966, (b) $\sum_{k=0}^{\infty} (-1)^k {{x^{2k+1}}\over
{2^{2k+2}(2k+1)!(2k+1)}}$

\noindent 21.  (a) converges conditionally; (b) converges absolutely

\noindent 22. (b), (d), (e), (f), (g), and (h) converge

\noindent 23.  (a) $R={{1}\over{2}}$, $I.C.=\left [ {{3}\over{2}}, 
{{5}\over{2}} \right ]$; (b) $R=1$, $I.C. = (-1,1)$

\section{Review of Volumes of Revolution: Sections 6.1--6.2}

\subsection*{Additional Practice Problems on Volumes}

\noindent 1.  Find the volume of the solid generated by 
revolving the
region bounded by the curve $y=\sin(x)$, the $x$-axis, and the 
lines $x=0$,
$x=\pi/2$ about the $y$-axis.

\medskip
\noindent 2.  Find the volume of the solid generated when the 
region bounded by the curves $y=4-x^2$ and $y=2-x$ is 
revolved about the $x$-axis.

\medskip
\noindent 3.  Find the volume of the solid generated when the 
region bounded by the curves $y=x^2-4$ and $y=2x-x^2$ is 
revolved about the line (a) $y=-4$ and (b) $x=2$.

\medskip
\noindent 4.  Use the method of cylindrical shells to find the 
volume of the 
solid generated when the region bounded 
by the curve $y=\sqrt{x}$, the $x$-axis, and the line $x=9$ is 
revolved about the $x$-axis. 

\subsection*{Answers to Additional Problems on Volumes}

\noindent 1.  $2 \pi$ 

\medskip
\noindent 2.  ${{108 \pi}\over{5}}$ cubic units

\medskip
\noindent 3.  (a) $45 \pi$ cubic units,
(b) $27 \pi$ cubic units

\medskip
\noindent 4.  ${{81 \pi}\over{2}}$ cubic units

\section{Problems From a Previous Final Exam Study Packet}

\subsection*{The Problem Set}

\begin{enumerate}

\item
 Sum the series
$$\sum_{k=2}^{\infty} {\frac{4^{2k} - 1}{17^{k-1}}}.$$
\medskip

\item
Find the sum of the series
$$\sum_{k=1}^{\infty} {\frac{1}{(2k-1)(2k+3)}}.$$
\medskip

\item
Determine whether the following series converge or diverge.  Justify your answers using the tests we discussed in class.
\begin{enumerate}
  \item $\sum_{k=1}^{\infty} {\frac{e^k}{(1+4e^k)^{3.2}}}$ 
\medskip
  \item $\sum_{k=2}^{\infty} \left ( {\frac{k-5}{k}} \right 
)^{k^2}$
\medskip
  \item $\sum_{k=1}^{\infty} {\frac{k^2 \cdot 2^{k+1}}{k!}}$
\medskip
   \item $\sum_{k=1}^{\infty} {\frac{1}{1 + 2 + 3 + ... + k}}$
 \medskip
\end{enumerate}
  
\item
Find the third degree Taylor polynomial of the function
$f(x)=\tan^{-1}(x)$ in powers of $x-1$.
\medskip

\item
Use a Taylor polynomial to estimate the value of $\sqrt{e}$ 
with an error of at most 0.01.  HINT: Choose $a=0$ and use the fact that 
$e<3$.

\item 
Use the MacLaurin series for $f(x)={\frac{1}{1-x}}$ to find a power series representation of the function 
$$g(x)={\frac{x}{(1-x)^3}}.$$
HINT: You will need to differentiate.
\medskip

\item 
Find the radius and interval of convergence of the power 
series
$$\sum_{k=1}^{\infty} {\frac{5^k}{\sqrt{k}}} (3-2x)^k.$$
\vfil\eject

\item
Determine whether each of the alternating series below 
converge absolutely, converge conditionally, or diverge.  Use the 
convergence tests from class to justify your answer. 
\begin{enumerate}
\item $$\sum_{k=1}^{\infty} (-1)^{k+1} {\frac{\ln k}{k^4}}$$
\medskip
\item $$\sum_{k=2}^{\infty} (-1)^k {\frac{4k^2}{k^3+1}}$$
\medskip
\end{enumerate}

\item 
Does the alternating series $\sum_{k=1}^{\infty} 
{\frac{(-1)^k}{k + \sqrt{k}}}$ converge absolutely, converge 
conditionally, or diverge?
\medskip

\item 
Evaluate each integral below using any of the methods we have 
learned.
\begin{enumerate}
\item $\int {\frac{\sin^3 x}{\cos x}} dx$
\medskip
\item $\int {\frac{x}{\sqrt{x^2+2x-3}}} dx$
\medskip
\item $\int {\frac{\cos x}{4+ \sin^2 x}} dx$
\medskip
\item $\int {\frac{1}{x (x^2+x+1)}} dx$
\medskip
\end{enumerate}

\item Evaluate the improper integral if it converges, or show 
that the integral diverges.
$$\int_0^3 {\frac{x}{(x^2-1)^{2/3}}} dx$$
\medskip

\item 
Does the integral
$$\int_0^{\infty} {\frac{dx}{e^x + e^{-x}}}$$
converge or diverge?  (HINT: Use the Integral Comparison Test.)
\medskip

\item 
Find the area bounded between the curves
$y=2 \cos x$ and $y=\sin (2x)$ on the interval $[-\pi,\pi]$.
\vfil\eject

\item 
Evaluate the integrals:

\begin{enumerate}

\item $\int \left ( \sqrt{x} - {\frac{1}{x^2}} \right )^2 dx$
\medskip

\item $\int {\frac{\log_3 x^4}{x}} dx$
\medskip

\item $\int {\frac{\sec (e^{-4x})}{e^{4x}}} dx$
\medskip

\item $\int_1^e {\frac{\sqrt{\ln x}}{x}} dx$
\medskip

\item $\int {\frac{x^2}{(ax^3+b)^2}}dx$
\medskip

\item $\int_{-5}^{0} \left ( x \sqrt{4-x} \right ) dx$
\medskip

\end{enumerate}


\item 
Find the general solution to the equation:
$$(y \ln x) y^{\prime} = {\frac{y^2+1}{x}}.$$
\medskip

\item 
Evaluate the following integrals.
\begin{enumerate}
\item $\int x^5 \ln (x) dx$
\medskip
\item $\int x^3 e^{x^2} dx$
\medskip
\item $\int (\ln x)^2 dx$
\medskip
\end{enumerate}

\item 
Evaluate the following integrals.
\begin{enumerate}
\item $\int x \tan^{-1} (x) dx$
\medskip
\item $\int {\frac{\cos^3 (x)}{\sin x}} dx$
\medskip
\item $\int \sqrt{9-x^2} dx$
\medskip
\end{enumerate}

\item 
 Find the area of the region bounded by the curves $y=5x+1$
and $y=x^2+3x-2$.
\medskip

\item 
Find the volume of the solid generated by revolving the
region bounded by the curve $y=\sin(x)$, the $x$-axis, and the lines $x=0$,
$x=\pi/2$ about the $y$-axis.
\medskip

\item 
Let  $f(x) = \int_0^x t \sin(t^3) dt$.  Use a MacLaurin 
series to find $f^{(11)}(0)$.
\medskip

\item 
Evaluate $\int \left ( e^{-5x} + {\frac{1}{1+4x^2}} 
\right ) dx.$
\medskip

\item 
Find the sum of the series:
$$1 - {\frac{\pi^2}{9 \cdot 2!}} + {\frac{\pi^4}{81 \cdot 4!}} + ... +
(-1)^n {\frac{\pi^{2n}}{3^{2n} (2n)!}} + ...$$
\medskip

\item 
Evaluate the integrals:
\begin{enumerate}
\item $\int {\frac{x^2}{\sqrt{4-x^6}}} dx$
\medskip
\item $\int_0^2 x |2x-1| dx$ 
\medskip
\end{enumerate}

\item 
Find $F^{\prime}(2)$ for the function
$$F(x) = \int_{\frac{8}{x}}^{x^2} \left ( {\frac{t}{1 - \sqrt{t}}} 
\right ) dt.$$
\medskip

\item 
Find the area bounded by the curves $f(x)=x^3+2x^2$ and 
$g(x)=x^2+2x$.
\medskip

\item 
Evaluate the integrals:
\begin{enumerate}
\item $\int {\frac{1}{x^2}} \sec \left ( {\frac{1}{x}} \right ) \tan 
\left ( {\frac{1}{x}} \right ) dx$
\medskip

\item $\int {\frac{1}{\ln(x^x)}} dx$
\medskip

\end{enumerate}

\item 
Evaluate the following integrals: 
\begin{enumerate}
\item $\int {\frac{e^{2x}}{\sqrt{4-3e^{2x}}}} dx$
\medskip

\item $\int_{-3}^{-2} {\frac{dx}{\sqrt{4 - (x+3)^2}}}$
\medskip

\end{enumerate}

\item 
Solve the initial value problem:
$$y^{\prime} = x \sqrt{ {\frac{1-y^2}{1-x^2}} }, \quad y(0)=0.$$
\medskip

\item 
Evaluate the following integrals.
\medskip
\begin{enumerate}
\item $\int \sin^5(2x) \cos^3 (2x) dx$
\medskip
\item $\int \tan^4(x) dx$ 
\medskip
\item $\int e^{2x} \sin(3x) dx$
\medskip
\item $\int {\frac{x^2}{(x^2+4)^{3/2}}} dx$ 
\medskip
\item $\int {\frac{\sqrt{1-x^2}}{x^4}} dx$
\medskip
\item $\int {\frac{x}{(4-x^2)^{3/2}}} dx$ 
\medskip
\item $\int {\frac{dx}{e^x \sqrt{e^{2x} - 9}}}$
\medskip
\end{enumerate}

\item 
Evaluate the following integrals.
\medskip
\begin{enumerate}
\item $\int {\frac{x+3}{(x-1)(x^2-4x+4)}} dx$
\medskip
\item $\int {\frac{x+4}{x^3+x}} dx$ 
\medskip
\end{enumerate}

\item 
Evaluate each of the following integrals.
\medskip
\begin{enumerate}
\item $\int {\frac{x+2}{x+1}} dx$
\medskip
\item  $\int \sqrt{25-x^2} dx$
\medskip
\item  $\int \tan^{3}(x) \sec^{4}(x) dx$
\medskip
\item  $\int x \tan^{-1}(x) dx$
\medskip
\item $\int {\frac{dx}{x \sqrt{1+x^2}}}$
\medskip
\item $\int {\frac{x+1}{x^2(x-1)}} dx$ 
\medskip
\item $\int {\frac{x+1}{x^2-4x+8}} dx$
\medskip
\end{enumerate}

\item  Evaluate the improper integrals if they converge, or show 
that the integral diverges.
\begin{enumerate}
\item $\int_1^3 {\frac{1}{(x^2-1)^{3/2}}} dx$
\medskip
\item $\int_0^{\infty} x^2 e^{-2x}dx$ 
\medskip
\end{enumerate}

\item  For what values of $p$ does the integral
$\int_4^{\infty} {\frac{dx}{x (\ln x)^p}}$ converge?
\medskip

\item Find the area bounded by the curve 
$y={\frac{1}{x^2+9}}$, the $x$-axis, and $x \ge 0$.
\medskip

\item Use series to write the repeating decimal 0.31313131... as a 
rational number.
\medskip

\item Find the sum of each convergent series below, or explain why 
the series diverges.

$$\sum_{k=7}^{\infty} {\frac{1}{(k-3)(k+1)}}, \quad
\sum_{k=0}^{\infty} (-1)^k, \quad
\sum_{k=2}^{\infty} {\frac{2^k + 1}{3^{k+1}}}$$
\medskip

\item Determine if each series below converges or diverges.  JUSTIFY 
YOUR ANSWER FULLY using either the nth term divergence test or the integral 
test. 

\begin{enumerate}
\item $$\sum_{k=1}^{\infty} {\frac{e^k}{4+e^{2k}}}$$
\medskip
\item $$\sum_{k=1}^{\infty} {\frac{5k^2+8}{7k^2+6k+1}}$$
\medskip
\end{enumerate}

\item Determine whether the following series converge or 
diverge.  Justify your answers using the tests we discussed in class.

\begin{enumerate}
\item  $\sum_{k=1}^{\infty} {\frac{3^{2k}}{8^k - 3}}$
\medskip
\item $\sum_{k=1}^{\infty} {\frac{k+2}{\sqrt{k^5 + 4}}}$
\medskip
\item $\sum_{k=2}^{\infty} {\frac{1}{k (\ln k)^3}}$
\end{enumerate}
\medskip
\item 
Determine whether the following series converge or diverge.  Justify your answers using the tests we discussed in class.

\begin{enumerate}
\item  $\sum_{k=1}^{\infty} {\frac{(2k)^k}{k!}}$
\medskip
\item $\sum_{k=1}^{\infty} \left ( {\frac{k}{k+1}} \right )^{2k^2}$
\medskip
\item $\sum_{k=1}^{\infty} k \tan \left ( {\frac{1}{k}} \right )$
\medskip
\item ${\frac{1}{1 \cdot 3}} + {{1}\over{3 \cdot 5}} + 
{\frac{1}{5 \cdot 7}} + ...$
\medskip 
\end{enumerate}

\item Determine whether the following series converge or 
diverge.  Justify your answers using the tests we discussed in class.

\begin{enumerate}
\item  $\sum_{n=2}^{\infty} {\frac{1}{n \sqrt{n^2-1}}}$
\medskip
\item $\sum_{n=1}^{\infty} {\frac{1 \cdot 3 \cdot 5 \cdot ... \cdot 
(2n-1)}{4^n 2^n n!}}$
\medskip
\end{enumerate}

\item Suppose $r > 0$.  Find the values of $r$, if any, 
for which $\sum_{k=1}^{\infty} {\frac{r^k}{k^r}}$ converges.
\medskip

\item Determine whether the following alternating series
converge absolutely, converge conditionally, or diverge.  Justify your
answers using the tests we discussed in class.

\begin{enumerate}
\item  $\sum_{k=2}^{\infty} (-1)^{k+1} {\frac{3k}{\sqrt{k^3+4}}}$
\medskip
\item $\sum_{k=2}^{\infty} (-1)^k {\frac{k}{k^4-1}}$
\medskip
\item  $\sum_{k=0}^{\infty} (-1)^{k} {\frac{k}{5^k+2^k}}$
\medskip
\item $\sum_{k=2}^{\infty} (-1)^k {\frac{1}{k \ln k \sqrt{\ln \ln k}}}$ 
\medskip 
\end{enumerate}

\item Find the radius and interval of convergence of the power 
series $f(x) = \sum_{k=1}^{\infty} {\frac{2^k}{k+1}} (x-3)^k$.
\medskip

\item For what values of $x$ can we replace $\cos x$ with $1 - 
{\frac{x^2}{2!}} + {\frac{x^4}{4!}}$ within an error range of no more 
that 0.001?
\medskip

\item  Find $f^{(7)}(0)$ for the function $f(x)=x \sin (x^2)$.
\medskip

\item Use a MacLaurin series to estimate $\int_0^1 e^{-x^2} dx$ 
within an error of no more than 0.01.
\medskip

\item  Find the volume of the solid generated when the region 
bounded by the curves $y=4-x^2$ and $y=2-x$ is 
revolved about the $x$-axis.
\medskip


\item Find the volume of the solid generated when the region 
bounded by the curves $y=x^2-4$ and $y=2x-x^2$ is 
revolved about the line $y=-4$.
\medskip

\item Find the volume of the solid generated when the region 
bounded by the curves $y=x^2-4$ and $y=2x-x^2$ is 
revolved about the line $x=2$.
\medskip

\item Use the method of cylindrical shells to find the volume of the 
solid generated when the region bounded 
by the curve $y=\sqrt{x}$, the $x$-axis, and the line $x=9$ is revolved 
about the $x$-axis. 
\medskip

\item Find a power series representation for the following
functions. When is your series valid?

\begin{enumerate}
\item  $f(x)={\frac{3x}{2+4x}}$ \smallskip
\item $g(x)=\int_0^x {\frac{\sin(t/2)}{2t}} dt$ \smallskip
\item  $h(x)=\tan^{-1}(x)$
\medskip 
\end{enumerate}

\item 

\begin{enumerate}
\item  Estimate $\cos 15^{\circ}$ using a fourth-degree Taylor
polynomial.
\medskip
\item Estimate $\int_0^1 e^{-2x^2} dx$ within an error of 0.01.
\medskip 
\end{enumerate}

\end{enumerate}

\subsection*{Answers}

\noindent 1.  $255 {\frac{15}{16}}$
\medskip

\noindent 2.  ${\frac{1}{3}}$
\medskip

\noindent 3. \\
(a) Converges by the integral test \hfil\break
(b) Converges by the root test \hfil\break
(c) Converges by the ratio test \hfil\break
(d) Converges by the basic comparison test, or using telescoping series
\medskip

\noindent 4.  $P_3(x) = {\frac{\pi}{4}} + {\frac{1}{2}} (x-1) - 
{\frac{1}{4}} (x-1)^2 + {\frac{1}{12}} (x-1)^3$
\medskip

\noindent 5.  $\sqrt{e} \approx f(0.5) = 1 + 0.5 + {\frac{(0.5)^2}{2}} + 
{\frac{(0.5)^3}{6}} = 1.6458.$
\medskip

\noindent 6.  ${\frac{1}{2}} \sum_{k=2}^{\infty} k(k-1)x^{k-1}$, $|x| < 1$
\medskip

\noindent 7.  $R={\frac{1}{10}}$, IC=$\left ( {\frac{7}{5}}, 
{\frac{8}{5}} \right ]$
\medskip

\noindent 8. \\
(a) Converges absolutely by the basic comparison test \hfil\break 
(b) Converges conditionally by the limit comparison and alternating series tests
\medskip

\noindent 9. converges conditionally  
\medskip

\noindent 10.\\
(a) $- \ln |\cos x | + {\frac{1}{2}} \cos^2 x + C$ \hfil\break
(b) $\sqrt{x^2+2x-3} - \ln | {\frac{x+1+\sqrt{x^2+2x-3}}{2}} | + C$
\hfil\break
(c) ${\frac{1}{2}} \tan^{-1} \left ( {\frac{\sin x}{2}} \right ) + C$
\hfil\break
(d) $\ln |x| - {\frac{1}{2}} \ln (x^2+x+1) - {\frac{\sqrt{3}}{3}} \tan^{-1}
\left ( {\frac{2x+1}{\sqrt{3}}} \right ) + C$
\medskip

\noindent 11.  Converges to ${\frac{9}{2}}$
\medskip

\noindent 12. Converges (compare to $\int_0^{\infty} {\frac{1}{e^x}} dx$)
\medskip

\noindent 13.  8 
\medskip

\noindent 14.\\

\noindent (a) $\frac{1}{2} x^2 + {\frac{4}{\sqrt{x}}} - {\frac{1}{3x^3}} + C$\\

\noindent (b) $\frac{2}{\ln 3} (\ln x)^2 + C$\\

\noindent (c) $-\frac{1}{4} \ln | \sec (e^{-4x}) + \tan (e^{-4x}) | + C$\\

\noindent (d) $\frac{2}{3}$\\

\noindent (e) $-\frac{1}{3a(ax^3+b)} + C$\\

\noindent (f)$-{\frac{506}{15}}$
\medskip

\noindent 15.  $y^2 = k(\ln x)^2 -1$, where  $k = e^{2C}$
\medskip

\noindent 16.\\
(a) ${\frac{x^6 \ln x}{6}} - {\frac{x^6}{36}} + C$

\noindent (b)  ${\frac{x^2 e^{x^2}}{2}} - {\frac{e^{x^2}}{2}} + C$  

\noindent (c)  $x (\ln x)^2 - 2x \ln x + 2x + C$
\medskip

\noindent 17.\\
(a) ${\frac{x^2 \tan^{-1} x}{2}} - {\frac{x}{2}} + {\frac{\tan^{-1} 
x}{2}} + C$
\medskip

\noindent (b) $\ln | \sin x| - {\frac{1}{2}} \sin^2 x + C$
\medskip

\noindent (c) ${\frac{9}{2}} \sin^{-1} \left ( {\frac{x}{3}} \right ) + {\frac{x 
\sqrt{9-x^2}}{2}} + C$
\medskip

\noindent 18.  ${\frac{32}{3}}$ 
\medskip

\noindent 19.  $2 \pi$ 
\medskip

\noindent 20.  $-{\frac{10!}{6}}$
\medskip

\noindent 21.  $-{\frac{1}{5}} e^{-5x} + {\frac{1}{2}} \tan^{-1} (2x) + C$
\medskip

\noindent 22.  ${\frac{1}{2}}$
\medskip

\noindent 23.\\  (a) ${\frac{1}{3}} \sin^{-1} \left ({\frac{x^3}{2}} 
\right) + C$
\medskip

\noindent (b) ${\frac{41}{12}}$ 
\medskip

\noindent 24.  -24
\medskip

\noindent 25.  ${\frac{37}{12}}$
\medskip

\noindent 26.  $- \sec \left ( {\frac{1}{x}} \right ) + C$, $\ln |\ln x| + C$
\medskip

\noindent 27.  $-{\frac{1}{3}} \sqrt{4-3e^{2x}}$, ${\frac{\pi}{6}}$
\medskip

\noindent 28. $y = \sin ( -\sqrt{1-x^2} + 1)$
\medskip

\noindent 29.\\
(a) ${\frac{1}{12}} \sin^6 (2x) -  {\frac{1}{16}} \sin^8 
(2x) + C$
\medskip

\noindent (b) ${\frac{1}{3}} \tan^3 (x) - \tan (x) + x + C$
\medskip

\noindent (c) ${\frac{2}{13}} e^{2x} \sin(3x) - {\frac{3}{13}} e^{2x} 
\cos(3x) + C$
\medskip 

\noindent (d) $\ln | {\frac{\sqrt{x^2+4}}{2}} + {\frac{x}{2}} | - 
{\frac{x}{\sqrt{x^2+4}}} + C$
\medskip

\noindent (e) $-{\frac{1}{3}} \cdot {\frac{(1-x^2)^{3/2}}{x^3}} + C$
\medskip

\noindent (f) ${\frac{1}{\sqrt{4-x^2}}} + C$
\medskip

\noindent (g) ${\frac{\sqrt{e^{2x}-9}}{9e^x}} + C$
\medskip

\noindent 30.\\
(a) $4 \ln \left | {\frac{x-1}{x-2}} \right | - {\frac{5}{x-2}} + 
C$
\medskip

\noindent (b) $4 \ln |x| - 2 \ln (x^2+1) + \tan^{-1} (x) + C$
\medskip

\noindent 31.\\
(a) $x + \ln|x+1| + C$ 
\medskip

\noindent (b)  ${\frac{25}{2}} \sin^{-1} \left ( {\frac{x}{5}} \right ) + 
{\frac{x \sqrt{25-x^2}}{2}} + C$
\medskip

\noindent (c)  ${\frac{1}{4}} \tan^4(x) + {\frac{1}{6}} \tan^6 (x) + C$ 
\medskip

\noindent (d)  ${\frac{x^2}{2}} \tan^{-1}(x) - {\frac{x}{2}} + {\frac{1}{2}} 
\tan^{-1}(x) + C$
\medskip

\noindent (e)  $- \ln \left | {\frac{\sqrt{1+x^2}}{x}} + {\frac{1}{x}} 
\right | + C$
\medskip

\noindent (f)  $-2 \ln|x| + {\frac{1}{x}} + 2 \ln|x-1| + C$ 
\medskip

\noindent (g)  ${\frac{1}{2}} \ln |x^2-4x+8| + {\frac{3}{2}} \tan^{-1} \left 
( {\frac{x-2}{2}} \right ) + C$
\medskip

\noindent 32. (a) diverges, (b) converges to ${\frac{1}{4}}$
\medskip

\noindent 33.  converges when $p>1$
\medskip

\noindent 34.  ${\frac{\pi}{6}}$ units$^2$
\medskip

\noindent 35.  ${\frac{31}{99}}$
\medskip

\noindent 36.  (a) $\approx 0.1899$, (b) diverges (look at the sequence of 
partial sums), (c) ${\frac{1}{2}}$
\medskip

\noindent 37.  (a) converges by the integral test, (b) diverges by the nth term 
test
\medskip

\noindent 38.  (a) diverges by basic comparison, (b) converges by basic 
comparison, (c) converges by the integral test
\medskip

\noindent 39.  (a) diverges by the ratio test, (b) converges by the root test, 
(c) diverges by the nth term test, (d) converges by limit comparison (or, you can 
show it is telescoping)
\medskip

\noindent 40.  (a) converges by limit comparison, (b) converges by ratio test
\medskip

\noindent 41.  converges when $0<r<1$
\medskip

\noindent 42.  (a) converges conditionally, (b) converges absolutely, (c) 
converges absolutely, (d) converges conditionally
\medskip

\noindent 43.  $\left [ {\frac{5}{2}}, {\frac{7}{2}} \right )$
\medskip

\noindent 44.  $x \in (-0.9467, 0.9467)$
\medskip

\noindent 45.  -840
\medskip

\noindent 46.  $\approx 0.743$
\medskip

\noindent 47.  ${\frac{108 \pi}{5}}$ cubic units
\medskip

\noindent 48.  $45 \pi$ cubic units
\medskip

\noindent 49.  $27 \pi$ cubic units
\medskip

\noindent 50.  ${\frac{81 \pi}{2}}$ cubic units
\medskip

\noindent 51.\\
(a) $3 \sum_{k=0}^{\infty} (-1)^k2^{k-1}x^{k+1}$, valid
for $|x|<{\frac{1}{2}}$ \hfil\break
(b) $\sum_{k=0}^{\infty} (-1)^k {\frac{x^{2k+1}}
{2^{2k+2}(2k+1)!(2k+1)}}$, valid for $x \neq 0$ \hfil\break
(c) $\sum_{k=0}^{\infty}
(-1)^k {\frac{x^{2k+1}}{2k+1}} + C$, valid for $|x|<1$ 
\medskip

\noindent 52.  (a) 0.966; (b) approximately 0.6165 (stop at $k=5$)

\section{A Few Extra Practice Problems that May Appear on the Final Exam -- KNOW THESE}

\begin{enumerate}

\item Evaluate the following improper integral: 
      \[
      I = \int_{e^2}^{\infty} \frac{\ln(\ln(x))}{x (\ln x)^{m+1}} dx, m > 0. 
      \]
\item Evaluate the following indefinite integral: 
      \[
      I = \int \frac{dx}{1+e^x}. 
      \]
\item Evaluate the next limit 
      \[
      L = \lim_{\alpha \rightarrow 0^{+}} \left( 
          \frac{1-e^{-\alpha v}}{\alpha}\right)^x, 
          \alpha > 0, v > 0. 
      \]
      HINT: Consider the Taylor series expansion of the exponential function. 

\end{enumerate}

\end{document}

