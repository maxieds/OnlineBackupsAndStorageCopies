\documentclass[12pt]{article}
\usepackage[utf8]{inputenc}
\usepackage{amsmath}
\usepackage{amssymb}
\usepackage{amsthm}
\usepackage{fullpage}
\setlength{\parskip}{1em}
\pagestyle{empty}

\begin{document}

\noindent
%Name:\underline{~~~~~~~~~~~~~~~~~~~~~~~~~~~~~~~~~~~}
MATH1552 Summer 2020
\hspace{2.2cm}
%\noindent
Worksheet 1 Summer
\hspace{2cm} Integral Calculus

\vspace{2mm}
%\hline


\begin{enumerate}
    \item Determine if the following statement is True or False. If the statement is false, provide a counterexample or provide a justification. 
    \begin{enumerate}
        \item I $F$ and $G$ are antiderivatives of $f$, then $F=G$.
%If $\lim_{n\to\infty}a_n=0$, then the sequence of partial sums $\{S_n\}$ converges. 
        
        {\bf Solution:} False.  It can be $F=G+c$.
%(1 point) Possible counterexample is $\{1/n\}$ where $\lim_{n\to\infty}\frac1n=0$, but the harmonic series diverges and hence the sequence of partial sums diverges. (3 points)
        
        \item The antiderivative of $\sec^2(3x)$ is $\frac{1}{3}\tan(3x)$.
%If $\lim_{n\to\infty}a_n=0$, then $\sum_na_n$ converges.
        
        {\bf Solution:} True. Use change of variable or find derivative of $\frac{1}{3}\tan(3x)$.
% False. (1 point) Possible counterexample is $\{1/n\}$ where $\lim_{n\to\infty}\frac1n=0$, but the harmonic series diverges. (3 points)
        
        \item The indenite integral of a function $f$ is the collection of all antiderivatives of $f$.
%If $\{a_n\}$ is monotonic, then it converges. 
        
        {\bf Solution:} True.
% False. (1 point) Possible counterexample is $\{n\}$ which is increasing, but divergent. (3 points)
        
        \item We know how to find the antiderivative of $e^{x^2}$, and it is $e^{x^2}$.
%If $\{a_n\}$ diverges, then $\lim_{n\to\infty}a_n=\infty$.

	{\bf Solution:} False. $\frac{d}{dx} e^{x^2} =2xe^{x^2}$. 

	\item $F$ and $G$ are antiderivatives of $f$ and $g$, then antiderivative of $FG$ is $fg$. 

	{\bf Solution:} False. Because of the product rule.
% $\frac{d}{dx} e^{x^2} =2xe^{x^2}$. 
        
   %     {\bf Solution:} False. (1 point) Possible counterexample is $\{(-1)^n\}$, which diverges and the limit as $n$ tends to infinity does not exist. (3 points)
        

    \end{enumerate}

\item Evaluate the following indenite integrals.
	\begin{enumerate}
	\item $\int (\sqrt[3]{x}-\frac{1}{x})^3 dx.$

	{\bf Solution.} $\int (\sqrt[3]{x}-\frac{1}{x})^3 dx= \int (x-\frac{3}{\sqrt[3]{x^2}} + \frac{3}{\sqrt[3]{x^5}} -\frac{1}{x^3})dx = \frac{x^2}{2} - 9x^{1/3} - \frac{9x^{-2/3}}{2} + \frac{x^{-2}}{2}.$

	\item $\int (3^{-x}+e^{-5x})dx.$

	{\bf Solution.} $\int  (3^{-x}+e^{-5x})dx= -\frac{3^{-x}}{\log(3)} -\frac{e^{-5x}}{5}$.
%\int (x-\frac{3}{\sqrt[3]{x^2}} + \frac{3}{\sqrt[3]{x^5}} -\frac{1}{x^3})dx = \frac{x^2}{2} - 9x^{1/3} - \frac{9x^{-2/3}}{2} + \frac{x^{-2}}{2}.$

	\item $\int \frac{e^{\sqrt{2}}+x^{\sqrt{2}}}{\sqrt{x}}.$

	{\bf Solution.} $\int \frac{e^{\sqrt{2}}+x^{\sqrt{2}}}{\sqrt{x}} dx= \frac{-e^{\sqrt{2}}\sqrt{x}}{2} +\frac{x^{\sqrt{2}+\frac{1}{2}}}{\sqrt{2}+\frac{1}{2}}$.
%  -\frac{3^{-x}}{\log(3)} -\frac{e^{-5x}}{5}.
	\end{enumerate}

\item (Optional): Evaluate $\int \sqrt{\tan(x)}dx$.

	{\bf Solution:} Let $\tan^2(u)=\tan(x)$. Then $2(\tan^2(u)+1)\tan(u)du=(\tan^2(x)+1)dx$. So 
$$\int \tan(u)\frac{2(\tan^2(u)+1)\tan(u)du}{\tan^4(u)+1}$$
Then use partial fraction.
\end{enumerate}

\end{document}