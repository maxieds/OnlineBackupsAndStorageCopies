\documentclass[12pt]{article}
\usepackage[utf8]{inputenc}
\usepackage{amsmath}
\usepackage{amssymb}
\usepackage{amsthm}
\usepackage{fullpage}
\setlength{\parskip}{1em}
\pagestyle{empty}

\begin{document}

\noindent
%Name:\underline{~~~~~~~~~~~~~~~~~~~~~~~~~~~~~~~~~~~}
MATH1552 Summer 2020
\hspace{2.2cm}
%\noindent
Worksheet 2 Summer
\hspace{2cm} Integral Calculus

\vspace{2mm}
%\hline


\begin{enumerate}
 \item   (Applying the Definite Integral ) A marketing company is trying a new campaign. The
campaign lasts for three weeks, and during this time, the company finds that it gains
customers as a function of time according to the formula:
$$C(t)=10t-4t^2+5$$
where $t$ is time in weeks and the number of customers is given in thousands.
\begin{enumerate}
\item Using the following form of the definite integral,
$$\int_{a}^{b}f(t)dt=\lim_{n\rightarrow \infty}\frac{b-a}{n}\sum_{i=1}^n f(a+\frac{(b-a)i}{n})$$
calculate the average number of customers gained during the three-week campaign.

\textbf{Solution.} We consider $t$ to be number of weeks.
\begin{align*}
\int_{0}^{3}(10t-4t^2+5)dt&=\lim_{n\rightarrow \infty}\frac{3}{n}\sum_{i=1}^n \left(10\left(\frac{3i}{n}\right)-4\left(\frac{3i}{n}\right)^2+5\right)\nonumber\\
&=\lim_{n\rightarrow \infty}\frac{90}{n^2}\sum_{i=1}^n i - \frac{108}{n^3}\sum_{i=1}^n i^2 +\frac{15}{n}\sum_{i=1}^n 1=\frac{90}{2}-\frac{108}{3}+15=24. 
\end{align*}

\item Also compute the average number of customers using the form
\begin{align*}
\int_{a}^{b}f(t)dt=\lim_{n\rightarrow \infty}\frac{b-a}{n}\sum_{i=1}^n f(a+\frac{(b-a)(i-1)}{n})
\end{align*}
Compare the results. What can you confirm from this comparison?

\textbf{Solution. }The result will be the same, and it confirms that $x_i^*$ can be any point in the interval $[x_{i-1},x_i]$.
\end{enumerate}

\item Explain why the following property is true:
$$\left\vert\int_{a}^{b}f(t)dt\right\vert\leq \int_{a}^b |f(t)|dt$$
Can you find an example where the inequality is strict?


\textbf{Solution. }The LHS is the net area, while the RHS is the total area. Please draw a picture and explain it. No need for proof.

\item Determine if each statement below is true or false.

	\begin{enumerate}
	\item We always set $x_i^*$ to be the right-hand endpoint of the $i$th interval.

\textbf{Solution.} False. $x_i^*$ can be any point in the interval $[x_{i-1},x_i]$.

	\item $$\sum_{i=1}^n i^3=\left(\frac{n(n+1)}{2}\right)^2$$

\textbf{Solution. }True. Please help them to prove it using induction.

	\item If $f(x)\geq 0$ on $[a,b]$, then $\int_{a}^b f(t)dt$ represents the total area bounded by $f$, $x=a$, $x=b$ and the $x$-axis.

\textbf{Solution. }True. Please help them understand the necessity of condition $f\geq 0$. 
	\end{enumerate}
\end{enumerate}

\end{document}