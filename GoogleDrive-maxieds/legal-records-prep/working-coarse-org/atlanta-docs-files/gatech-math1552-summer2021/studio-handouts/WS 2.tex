\documentclass[12pt]{article}
\usepackage[utf8]{inputenc}
\usepackage{amsmath}
\usepackage{amssymb}
\usepackage{amsthm}
\usepackage{fullpage}
\setlength{\parskip}{1em}
\pagestyle{empty}

\begin{document}

\noindent
%Name:\underline{~~~~~~~~~~~~~~~~~~~~~~~~~~~~~~~~~~~}
MATH1552 Summer 2020
\hspace{2.2cm}
%\noindent
Worksheet 2 Summer
\hspace{2cm} Integral Calculus

\vspace{2mm}
%\hline


\begin{enumerate}
    \item (Applying the Riemann Sum) You are driving when all of a sudden, you see traffic
stopped in front of you. You slam the brakes to come to a stop. While your brakes are
applied, the velocity of the car is measured, and you obtain the following measurements:

Time since applying breaks (sec) : 0 \quad 1 \quad 2 \quad 3 \quad 4 \quad 5

Velocity of car (in ft/sec) :\quad \quad  \quad  88 \quad 60 \quad 40 \quad 25 \quad 10 \quad 0

    \begin{enumerate}
        \item Plot the points on a curve of velocity vs. time.

        \item Using the points given, determine upper and lower bounds for the total distance traveled
before the car came to a stop.


    \end{enumerate}

\item Estimate the area under the graph of $f(x) = 10-x^2$ between the lines $x = -3$ and
$x = 2$ using $n = 5$ equally spaced subintervals, by finding:

	\begin{enumerate}
	\item The upper sum, $U_f$.

	\item The lower sum, $L_f$.

	\end{enumerate}

\item Determine if each statement below is true or false.

	\begin{enumerate}
	\item To find the upper sum $U_f$ of a function $f$ on $[a, b]$, after partitioning the interval into
$n$ pieces, evaluate $f$ at the right-hand endpoint of each subinterval.

	\item When the interval $[a, b]$ is partitioned into $n$ pieces, there are exactly $n$ endpoints.

	\item A partition of the interval $[a, b]$ does not need to be evenly spaced in order to calculate
a Riemann Sum.

	\item If $f$ is positive and continuous on $[a, b]$, and $A$ is the actual area bounded by $f$, $x = a$,
$x = b$, and the $x$-axis, then $L_f < A < U_f$.
	\end{enumerate}
\end{enumerate}

\end{document}