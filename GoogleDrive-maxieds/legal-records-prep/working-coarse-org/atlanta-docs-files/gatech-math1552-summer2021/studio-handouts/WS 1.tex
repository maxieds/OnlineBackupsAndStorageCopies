\documentclass[12pt]{article}
\usepackage[utf8]{inputenc}
\usepackage{amsmath}
\usepackage{amssymb}
\usepackage{amsthm}
\usepackage{fullpage}
\setlength{\parskip}{1em}
\pagestyle{empty}

\begin{document}

\noindent
%Name:\underline{~~~~~~~~~~~~~~~~~~~~~~~~~~~~~~~~~~~}
MATH1552 Summer 2020
\hspace{2.2cm}
%\noindent
Worksheet 1 Summer
\hspace{2cm} Integral Calculus

\vspace{2mm}
%\hline


\begin{enumerate}
    \item Determine if the following statement is True or False. If the statement is false, provide a counterexample or provide a justification. 
    \begin{enumerate}
        \item I $F$ and $G$ are antiderivatives of $f$, then $F=G$.
%If $\lim_{n\to\infty}a_n=0$, then the sequence of partial sums $\{S_n\}$ converges. 
        
    %    {\bf Solution:} False. (1 point) Possible counterexample is $\{1/n\}$ where $\lim_{n\to\infty}\frac1n=0$, but the harmonic series diverges and hence the sequence of partial sums diverges. (3 points)
        
        \item The antiderivative of $\sec^2(3x)$ is $\frac{1}{3}\tan(3x)$.
%If $\lim_{n\to\infty}a_n=0$, then $\sum_na_n$ converges.
        
     %   {\bf Solution:} False. (1 point) Possible counterexample is $\{1/n\}$ where $\lim_{n\to\infty}\frac1n=0$, but the harmonic series diverges. (3 points)
        
        \item The indenite integral of a function $f$ is the collection of all antiderivatives of $f$.
%If $\{a_n\}$ is monotonic, then it converges. 
        
     %   {\bf Solution:} False. (1 point) Possible counterexample is $\{n\}$ which is increasing, but divergent. (3 points)
        
        \item We know how to find the antiderivative of $e^{x^2}$, and it is $e^{x^2}$.
%If $\{a_n\}$ diverges, then $\lim_{n\to\infty}a_n=\infty$.

	\item $F$ and $G$ are antiderivatives of $f$ and $g$, then antiderivative of $FG$ is $fg$. 
        
   %     {\bf Solution:} False. (1 point) Possible counterexample is $\{(-1)^n\}$, which diverges and the limit as $n$ tends to infinity does not exist. (3 points)
        

    \end{enumerate}

\item Evaluate the following indenite integrals.
	\begin{enumerate}
	\item $\int (\sqrt[3]{x}-\frac{1}{x})^3 dx.$

	\item $\int (3^{-x}+e^{-5x})dx.$

	\item $\int \frac{e^{\sqrt{2}}+x^{\sqrt{2}}}{\sqrt{x}}.$
	\end{enumerate}

\item (Optional): Evaluate $\int \sqrt{\tan(x)}dx$.
\end{enumerate}

\end{document}