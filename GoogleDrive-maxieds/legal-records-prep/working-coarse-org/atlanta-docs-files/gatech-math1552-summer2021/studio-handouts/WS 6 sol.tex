\documentclass[12pt]{article}
\usepackage[utf8]{inputenc}
\usepackage{amsmath}
\usepackage{amssymb}
\usepackage{amsthm}
\usepackage{fullpage}
\setlength{\parskip}{1em}
\pagestyle{empty}

\begin{document}

\noindent
%Name:\underline{~~~~~~~~~~~~~~~~~~~~~~~~~~~~~~~~~~~}
MATH1552 Summer 2020
\hspace{2.2cm}
%\noindent
Worksheet 6 Summer
\hspace{2cm} Integral Calculus

\vspace{2mm}
%\hline


\begin{enumerate}

\item Evaluate the following integrals using the method of substitution.



\begin{enumerate}
\item $\int \frac{1}{\ln(x^x)}dx$.

\textbf{Solution. } Let $u=\ln(x)$. Please remind them the logarithm properties. 

\item $\int \frac{e^{2x}}{\sqrt{3-4e^{2x}}}dx$.

\textbf{Solution. } Consider $u=e^{2x}$.

\item $\int \frac{1}{\sqrt{4-(x+3)^2}}dx$.

\textbf{Solution. } Please do it in 2 steps. First let $u=x+3$. Then pick $2\sin(v)=u$. 
\end{enumerate}

\item Suppose that $y = f(x)$ and $y = g(x)$ are both continuous functions on the interval $[a ,b]$.
Determine if each statement below is always true or sometimes false.

\begin{enumerate}
\item Suppose that $f(c) > g(c)$ for some number $c\in (a, b)$. Then the area bounded by $f, g$,
$x = a$, and $x = b$ can be found by evaluating the integral $\int_a^b (f(x)-g(x))dx$.

\textbf{Comment. } False. Please help them understand by showing counterexample. Also, explain the case when $f(x)>g(x)$ for all $x$.

\item If $\int_a^b (f(x)-g(x))dx$  evaluates to -5, then the area bounded by $f, g, x = a$, and $x = b$
is 5.

\textbf{Solution. } False. There might be intersection points.

\item  If $f(x) > g(x)$ for every $x \in [a, b]$, then $\int_a^b |f(x)-g(x)|dx =\int_a^b (f(x)-g(x))dx$

\textbf{Solution. } True. 

\end{enumerate}

\item Find the area bounded by the region between the curves $f(x) = x^3 + 2x^2$ and $g(x) =
x^2 + 2x$.

\textbf{Comment. } Please use the 3 steps: finding intersection points, finding the larger function in each subinterval, and computing the subintervals. The final answer is $37/12$.

\item Find the area bounded by the region enclosed by the three curves $y = x^3$, $y = -x$, and
$y = -1$.

\textbf{Comment. } Final answer: $5/4$

\item Find the area bounded by the curves $y = \cos(x)$ and $y = \sin(2x)$ on the interval $[0,\frac{\pi}{2}]$.

\textbf{Comment. } Final answer: $1/2$.

\item Find the area of the triangle with vertices at the points $(0,1)$, $(3,4)$, and $(4,2)$. USE
CALCULUS.

\textbf{Comment. } Final answer: $4.5$, please explain that although it seems not the bext way to find the area, even in this case it might have computational advantage. 

\item For each function below: (i) determine which method to use to evaluate the function
(formula, u-substitution, or integration by parts, and (ii) evaluate the integral.
\begin{enumerate}
\item $\int \frac{\sqrt{\ln(x)}}{x}dx$.

\textbf{Solution. } Using $u-$sub $u=\ln(x)$.

\item $\int \left(\ln(x)\right)^2dx$.

\textbf{Solution. } We need to take integration by parts. $u=\ln^2(x)$ and $dv=dx$.
\end{enumerate}


\end{enumerate}

\end{document}