\documentclass[12pt]{article}
\usepackage[utf8]{inputenc}
\usepackage{amsmath}
\usepackage{amssymb}
\usepackage{amsthm}
\usepackage{fullpage}
\setlength{\parskip}{1em}
\pagestyle{empty}

\begin{document}

\noindent
%Name:\underline{~~~~~~~~~~~~~~~~~~~~~~~~~~~~~~~~~~~}
MATH1552 Summer 2020
\hspace{2.2cm}
%\noindent
Worksheet 5 Summer
\hspace{2cm} Integral Calculus

\vspace{2mm}
%\hline


\begin{enumerate}

\item Evaluate the integrals:

\begin{enumerate}
\item $\int_{2}^5 \frac{3x-5}{x^3}dx$.

\item $\int_{3}^5 \frac{dx}{x^2(x-3)}$.

\item $\int_{\pi}^{7\pi/2} \frac{\cot(x)+\sin^2(x)}{4}$.
\end{enumerate}


\item Find $F'(3)$ where
$$F(x) = \int_{\cos(4\pi x)}^{e^{1/x}} \frac{3x^2}{x+2} dx$$

\item (Optional) Let $f(1/x)=f(x)$ and  $f$ be an odd function. If $\int_{1/2}^{1/4} f(x)\frac{dx}{x^2} =3$. Then compute
$$\int_{-4}^2 (f(x)+3x^2 -5)dx$$

\item a) Given the function below, evaluate $\int_{1}^9 f(x)dx$. 
\begin{align*}
f(x)=\begin{cases}
x^2+4 & x<4\\
\sqrt{x} -x & x\geq 4.
\end{cases}
\end{align*}

b) Would you get the same answer to part (a) if you evaluated $F(9) -  F(1)$? What does
this tell you about the FTC and continuity?

\item (a) Evaluate the expressions:
$$\int x(x+1)dx \quad \text{ and }\quad \int xdx \int (x+1)dx$$

b) Looking at your answer in part (a), what, if anything, can you say in general about $\int f(x)g(x)dx$?

\item For each integral below, determine if we can evaluate the integral using the method of
u-substitution. If the answer is "yes", detect $u$.

a) $\int \frac{1}{x^2}\sec(\frac{1}{x})\tan(\frac{1}{x})dx.$

b) $\int x\csc^2(x)dx$.

c) $\int e^{x^2}dx$.

\item Determine if each statement below is true or false.

\begin{enumerate}

\item If $f$ is a continuous function, then the function $F(x) = \int_a ^x f(t)dt$ is an anti-derivative
of $f$.

\item If $F$ is an anti-derivative of $f$, then $\int_a^b f(t)dt$ represents the slope of the secant line of
$F(x)$ on the interval $[a, b]$.

\item $\frac{d}{dx}\left(\int_{a}^b f(t)dt\right) = f(b)$.

\item Given that f is continuous on $[a, b]$ and $F'(x) = f(x)$, then $F(b) - F(a)$ represents the
net area bounded by the graph of $y = f(x)$, the lines $x = a$, $x = b$, and the $x$-axis.

\end{enumerate}


\end{enumerate}

\end{document}