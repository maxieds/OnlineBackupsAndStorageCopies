%%%%%%%%%%%%%%%%%%%%%%%%%%%%%%%%%%%%%%%%%
% Beamer Presentation
% LaTeX Template
% Version 1.0 (10/11/12)
%
% This template has been downloaded from:
% http://www.LaTeXTemplates.com
%
% License:
% CC BY-NC-SA 3.0 (http://creativecommons.org/licenses/by-nc-sa/3.0/)
%
%%%%%%%%%%%%%%%%%%%%%%%%%%%%%%%%%%%%%%%%%

%----------------------------------------------------------------------------------------
%	PACKAGES AND THEMES
%----------------------------------------------------------------------------------------

\PassOptionsToPackage{prologue}{xcolor}
%\documentclass[notes,usenames,svgnames,dvipsnames,11pt]{beamer}
\documentclass[usenames,svgnames,dvipsnames,11pt]{beamer}

\usepackage{amsthm,amsmath,amssymb,amscd}
\usepackage{geometry}
\usepackage{longtable}
\usepackage{listings}

%\usepackage{CustomColors}

\definecolor{indiagreen}{rgb}{0.07,0.53,0.03}
\definecolor{GATechBlue}{rgb}{0.0,0.18823529411764706,0.3411764705882353}%{003057}
\definecolor{GATechGold}{rgb}{0.7019607843137254,0.6392156862745098,0.4117647058823529}%{B3A369​}
\definecolor{GATechBuzzGold}{rgb}{0.9176470588235294,0.6666666666666666,0.0}%{EAAA00}
\definecolor{SlideBGLight}{rgb}{0.902,0.918,0.933}

\mode<presentation> {

% The Beamer class comes with a number of default slide themes
% which change the colors and layouts of slides. Below this is a list
% of all the themes, uncomment each in turn to see what they look like.

%\usetheme{default}
\usetheme{AnnArbor}
%\usetheme{Antibes}
%\usetheme{Bergen}
%\usetheme{Berkeley}
%\usetheme{Berlin}
%\usetheme{Boadilla}
%\usetheme{CambridgeUS}
%\usetheme{Copenhagen}
%\usetheme{Darmstadt}
%\usetheme{Dresden}
%\usetheme{Frankfurt}
%\usetheme{Goettingen}
%\usetheme{Hannover}
%\usetheme{Ilmenau}
%\usetheme{JuanLesPins}
%\usetheme{Luebeck}
%\usetheme{Madrid}
%\usetheme{Malmoe}
%\usetheme{Marburg}
%\usetheme{Montpellier}
%\usetheme{PaloAlto}
%\usetheme{Pittsburgh}
%\usetheme{Rochester}
%\usetheme{Singapore}
%\usetheme{Szeged}
%\usetheme{Warsaw}

% As well as themes, the Beamer class has a number of color themes
% for any slide theme. Uncomment each of these in turn to see how it
% changes the colors of your current slide theme.

%\usecolortheme{albatross}
%\usecolortheme{beaver}
%\usecolortheme{beetle}
%\usecolortheme{crane}
%\usecolortheme{dolphin}
%\usecolortheme{dove}
%\usecolortheme{fly}
%\usecolortheme{lily}
%\usecolortheme{orchid}
%\usecolortheme{rose}
%\usecolortheme{seagull}
%\usecolortheme{seahorse}
%\usecolortheme{whale}
%\usecolortheme{wolverine}

%\setbeamertemplate{footline} % To remove the footer line in all slides uncomment this line
%\setbeamertemplate{footline}[page number] % To replace the footer line in all slides with a simple slide count uncomment this line

%\setbeamertemplate{navigation symbols}{} % To remove the navigation symbols from the bottom of all slides uncomment this line

\setbeamercolor*{structure}{bg=GATechBlue,fg=GATechGold}

\setbeamercolor*{palette primary}{use=structure,fg=white,bg=structure.fg}
\setbeamercolor*{palette secondary}{use=structure,fg=white,bg=GATechGold!85!black}
\setbeamercolor*{palette tertiary}{use=structure,fg=white,bg=GATechGold!70!black}
\setbeamercolor*{palette quaternary}{fg=white,bg=GATechGold!65!white}

\setbeamercolor{background canvas}{bg=SlideBGLight}

\setbeamercolor{frametitle}{bg=GATechBlue,fg=GATechBuzzGold}
\setbeamercolor*{titlelike}{bg=GATechBlue,fg=GATechBuzzGold}

\defbeamertemplate{itemize item}{bulletpoint}{\usebeamerfont*{itemize item enumitem}\raise1.05pt\hbox{\color{GATechGold!70!black}{$\blacktriangleright$}}}
\setbeamertemplate{items}[bulletpoint]

\setbeamercolor{section in toc}{fg=black}
\setbeamercolor{subsection in toc}{fg=black}

\setbeamercolor{bibliography item}{parent=palette primary}
\setbeamercolor{bibliography entry author}{fg=GATechBlue}
\setbeamercolor{bibliography entry title}{fg=GATechBlue}
\setbeamercolor{bibliography entry note}{fg=GATechBlue}

}

\usepackage{graphicx} % Allows including images
\usepackage{booktabs} % Allows the use of \toprule, \midrule and \bottomrule in tables
\usepackage{fancyvrb}
\usepackage{inconsolata}

\newcommand{\Iverson}[1]{\ensuremath{\left[#1\right]_{\delta}}} 

\DeclareMathOperator{\DGF}{DGF} 
\DeclareMathOperator{\ds}{ds} 
\DeclareMathOperator{\Id}{Id}
\DeclareMathOperator{\sq}{sq}

\newcommand{\ceiling}[1]{\ensuremath{\left\lceil #1 \right\rceil}} 
\newcommand{\ImportantMarker}{%\textcolor{GATechGold}{$\mathbf{\Leftarrow}$}\ 
                              \textcolor{GATechGold}{\textbf{[!! \underline{IMPORTANT} !!]}}}

\newcommand{\Floor}[2]{\ensuremath{\left\lfloor \frac{#1}{#2} \right\rfloor}}
\newcommand{\Ceiling}[2]{\ensuremath{\left\lceil \frac{#1}{#2} \right\rceil}}                              

\newcommand{\gkpSI}[2]{\ensuremath{\genfrac{\lbrack}{\rbrack}{0pt}{}{#1}{#2}}} 
\newcommand{\gkpSII}[2]{\ensuremath{\genfrac{\lbrace}{\rbrace}{0pt}{}{#1}{#2}}}

\newcommand{\TitleBoxed}[1]{
     \begin{beamercolorbox}[sep=8pt,center,shadow=true,rounded=true]{title}
          \usebeamerfont{title}#1\par%
     \end{beamercolorbox}
}

%\setbeamertemplate{note page}[plain]
\setbeamerfont{note page}{family*=pplx,size=\footnotesize} % Palatino for notes

%----------------------------------------------------------------------------------------
%	TITLE PAGE
%----------------------------------------------------------------------------------------

\title[Sandia 2022 -- MathBio Software]{
     Software work in mathematical biology at Georgia Tech
} 

\author{Maxie Dion Schmidt} % Your name
\institute[] 
{
%Georgia Institute of Technology \\ 
%School of Mathematics \\ % Your institution for the title page
%Atlanta, GA 30318, USA \\ 
%\smallskip
\texttt{maxieds@gmail.com} \\ 
\url{http://people.math.gatech.edu/~mschmidt34} \\ 
\url{https://github.com/maxieds}
}
\date[Spring 2022]{Sandia National Labs \\ Technical Presentation \\ Spring 2022} % Date, can be changed to a custom date

\begin{document}

\begin{frame}
\titlepage % Print the title page as the first slide
\end{frame} 

%----------------------------------------------------------------------------------------
%	PRESENTATION SLIDES
%----------------------------------------------------------------------------------------

%------------------------------------------------

\section{Introduction} 

\begin{frame}
\frametitle{Introduction -- Applications of RNA research}
\begin{itemize} 

\item RNA molecules are utilized in cancer research and therapy 
\item mRNA vaccines are newly available to the public but have been studied for decades
\item Allowed for rapid development of a COVID-19 vaccine
\item RNA may eventually play a central role in medical applications 

\end{itemize}

\end{frame}

\begin{frame}
\frametitle{What is mathematical biology?}
\begin{itemize} 

\item Mathematical biology (MathBio) uses mathematical models as 
      theoretical abstractions of the natural structure of living organisms 
\item Software engineering RA in the 
      \emph{Georgia Tech Discrete Mathematics and Molecular Biology} group (gtDMMB)

\end{itemize}

\end{frame}

\section{RNA basics}

\begin{frame}
\TitleBoxed{
     \Huge{\centerline{RNA basics}}
}
\end{frame}

\begin{frame}
\frametitle{RNA basics}
\begin{itemize} 

\item RNA is a single-stranded molecule similar to DNA 

\end{itemize}

\begin{center}
\includegraphics[height=0.8\textheight]{presentation-images/RNAStructVisualization-FivePrimeThreePrime.png} \\ 
\end{center}

\end{frame}

\begin{frame}
\frametitle{RNA secondary structures}
\begin{itemize} 

\item RNA base sequence and can have more than one 
      2D or 3D structure 
\item Obtaining the 1D structure for organisms is easy via modern technology
\item Understanding 3D molecular conformations is still comparatively hard 

\end{itemize}

\begin{center}
\includegraphics[height=0.56\textheight]{presentation-images/YeastCerivisiae3DViz.png} \\ 
\textbf{(Yeast 3D structure)}
\end{center}

\end{frame}

\begin{frame}[fragile]
\frametitle{Arc diagrams -- Discussion example -- S.~Cerevisiae (yeast)}

\noindent
{\small\tt 
GGUUGCGGCCAUAUCUACCAGAAAGCACCGUUUCCCGUCCGAUCAACUGUGUUAAGCUGGUAGA \\[0.05cm]
{\color{red}(((((((((}....{\color{orange}((((((((}...{\color{yellow!75!black}((((}.{\color{green}(((}{\color{blue}((}......{\color{blue}))}..{\color{green})))}.{\color{yellow!75!black}))))}....{\color{orange}))))))))} \\[0.1cm]
GCCUGACCGAGUAGUGUAUGGGUGACCAUACGCGAAACUCAGGUGCUGCAAUCU \\[0.05cm]
.....{\color{orange}(((((((}.{\color{yellow!75!black}((((((((}....{\color{yellow!75!black}))))))))}...{\color{orange}))))}.{\color{orange})))}{\color{red})))))))))}.
}

\bigskip

\begin{minipage}{0.4\textwidth}
\begin{center}
\includegraphics[width=0.8\textwidth]{presentation-images/SCerevisiae-PhotoRealistic-MicroscopicImages.jpg} \\ 
\textbf{(Actual microscopic views)}
\end{center}
\end{minipage}\hfil
\begin{minipage}{0.58\textwidth}
\begin{center}
\includegraphics[height=0.44\textheight]{presentation-images/SCerevisiae-SecondaryStructure-RadialView.png} \\ 
\textbf{(Radial view of 2D MFE structure)} 
\end{center}
\end{minipage}

\end{frame}

\begin{frame}
\frametitle{RNA secondary structures}
\begin{itemize} 

\item Understanding the 2D secondary structures (base pairings) 
      remains a crucial component of ribonomics research
\item RNA folding prediction programs generate 2D structures given the 1D base sequence  
\item Terminology: MFE and thermodynamic equilibrium

\end{itemize}

\end{frame}

\section{GTFoldPython -- GTFP}

\begin{frame}
\TitleBoxed{
     \Huge{\centerline{GTFoldPython software project}}
}
\end{frame}

\begin{frame}
\frametitle{GTFold -- Overview I}

\begin{itemize} 

\item Accurate and efficient RNA secondary structure prediction is an important 
      open problem in computational molecular biology 
\item GTFold is the first implementation of RNA secondary structure prediction by thermodynamic
	optimization for \textbf{modern multi-core computers}
\item The speedup is useful to researchers working with very long 
      RNA sequences, such as RNA viral genomes in recent applications

\end{itemize}

\begin{minipage}{0.5\textwidth}
\begin{center}
\includegraphics[width=0.8\textwidth]{presentation-images/GTFold-RunningTimesChart-ByBigThreeProgs.png} \\ 
\end{center}
\end{minipage}\hfil
\begin{minipage}{0.5\textwidth}
\begin{center}
\includegraphics[width=0.8\textwidth]{presentation-images/GTFold-RunningTimesChart-ByThreads.png} \\ 
\end{center}
\end{minipage}

\end{frame}

\begin{frame}
\frametitle{GTFold -- Overview II}

\begin{itemize} 

\item Original GTFold produced command line only utilities 
\item Motivated by the need for a Python interface to get the same GTFold data 
\item Started work in the Fall of 2019 writing Python3 bindings for GTFold 
\item Hardest part of development work? Requirements for highly robust custom 
      cross platform build scripts 

\end{itemize}

\end{frame}

\begin{frame}
\frametitle{GTFoldPython -- Introduction}

\begin{itemize} 

\item GTFoldPython (GTFP): Python3 bindings based around the 
      original GTFold sources written in C++
\item Backend uses the Python3 C API 
\item Frontend interface is a wrapper library that uses 
      \texttt{CTypes} to call the C API functions

\end{itemize}

\begin{center}
\includegraphics[height=0.5\textheight]{presentation-images/GTFoldPython-InterfaceStructure.png}
\end{center}

\end{frame}

\begin{frame}[t,fragile]
\frametitle{GTFoldPython -- Example -- Find MFE and MFE structure}

\small\noindent
\textbf{External Python3 script source:}
\begin{lstlisting}[language=Python,basicstyle=\tiny\ttfamily,keywordstyle=\bfseries\color{green!40!black},
                   commentstyle=\itshape\color{purple!40!black},identifierstyle=\color{blue!63!green},
                   stringstyle=\color{orange},frame=none,keepspaces=true,numbers=left,xleftmargin=0.28cm]
import sys, os
from GTFoldPythonImportAll import *
\end{lstlisting}

\end{frame}

\begin{frame}[t,fragile]
\frametitle{GTFoldPython -- Example -- Find MFE and MFE structure}

\small\noindent
\textbf{External Python3 script source:}
\begin{lstlisting}[language=Python,basicstyle=\tiny\ttfamily,keywordstyle=\bfseries\color{green!40!black},
                   commentstyle=\itshape\color{purple!40!black},identifierstyle=\color{blue!63!green},
                   stringstyle=\color{orange},frame=none,keepspaces=true,numbers=left,xleftmargin=0.28cm]
import sys, os
from GTFoldPythonImportAll import *

GTFP.Init()
GTFP.Config(quiet = False, debugging = False, verbose = False, stdmsgout = "stderr")
\end{lstlisting}

\end{frame}

\begin{frame}[t,fragile]
\frametitle{GTFoldPython -- Example -- Find MFE and MFE structure}

\small\noindent
\textbf{External Python3 script source:}
\begin{lstlisting}[language=Python,basicstyle=\tiny\ttfamily,keywordstyle=\bfseries\color{green!40!black},
                   commentstyle=\itshape\color{purple!40!black},identifierstyle=\color{blue!63!green},
                   stringstyle=\color{orange},frame=none,keepspaces=true,numbers=left,xleftmargin=0.28cm]
import sys, os
from GTFoldPythonImportAll import *

GTFP.Init()
GTFP.Config(quiet = False, debugging = False, verbose = False, stdmsgout = "stderr")

baseSeqFPCons = "GCAUUGGAGAUGGCAUUCCUCCAUUAACAAACCGCUGCGCCCGUAGCAGCUGAUGAUGCCUACAGA"
consListFP = GTFPUtils.ReadFPConstraintsFromFile("../Testing/TestData/tRNA/yeast.fa.cons")
\end{lstlisting}

\end{frame}

\begin{frame}[t,fragile]
\frametitle{GTFoldPython -- Example -- Find MFE and MFE structure}

\small\noindent
\textbf{External Python3 script source:}
\begin{lstlisting}[language=Python,basicstyle=\tiny\ttfamily,keywordstyle=\bfseries\color{green!40!black},
                   commentstyle=\itshape\color{purple!40!black},identifierstyle=\color{blue!63!green},
                   stringstyle=\color{orange},frame=none,keepspaces=true,numbers=left,xleftmargin=0.28cm]
import sys, os
from GTFoldPythonImportAll import *

GTFP.Init()
GTFP.Config(quiet = False, debugging = False, verbose = False, stdmsgout = "stderr")

baseSeqFPCons = "GCAUUGGAGAUGGCAUUCCUCCAUUAACAAACCGCUGCGCCCGUAGCAGCUGAUGAUGCCUACAGA"
consListFP = GTFPUtils.ReadFPConstraintsFromFile("../Testing/TestData/tRNA/yeast.fa.cons")

(mfe, mfeDOTStruct) = GTFP.GetMFEStructure(baseSeqFPCons, consListFP)
print("MFE %1.3f => MFE DOT STRUCT\n \"%s\"\n\n" % (mfe, mfeDOTStruct))
\end{lstlisting}

\end{frame}

\begin{frame}[t,fragile]
\frametitle{GTFoldPython -- Example -- Find MFE and MFE structure}

\small\noindent
\textbf{External Python3 script source:}
\begin{lstlisting}[language=Python,basicstyle=\tiny\ttfamily,keywordstyle=\bfseries\color{green!40!black},
                   commentstyle=\itshape\color{purple!40!black},identifierstyle=\color{blue!63!green},
                   stringstyle=\color{orange},frame=none,keepspaces=true,numbers=left,xleftmargin=0.28cm]
import sys, os
from GTFoldPythonImportAll import *

GTFP.Init()
GTFP.Config(quiet = False, debugging = False, verbose = False, stdmsgout = "stderr")

baseSeqFPCons = "GCAUUGGAGAUGGCAUUCCUCCAUUAACAAACCGCUGCGCCCGUAGCAGCUGAUGAUGCCUACAGA"
consListFP = GTFPUtils.ReadFPConstraintsFromFile("../Testing/TestData/tRNA/yeast.fa.cons")

(mfe, mfeDOTStruct) = GTFP.GetMFEStructure(baseSeqFPCons, consListFP)
print("MFE %1.3f => MFE DOT STRUCT\n \"%s\"\n\n" % (mfe, mfeDOTStruct))
\end{lstlisting}

\medskip\hrule\medskip

\small\noindent
\textbf{Terminal output printed upon invoking the script above:}
\begin{lstlisting}[language=bash,basicstyle=\tiny\ttfamily,keywordstyle=\bfseries\color{green!40!black},
                   commentstyle=\itshape\color{purple!40!black},identifierstyle=\color{blue!63!green},
                   stringstyle=\color{orange},frame=none,keepspaces=true,numbers=left,xleftmargin=0.28cm]
MFE -17.200 => MFE DOT STRUCT 
"(((((((((.........))))...........(((((.......)))))....)))))......."
\end{lstlisting}

\end{frame}

\section{RNAStructViz}

\begin{frame}
\TitleBoxed{
     \Huge{\centerline{The RNAStructViz application}}
}
\end{frame}

\begin{frame}
\frametitle{RNAStructViz: Graphical base pairing analysis}
\begin{itemize} 

\item RNAStructViz was a project developed by 
	Professor Christine Heitsch and Dr.~S.~Cheney (SE, Google)
\item First project with the gtDMMB group (Summer of 2018):
	\begin{itemize}
	\item Modernize the C++ source
        \item Add support for enhanced graphics using the \texttt{cairo} library
	\item Re-write the dated build scripts 
	\item Improve and support the project in the long term 
	\end{itemize}
\item Key features of RNAStructViz: 
	\begin{itemize}
	\item Visualization of RNA secondary structures via arc diagrams
	\item Comparison of the common (distinctive) pairs between 
	      multiple structures at the same time 
	\end{itemize}

\end{itemize}

\end{frame}

\begin{frame}[fragile]
\frametitle{RNAStructViz Screenshot -- Loading sample structures I}

\begin{center}
\includegraphics[height=0.85\textheight]{presentation-images/RNAStructViz-LoadingSampleStructures-v1.png}
\end{center}

\end{frame}

\begin{frame}[fragile]
\frametitle{RNAStructViz Screenshot -- Loading sample structures II}

\begin{center}
\includegraphics[height=0.85\textheight]{presentation-images/RNAStructViz-LoadingSampleStructures-v2.png}
\end{center}

\end{frame}

\begin{frame}[fragile]
\frametitle{RNAStructViz Screenshot -- Arc diagram window}

\begin{center}
\includegraphics[height=0.88\textheight]{presentation-images/RNAStructViz-ArcDiagramWindow-OneStructure-v1.png}
\end{center}

\end{frame}

\begin{frame}
\frametitle{Arc diagram window -- Discussion}

\begin{itemize} 

\item The bases from position \texttt{\#1} to \texttt{\#LengthOfBaseSequenceString} at 
      equidistant spacings around a circle 
\item The sequentially numbered base pairs are ordered around the circle counter-clockwise 
      starting from the bottom 
\item An arc connecting paired bases is drawn within the circle 

\end{itemize}

\begin{center}
\includegraphics[height=0.63\textheight]{presentation-images/RNAStructViz-ArcDiagramWindow-OneStructure-v1.png}
\end{center}

\end{frame}

\begin{frame}[fragile]
\frametitle{Arc diagram window -- Zoom select}

\begin{center}
\includegraphics[height=0.87\textheight]{presentation-images/RNAStructViz-ArcDiagramWindow-OneStructure-v2_Zooming.png}
\end{center}

\end{frame}

\begin{frame}[fragile]
\frametitle{Arc diagram zoom -- Radial layout visualization}

\begin{center}
\includegraphics[height=0.87\textheight]{presentation-images/RNAStructViz-RadialLayout-Example.png}
\end{center}

\end{frame}

\begin{frame}[fragile]
\frametitle{Arc diagram zoom -- CT segment visualization}

\begin{center}
\includegraphics[height=0.87\textheight]{presentation-images/RNAStructViz-CTViewer-Example.png}
\end{center}

\end{frame}

\begin{frame}[fragile]
\frametitle{Arc diagram window -- Comparing multiple structures}

\begin{center}
\includegraphics[height=0.87\textheight]{presentation-images/RNAStructViz-ArcDiagramWindow-ThreeStructures.png}
\end{center}

\end{frame}

\begin{frame}[fragile]
\frametitle{RNAStructViz Screenshot -- Statistics window}

\begin{center}
\includegraphics[width=\textwidth]{presentation-images/RNAStructViz-StatsWindow-v1.png}
\end{center}

\end{frame}

\section{Conclusions}

\begin{frame}
\TitleBoxed{
     \Huge{\centerline{Wrapping up}}
}
\end{frame}

\begin{frame}
\frametitle{Summary of accomplishments with gtDMMB software I}
\begin{itemize} 

\item Success modernizing and enhancing the source code for 
      these projects in computational and mathematical biology 
\item Success in modernizing and extending build scripts to support installation on MacOS, 
      Linux and Unix-based systems 
\item A few of the software projects we worked on:
	\begin{itemize}
	\item RNAStructViz 
	\item GTFold (CMake for MacOS and Linux)
	\item GTFoldPython
	\end{itemize}

\end{itemize}

\end{frame}

\begin{frame}
\frametitle{Summary of accomplishments with gtDMMB software II}
\begin{itemize} 

\item Application note re-introducing our new work on RNAStructViz 
      published in \emph{Bioinformatics} in 2021
\item Sister RNA labs that helped with testing and/or use our software include: 
	\begin{itemize}
	\item Computational RNA Genomics Lab at University of California Davis
	\item Laederach Lab at the University of North Carolina at Chapel Hill
	\item Mathews Lab at the University of Rochester
	\end{itemize}

\end{itemize}

\end{frame}

\begin{frame}
\frametitle{Concluding remarks} 

     \Huge{\centerline{The End}}\smallskip
     \Large{\centerline{Questions?}}\smallskip
     \Large{\centerline{Comments?}}\smallskip
     \Large{\centerline{Feedback?}}\bigskip
     \Huge{\centerline{Thank you for your time!}} 

\end{frame}

\begin{frame}[fragile]
\frametitle{GTFoldPython -- Comparison -- Python C API code}

\begin{lstlisting}[language=C,basicstyle=\tiny\ttfamily,keywordstyle=\bfseries\color{green!40!black},
                   commentstyle=\itshape\color{purple!40!black},identifierstyle=\color{blue!63!green},
                   stringstyle=\color{orange},frame=none,keepspaces=true,numbers=left,xleftmargin=0.28cm]
PyObject * GetMFEStructure(const char *baseSeq, ConsListCType_t consList, int consLength) {
     /* Error checking omitted ... */
     MFEStructRuntimeArgs_t rtArgs;
     InitMFEStructRuntimeArgs(&rtArgs);
     rtArgs.baseSeq = baseSeq;
     SetRTArgsSequenceLength(rtArgs, strlen(baseSeq));
     if(ParseGetMFEStructureArgs(consList, consLength, &rtArgs) != GTFPYTHON_ERRNO_OK) {
          FreeMFEStructRuntimeArgs(&rtArgs);
	  return ReturnPythonNone();
     }
     if(InitGTFoldMFEStructureData(&rtArgs) != GTFPYTHON_ERRNO_OK) {
          FreeMFEStructRuntimeArgs(&rtArgs);
	  return ReturnPythonNone();
     }
     double mfe = ComputeMFEStructure(&rtArgs);
     if(GetLastErrorCode() != GTFPYTHON_ERRNO_OK) {
          return ReturnPythonNone();
     }
     if(WRITEAUXFILES) {
          ConfigureOutputFileSettings();
          save_ct_file(outputFile, baseSeq, mfe);
     }
     char *dbMFEStruct = ComputeDOTStructureResult(rtArgs.numBases);
     PyObject *mfeTupleRes = PrepareMFETupleResult(mfe, dbMFEStruct);
     Free(dbMFEStruct);
     FreeMFEStructRuntimeArgs(&rtArgs);
     FreeGTFoldMFEStructureData(rtArgs.numBases);
     if(mfeTupleRes == NULL) {
          return ReturnPythonNone();
     }
     return mfeTupleRes;
}
\end{lstlisting}

\end{frame}

\begin{frame}[fragile]
\frametitle{GTFoldPython -- Comparison -- Wrapper library code}

\begin{lstlisting}[language=Python,basicstyle=\tiny\ttfamily,keywordstyle=\bfseries\color{green!40!black},
                   commentstyle=\itshape\color{purple!40!black},identifierstyle=\color{blue!63!green},
                   stringstyle=\color{orange},frame=none,keepspaces=true,numbers=left,xleftmargin=0.28cm]
    ## Library initialization code: 
    if GTFPConfig.PLATFORM_DARWIN:
        GTFoldPython._libGTFoldHandle = ctypes.cdll.LoadLibrary("GTFoldPython.dylib")
    else:
        GTFoldPython._libGTFoldHandle = ctypes.PyDLL("GTFoldPython.so", 
	                                mode=ctypes.RTLD_GLOBAL, use_errno=True)
    @staticmethod
    def _WrapCTypesFunction(funcname, restype=None, argtypes=None):
        return GTFoldPython._libGTFoldHandle.__getattr__(funcname)
        
    @staticmethod
    def GetMFEStructure(baseSeq, consList = []):
        """Get the MFE and MFE structure (in DOTBracket structure notation)
        :param baseSeq: A string of valid bases (ATGU/X) 
        :param consList: A list of constraints on the MFE structure
        :return: A tuple (MFE as double, MFE structure as string in DOTBracket notation)
        :rtype: tuple
        """
        GTFoldPython._ConstructLibGTFold()
        resType = ctypes.py_object
        argTypes = [ GTFPTypes.CStringType, 
	             GTFPTypes.FPConstraintsListType(consList), 
		     ctypes.c_int ]
        libGTFoldFunc = GTFoldPython._WrapCTypesFunction("GetMFEStructure", resType, argTypes)
        (mfe, mfeStruct) = libGTFoldFunc(GTFPTypes.CString(baseSeq), 
	                                 GTFPTypes.FPConstraintsList(consList), 
                                         len(consList))
        return (float(mfe), str(mfeStruct))
\end{lstlisting}

\end{frame}

\end{document} 
