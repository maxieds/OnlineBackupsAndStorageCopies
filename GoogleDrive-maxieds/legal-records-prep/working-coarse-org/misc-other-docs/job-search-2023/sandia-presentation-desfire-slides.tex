%%%%%%%%%%%%%%%%%%%%%%%%%%%%%%%%%%%%%%%%%
% Beamer Presentation
% LaTeX Template
% Version 1.0 (10/11/12)
%
% This template has been downloaded from:
% http://www.LaTeXTemplates.com
%
% License:
% CC BY-NC-SA 3.0 (http://creativecommons.org/licenses/by-nc-sa/3.0/)
%
%%%%%%%%%%%%%%%%%%%%%%%%%%%%%%%%%%%%%%%%%

%----------------------------------------------------------------------------------------
%	PACKAGES AND THEMES
%----------------------------------------------------------------------------------------

\PassOptionsToPackage{prologue}{xcolor}
%\documentclass[notes,usenames,svgnames,dvipsnames,11pt]{beamer}
\documentclass[usenames,svgnames,dvipsnames,11pt]{beamer}

\usepackage{amsthm,amsmath,amssymb,amscd}
\usepackage{geometry}
\usepackage{longtable}

%\usepackage{CustomColors}

\definecolor{indiagreen}{rgb}{0.07,0.53,0.03}
\definecolor{GATechBlue}{rgb}{0.0,0.18823529411764706,0.3411764705882353}%{003057}
\definecolor{GATechGold}{rgb}{0.7019607843137254,0.6392156862745098,0.4117647058823529}%{B3A369​}
\definecolor{GATechBuzzGold}{rgb}{0.9176470588235294,0.6666666666666666,0.0}%{EAAA00}
\definecolor{SlideBGLight}{rgb}{0.902,0.918,0.933}

\mode<presentation> {

% The Beamer class comes with a number of default slide themes
% which change the colors and layouts of slides. Below this is a list
% of all the themes, uncomment each in turn to see what they look like.

%\usetheme{default}
\usetheme{AnnArbor}
%\usetheme{Antibes}
%\usetheme{Bergen}
%\usetheme{Berkeley}
%\usetheme{Berlin}
%\usetheme{Boadilla}
%\usetheme{CambridgeUS}
%\usetheme{Copenhagen}
%\usetheme{Darmstadt}
%\usetheme{Dresden}
%\usetheme{Frankfurt}
%\usetheme{Goettingen}
%\usetheme{Hannover}
%\usetheme{Ilmenau}
%\usetheme{JuanLesPins}
%\usetheme{Luebeck}
%\usetheme{Madrid}
%\usetheme{Malmoe}
%\usetheme{Marburg}
%\usetheme{Montpellier}
%\usetheme{PaloAlto}
%\usetheme{Pittsburgh}
%\usetheme{Rochester}
%\usetheme{Singapore}
%\usetheme{Szeged}
%\usetheme{Warsaw}

% As well as themes, the Beamer class has a number of color themes
% for any slide theme. Uncomment each of these in turn to see how it
% changes the colors of your current slide theme.

%\usecolortheme{albatross}
%\usecolortheme{beaver}
%\usecolortheme{beetle}
%\usecolortheme{crane}
%\usecolortheme{dolphin}
%\usecolortheme{dove}
%\usecolortheme{fly}
%\usecolortheme{lily}
%\usecolortheme{orchid}
%\usecolortheme{rose}
%\usecolortheme{seagull}
%\usecolortheme{seahorse}
%\usecolortheme{whale}
%\usecolortheme{wolverine}

%\setbeamertemplate{footline} % To remove the footer line in all slides uncomment this line
%\setbeamertemplate{footline}[page number] % To replace the footer line in all slides with a simple slide count uncomment this line

%\setbeamertemplate{navigation symbols}{} % To remove the navigation symbols from the bottom of all slides uncomment this line

\setbeamercolor*{structure}{bg=GATechBlue,fg=GATechGold}

\setbeamercolor*{palette primary}{use=structure,fg=white,bg=structure.fg}
\setbeamercolor*{palette secondary}{use=structure,fg=white,bg=GATechGold!85!black}
\setbeamercolor*{palette tertiary}{use=structure,fg=white,bg=GATechGold!70!black}
\setbeamercolor*{palette quaternary}{fg=white,bg=GATechGold!65!white}

\setbeamercolor{background canvas}{bg=SlideBGLight}

\setbeamercolor{frametitle}{bg=GATechBlue,fg=GATechBuzzGold}
\setbeamercolor*{titlelike}{bg=GATechBlue,fg=GATechBuzzGold}

\defbeamertemplate{itemize item}{bulletpoint}{\usebeamerfont*{itemize item enumitem}\raise1.05pt\hbox{\color{GATechGold!70!black}{$\blacktriangleright$}}}
\setbeamertemplate{items}[bulletpoint]

\setbeamercolor{section in toc}{fg=black}
\setbeamercolor{subsection in toc}{fg=black}

\setbeamercolor{bibliography item}{parent=palette primary}
\setbeamercolor{bibliography entry author}{fg=GATechBlue}
\setbeamercolor{bibliography entry title}{fg=GATechBlue}
\setbeamercolor{bibliography entry note}{fg=GATechBlue}

}

\usepackage{graphicx} % Allows including images
\usepackage{booktabs} % Allows the use of \toprule, \midrule and \bottomrule in tables
\usepackage{fancyvrb}
\usepackage{inconsolata}

\newcommand{\Iverson}[1]{\ensuremath{\left[#1\right]_{\delta}}} 

\DeclareMathOperator{\DGF}{DGF} 
\DeclareMathOperator{\ds}{ds} 
\DeclareMathOperator{\Id}{Id}
\DeclareMathOperator{\sq}{sq}

\newcommand{\ceiling}[1]{\ensuremath{\left\lceil #1 \right\rceil}} 
\newcommand{\ImportantMarker}{%\textcolor{GATechGold}{$\mathbf{\Leftarrow}$}\ 
                              \textcolor{GATechGold}{\textbf{[!! \underline{IMPORTANT} !!]}}}

\newcommand{\Floor}[2]{\ensuremath{\left\lfloor \frac{#1}{#2} \right\rfloor}}
\newcommand{\Ceiling}[2]{\ensuremath{\left\lceil \frac{#1}{#2} \right\rceil}}                              

\newcommand{\gkpSI}[2]{\ensuremath{\genfrac{\lbrack}{\rbrack}{0pt}{}{#1}{#2}}} 
\newcommand{\gkpSII}[2]{\ensuremath{\genfrac{\lbrace}{\rbrace}{0pt}{}{#1}{#2}}}

\newcommand{\TitleBoxed}[1]{
     \begin{beamercolorbox}[sep=8pt,center,shadow=true,rounded=true]{title}
          \usebeamerfont{title}#1\par%
     \end{beamercolorbox}
}

%\setbeamertemplate{note page}[plain]
\setbeamerfont{note page}{family*=pplx,size=\footnotesize} % Palatino for notes

%----------------------------------------------------------------------------------------
%	TITLE PAGE
%----------------------------------------------------------------------------------------

\title[Sandia 2022 -- Embedded DESFire]{
     A recent open source embedded implementation of 
     the DESFire specification designed for on-the-fly logging with 
     NFC based systems
} 

\author{Maxie Dion Schmidt} % Your name
\institute[] 
{
%Georgia Institute of Technology \\ 
%School of Mathematics \\ % Your institution for the title page
%Atlanta, GA 30318, USA \\ 
%\smallskip
\texttt{maxieds@gmail.com} \\ 
\url{http://people.math.gatech.edu/~mschmidt34} \\ 
\url{https://github.com/maxieds}
}
\date[Spring 2022]{Sandia National Labs \\ Technical Presentation \\ Spring 2022} % Date, can be changed to a custom date

\begin{document}

\begin{frame}
\titlepage % Print the title page as the first slide
\end{frame} 

%----------------------------------------------------------------------------------------
%	PRESENTATION SLIDES
%----------------------------------------------------------------------------------------

%------------------------------------------------

\section{Introduction} 

\begin{frame}
\frametitle{High-level overview -- NFC}
\begin{itemize} 

\item Near Field Communication (NFC) protocol over short-distance RFID @ 13.56MHz
      (wirless communication over a proximity of typically less than 10 centimeters)
\item NFC enables contactless data exchanges between passive tags (PICC) and active hosts (PCD)
\item Common in applications like physical authentication with door readers, university ID cards, 
      bus passes, to exchange credentials 
      renting bikes or motorized scooters, and to charge limited credit transactions to vending machines 
      and other virtual payment kiosks
\item Often encountered tag types include: MIFARE Classic, MIFARE Ultralight, NTAG and others over 
      standardized ISO protocols and wrapped instruction sets

\end{itemize}

\end{frame}

\begin{frame}
\frametitle{High-level overview -- DESFire tags and Chameleon Mini}

\begin{itemize} 

\item DESFire type cards provide modern cryptographic algorithms and 
      have a more sophisticated feature set than other common NFC tags
\item The Chameleon Mini (RevG) devices are used for pentesting and in 
      security applications as tag emulators and NFC data loggers
\item DESFire emulation support for the Chameleon Mini has been a frequently requested, 
      however complicated to deliver, feature for years
\item The first testing releases came together in the Fall of 2020 
\item More enhancements and improvements are made over the Spring of 2022 through funding of the project 
      at GA Tech

\end{itemize} 

\end{frame}

\begin{frame}
\frametitle{High-level overview -- Project big picture and takeaways}

\begin{itemize} 

\item \textbf{Significance:} First of its kind functional embedded proof-of-concept 
      DESFire stack that is freely available as 
      OSS to researchers, security experts and end users alike
\item \textbf{Limitations:} Small R\&D budget for testing and lack of standardized default 
      data transfer modes to ensure interoperability amongst door readers in applications 
\item \textbf{Key Challenges:} Proprietary specs (lack of documentation) 
      and tight memory restrictions working on an 
      embedded platform 

\end{itemize} 

\end{frame}

\begin{frame}
\TitleBoxed{
     \Huge{\centerline{Outline of topics}}
}
\end{frame}

\begin{frame}
\frametitle{Presentation outline -- Topics}
\begin{itemize} 

\item Origins of the project -- How I got involved developing software for NFC devices 
\item The Chameleon Mini device hardware profile and embedded software features
\item Overview of key features of the proprietary DESFire command set 
\item Key features and challenges in writing the embedded DESFire implmentation (with examples)

\end{itemize} 

\end{frame}

\section{Origins of the project}

\begin{frame}
\TitleBoxed{
     \Huge{\centerline{Origins of the project}}
}
\end{frame}

\begin{frame}
\frametitle{Origins of the project}

\begin{itemize}

\item I moved to GA Tech in 2017 as a Ph.D. student in the School of Math
\item Shortly after arriving on campus I was issued a student ID with an integrated DESFire EV1 tag
\item This was also around the time I had purchased my first developer grade NFC-enabled Android phone
\item I decided that I wanted the physical authentication to doors on campus to work not only with the 
      standard issue university ID card but also with my phone

\end{itemize}

\end{frame}

\begin{frame}
\frametitle{Origins of the project -- Initial exploration}

\begin{center}
\includegraphics[width=\textwidth]{presentation-images/AndroidHCEBuzzcardAppForAtpiQMobileDoorReaders.jpeg}
\end{center}

\end{frame}

\begin{frame}
\frametitle{Origins of the project -- Enter the Chameleon Mini}

\begin{itemize}

\item Exploration with Android OS application development and limitations of low-level NFC data exchange 
      transparency on the stock MotoDroid led me to seek external hardware to help reverse engineer the 
      bytes I would need to exchange from phone to door reader, and vice versa
\item Reading online led me to purchase a Chameleon Mini (RevG) device 

\end{itemize}

\bigskip
\minipage{0.38\textwidth}
\begin{center}
$\vcenter{\hbox{\includegraphics[width=\textwidth]{presentation-images/ChameleonMiniPhotoSchoolIDCardSmall.png}}}$
\end{center}
\endminipage\hfil
\minipage{0.56\textwidth}
\begin{center}
$\vcenter{\hbox{\includegraphics[width=\textwidth]{presentation-images/ChameleonLogsExample.png}}}$
\end{center}
\endminipage

\end{frame}

\begin{frame}
\frametitle{Origins of the project -- Chameleon Mini Live Debugger}

\begin{itemize}

\item Viewing the logging output and seeing that it could be printed over a serial USB terminal in realtime 
      made me start on another long-term OSS project 
\item The Chameleon Mini Live Debugger (CMLD) is an Android OS application that 
      facilitates controlling the Chameleon Mini devices and 
      features a window for viewing the live logging data it generates 
\item It has grown and been restyled and refactored over the years (since 2017)
\item The most recent experimental features in the CMLD include a hybrid built-in scripting language to 
      control and interface to the Chameleon Mini

\end{itemize}

\end{frame}

\begin{frame}
\frametitle{Origins of the project -- Early prototype of the CMLD}

\minipage{0.35\textwidth}
\begin{center}
\includegraphics[height=0.76\textheight]{presentation-images/CMLDInitialLoggerWindowPrototype.png}
\end{center}
\endminipage
\minipage{0.65\textwidth}
\begin{center}
$\vcenter{\hbox{\includegraphics[width=\textwidth]{presentation-images/AndroidStudioCMLDExample.png}}}$
\end{center}
\endminipage

\end{frame}

\section{Chameleon Mini Hardware}

\begin{frame}
\TitleBoxed{
     \Huge{\centerline{The Chameleon Mini device}}
}
\end{frame}

\begin{frame}
\frametitle{The Chameleon Mini device profile -- Hardware}

\begin{itemize} 

\item On-board integration of a modern AVR chip (\texttt{ATxmega128A4U}) 
\item Memory: 128Kb of FLASH, 8Kb of SRAM, and 2Kb of EEPROM spaces and 
      support for faster FRAM-based memory access
\item Accelerated hardware support for AES and DES cryptographic engines
\item Embedded firmware and flashable bootloader support to memory map the integrated RF hardware on the PCB
\item Serial data transfer over wired micro-USB 

\end{itemize} 

\end{frame}


\begin{frame}
\frametitle{The Chameleon Mini device profile -- Software}

\begin{itemize} 

\item Embedded OSS firmware and bootloader sources in C and ASM compiled with \texttt{avr-gcc} that are 
      flashed to the device over USB 
\item Convenient serial terminal that has a human-readable command set for easy on-the-fly configuration of 
      emulated tags 
\item Ability to act as a PICC, PCD or bidirectional NFC packet sniffer 
      depending on the active configuration set in one of the eight 
      8Kb sized partitions of the onboard (FRAM) memory 
\item Logging of time-stamped communication details and status events to internal FRAM memory or 
      LIVE mode printed to the serial USB 

\end{itemize} 

\end{frame}

\begin{frame}
\frametitle{Chameleon Mini Firmware Organization}

\begin{center}
\includegraphics[width=\textwidth]{presentation-images/ChameleonMiniFirmwareBlockDiagram}
\end{center}

\end{frame}

\section{DESFire NFC Tags} 

\begin{frame}
\TitleBoxed{
     \Huge{\centerline{DESFire tags}}
}
\end{frame}

\begin{frame}
\frametitle{Key Features}

\begin{itemize}
\item Multiple nested and semi-interoperable generations of DESFire tags: 
      Legacy Mifare DESFire, EV1, EV2, EV3 and Light variants 
\item Larger scale integrated memory storage sizes than most contactless NFC tags 
      (usually 2Kb, 4Kb or 8Kb if not larger these days) 
\item Standard use of modern cryptographic algorithms for 
      secure data exchange (legacy DES/3DES/AES-128/AES-256)
\item Data messages optionally padded with crytographically hashed bytes to ensure data 
      integrity over the physical interface using $2$-byte CRC checksums or $4$-byte MAC'ed trailers
\item Implementations are complicated by proprietary handling of most DESFire tag specs by the 
      manufacturers 
\end{itemize}

\end{frame}

\begin{frame}
\frametitle{Filesystem: Organization and internal storage types}

\begin{itemize}
\item Files grouped by allocations of the physical IC memory into 
      top-level subdirectories called applications indexed by unique 
      application identifier (AID)
\item \textbf{Native file types:} Standard data files (type $0$), backup data files (type $1$), 
      value files (type $2$), linear record files (type $3$), and cyclic record files (type $4$)
\item Each file has $2$-bytes of associated access rights to indicate one of 
      read/write/read and write/change.
\item Access permissions on the files provide more secure protections for storage of secret binary key data 
\end{itemize}

\end{frame}

\begin{frame}[fragile]
\frametitle{Commands and native instruction support}

\begin{itemize}
\item Formatting of raw data to communicate instructions is performed by sending 
      unpadded native commands or by communicating ISO standardized wrapped APDU messages
\item Think of APDU messages as an ``assembly language`` for NFC exchanges 
\end{itemize}
\medskip 
\hrule
\medskip

     \scriptsize
     \textbf{\large{PCD-to-PICC wrapped APDU data exchange format:}}
     \begin{center}
     \begin{tabular}{|c|c|c|c|c|l|c|} 
     \hline
     \textbf{CLA} & \textbf{INS} & $\mathbf{P_1}$ & $\mathbf{P_2}$ & $\mathbf{L_c}$ & 
     \textbf{Data Bytes} & $\mathbf{L_e}$ \\ 
     \hline
     \texttt{0x90} & command code & \texttt{0x00} & \texttt{0x00} & variable length of data & 
     command data & \texttt{0x00} \\ 
     \hline
     \end{tabular}
     \end{center}
     \bigskip 
     \textbf{\large{PICC-to-PCD format:}}
     \begin{center}
     \begin{tabular}{|l|c|c|} 
     \hline 
     \textbf{Data Bytes} & \textbf{SW1} & \textbf{SW2 (Status)} \\ 
     \hline 
     DESFire command response data & 
     \texttt{0x91} & \texttt{0xYY} \\ 
     \hline 
     \end{tabular} 
     \end{center} 

\end{frame}

\begin{frame}[fragile]
\frametitle{Supported command codes -- Some examples}

     \scriptsize
     \begin{longtable}{|l|c|p{0.575\linewidth}|} 
     \hline 
     \textbf{Command Long Name} & \textbf{INS} & \textbf{Description} \\ 
     \hline 
     \texttt{AUTHENTICATE} & \texttt{0x0A} & Legacy mode authentication \\ 
     \texttt{AUTHENTICATE\_ISO} & \texttt{0x1A} & ISO authentication with 3DES \\ 
     \texttt{AUTHENTICATE\_AES} & \texttt{0xAA} & Standard AES authentication \\ 
     \texttt{CHANGE\_KEY\_SETTINGS} & \texttt{0x54} & Modify PICC master key properties \\ 
     \texttt{SET\_CONFIGURATION} & \texttt{0x5C} & Used to configure DESFire card or application specific attributes \\ 
     \texttt{CHANGE\_KEY} & \texttt{0xC4} & Changes the key data stored on the PICC \\ 
     \texttt{GET\_KEY\_VERSION} & \texttt{0x64} & Returns the active key version stored on the PICC \\ 
     \texttt{CREATE\_APPLICATION} & \texttt{0xCA} & Creates new applications by unique AID \\ 
     \texttt{DELETE\_APPLICATION} & \texttt{0xDA} & Non-restorable deletion operation \\ 
     \texttt{GET\_APPLICATION\_IDS} & \texttt{0x6A} & Returns a list of all AID codes stored on the PICC \\ 
     \texttt{FREE\_MEMORY} & \texttt{0x6E} & Returns the total free memory on the tag in bytes \\ 
     \texttt{GET\_DF\_NAMES} & \texttt{0x6D} & Obtain the ISO7816-4 DF names associated with the tag \\ 
     \texttt{GET\_KEY\_SETTINGS} & \texttt{0x45} & Get permissions data and format for PICC and application master keys \\ 
     \texttt{SELECT\_APPLICATION} & \texttt{0x5A} & Select a specific application by AID for further access \\ 
     \hline     
     \end{longtable}

\hrule\medskip
\noindent\it\scriptsize 
A more complete listing of supported commands is found in the source code; 
alternately, the data sheets linked in the bibliography archived by 
incidentally lucky NFC researchers give command syntax and argument details 
precisely.

\end{frame}

\begin{frame}[fragile]
\frametitle{Data exchanges with the Chameleon DESFire configuration}

\begin{Verbatim}[numbers=left,fontsize=\scriptsize,frame=single]
>>> Select Application By AID:
    -> 90 5a 00 00 03 00 00 00 | 00 
    <- 91 00 
>>> Start AES Authenticate:
    -> 90 aa 00 00 01 00 00 
    <- 54 b8 9e fe 19 9b c6 a5 | fd 8f 00 be c1 23 99 c0 | 91 af 
    -> 90 af 00 00 10 df a0 79 | 13 59 ac 4c 75 5f 81 69 | 
       bc 9c 3e c6 7e 00 
    <- a9 e2 79 42 11 63 9c 14 | 07 b3 02 2f 2e 4b 2e c5 | 91 00 
>>> Get AID List From Device:
    -> 90 6a 00 00 00 00 
    <- 77 88 99 01 00 34 91 00 
>>> CreateApplication command:
    -> 90 ca 00 00 05 77 88 99 | 0f 03 00 
    <- 91 de 
>>> Get AID List From Device:
    -> 90 6a 00 00 00 00 
    <- 77 88 99 01 00 34 91 00 
\end{Verbatim}

\medskip\hrule\medskip
\noindent\it\scriptsize 
More complete examples of data exchanges using these commands are found in the 
\texttt{LibNFC} testing code within the 
Chameleon mini main firmware on the \textit{GitHub/emsec/ChameleonMini} repository 
(in the \texttt{Software/DESFireLibNFCTesting} directory). 

\end{frame}

\section{OSS Embedded DESFire}

\begin{frame}
\TitleBoxed{
     \huge{\centerline{An Embedded Open Source DESFire}}
     \huge{\centerline{Stack for the Chameleon Mini}}
}
\note[item]{No notes: Title slide only}
\end{frame}

\begin{frame}
\frametitle{Extensions of the firmware sources to support DESFire tags}

\begin{itemize}
\item New native AES support using hardware acceleration support
\item Extensions of prior work to add hardware based DES and 3DES support to the firmware 
\item Minor changes to the codec layer of the firmware to support DESFire tags 
\item Enhancements and bug fixes to the \texttt{LIVE} logging functionality of the 
      Chameleon RevG devices 
\item New default customized extension of the Chameleon terminal commands to 
      enhance DESFire configuration support for users (see next slide)
\end{itemize}

\end{frame}

\begin{frame}[fragile]
\frametitle{Terminal configuration of DESFire emulation support}

\minipage{0.4\textwidth}
\begin{Verbatim}[fontsize=\scriptsize]
> CONFIG=MF_DESFIRE
> DF_SETHDR=ATS 0675F7B102
> UID=2377000B99BF98

---------------------------------

DF_SETHDR=ATS xxxxxxxxxx
DF_SETHDR=HardwareVersion xxxx
DF_SETHDR=SoftwareVersion xxxx
DF_SETHDR=BatchNumber xxxxxxxxxx
DF_SETHDR=ProductionDate xxxx
\end{Verbatim}
\endminipage
\minipage{0.3\textwidth}
\begin{center}
\includegraphics[height=0.78\textheight]{presentation-images/CMLDDESFireLoggerWindow.png}
\end{center}
\endminipage
\minipage{0.3\textwidth}
\begin{center}
\includegraphics[height=0.78\textheight]{presentation-images/CMLDEDESFireSpecificCmdButtonInterface.png}
\end{center}
\endminipage

\end{frame}

\begin{frame}[fragile]
\frametitle{DESFire emulation support -- Anti-collision loop}

\begin{center}
\minipage{0.7\textwidth}
\begin{Verbatim}[fontsize=\scriptsize]
     NFC reader: SCM Micro / SCL3711-NFC&RW opened

     Sent bits:     26 (7 bits)
     Received bits: 03  44  
     Sent bits:     93  20  
     Received bits: 88  23  77  00  dc  
     Sent bits:     93  70  88  23  77  00  dc  4b  b3  
     Received bits: 04  
     Sent bits:     95  20  
     Received bits: 0b  99  bf  98  b5  
     Sent bits:     95  70  0b  99  bf  98  b5  2f  24  
     Received bits: 20  
     Sent bits:     e0  50  bc  a5  
     Received bits: 75  77  81  02  80  
     Sent bits:     50  00  57  cd  

     Found tag with
     UID: 2377000b99bf98
     ATQA: 4403
     SAK: 20
     ATS: 75  77  81  02  80
\end{Verbatim}
\endminipage
\end{center}

\end{frame}

\begin{frame}[fragile]
\frametitle{Terminal screenshot of the DESFire source code}

\begin{center}
\includegraphics[width=0.9\textwidth]{presentation-images/DESFireSourceCode-TerminalScreenshot-Example.png}
\end{center}

\end{frame}

\begin{frame}
\frametitle{ISO authentication with the PM3 (Support added in 2022)}

\begin{center}
\includegraphics[height=0.88\textheight]{presentation-images/DESFire-PM3-ISOAuthentication.png}
\end{center}

\end{frame}

\begin{frame}
\frametitle{Challenges with the implementation during development}

\begin{itemize}
\item Approximately six to eight months of active development were required to complete the inital stages of the project 
\item Forced by local embedded system 
      constraints to carefully optimize and organize our use of 
      the embedded AVR memory to resolve 
      insufficient memory type exceptions 
\item The speedup in computations for AES and 3DES operations 
      provides an order of magnitude improvement 
      compared to existing OSS libraries for AVR chips 
\item A complicated nested, quasi-linked pointer based structure was required to efficiently store the 
      filesystem entries and tag accounting metadata 
\end{itemize}

\end{frame}

\section{Credits and Concluding Discussion}

\begin{frame}
\TitleBoxed{
     \Huge{\centerline{Concluding Remarks}}
}
\end{frame}

\begin{frame}
\frametitle{Summary and accomplishments}

\begin{itemize} 

\item Chameleon DESFire configuration can exchange data with most external 
      USB NFC readers, LibNFC and is compatible with PM3 devices
\item Back-of-the-envelope estimate is that DESFire support on the Chameleon Mini is by far the 
      most requested single feature from users \\ 
      \begin{itemize}
      \item \textbf{GitHub/emsec/ChameleonMini watchers:} 129
      \item \textbf{GitHub/emsec/ChameleonMini stars:} 1344
      \item \textbf{GitHub/emsec/ChameleonMini forks:} 342
      \end{itemize}
\item Users of the official CMLD applications on Google Play Store peaked at 
      $\approx 500$ (free version) and $\approx 50$ (paid version) internationally -- 
      Not too shabby considering 
      that the KAOS devices are priced at $\approx 110$ USD 
      (historically higher at $\approx 165+$ USD when the RevG was state-of-the-art hardware)
      \begin{itemize}
      \item \textbf{GitHub/maxieds/ChameleonMiniLiveDebugger stars:} 76
      \item \textbf{GitHub/maxieds/ChameleonMiniLiveDebugger forks:} 15
      \end{itemize}

\end{itemize} 

\end{frame}

\begin{frame}
\frametitle{Funding sources and support for the project}

\begin{itemize} 

\item Initial sources for the DESFire Chameleon firmware are due to 
      Dmitry Janushkevich (\textbf{@devzzo}) (2017)
\item Professor Josephine Yu in the School of Math at GA Tech in the US
\item Georgia Tech for supporting me as a RA in the Spring of 2022 through the university's 
      COVID-19 relief funding 
\item The original Kasper and Oswald (KAOS) developers of the Chameleon Mini hardware and software 
\item David Oswald from the University of Birmingham in the UK

\end{itemize} 

\end{frame}

\section{Bibliography} 

\begin{frame}[t,allowframebreaks] 
\frametitle{References} 

\footnotesize 
\begin{thebibliography}{10}

\bibitem{ANDROID-HCE-DESFIRE} 
Android HCE DESFire: A software implementation of Desfire in an Android app. 
\url{https://github.com/jekkos/android-hce-desfire}

\bibitem{EMSEC-CMINI-FIRMWARE} 
Chameleon Mini Firmware (authoritative sources). 
\url{https://github.com/emsec/ChameleonMini}
%\url{https://github.com/emsec/ChameleonMini/pull/287}

\bibitem{ISOIEC-STANDARDS}
ISO/IEC 14443, 15693 and 7816 Standards. 
Identification Cards - Contactless Integrated Circuit Cards. 
\url{www.iso.org}

\bibitem{KAOS-CHAMELEON}
Kasper T., von Maurich I., Oswald D., Paar C. (2011) 
Chameleon: A Versatile Emulator for Contactless Smartcards. In: Rhee KH., Nyang D. (eds) 
Information Security and Cryptology - ICISC 2010. ICISC 2010.
Lecture Notes in Computer Science, vol 6829. Springer, Berlin, Heidelberg. 
\url{https://doi.org/10.1007/978-3-642-24209-0_13} 

\bibitem{KAOS-CHAMELEON-SHORT-SLIDES}
Kasper, T. and Oswald, D. Presentation slides on the history of the Chameleon Mini devices. 
\url{https://raw.github.com/wiki/emsec/ChameleonMini/Images/160110_ChameleonMini_history_smaller.pdf}

\bibitem{LIBFREEFARE} 
LibFreeFare: A convenience API for NFC cards manipulations on top of LibNFC. 
\url{https://github.com/nfc-tools/libfreefare}

\bibitem{LIBNFC}
LibNFC: A platform independent NFC library. 
\url{https://github.com/nfc-tools/libnfc}

\bibitem{MICROCHIP-ATXMEGA128A4U}
Microchip. ATxmega1284U Data Sheet. 
\url{https://ww1.microchip.com/downloads/en/DeviceDoc/ATxmega128-64-32-16A4U-DataSheet-DS40002166A.pdf} 

\bibitem{NXP-MFDESFIRE-DATASHEET-2008}
NXP Semiconductors. MIFARE DESFire Functional specification. 
Publicly available MF3ICD81 datasheet (2008). 
\url{https://tinyurl.com/kwweanp9}

\bibitem{PHILIPSCO-MFDESFIRE-DATASHEET-2004}
Philips Semicondictors. 
Mifare DESFire: Contactless multi-application IC with DES and 3DES security. 
Publicly available MF3-IC-D40 datasheet (2004). 
\url{https://tinyurl.com/5era3dx2}

\bibitem{PROXMARK3}
Proxmark III. A Radio Frequency IDentification Tool. 
\url{http://www.proxmark.org}

\bibitem{CMINIFW-DESFIRE-DEVSOURCES}
Schmidt, M. D. 
Chameleon Mini DESFire Stack (development sources). 
\url{https://github.com/maxieds/ChameleonMiniDESFireStack}

\bibitem{MDS-CMLD} 
Schmidt, M. D. 
Chameleon Mini Live Debugger. 
\url{https://github.com/maxieds/ChameleonMiniLiveDebugger}

\end{thebibliography}

\end{frame} 

%----------------------------------------------------------------------------------------

\end{document} 
