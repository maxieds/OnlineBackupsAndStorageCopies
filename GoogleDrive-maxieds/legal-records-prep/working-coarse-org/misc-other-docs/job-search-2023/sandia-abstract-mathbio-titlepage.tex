\documentclass[reqno,11pt]{article} 

\usepackage{amsmath,amsthm,amssymb,amscd}
\usepackage{url}
\usepackage{dsfont}
\usepackage{mathabx}

\usepackage{listings}
\lstset{
basicstyle=\small\ttfamily,
columns=flexible,
breaklines=true,
extendedchars=true,
literate={•}{$\bullet$}1 {▄}{$\blacksquare$}1 {□}{$\square$}1
}

\usepackage{url}
\usepackage{graphicx}

\newtheorem{theorem}{Theorem}[section]
\newtheorem{definition}{Definition}[section]
\newtheorem{lemma}{Lemma}[section]
\newtheorem{prop}{Proposition}[section]
\newtheorem{conj}{Conjecture}[section]
\newtheorem{claim}{Claim}[section]
\newtheorem{procedure}{Procedure}[section]

\theoremstyle{remark}
\newtheorem{problem}{Problem Statement}[section]
\newtheorem{example}{Example}[section]


\usepackage[top=0.5in,bottom=0.65in,left=0.75in,right=0.75in]{geometry} 
\usepackage[square,sort,comma,numbers]{natbib}

\DeclareMathOperator{\Int}{int}

\newcommand{\PP}[1]{\ensuremath{\mathbb{P}\left(#1\right)}} 
\newcommand{\Iverson}[1]{\ensuremath{\left[#1\right]_{\delta}}}
\newcommand{\Floor}[2]{\ensuremath{\left\lfloor \frac{#1}{#2} \right\rfloor}}

\title{Software work in mathematical biology}
\author{Maxie Dion Schmidt} 
\date{\today}

\setlength{\parskip}{0.35em}
\setlength{\parindent}{0in}

\usepackage{tocloft}
\setlength{\cftsecindent}{0.25em}
\renewcommand{\numberline}[1]{#1~}
\newcommand{\xmark}{\ensuremath{\mathcal{X}}}

\UseRawInputEncoding

\begin{document}

\maketitle 
\medskip\hrule\medskip 

\begin{abstract}
\noindent
Ribonucleic acid (RNA) and similar biological molecules are easily sequenced. 
Cutting-edge research on RNA structures and sequencing is utilized in medical 
applications from cancer research and therapy, mRNA-based vaccines, and may in time 
grow to provide the central role of diagnostics and therapeutic treatments to 
remedy many uncured diseases. 
The biological function of RNA sequences is dictated by the physical structures they fold into. 
Predicting, comparing and visualizing those primary and secondary structures from their 
base sequence data is an important and computationally complex scientific problem. 
Moreover, as the cellular roles for RNA molecules continue to grow, 
so does the importance of gaining functional insight from structural analyses. 
In this technical presentation we will discuss my applied software 
work in mathematical biology. 
This work was supported working with the gtDMMB group of Professor Christine Heitsch 
at Georgia Tech from 2018--2022. 
The projects we developed have resulted in cross-platform open source software for the 
biology community. 

We focus on two primary open source software projects updated over the last few years. 
The first application, RNAStructViz, 
is a GUI-based application written in C++ for visualizing RNA secondary structures. 
RNA structures can be input in several standardized text formats for base sequences. 
The application then displays arc diagram views of base pairings across up to three 
folded structures concurrently. 
These arc diagrams are displayed on screen using vector graphics libraries and can be exported 
in publication quality image formats to SVG and PNG. 
The second application is GTFold, a fast, scalable multicore software for predicting RNA secondary structure that 
is one to two orders of magnitude faster than the de facto standard programs 
and which achieves comparable accuracy of prediction. 
This software was originally written in C++ as a suite of command line utilities. 
Over the last few years, we have worked to develop the GTFoldPython bindings and wrapper libraries 
to use the functionality provided by GTFold in Python3. 

\smallskip\smallskip
\noindent
\textbf{RNAStructViz software documentation:} \url{https://github.com/gtDMMB/RNAStructViz/wiki} \\ 
\textbf{RNAStructViz application note (2021):} \url{https://doi.org/10.1093/bioinformatics/btab197} \\ 
\textbf{GTFoldPython software documentation:} \url{https://github.com/gtDMMB/GTFoldPython/wiki} \\ 
\textbf{GTFold software:} \url{https://github.com/gtDMMB/gtfold} \\ 
\textbf{GTFold publication (2012):} \url{https://doi.org/10.1186/1756-0500-5-341} \\ 

\end{abstract}

\end{document}
